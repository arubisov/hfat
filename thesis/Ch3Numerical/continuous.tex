\chapter{Stochastic Optimal Control}
\fxnote{write intro section.}
Hello and welcome to this limited edition chapter on stochastic optimal control. Read on if you dare subject yourself to the infinite wisdom contained herein.

\section{Continuous Time}
\fxnote{better lead in to this section.}
Below we list the processes involved in the optimization problem:

\begin{tabular}{lll}
Imbalance \& Midprice Change & $\bZ_t = (\rho_t, \Delta_t) $ & CTMC with generator $G$ \\
Imbalance & $\rho_t = \bZ_t^{(1)}$ & LOB imbalance at time $t$ \\
Midprice & $S_t$ & evolves according to CTMC \\
Midprice Change & $\Delta_t = \bZ_t^{(2)} = S_t - S_{t-s}$ & $s$ a pre-determined interval \\
Bid-Ask half-spread & $\xi_t$ & constant? \\
LOB Shuffling & $N_t$ & Poisson with rate $\lambda(\bZ_t)$ \\
$\Delta\text{Price:}$ LOB shuffled & $\{ \eta_{0,z}, \eta_{1,z}, \dots \} \sim F_{z}$ & i.i.d. with $z = (k,l)$, where \\
& & $k \in \{ \text{\#bins} \}, \; l \in \{ \Delta \$ \}$ \\
Other Agent MOs & $K^{\pm}_t$ & Poisson with rate $\mu^{\pm}(\bZ_t)$ \\
LO posted depth & $\delta^{\pm}_t$ & our $\cF$-predictable controlled processes \\
Our LO fill count & $L^{\pm}_t$ & $\cF$-predictable, non-Poisson \\
Our MOs & $M^{\pm}_t$ & our controlled counting process \\
Our MO execution times & $\btau^\pm = \{ \tau^\pm_k : k = 1, \dots \}$ & increasing sequence of $\cF$-stopping times \\
Cash & $X^{\btau, \delta}_t$ & depends on our processes $M$ and $\delta$ \\
Inventory & $Q^{\btau, \delta}_t$ & depends on our processes $M$ and $\delta$
\end{tabular}

$L^{\pm}_t$ are counting processes (not Poisson) satisfying the relationship that if at time $t$ we have a sell limit order posted at a depth $\delta^{-}_t$, then our fill probability is $e^{-\kappa \delta^{-}_t}$ conditional on a buy market order arriving; namely:
\begin{align}
\P [\d L^{-}_t = 1 \, | \, \d K^+_t = 1] & = e^{-\kappa \delta^{-}_t} \\
\P [\d L^{+}_t = 1 \, | \, \d K^{-}_t = 1] & = e^{-\kappa \delta^{+}_t}
\end{align}

The midprice $S_t$ evolves according to the Markov chain and hence is Poisson with rate $\lambda$ and jump size $\eta$, both of which depend on the state of the Markov chain. This Poisson process is all-inclusive in the sense that it accounts for any midprice change, be it from executions, cancellations, or order modifications with the LOB. Thus, the stock midprice $S_t$ evolves according to the SDE:
\begin{equation}\label{eq:stockprocess}
\d S_t = \eta_{N_{t^-},Z_{t^-}} \d N_t
\end{equation}
and additionally satisfies:
\begin{equation}\label{eq:stockintegral} 
S_t = S_{t_0} + \int\limits_{t_0 + s}^t \Delta_u \du
\end{equation}

In executing market orders, we assume that the size of the MOs is small enough to achieve the best bid/ask price, and not walk the book. Hence, our cash process evolves according to:
\begin{equation}\label{eq:cashprocess}
\begin{split}
\d X^{\btau, \delta}_t = 	&\underbrace{(S_t + \xi_t + \delta^{-}_t) \d L^{-}_t}_{\text{sell limit order}} - \underbrace{(S_t - \xi_t - \delta^{+}_t) \d L^{+}_t}_{\text{buy limit order}} \\
						&+ \underbrace{(S_t - \xi_t) \d M^{-}_t}_{\text{sell market order}} - \underbrace{(S_t + \xi_t) \d M^{+}_t}_{\text{buy market order}}
\end{split}
\end{equation}

Based on our execution of limit and market orders, our inventory satisfies:
\begin{equation}\label{eq:inventory}
Q^{\btau, \delta}_0 = 0, \qquad Q^{\btau, \delta}_t = L^+_t + M^+_t - L^-_t - M^-_t
\end{equation}

We define a new variable for our net present value (NPV) at time $t$, call it $W^{\btau, \delta}_t$, and hence $W^{\btau, \delta}_T$ at terminal time $T$ is our `terminal wealth'. In algorithmic trading, we want to finish the trading day with zero inventory, and assume that at the terminal time $T$ we will submit a market order (of a possibly large volume) to liquidate remaining stock. Here we do not assume that we can receive the best bid/ask price - instead, the price achieved will be $(S - \mathrm{sgn}(Q)\xi - \alpha Q)$, where $\mathrm{sgn}(Q)\xi$ represents crossing the spread in the direction of trading, and $\alpha Q$ represents receiving a worse price linearly in $Q$ due to walking the book. Hence, $W^{\btau, \delta}_t$ satisfies:
\begin{equation}\label{eq:terminalwealth}
W^{\btau, \delta}_t = \underbrace{\vphantom{\left( Q^{\btau, \delta}_t) \right)}X^{\btau, \delta}_t}_{\text{cash}}+ \underbrace{Q^{\btau, \delta}_t \left( S_t - \mathrm{sgn}(Q^{\btau, \delta}_t)\xi_t \right)}_{\text{book value of assets}} - \underbrace{\alpha \left( Q^{\btau, \delta}_t \right)^2}_{\text{liquidation penalty}}
\end{equation}
The set of admissible trading strategies is the product of the sets $\mathcal{T}$, the set of all $\cF$-stopping times, and $\cA$, the set of all $\cF$-predictable, bounded-from-below depths $\delta$. We only consider $\delta^\pm \geq 0$, since at $\delta=0$ our fill probability is $e^{-\kappa\delta}=1$, so we cannot increase the chance of our limit order being filled by posting any lower than at-the-touch; doing so would only diminish our profit.

For deriving an optimal trading strategy via dynamic programming, I will consider the performance criteria that maximizes terminal wealth. With the above notation, the performance criteria function can be written 
\begin{equation}\label{eq:performancecriteria}
H^{\btau, \delta}(t,x,s,\bz,q) = \E \left[ W_T^{\btau, \delta} \right]\vphantom{\int\limits_t^T}
\end{equation}
And the value function, in turn, is given by
\begin{equation}\label{eq:valuefunction}
H(t,x,s,\bz,q) = \sup\limits_{\btau \in \mathcal{T}_{[t,T]}} \sup\limits_{\delta \in \cA_{[t,T]}} H^{\btau, \delta}(t,x,s,\bz,q)
\end{equation}

\subsection{Dynamic Programming}
The following theorems establish the dynamic programming method we will utilize to solve this type of problem:
\begin{thm}[\cite{STA4505}] \textbf{Dynamic Programming Principle for Optimal Stopping and Control.}
If an agent's performance criteria for a given admissible control $\bu$ and admissible stopping time $\tau$ are given by
\[ H^{\tau, \bu}(t,\bx) = \E_{t,\bx} [ G (X^{\bu}_\tau)] \]
and the value function is
\[ H(t,\bx) = \sup\limits_{\tau \in \mathcal{T}_{[t,T]}} \sup\limits_{\bu \in \cA_{[t,T]}} H^{\tau, \bu}(t,\bx) \]
then the value function satisfies the Dynamic Programming Principle
\begin{equation}\label{eq:thmDPP}
H(t,\bx) = \sup\limits_{\tau \in \mathcal{T}_{[t,T]}} \sup\limits_{\bu \in \cA_{[t,T]}} \E_{t,\bx} \left[ G (X^{\bu}_\tau) \indicator_{\tau<\theta} + H(\theta, X^{\bu}_\theta)\indicator_{\tau \geq \theta} \right]
\end{equation}
for all $(t,\bx) \in [0,T] \times \R^m$ and all stopping times $\theta \leq T$.
\end{thm}
\begin{thm}[\cite{STA4505}] \textbf{Dynamic Programming Equation for Optimal Stopping and Control.}
Assume that the value function $H(t,\bx)$ is once differentiable in $t$, all second-order derivatives in $\bx$ exist, and that \myfunction{G}{\R^m}{\R} is continuous. Then $H$ solves the quasi-variational inequality
\begin{equation}\label{eq:thmDPE}
0 = \max \left\lbrace \partial_t H + \sup \limits_{\bu \in \cA_t} \cL^{\bu}_t H \; ; \; G - H \right\rbrace
\end{equation}
on $\mathcal{D}$, where $\mathcal{D} = [0,T] \times \R^m$.
\end{thm}

\subsection{Maximizing Terminal Wealth (Continuous)}
In this section we solve the DPE that results from using the maximal terminal wealth performance criteria. The quasi-variational inequality in equation \ref{eq:thmDPE} can be interpreted as follows: the max operator is choosing between posting limit orders or executing market orders; the second term, $G-H$, is the stopping region and represents the value derived from executing a market order; and the first term is the continuation region, representing the value of posting limit orders. We'll use the shorthand $H(\cdot) = H(t,x,s,\bz,q)$ and solve for $\d H$ inside the continuation region, hence $\d M^{\pm} = 0$, in order to then extract out the infinitesimal generator.

\begin{align}
\d H (t,x,s,\bz,q) & = \sum\limits_i \partial_{x_i} H \dx_i \\
& = \partial_t H \dt + \partial_{K^{\pm}} H \d {K^{\pm}} + \partial_{\bZ} H \d {\bZ} \\
\begin{split}
& = \partial_t H \dt + \bigl\lbrace e^{ -\kappa \delta^{-}} \E \bigl[ H(t,x+(s+\xi+\delta^-),s,\bz,q-1) - H(\cdot) \bigr] \bigr\rbrace \d K^+\\
& \hphantom{{}={} \partial_t H \dt} + \bigl\lbrace e^{ -\kappa \delta^{+}} \E \bigl[ H(t,x-(s-\xi-\delta^+),s,\bz,q+1) - H(\cdot) \bigr] \bigr\rbrace \d K^-\\
& \hphantom{{}={}} + \sum_{\bj} \E \left[ H(t,x,s+\eta_{0,\bj},\bj,q) - H(\cdot) \right] \d Z_{\bz,\bj}
\end{split}
\end{align}
\fxnote{section on compensated processes, especially for Markov Chain}

Substitute in the following identities for the compensated processes
\begin{align} 
\d M^{\pm} & = \d \tilde{K}^{\pm} + \mu^{\pm}(\bz) \dt \\
\d Z_{\bz,\bj}  & = \d \tilde{Z}_{\bz,\bj}  + G_{\bz,\bj} \dt 
\end{align}
\begin{align}
\begin{split}
{}\phantom{\d H (t,x,s,\bz,q)} & = \partial_t H \dt + \biggl\lbrace \mu^+(\bz) e^{ -\kappa \delta^{-}} \E \bigl[ H(t,x+(s+\xi+\delta^-),s,\bz,q-1) - H(\cdot) \bigr] \\
& \hphantom{{}={} \partial_t H \dt +} + \mu^-(\bz) e^{ -\kappa \delta^{+}} \E \bigl[ H(t,x-(s-\xi-\delta^+),s,\bz,q+1) - H(\cdot) \bigr] \\
& \hphantom{{}={}} + \sum_{\bj} G_{\bz,\bj} \E \left[ H(t,x,s+\eta_{0,\bj},\bj,q) - H(\cdot) \right]  \biggr\rbrace \dt \\
& \hphantom{{}={} \partial_t H \dt} + \bigl\lbrace e^{ -\kappa \delta^{-}} \E \bigl[ H(t,x+(s+\xi+\delta^-),s,\bz,q-1) - H(\cdot) \bigr] \bigr\rbrace \d \tilde{K}^+\\
& \hphantom{{}={} \partial_t H \dt} + \bigl\lbrace e^{ -\kappa \delta^{+}} \E \bigl[ H(t,x-(s-\xi-\delta^+),s,\bz,q+1) - H(\cdot) \bigr] \bigr\rbrace \d \tilde{K}^-\\
& \hphantom{{}={}} + \sum_{\bj} \E \left[ H(t,x,s+\eta_{0,\bj},\bj,q) - H(\cdot) \right] \d \tilde{Z}_{\bz,\bj}
\end{split}
\end{align}
From which we can see that the infinitesimal generator is given by
\begin{equation}\label{eq:infgen}
\begin{split}
\cL^{\delta}_t H & = \mu^+(\bz) e^{ -\kappa \delta^{-}} \E \bigl[ H(t,x+(s+\xi+\delta^-),s,\bz,q-1) - H(\cdot) \bigr] \\
& \quad + \mu^-(\bz) e^{ -\kappa \delta^{+}} \E \bigl[ H(t,x-(s-\xi-\delta^+),s,\bz,q+1) - H(\cdot) \bigr] \\
& \quad + \sum_{\bj} G_{\bz,\bj} \E \left[ H(t,x,s+\eta_{0,\bj},\bj,q) - H(\cdot) \right]
\end{split}
\end{equation}
Now, our DPE has the form
\begin{equation}\label{eq:DPEmaxprofit}
\begin{split}
0 = \max \biggl\lbrace \partial_t H + \sup \limits_{\bu \in \cA_t} \cL^{\bu}_t H \; ; \; & H(t,x-(s+\xi), s, \bz, q+1) - H(\cdot) \; ; \\
&  H(t,x+(s-\xi), s, \bz, q-1) - H(\cdot) \biggr\rbrace
\end{split}
\end{equation}
with boundary conditions
\begin{align}
H(T, x, s, \bz, q) & = x + q(s - \mathrm{sgn}(q)\xi) - \alpha q^2 \\
H(t, x, s, \bz, 0) & = x
\end{align}
The three terms over which we are maximizing represent the continuation regions and stopping regions of the optimization problem. The first term, the continuation region, represents the limit order controls; the second and third terms, each a stopping region, represent the value gain from executing a buy market order and a sell market order, respectively.

Let's introduce the ansatz $H(\cdot) = x + q(s - \mathrm{sgn}(q)\xi) + h(t,\bz,q)$. The first two terms are the wealth plus book value of assets, hence a mark-to-market of the current position, whereas the $h(t,\bz,q)$ captures value due to the optimal trading strategy. The corresponding boundary conditions on $h$ are
\begin{align}
h(T, \bz, q) & = - \alpha q^2 \\
h(t, \bz, 0) & = 0
\end{align}
Substituting this ansatz into equation \ref{eq:infgen}, we get:
\begin{equation}
\begin{split}
\cL^{\delta}_t H & = \mu^+(\bz) e^{ -\kappa \delta^{-}} \bigl[ \delta^- + \xi [ 1 + \mathrm{sgn}(q-1) + q(\mathrm{sgn}(q) - \mathrm{sgn}(q-1) ) ] \\
& \hphantom{ {}={} \mu^+(\bz) e^{ -\kappa \delta^{-}} \bigl[ } + h(t,\bz,q-1) - h(t,\bz,q) \bigr] \\
& \quad + \mu^-(\bz) e^{ -\kappa \delta^{+}} \bigl[ \delta^+ + \xi [ 1 - \mathrm{sgn}(q+1) + q(\mathrm{sgn}(q) - \mathrm{sgn}(q+1) ) ] \\
& \hphantom{\quad + \mu^-(\bz) e^{ -\kappa \delta^{+}} \bigl[} + h(t,\bz,q+1) - h(t,\bz,q) \bigr] \\
& \quad + \sum_{\bj} G_{\bz,\bj} \left[q \E [\eta_{0,\bj}] +h(t,\bj,q) - h(t,\bz, q) \right]
\end{split}
\end{equation}
We can further simplify the factors of $\xi$; for example, in the case of the $\delta^+$ term, we can write
\begin{align*}
1 - \mathrm{sgn}(q+1) + q(\mathrm{sgn}(q) - \mathrm{sgn}(q+1) )  & = 
1 - (- \indicator_{q \leq -2} + \indicator_{q \geq 0}) +  \indicator_{q = -1} \\
& = 1 + ( \indicator_{q \leq -1} -  \indicator_{q \geq 0} )\\
& = 2 \cdot \indicator_{q \leq -1}
\end{align*}
This gives us the simplified infinitesimal generator term
\begin{equation}
\begin{split}
\cL^{\delta}_t H & = \mu^+(\bz) e^{ -\kappa \delta^{-}} \bigl[ \delta^- + 2 \xi \cdot \indicator_{q \geq 1} + h(t,\bz,q-1) - h(t,\bz,q) \bigr] \\
& \quad + \mu^-(\bz) e^{ -\kappa \delta^{+}} \bigl[ \delta^+ + 2 \xi \cdot \indicator_{q \leq -1} + h(t,\bz,q+1) - h(t,\bz,q) \bigr] \\
& \quad + \sum_{\bj} G_{\bz,\bj} \left[q \E [\eta_{0,\bj}] +h(t,\bj,q) - h(t,\bz, q) \right]
\end{split}
\end{equation}
In the DPE, the first term requires finding the supremum over all $\delta^\pm$ of the infinitesimal generator. For this we can set the partial derivatives with respect to both $\delta^+$ and $\delta^-$ equal to zero to solve for the optimal posting depth, which we denote with a superscript asterisk. For $\delta^+$ we get:
\begin{align}
0 & = \partial_{\delta^+} \biggl[ e^{ -\kappa {\delta^+}^*} \bigl[ {\delta^+}^* +  2 \xi \cdot \indicator_{q \leq -1} + h(t,\bz,q+1) - h(t,\bz,q) \bigr] \biggr] \\
& = -\kappa e^{ -\kappa {\delta^+}^*} \bigl[ {\delta^+}^* +2 \xi \cdot \indicator_{q \leq -1} + h(t,\bz,q+1) - h(t,\bz,q) \bigr] + e^{ -\kappa {\delta^+}^*} \\
& = e^{ -\kappa \delta^{+}} \bigl[ -\kappa \bigl( {\delta^+}^* +2 \xi \cdot \indicator_{q \leq -1} + h(t,\bz,q+1) - h(t,\bz,q) \bigr) + 1 \bigr] \\
\intertext{Since $e^{ -\kappa {\delta^+}^*}>0$, the term inside the square braces must be equal to zero:}
0 & = -\kappa \bigl( {\delta^+}^* +2 \xi \cdot \indicator_{q \leq -1} + h(t,\bz,q+1) - h(t,\bz,q) \bigr) + 1 \\
{\delta^+}^* & = \frac{1}{\kappa} - 2 \xi \cdot \indicator_{q \leq -1} - h(t,\bz,q+1) + h(t,\bz,q) 
\end{align}

Recalling that our optimal posting depths are to be non-negative, we thus find that the optimal buy limit order posting depth can be written in feedback form as
\begin{equation}\label{eq:optbuydepthfeedback}
{\delta^+}^* = \max \left\lbrace 0 \; ; \; \frac{1}{\kappa} - 2 \xi \cdot \indicator_{q \leq -1} - h(t,\bz,q+1) + h(t,\bz,q) \right\rbrace
\end{equation}
We can follow similar steps to solve for the optimal sell limit order posting depth
\begin{equation}\label{eq:optselldepthfeedback}
{\delta^-}^* = \max \left\lbrace 0 \; ; \; \frac{1}{\kappa} - 2 \xi \cdot \indicator_{q \geq 1} - h(t,\bz,q-1) + h(t,\bz,q) \right\rbrace
\end{equation}
Turning our attention to the stopping regions of the DPE, we can use the ansatz to simplify the expressions:
\begin{align}
\begin{split}
& H(t,x-(s+\xi), s, \bz, q+1) - H(\cdot) \\
& \qquad = x - s - \xi + (q+1)(s - \sgn(q+1)\xi) + h(t, \bz, q+1) \\
& \hphantom{\qquad {}={}} - \bigl[ x + q(s - \sgn(q)\xi) + h(t,\bz,q) \bigr]
\end{split} \\
& \qquad = -\xi \bigl[ (q+1)\sgn(q+1) - q\sgn(q) + 1 \bigr] + h(t, \bz, q+1) - h(t,\bz,q)  \\
& \qquad = - 2 \xi \cdot \indicator_{q \geq 0} + h(t, \bz, q+1) - h(t,\bz,q)
\intertext{and similarly,}
& H(t,x+(s-\xi), s, \bz, q-1) - H(\cdot) = -2 \xi \cdot \indicator_{q \leq 0} + h(t, \bz, q-1) - h(t,\bz,q)
\end{align}
Substituting all these results and simplifications into the DPE, we find that $h$ satisfies
\begin{equation}\label{eq:PDEcase1}
\begin{split}
0 = \max \biggl\lbrace & \partial_t h + \mu^+(\bz) e^{ -\kappa {\delta^-}^*} \left( {\delta^-}^* + 2\xi \indicator_{q \geq 1} + h(t,\bz,q-1) - h(t,\bz,q) \right)  \\
& \quad + \mu^-(\bz) e^{ -\kappa {\delta^+}^*} \left( {\delta^+}^* + 2 \xi \cdot \indicator_{q \leq -1} + h(t,\bz,q+1) - h(t,\bz,q) \right) \\
& \quad + \sum_{\bj} G_{\bz,\bj} \left[ ql \E \left[ \eta_{0,\bj} \right] + h(t,\bj,q) - h(t,\bz,q) \right] \; ; \\
& -2 \xi \cdot \indicator_{q \geq 0} + h(t, \bz, q+1) - h(t,\bz,q)   \; ; \\
& -2 \xi \cdot \indicator_{q \leq 0} + h(t, \bz, q-1) - h(t,\bz,q)  \biggr\rbrace
\end{split}
\end{equation}
Looking at the simplified feedback form in the stopping region, we see that a buy market order will be executed at time $\tau^+_q$ whenever
\begin{equation}\label{eq:buyMOfeedbackcase1}
h(\tau^+_q, \bz, q+1) - h(\tau^+_q,\bz,q) = 2 \xi \cdot \indicator_{q \geq 0}
\end{equation}
and a sell market order whenever
\begin{equation}\label{eq:sellMOfeedbackcase1}
h(\tau^+_q, \bz, q-1) - h(\tau^+_q,\bz,q) = 2 \xi \cdot \indicator_{q \leq 0}
\end{equation}
Consider than when our inventory is positive, we can purchase a stock at $s+\xi$, but it will be marked-to-market at $s-\xi$, resulting in a value difference of $2\xi$. With negative inventory, we will still purchase at $s_\xi$, but will now also value at $s+\xi$ because our overall position is still negative, producing no value difference. In particular, with negative inventory, we will execute a buy market order so long as it does not change our value function; and with zero or positive inventory, only if it increases the value function by the value of the spread. The opposite holds for sell market orders. Together, these indicate a penchant for using market orders to drive inventory levels back toward zero when it has no effect on value, and using them to gain extra value only when the expected gain is equal to the size of the spread. This is reminiscent of what we saw in the exploratory data analysis: if a stock is worth $S$, we can purchase it at $S+\xi$ and immediately be able to sell it at $S-\xi$, at a loss of $2 \xi$; this was the most significant source of loss in the naive trading market order strategy. Hence we need to expect our value to increase by at least $2\xi$ when executing market orders for gain.

The variational inequality in \eqref{eq:PDEcase1} yields that whilst in the continuation region, we instead have
\begin{align}
h(\tau^+_q, \bz, q+1) - h(\tau^+_q,\bz,q) & \leq 2 \xi \cdot \indicator_{q \geq 0} \\
h(\tau^+_q, \bz, q-1) - h(\tau^+_q,\bz,q) & \leq 2 \xi \cdot \indicator_{q \leq 0}
\end{align}
Taken together, these inequalities yield
\begin{align}
-2\xi \cdot \indicator_{q \geq 0} \leq h(t,\bz,q) - h(t,\bz,q+1) \leq 2\xi \cdot \indicator_{q \leq -1} \label{eq:MOineq1}\\
-2\xi \cdot \indicator_{q \leq 0} \leq h(t,\bz,q) - h(t,\bz,q-1) \leq 2\xi \cdot \indicator_{q \geq 1} \label{eq:MOineq2}
\end{align}
or alternatively,
\begin{align}
h(t,\bz,q) \leq h(t,\bz,q+1) \leq h(t,\bz,q) + 2\xi, \qquad q \geq 0 \\
h(t,\bz,q) \leq h(t,\bz,q-1) \leq h(t,\bz,q) + 2\xi, \qquad q \leq 0 
\end{align}
\fxnote{insert the little bubbles with sell and buy at the inequatlity signs. sell, buy, buy, sell left right top down.}

Recalling the boundary condition $h(t,\bz,0) = 0$, this tells us that the function $h$ is non-negative everywhere. Furthermore, noting the feedback form of our optimal buy limit order depth given in equation \eqref{eq:optbuydepthfeedback}, together with the inequalities in \eqref{eq:MOineq1} and \eqref{eq:MOineq2}, we obtain bounds on our posting depths given by
\begin{align}
{\delta^+}^* & = \frac{1}{\kappa} - 2 \xi \cdot \indicator_{q \leq -1} - h(t,\bz,q+1) + h(t,\bz,q) \\
& \geq \frac{1}{\kappa} - 2 \xi \cdot \indicator_{q \leq -1} - 2 \xi \cdot \indicator_{q \geq 0} \\
& = \frac{1}{\kappa} - 2\xi \\
{\delta^+}^* & \leq \frac{1}{\kappa} - 2 \xi \cdot \indicator_{q \leq -1} + 2 \xi \cdot \indicator_{q \leq -1} \\
& = \frac{1}{\kappa}
\end{align}
Combined with the identical conditions on the sell depth, we have the conditions
\begin{equation}\label{eq:deltaslowerboundcase1}
\boxed{ \frac{1}{\kappa} - 2\xi  \leq {\delta^\pm}^* \leq \frac{1}{\kappa} }
\end{equation}
A possible interpretation of the unexpected upper bound on the posting depth is that if the calculated buy (resp. sell) depth is `sufficiently' large so as to indicate a disposition against buying (resp. selling), then it is actually optimal to sell (resp. buy) instead.