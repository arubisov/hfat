\chapter{Exploratory Data Analysis}

\section{Modelling Imbalance: Continuous Time Markov Chain}
The aim of this research project is to utilize the LOB volume imbalance $I(t)$ in an algorithmic trading application; hence, a suitable choice of model for $I(t)$ must be made. Rather than modelling imbalance directly as a real-valued process, an alternative approach, and that which is utilized herein, is to discretize the imbalance value $I(t)$ into subintervals, or bins, and fit the resulting process to a continuous-time Markov chain.

The following definitions and properties are adapted from \cite{STAT455}:
\begin{defn}
A continuous-time stochastic process $\left\lbrace X(t) \; | \; t \geq 0 \right\rbrace$ with state space $S$ is called a \emph{continuous-time Markov chain (CTMC)} if it has the Markov property; namely, that 
\begin{equation}
\P \left[ X(t) = j \; | \; X(s) = i, X(t_{n-1} = i_{n-1}, \dots , X(t_1) = i_1 \right] = \P \left[ X(t) = j \; | \; X(s) = i \right]
\end{equation}
where for any integer $n \geq 1$, $0 \leq t_1 \leq \dots \leq t_{n-1} \leq s \leq t$ is any non-decreasing sequence of $n+1$ times, and $i_1,\dots,i_{n-1}, i, j \in S$ are any $n+1$ states.
\end{defn}
\begin{defn} A CTMC $X(t)$ is \emph{time homogeneous} if for any $s \leq t$ and any states $i,j \in S$,
\begin{equation}
\P \left[ X(t) = j \; | \; X(s) = i \right] = \P \left[ X(t-s) = j \; | \; X(0) = i \right]
\end{equation}
\end{defn}
\begin{defn}
The key quantities that determine a CTMC $X(t)$ are the \emph{transition rates} $q_{ij}$, which specify the rate at which $X$ jumps from state $i$ to $j$. Conditional on leaving state $i$, $X$ transitions to state $j$ with \emph{conditional transition probability} $p_{ij}$. The amount of time that $X$ spends in state $i$, called the \emph{holding time}, is exponentially distributed with rate $v_i$. These quantities are related by:
\begin{align}
v_i & = \sum_{\substack{j \in S \\ j \neq i}} q_{ij} \\
q_{ij} & = v_i \cdot p_{ij} \\
p_{ij} & = \frac{q_{ij}}{v_i}
\end{align} 
\end{defn}
\begin{defn}
A CTMC $X(t)$ has an \emph{infinitesimal generator matrix} $\mat{G}$, whose entries are
\begin{align}
g_{ij} &= q_{ij}, \qquad i \neq j \\
g_{ii} & = -v_i
\end{align}
If $X(t)$ has transition probabilities $P_{ij}(t) = \P \left[ X(t) = j \; | \; X(0) = i \right]$ and matrix $\mat{P}(t) = \lbrace P_{ij}(t) \rbrace$, then $\mat{P}(t)$ and $\mat{G}$ are related by
\begin{align}
\dot{\mat{P}}(t) & = \mat{G} \cdot \mat{P}(t) \\
\mat{P}(t)  &= e^{\mat{G}t} \label{eq:CTMCPG}
\end{align}
\end{defn}
\fxnote{confirm we're using left stochastic or righ stochastic matrices? a few formulas depend on this.}

Conditional on $Z(t) = k$, we assume the arrival of buy and sell market orders follow independent Poisson processes with intensities $\lambda_k^\pm$, where $\lambda_k^+$ ($\lambda_k^-$) is the rate of arrivals of market buys (resp. sells). Such processes are called \textit{Markov-modulated Poisson processes}, as the Poisson intensities are themselves stochastic processes determined by the state of the Markov chain.

In the sections that follow, I derive maximum likelihood estimations for the parameters of the CTMC, and evaluate the fit of the model to the data.

\section{Maximum Likelihood Estimate of a Markov-modulated Poisson Process}

\subsection{Maximum Likelihood Estimation of \texorpdfstring{$G$}{G}}

Let $\mat{G}$ be the generator matrix for a CTMC $Z(t)$ with state space $K$. From observations, e.g. the fictional events in the timeline given in Figure \ref{introtimeline}, we want to estimate the entries of $\mat{G}$. Since the holding time in a given state $i$ has pdf $f(t;v_i) = v_i e^{-v_i t}$, the likelihood function (allowing for repetition of terms) would therefore be:
\begin{align}
\mathcal{L}(\mat{G}) &= (v_{i} e^{-v_{i}(\tau_2 - \tau_1)} p_{ij}) (v_{j} e^{-v_{j}(\tau_3 - \tau_2)} p_{ji}) (v_{i} e^{-v_{i}(\tau_4 - \tau_3)} p_{ik}) \dots \\
&= \prod\limits_{i=1}^{K} \prod\limits_{i \neq j} (v_{i}p_{ij})^{N_{ij}(T)} e^{-v_{i}H_i(T)} \\
&= \prod\limits_{i=1}^{K} \prod\limits_{i \neq j} (q_{ij})^{N_{ij}(T)} e^{-v_{i}H_i(T)}
\intertext{where:}
N_{ij}(T) & \equiv \mbox{number of transitions from regime $i$ to $j$ up to time $T$} \nonumber \\
H_{i}(T) & \equiv \mbox{holding time in regime $i$ up to time $T$} \nonumber
\intertext{So that the log-likelihood becomes:} 
\ln \mathcal{L}(\mat{G}) & = \sum\limits_{i=1}^{K} \sum\limits_{i \neq j} \left[ N_{ij}(T) \ln(q_{ij}) - v_{i} H_i(T) \right] \\
&= \sum\limits_{i=1}^{K} \sum\limits_{i \neq j} \left[ N_{ij}(T) \ln(q_{ij}) - \left( \sum\limits_{i \neq k} q_{ik} H_i(T) \right) \right]
\end{align}
To get a maximum likelihood estimate $\hat{q}_{ij}$ for transition rates and therefore the matrix $\mat{G}$, we take the partial derivative of $\ln \mathcal{L}(\mat{G})$ and set it equal to zero:
\begin{equation}
\dfrac{\partial \ln \mathcal{L}(\mat{G})}{\partial q_{ij}} = \dfrac{N_{ij}(T)}{q_{ij}} - H_i(T) = 0
\end{equation}
\begin{equation}
\Rightarrow \hat{q}_{ij} = \dfrac{N_{ij}(T)}{H_i(T)}
\end{equation}

\subsection{Maximum Likelihood Estimation of \texorpdfstring{$\lambda^{\pm}_k$}{lpmk}}

Now we want to derive an estimate for the intensity of the Poisson process of market order arrivals conditional on being in state $k$. We'll look first at just the buy market orders for some regime $k$, as the sell case is identical. Let the buy market order arrival times be indexed by $b_i$. Since we're assuming that the arrival process is Poisson with the same intensity throughout trials, we can consider the inter-arrival time of events conditional on being in state $k$. Then the MLE derivation follows just as for the generator matrix:
\begin{align}
\mathcal{L}(\lambda^{+}_k ; b_1, \dots, b_N) &= \prod\limits_{i=2}^{N} \lambda^{+}_k e^{-\lambda^{+}_k (b_{i} - b_{i-1})} \\
&= (\lambda^{+}_k)^{N^{+}_k(T)} e^{-\lambda^{+}_k H_k(T)}
\intertext{where:}
N^{+}_{k}(T) & \equiv \mbox{number of market order arrivals in regime $k$ up to time $T$} \nonumber \\
H_{k}(T) & \equiv \mbox{holding time in regime $k$ up to time $T$} \nonumber
\intertext{So that the log-likelihood becomes:} 
\ln \mathcal{L}(\lambda^{+}_k) & = N^{+}_k(T) \ln(\lambda^{+}_k) -\lambda^{+}_k H_k(T)
\intertext{And the ML estimate for $\hat{\lambda}^{+}_k$ is:} 
\dfrac{\partial \ln\mathcal{L} }{\partial \lambda^{+}_k} & = 
\dfrac{N^{+}_k(T)}{\lambda^{+}_k} - H_k(T) = 0
\end{align}
\begin{equation}
\Rightarrow \hat{\lambda}^{+}_k = \dfrac{N^{+}_k(T)}{H_k(T)}
\end{equation}

\section{2-Dimensional CTMC}
Next we consider a CTMC that jointly models the imbalance bin and the price change over a subsequent interval. That is, the CTMC models the joint distribution $(I(t), \Delta S(t))$ where $I(t) \in \lbrace 1,2,\dots,\#_{bins} \rbrace$ is the bin corresponding to imbalance averaged over the interval $[t-\Delta t_I, t]$, and $\Delta S(t) = \sgn(S(t+\Delta t_S)-S(t)) \in \lbrace -1, 0, 1 \rbrace$.  The pair $(I(t), \Delta S(t))$ is then reduced into one dimension with a simple encoding which we will denote $\varphi(I(t),S(t))$; for example, using 3 bins:

\begin{table}[H]
\centering
\ra{1.2}
\begin{tabular}{@{}rrrcrrrcrrr@{}}
\toprule
$Z(t)$ & Bin $I(t)$ & $\Delta S(t)$ & \phantom{abc} & $Z(t)$ & Bin $I(t)$ & $\Delta S(t)$ & \phantom{abc} & $Z(t)$ & Bin $I(t)$ & $\Delta S(t)$ \\
\cmidrule{1-3} \cmidrule{5-7} \cmidrule{9-11}
1 & Bin 1 & $<0$ && 4 & Bin 1 & $0$ && 7 & Bin 1 & $>0$ \\
2 & Bin 2 & $<0$ && 5 & Bin 2 & $0$ && 8 & Bin 2 & $>0$ \\
3 & Bin 3 & $<0$ && 6 & Bin 3 & $0$ && 9 & Bin 3 & $>0$ \\
\bottomrule
\end{tabular}
\caption{$\varphi(I(t),S(t))$: 1-Dimensional Encoding of 2-Dimensional CTMC}
\end{table}

It is crucial to note that the value $\Delta S(t)$ contains the price change from time $t$ over the \textit{future} $\Delta t_S$ seconds - hence in real-time one cannot know the state of the Markov Chain. However, the analytic results do prove enlightening: from the resulting timeseries we estimate a generator matrix $\mat{G}$, and transform it into a one-step transition probability matrix $\mat{P} = e^{\mat{G}\Delta t_I}$. The entries of $\mat{P}$ are the conditional probabilities 
\begin{align}
\mat{P}_{ij} & = \mathbb{P}\left[ \varphi( I_{[t-\Delta t_I, t]}, \Delta S_{[t,t+\Delta t_S]}) = j \; | \; \varphi( I_{[t-2\Delta t_I, t-\Delta t_I]}, \Delta S_{[t-\Delta t_I, t]} ) = i \right] \label{eq:POneStepUgly} \\
\intertext{which can be expressed semantically as}
& = \mathbb{P}\left[ \varphi( \rho_{curr}, \Delta S_{future}) = j \; | \; \varphi( \rho_{prev}, \Delta S_{curr} ) = i \right] \label{eq:POneStepNice} \\
\intertext{Since we can easily decode the 1-dimensional Markov state back into two dimensions, we can think of $\mat{P}$ as being four-dimensional and re-write its entries as}
& = \mathbb{P}\left[ \rho_{curr} = i,  \Delta S_{future} = j \; | \; \rho_{prev} = k, \Delta S_{curr} = m \right] \\
& = \mathbb{P}\left[ \rho_{curr} = i,  \Delta S_{future} = j \; | \; B \right]
\end{align}
where we're using the shorthand $B = (\rho_{prev} \in k, \Delta S_{curr} \in m)$ to represent the states in the previous timestep. Applying Bayes' Rule:
\begin{equation}\label{eq:POneStepBayes}
\mathbb{P}\left[ \Delta S_{future} \in j \; | \; B, \rho_{curr} \in i \right] = \dfrac{\mathbb{P}\left[ \rho_{curr} \in i, \Delta S_{future} \in j \; | \; B \right]}{\mathbb{P}\left[ \rho_{curr} \in i \; | \; B \right]}
\end{equation}
where the right-hand-side numerator is each individual entry of the one-step probability matrix $\mat{P}$, and the denominator can be computed from $\mat{P}$ by:
\begin{equation}\label{eq:POneStepBayesDenom}
\mathbb{P}\left[ \rho_{curr} \in i \; | \; B \right] = \sum\limits_j \mathbb{P}\left[ \rho_{curr} \in i,  \Delta S_{future} \in j \; | \; B \right]
\end{equation}
This result is of great interest to us: the left-hand-side value is the probability of seeing a given price change over the immediate future time interval conditional on past imbalances and the most recent price change, and therefore allows us to predict future price moves. We'll denote by $\mat{Q}$ the matrix containing all values given by \eqref{eq:POneStepBayes}.

The following $\mat{Q}$ matrix was obtained using data for \texttt{MMM} from 2013-05-15, averaging imbalance timewindow $t_I = 1000\text{ms}$, $K=3$ imbalance bins, and price change timewindow $t_S = 1000\text{ms}$:
\begin{table}[H]
\centering
\ra{1.2}
\begin{tabular}{@{} *{10}{r} @{}}
\toprule
& \multicolumn{3}{c}{$\Delta S_{curr} < 0$} & \multicolumn{3}{c}{$\Delta S_{curr} = 0$} & \multicolumn{3}{c}{$\Delta S_{curr} > 0$} \\
\cmidrule(lr){2-4} \cmidrule(lr){5-7} \cmidrule(lr){8-10}
&  $\rho_{n} = 1$ & $\rho_{n} = 2$ & $\rho_{n} = 3$ & $\rho_{n} = 1$ & $\rho_{n} = 2$ & $\rho_{n} = 3$ & $\rho_{n} = 1$ & $\rho_{n} = 2$ & $\rho_{n} = 3$ \\
\midrule
\multicolumn{10}{l}{$\Delta S_{future} < 0$} \\
$\rho_{n-1} = 1$ & \bf 0.53 & 0.15 & 0.12 & 0.05 & 0.10 & 0.14 & 0.08 & 0.13 & 0.14 \\
$\rho_{n-1} = 2$ & 0.10 & \bf 0.58 & 0.14 & 0.07 & 0.04 & 0.10 & 0.13 & 0.06 & 0.12 \\
$\rho_{n-1} = 3$ & 0.08 & 0.12 & \bf 0.52 & 0.09 & 0.06 & 0.03 & 0.11 & 0.10 & 0.05 \\[0.6ex]
\multicolumn{10}{l}{$\Delta S_{future} = 0$} \\
$\rho_{n-1} = 1$ & 0.41 & 0.75 & 0.78 & \bf 0.91 & 0.84 & 0.79 & 0.42 & 0.79 & 0.77 \\
$\rho_{n-1} = 2$ & 0.79 & 0.36 & 0.71 & 0.83 & \bf 0.92 & 0.82 & 0.75 & 0.37 & 0.78 \\
$\rho_{n-1} = 3$ & 0.79 & 0.74 & 0.40 & 0.81 & 0.83 & \bf 0.91 & 0.70 & 0.76 & 0.39 \\[0.6ex]
\multicolumn{10}{l}{$\Delta S_{future} > 0$} \\
$\rho_{n-1} = 1$ & 0.06 & 0.10 & 0.09 & 0.04 & 0.06 & 0.07 & \bf 0.50 & 0.09 & 0.09 \\
$\rho_{n-1} = 2$ & 0.10 & 0.06 & 0.15 & 0.10 & 0.04 & 0.08 & 0.12 & \bf 0.57 & 0.10 \\
$\rho_{n-1} = 3$ & 0.13 & 0.14 & 0.08 & 0.10 & 0.11 & 0.05 & 0.19 & 0.14 & \bf 0.56 \\
\bottomrule
\end{tabular}
\caption{The $\mat{Q}$ matrix: conditional probabilities of future price changes, conditioned on current imbalance, current price change, and previous imbalance.}
\label{tbl:Qmatrix}
\end{table}

Immediately evident from $\mat{Q}$ is that in most cases we are expecting no price change. In fact, the only cases in which the probability of a price change is $>0.5$ show evidence of \textit{momentum}; for example, the way to interpret the value in row 1, column 1 is: if $\rho_{prev} = \rho_{curr} = 1$ and previously we saw a downward price change, then we expect to again see a downward price change. In fact, the best way to summarize the matrix is:
\begin{equation}\label{eq:EDAKeyInsight}
\mathbb{P} \left[ \Delta S_{future} = \Delta S_{curr} \; | \; \rho_{curr} = \rho_{prev} \right] > 0.5
\end{equation} 

\section{Cross-Validation of CTMC Parameters}
To cross-validate the CTMC calibration, we conduct a time-homogeneity test similar to that done in \cite{Tan02}. The null hypothesis is given by \cite{Weiss10}:
\begin{equation}
H_0 = \forall i,j \in S \; : \; \exists q_{ij} \in \R^+ \; : \; q_{ij}(t) \equiv q_{ij} \forall t \in [0,T]
\end{equation}
whereas the alternative hypothesis states that transition rates/probabilities are time-dependent. To test the hypothesis, we fix an imbalance averaging time $\Delta t_I$, number of imbalance bins, and calculate the MLE estimate of the infinitesimal generator matrix $\mat{G}$ on the full timeseries. For a chosen error threshold $\epsilon$, we use the relationship in \eqref{eq:CTMCPG} to calculate the number of timesteps $n_{conv}$ of size $\Delta t_I$ such that
\begin{equation}\label{eq:crossvalidnconv}
|| \mat{P}\left((n_{conv}+1)\Delta t_I \right) - \mat{P} \left( n_{conv}\Delta t_I \right) || < \epsilon
\end{equation}
This value $n_{conv}$ determines the size of the cross-validation timewindow into which to partition the full timeseries, yielding $K$ equal subintervals of length $n$. For comparison, we also partitioned the timeseries into 8, 4, and 2 equal intervals. For each ``removed series'' $k \in \{ 1,\dots,K \}$, we recalibrate a CTMC generator matrix $\mat{G}_{k}$. Finally, we test whether the one-step transition probabilities $p_{ij}^k$ contained in $\mat{P}_k \left(\Delta t_I \right)$ are statistically different from those of the full period. The asymptotically equivalent test statistic to the likelihood ratio test statistic is:
\begin{equation}
D = -2 \ln (\mathcal{L})  = 2 \sum_k \sum_{i,j} n_{i,j}^k \left[ \ln(p_{ij}^k)  - \ln(p_{ij})   \right]
\end{equation}
where $n_{ij}^k$ is the number of observed transitions from state $i$ to $j$ in subinterval $k$. This test statistic has a $\chi^2$ distribution with $(K-1)(3 \cdot \#_{bins})(3 \cdot \#_{bins} - 1)$. The tests were run for each ticker for each trading day of 2013, and averaged over the year. The following table summarizes the $p$-value scores for the tests:
\begin{table}[H]
\centering
\ra{1.2}
\begin{tabular}{@{}rrrrrr@{}}
\toprule
 &  & \multicolumn{4}{c}{subintervals} \\ 
\cmidrule{3-6}
$\Delta t_I$ & $n_{conv}$ & $n_{conv}$ & 8 & 4 & 2 \\
\midrule
\multicolumn{6}{c}{\texttt{FARO}} \\
$\#_{bins}=3$ &&&&& \\
100ms &  4933 & 0.000 & 0.000 & 0.000 & 0.003 \\
1000ms &  727 & 0.000 & 0.002 & 0.001 & 0.005 \\
10000ms & 149 & 0.000 & 0.005 & 0.010 & 0.017 \\
$\#_{bins}=5$ &&&&& \\
100ms &  6450 & 0.000 & 0.001 & 0.002 & 0.004 \\
1000ms &  941 & 0.000 & 0.001 & 0.003 & 0.006 \\
10000ms & 187 & 0.000 & 0.000 & 0.000 & 0.005 \\
\multicolumn{6}{c}{\texttt{NTAP}} \\
$\#_{bins}=3$ &&&&& \\
100ms & 1320 & 0.000 & 0.000 & 0.000 & 0.000 \\
1000ms & 237 & 0.000 & 0.000 & 0.000 & 0.000 \\
10000ms & 72 & 0.000 & 0.006 & 0.003 & 0.007 \\
$\#_{bins}=5$ &&&&& \\
100ms & 1777 & 0.000 & 0.000 & 0.000 & 0.000 \\
1000ms & 308 & 0.000 & 0.001 & 0.000 & 0.001 \\
10000ms & 87 & 0.000 & 0.000 & 0.002 & 0.010 \\
\bottomrule
\end{tabular} \hfill
\begin{tabular}{@{}rrrrrr@{}}
\toprule
 &  & \multicolumn{4}{c}{subintervals} \\ 
\cmidrule{3-6}
$\Delta t_I$ & $n_{conv}$ & $n_{conv}$ & 2 & 4 & 8 \\
\midrule
\multicolumn{6}{c}{\texttt{ORCL}} \\
$\#_{bins}=3$ &&&&& \\
100ms & 1803 & 0.000 & 0.000 & 0.000 & 0.000 \\
1000ms & 303 & 0.000 & 0.000 & 0.000 & 0.001 \\
10000ms & 84 & 0.000 & 0.007 & 0.005 & 0.010 \\
$\#_{bins}=5$ &&&&& \\
100ms &  2503 & 0.000 & 0.000 & 0.000 & 0.000 \\
1000ms &  404 & 0.000 & 0.001 & 0.002 & 0.003 \\
10000ms & 103 & 0.000 & 0.000 & 0.001 & 0.009 \\
\multicolumn{6}{c}{\texttt{INTC}} \\
$\#_{bins}=3$ &&&&& \\
100ms &  2545 & 0.000 & 0.000 & 0.000 & 0.001 \\
1000ms &  408 & 0.000 & 0.001 & 0.001 & 0.002 \\
10000ms & 105 & 0.000 & 0.004 & 0.006 & 0.009 \\
$\#_{bins}=5$ &&&&& \\
100ms &  3498 & 0.000 & 0.001 & 0.001 & 0.001 \\
1000ms &  771 & 0.000 & 0.001 & 0.002 & 0.002 \\
10000ms & 133 & 0.000 & 0.000 & 0.000 & 0.007 \\
\bottomrule
\end{tabular}
\caption{$\chi^2$-test $p$-values for testing the time homogeneity hypothesis. Tests were run for each ticker for each trading day of 2013, and averaged over the year. For calculating $n_{conv}$, the converge error threshold was $\epsilon = 1\times 10^{-10}$.}
\label{tbl:pvalues}
\end{table}
Considering the standard cutoff $p$-value of 0.05, the cross-validation results show a strong case for the rejection of the homogeneity hypothesis. However, utilizing a non-homogeneous model falls outside of the scope of this research project, and instead suggests possible extensions to this research wherein the trading day is broken down into subintervals to better account for fluctuations and patterns in trading activity - perhaps early morning, mid-day, and final hour of trading.

\section{Naive Trading Strategies}
Utilizing the key insight drawn from \eqref{eq:EDAKeyInsight}, we implemented several naive trading strategies, descriptions of which follow:

\subsection{Naive Trading Strategy}  Using the conditional probabilities obtained from $\mat{Q}$, we will execute a buy (resp. sell) market order if the probability of an upward (resp. downward) price change is $> 0.5$.
\begin{algorithm}[H]
\caption{Naive Trading Strategy}
\begin{algorithmic}[1]
\State $cash = 0$
\State $asset = 0$
\For{$t=2 \; : \; \texttt{length}(timeseries)$}
	\If {$\mathbb{P} \left[ \Delta S_{curr} < 0 \; | \; \rho_{curr}, \rho_{prev}, \Delta S_{prev} \right] > 0.5$}
		\State $cash \pluseq data.BuyPrice(\textit{t})$
		\State $asset \mineq 1$
	\ElsIf {$\mathbb{P} \left[ \Delta S_{curr} > 0 \; | \; \rho_{curr}, \rho_{prev}, \Delta S_{prev} \right] > 0.5$}
		\State $cash \mineq data.SellPrice(\textit{t})$	
		\State $asset \pluseq 1$
	\EndIf
\EndFor
\If {$asset > 0$} 
	\State $cash \pluseq asset \times data.BuyPrice(\textit{t})$
\ElsIf {$asset < 0$} 
	\State $cash \pluseq asset \times data.SellPrice(\textit{t})$	
\EndIf
\end{algorithmic}
\end{algorithm}

\subsection{Naive+ Trading Strategy} If we anticipate no midprice change then we'll additionally keep limited orders posted at the touch, front of the queue. We'll track MO arrival, assume we always get executed, and immediately repost the limit orders.
\begin{algorithm}[H]
\caption{Naive+ Trading Strategy}
\begin{algorithmic}[1]
\State $cash = 0$
\State $asset = 0$
\For{$t=2 \; : \; \texttt{length}(timeseries)$}
	\If {$\mathbb{P} \left[ \Delta S_{curr} = 0 \; | \; \rho_{curr}, \rho_{prev}, \Delta S_{prev} \right] > 0.5$}
		\State $LO_{posted} = \texttt{True}$
	\Else
		\State $LO_{posted} = \texttt{False}$
	\EndIf
	\If {$LO_{posted}$}
		\For{$MO \in ArrivedMarketOrders(t,t+1)$}		
			\If {$MO == Sell$}
				\State $cash \mineq data.BuyPrice(\textit{t})$	
				\State $asset \pluseq 1$
			\ElsIf {$MO == Buy$}
				\State $cash \pluseq data.SellPrice(\textit{t})$
				\State $asset \mineq 1$
			\EndIf
		\EndFor
	\EndIf
\EndFor
\If {$asset > 0$} 
\State $cash \pluseq asset \times data.BuyPrice(\textit{t})$
\ElsIf {$asset < 0$} 
\State $cash \pluseq asset \times data.SellPrice(\textit{t})$	
\EndIf
\end{algorithmic}
\end{algorithm}

\subsection{Naive++ Trading Strategy} We won't execute market orders or keep limit orders at the touch. Using the conditional probabilities obtained from $\mat{Q}$, if we expect a downward (resp. upward) price change then we'll add a limit order to the sell (resp. buy) side, and hopefully pick up an agent who is executing a market order going against the price change momentum. 
\begin{algorithm}[H]
\caption{Naive++ Trading Strategy}
\begin{algorithmic}[1]
\State $cash = 0$
\State $asset = 0$
\For{$t=2 \; : \; \texttt{length}(timeseries)$}
	\State $LOBuy_{posted} = \texttt{False}$
	\State $LOSell_{posted} = \texttt{False}$
	\If {$\mathbb{P} \left[ \Delta S_{curr} < 0 \; | \; \rho_{curr}, \rho_{prev}, \Delta S_{prev} \right] > 0.5$}
		\State $LOSell_{posted} = \texttt{True}$
	\ElsIf {$\mathbb{P} \left[ \Delta S_{curr} > 0 \; | \; \rho_{curr}, \rho_{prev}, \Delta S_{prev} \right] > 0.5$}
		\State $LOBuy_{posted} = \texttt{True}$
	\EndIf
	\For{$MO \in ArrivedMarketOrders(t,t+1)$}		
		\If {$MO == Sell \; \land \; LOBuy_{posted}$}
			\State $cash \mineq data.BuyPrice(\textit{t})$	
			\State $asset \pluseq 1$
		\ElsIf {$MO == Buy \; \land \; LOSell_{posted}$}
			\State $cash \pluseq data.SellPrice(\textit{t})$
			\State $asset \mineq 1$
		\EndIf
	\EndFor
\EndFor
\If {$asset > 0$} 
\State $cash \pluseq asset \times data.BuyPrice(\textit{t})$
\ElsIf {$asset < 0$} 
\State $cash \pluseq asset \times data.SellPrice(\textit{t})$	
\EndIf
\end{algorithmic}
\end{algorithm}