\chapter{Exploratory Data Analysis}

\section[Modelling Imbalance: Continuous Time Markov Chain]{Modelling Imbalance: \texorpdfstring{\\}{} Continuous Time Markov Chain}
The aim of this research project is to use the LOB volume imbalance $I(t)$ in an algorithmic trading application; hence, a suitable choice of model for $I(t)$ must be made. Rather than modelling imbalance directly as a real-valued process, an alternative approach, and that which is used herein, is to discretize the imbalance value $I(t)$ into subintervals, or bins, and fit the resulting process to a continuous-time Markov chain.

The following definitions and properties are adapted from \citet{STAT455}:
\begin{defn}
A continuous-time stochastic process $\left\lbrace X(t) \; | \; t \geq 0 \right\rbrace$ with state space $S$ is called a \emph{continuous-time Markov chain (CTMC)} if it has the Markov property; namely, that 
\begin{equation}
\P \left[ X(t) = j \; | \; X(s) = i, X(t_{n-1} = i_{n-1}, \dots , X(t_1) = i_1 \right] = \P \left[ X(t) = j \; | \; X(s) = i \right]
\end{equation}
where for any integer $n \geq 1$, $0 \leq t_1 \leq \dots \leq t_{n-1} \leq s \leq t$ is any non-decreasing sequence of $n+1$ times, and $i_1,\dots,i_{n-1}, i, j \in S$ are any $n+1$ states.
\end{defn}
\begin{defn} A CTMC $X(t)$ is \emph{time homogeneous} if for any $s \leq t$ and any states $i,j \in S$,
\begin{equation}
\P \left[ X(t) = j \; | \; X(s) = i \right] = \P \left[ X(t-s) = j \; | \; X(0) = i \right]
\end{equation}
\end{defn}
\begin{defn}
The key quantities that determine a CTMC $X(t)$ are the \emph{transition rates} $q_{ij}$, which specify the rate at which $X$ jumps from state $i$ to $j$. Conditional on leaving state $i$, $X$ transitions to state $j$ with \emph{conditional transition probability} $p_{ij}$. The amount of time that $X$ spends in state $i$, called the \emph{holding time}, is exponentially distributed with rate $v_i$. These quantities are related by
\begin{align}
v_i & = \sum_{\substack{j \in S \\ j \neq i}} q_{ij} \\
q_{ij} & = v_i \cdot p_{ij} \\
p_{ij} & = \frac{q_{ij}}{v_i}
\end{align} 
\end{defn}
\begin{defn}
A CTMC $X(t)$ has an \emph{infinitesimal generator matrix} $\mat{G}$, whose entries are
\begin{align}
g_{ij} &= q_{ij}, \qquad i \neq j \\
g_{ii} & = -v_i
\end{align}
If $X(t)$ has transition probabilities $P_{ij}(t) = \P \left[ X(t) = j \; | \; X(0) = i \right]$ and matrix $\mat{P}(t) = \lbrace P_{ij}(t) \rbrace$, then $\mat{P}(t)$ and $\mat{G}$ are related by
\begin{align}
\dot{\mat{P}}(t) & = \mat{G} \cdot \mat{P}(t) \\
\mat{P}(t)  &= e^{\mat{G}t} \label{eq:CTMCPG}
\end{align}
\end{defn}

Conditional on $X(t) = k$, we assume the arrival of buy and sell market orders follow independent Poisson processes with intensities $\lambda_k^\pm$, where $\lambda_k^+$ ($\lambda_k^-$) is the rate of arrivals of market buys (sells). Such processes are called \textit{Markov-modulated Poisson processes}, as the Poisson intensities are themselves stochastic processes determined by the state of the Markov chain. Thus, a timeline of observations of arrivals of buy/sell market orders and of regime switches might look as in \autoref{introtimeline}.

\begin{figure}
  \tikzsetnextfilename{Ch2/LOBtimeline}
  \resizebox{\linewidth}{!}{% Limit Order Book timeline by Anton
%

\begin{tikzpicture}[scale=1.5]
	[triangle/.style = {fill=blue!20, regular polygon, regular polygon sides=3 },
	border rotated/.style = {shape border rotate=180}]
	
    \draw [>=latex,->] (0,0) -- (10,0) node[draw=none,fill=none,shift=(right:0.5)] {time};
    \draw[mark options={fill=black}, mark size=+3pt] plot[mark=*] coordinates {(.5,0)} node[shift=(down:0.5), align=center] {$\tau_1$};
    \draw[mark options={fill=black}, mark size=+3pt] plot[mark=*] coordinates {(4,0)} node[shift=(down:0.5), align=center] {$\tau_2$};
    \draw[mark options={fill=black}, mark size=+3pt] plot[mark=*] coordinates {(5,0)} node[shift=(down:0.5), align=center] {$\tau_3$};
    \draw[mark options={fill=black}, mark size=+3pt] plot[mark=*] coordinates {(8,0)} node[shift=(down:0.5), align=center] {$\tau_4$};
    
    
	\draw[mark options={fill=blue}, mark size =+3pt, shift=(up:0.1)] plot[mark=diamond*] coordinates {(.75,0)} node[shift=(up:0.7), align=center] {buy \\ $b_1$};
	\draw[mark options={fill=red}, mark size =+3pt, shift=(down:0.1)] plot[mark=diamond*] coordinates {(1.25,0)} node[shift=(down:0.7), align=center] {$s_2$ \\ sell};
	\draw[mark options={fill=blue}, mark size =+3pt, shift=(up:0.1)] plot[mark=diamond*] coordinates {(2.5,0)} node[shift=(up:0.7), align=center] {buy \\ $b_3$};
	\draw[mark options={fill=red}, mark size =+3pt, shift=(down:0.1)] plot[mark=diamond*] coordinates {(3,0)} node[shift=(down:0.7), align=center] {$s_4$ \\ sell};
	\draw[mark options={fill=red}, mark size =+3pt, shift=(down:0.1)] plot[mark=diamond*] coordinates {(3.5,0)} node[shift=(down:0.7), align=center] {$s_5$ \\ sell};

%%% REGIME SWITCH
	
	\draw[mark options={fill=blue}, mark size =+3pt, shift=(up:0.1)] plot[mark=diamond*] coordinates {(4.25,0)} node[shift=(up:0.4), align=center] {$b_6$};
	\draw[mark options={fill=blue}, mark size =+3pt, shift=(up:0.1)] plot[mark=diamond*] coordinates {(4.50,0)} node[shift=(up:0.7), align=center]
{buy \\ $b_7$};
	\draw[mark options={fill=blue}, mark size =+3pt, shift=(up:0.1)] plot[mark=diamond*] coordinates {(4.75,0)} node[shift=(up:0.4), align=center] {$b_8$};
	
%%% REGIME SWITCH

	\draw[mark options={fill=blue}, mark size =+3pt, shift=(up:0.1)] plot[mark=diamond*] coordinates {(5.25,0)} node[shift=(up:0.7), align=center] {buy \\ $b_9$};
	\draw[mark options={fill=red}, mark size =+3pt, shift=(down:0.1)] plot[mark=diamond*] coordinates {(5.75,0)} node[shift=(down:0.7), align=center] {$s_{10}$ \\ sell};
	\draw[mark options={fill=blue}, mark size =+3pt, shift=(up:0.1)] plot[mark=diamond*] coordinates {(7,0)} node[shift=(up:0.7), align=center] {buy \\ $b_{11}$};
	
%%% REGIME SWITCH

	\draw[mark options={fill=red}, mark size =+3pt, shift=(down:0.1)] plot[mark=diamond*] coordinates {(9,0)} node[shift=(down:0.7), align=center] {$s_{12}$ \\ sell};
	
%%% BRACES
	
	\draw [decorate, decoration = {brace, amplitude = 10pt, mirror}]
	(0.55,-1) -- (3.95,-1) node [black, midway, yshift = -0.6cm] {regime $i$};
	\draw [decorate, decoration = {brace, amplitude = 10pt, mirror}]
	(4.05,-1) -- (4.95,-1) node [black, midway, yshift = -0.6cm] {regime $j$}; 
	\draw [decorate, decoration = {brace, amplitude = 10pt, mirror}]
	(5.05,-1) -- (7.95,-1) node [black, midway, yshift = -0.6cm] {regime $i$}; 
	\draw [decorate, decoration = {brace, amplitude = 10pt, mirror}]
	(8.05,-1) -- (9.95,-1) node [black, midway, yshift = -0.6cm] {regime $k$}; 
\end{tikzpicture}
}
\caption[Hypothetical timeline of market orders and imbalance regime switches]{Hypothetical timeline of market orders arriving during changing order imbalance regimes. The $\tau$ nodes represent regime switch times, and appear in unequally spaced intervals. Regime $i$ occurs twice, and market order arrivals behave similarly in both instances.}
\label{introtimeline}
\end{figure}

In the sections that follow, we derive maximum likelihood estimations for the parameters of the CTMC, and evaluate the fit of the model to the data.

\section[Maximum Likelihood Estimate of a Markov-Modulated Poisson Process]{Maximum Likelihood Estimate \texorpdfstring{\\}{} of a Markov-Modulated Poisson Process}

\subsection{Infinitesimal Generator Matrix}

Let $\mat{G}$ be the generator matrix for a CTMC $X(t)$ with state space $K$. From observations, e.g. the fictional events in the timeline given in \autoref{introtimeline}, we want to estimate the entries of $\mat{G}$. Since the holding time in a given state $i$ has probability density function $f(t;v_i) = v_i e^{-v_i t}$, the likelihood function (allowing for repetition of terms) is therefore
\begin{align}
\mathcal{L}(\mat{G}) &= (v_{i} e^{-v_{i}(\tau_2 - \tau_1)} p_{ij}) (v_{j} e^{-v_{j}(\tau_3 - \tau_2)} p_{ji}) (v_{i} e^{-v_{i}(\tau_4 - \tau_3)} p_{ik}) \dots \\
&= \prod\limits_{i=1}^{K} \prod\limits_{i \neq j} (v_{i}p_{ij})^{N_{ij}(T)} e^{-v_{i}H_i(T)} \\
&= \prod\limits_{i=1}^{K} \prod\limits_{i \neq j} (q_{ij})^{N_{ij}(T)} e^{-v_{i}H_i(T)}
\intertext{where}
N_{ij}(T) & \equiv \mbox{number of transitions from regime $i$ to $j$ up to time $T$} \nonumber \\
H_{i}(T) & \equiv \mbox{holding time in regime $i$ up to time $T$} \nonumber
\intertext{Therefore the log-likelihood becomes} 
\ln \mathcal{L}(\mat{G}) & = \sum\limits_{i=1}^{K} \sum\limits_{i \neq j} \left[ N_{ij}(T) \ln(q_{ij}) - v_{i} H_i(T) \right] \\
&= \sum\limits_{i=1}^{K} \sum\limits_{i \neq j} \left[ N_{ij}(T) \ln(q_{ij}) - \left( \sum\limits_{i \neq k} q_{ik} H_i(T) \right) \right]
\end{align}
To get a maximum likelihood estimate $\hat{q}_{ij}$ for transition rates and therefore the matrix $\mat{G}$, we take the partial derivative of $\ln \mathcal{L}(\mat{G})$ and set it equal to zero:
\begin{equation}
\dfrac{\partial \ln \mathcal{L}(\mat{G})}{\partial \hat{q}_{ij}} = \dfrac{N_{ij}(T)}{q_{ij}} - H_i(T) = 0
\end{equation}
\begin{equation}\label{eq:MLEG}
\Rightarrow \hat{q}_{ij} = \dfrac{N_{ij}(T)}{H_i(T)}
\end{equation}

\subsection{Arrival Rates}

Now we want to derive an estimate for the intensity of the Poisson process of market order arrivals conditional on being in state $k$. We'll look at just the buy market orders for some regime $k$, as the sell case is identical. Let the buy market order arrival times be indexed by $b_i$. Since we're assuming that the arrival process is Poisson with the same intensity throughout trials, we can consider the inter-arrival time of events conditional on being in state $k$. The MLE derivation then follows just as it did for the generator matrix.
\begin{align}
\mathcal{L}(\lambda^{+}_k ; b_1, \dots, b_N) &= \prod\limits_{i=2}^{N} \lambda^{+}_k e^{-\lambda^{+}_k (b_{i} - b_{i-1})} \\
&= (\lambda^{+}_k)^{N^{+}_k(T)} e^{-\lambda^{+}_k H_k(T)}
\intertext{where}
N^{+}_{k}(T) & \equiv \mbox{number of market order arrivals in regime $k$ up to time $T$} \nonumber \\
H_{k}(T) & \equiv \mbox{holding time in regime $k$ up to time $T$} \nonumber
\intertext{Therefore the log-likelihood becomes:} 
\ln \mathcal{L}(\lambda^{+}_k) & = N^{+}_k(T) \ln(\lambda^{+}_k) -\lambda^{+}_k H_k(T)
\intertext{By setting the partial derivative of $\ln \mathcal{L}$ with respect to ${\lambda}^{+}_k$ equal to zero, we get that the ML estimate for $\hat{\lambda}^{+}_k$ is:} 
\dfrac{\partial \ln\mathcal{L} }{\partial \lambda^{+}_k} & = 
\dfrac{N^{+}_k(T)}{\hat{\lambda}^{+}_k} - H_k(T) = 0
\end{align}
\begin{equation}\label{eq:MLElambda}
\Rightarrow \hat{\lambda}^{+}_k = \dfrac{N^{+}_k(T)}{H_k(T)}
\end{equation}

\section{2-Dimensional CTMC}
\label{sec:2DCTMC}
Next we consider a CTMC $Z(t)$ that jointly models the imbalance bin $\rho(t)$ and the price change $\Delta S(t)$. The raw imbalance timeseries is very erratic, so to smooth it we take the time-weighted average of imbalance over the past time inteval $\Delta t_I$. We compute price change as the \emph{sign} of the change in midprice of the \emph{future} time interval $\Delta t_S$. These time intervals are illustrated in \autoref{fig:timewindows}.

\begin{figure}[H]
  \centering
  \tikzsetnextfilename{Ch2/timewindows}
  % Limit Order Book timeline by Anton
%
\colorlet{buyLOcolor}{black!25}%
\colorlet{sellLOcolor}{black!90}%

\begin{tikzpicture}
    \draw [>=latex,->] (0,0) -- (10,0) node[draw=none,fill=none,shift=(right:0.5)] {time};
    \draw[mark options={fill=black}, mark size=+3pt] plot[mark=*] coordinates {(1,0)} node[shift=(down:0.5), align=center] {$t-\Delta t_I$};
    \draw[mark options={fill=black}, mark size=+3pt] plot[mark=*] coordinates {(5,0)} node[shift=(down:0.5), align=center] {$t$};
    \draw[mark options={fill=black}, mark size=+3pt] plot[mark=*] coordinates {(8,0)} node[shift=(down:0.5), align=center] {$t+\Delta t_S$};
    
%%% BRACES

	
	\draw [decorate, decoration = {brace, amplitude = 10pt}] (1.1,0.25) -- (4.9,0.25) node [black, midway, yshift = -0.6cm] {};
	\node at (2.5,1) [anchor=south, text width=5cm, align=left, font=\footnotesize] {$\rho(t)$ is the imbalance bin of the time-weighted average of $I(t)$ over this past interval.};		
	\draw [decorate, decoration = {brace, amplitude = 10pt}] (5.1,0.25) -- (7.9,0.25) node [black, midway, yshift = -0.6cm] {}; 
	\node at (5.5,1) [anchor=south west, text width=5cm, align=left, font=\footnotesize] {$\Delta S(t)$ is the sign of the midprice change over this future interval. \vphantom{$I(t)$}};		
\end{tikzpicture}

\caption{Time intervals for time-weighted averaging of imbalance and for price change.}
\label{fig:timewindows}
\end{figure}

Thus, the CTMC models the joint distribution $(\rho(t), \Delta S(t))$ where 
\[ \rho(t) \in \lbrace 1,2,\dots,\#_{bins} \rbrace \]
is the bin corresponding to imbalance averaged over the interval $[t-\Delta t_I, t]$, and
\[ \Delta S(t) = \sgn(S(t+\Delta t_S)-S(t)) \in \lbrace -1, 0, 1 \rbrace \]
For simplicity of computation, the pair $(\rho(t), \Delta S(t))$ is then reduced into one dimension with a simple encoding function $\varphi$; for example, using 3 bins:
\begin{table}[H]
\centering
\ra{1.2}
\begin{tabular}{@{} *{11}{c} @{}}
\toprule
$Z(t)$ & Bin $\rho(t)$ & $\Delta S(t)$ & \phantom{abc} & $Z(t)$ & Bin $\rho(t)$ & $\Delta S(t)$ & \phantom{abc} & $Z(t)$ & Bin $\rho(t)$ & $\Delta S(t)$ \\
\cmidrule{1-3} \cmidrule{5-7} \cmidrule{9-11}
1 & Bin 1 & $-1$ && 4 & Bin 1 & $0$ && 7 & Bin 1 & $+1$ \\
2 & Bin 2 & $-1$ && 5 & Bin 2 & $0$ && 8 & Bin 2 & $+1$ \\
3 & Bin 3 & $-1$ && 6 & Bin 3 & $0$ && 9 & Bin 3 & $+1$ \\
\bottomrule
\end{tabular}
\caption[1-dimensional encoding of 2-dimensional CTMC]{$\varphi(\rho(t),S(t))$: 1-dimensional encoding of 2-dimensional CTMC.}
\end{table}

\subsection{Cross-Validation}
We cross-validate the CTMC calibration by means of a time-homogeneity test similar to that done in \citet{Tan02}. The null hypothesis is given by \citep{Weiss10}
\begin{equation}
H_0 = \forall i,j \in S \; : \; \exists q_{ij} \in \R^+ \; : \; q_{ij}(t) \equiv q_{ij} \; \forall t \in [0,T]
\end{equation}
whereas the alternative hypothesis states that transition rates/probabilities are time-dependent. To test the hypothesis, we fix $\Delta t_I = \Delta t_S$ at some value, choose a number of imbalance bins, and calculate the maximum likelihood estimate of the infinitesimal generator matrix $\mat{G}$ on the full timeseries using \eqref{eq:MLEG}. For a chosen error threshold $\epsilon$, we use the relationship in \eqref{eq:CTMCPG} to calculate the number of timesteps $n_{conv}$ of size $\Delta t_I$ such that
\begin{equation}\label{eq:crossvalidnconv}
|| \mat{P}\left((n_{conv}+1)\Delta t_I \right) - \mat{P} \left( n_{conv}\Delta t_I \right) || < \epsilon
\end{equation}
This value $n_{conv}$ determines the size of the cross-validation timewindow into which to partition the full timeseries, yielding $K$ equal subintervals of length $n_{conv}$. For each ``removed series'' $k \in \{ 1,\dots,K \}$, we recalibrate a CTMC generator matrix $\mat{G}_{k}$. Finally, we test whether the one-step transition probabilities $p_{ij}^k$ contained in $\mat{P}_k \left(\Delta t_I \right)$ are statistically different from those of the full period. For comparison, we also partitioned the timeseries into 8, 4, and 2 equal intervals. The asymptotically equivalent test statistic to the likelihood ratio test statistic is
\begin{equation}
D = -2 \ln (\mathcal{L})  = 2 \sum_k \sum_{i,j} n_{i,j}^k \left[ \ln(p_{ij}^k)  - \ln(p_{ij})   \right]
\end{equation}
where $n_{ij}^k$ is the number of observed transitions from state $i$ to $j$ in subinterval $k$. This test statistic has a $\chi^2$ distribution with $(K-1)(3 \cdot \#_{bins})(3 \cdot \#_{bins} - 1)$. The tests were run for each ticker for each trading day of 2013, and averaged over the year. \autoref{tbl:pvalues} shows the $p$-value scores for the tests.
\begin{table}
\centering
\ra{1.2}
\begin{adjustbox}{max width=\textwidth}
\begin{tabular}{@{} *{6}{r} r *{5}{r} @{}}
\toprule
 &  & \multicolumn{4}{c}{subintervals} & \hphantom{abc} & \multicolumn{4}{c}{subintervals} \\ 
\cmidrule{3-6} \cmidrule{9-12}
$\Delta t_I$ & $n_{conv}$ & $K$ & 8 & 4 & 2 & & $n_{conv}$ & $K$ & 8 & 4 & 2 \\
\midrule
& \multicolumn{5}{c}{\texttt{FARO}} & & \multicolumn{5}{c}{\texttt{ORCL}}\\
\multicolumn{12}{l}{$\#_{bins}=3$} \\
100ms &  4933 & 0.000 & 0.000 & 0.000 & 0.003 & & 1803 & 0.000 & 0.000 & 0.000 & 0.000 \\
1000ms &  727 & 0.000 & 0.002 & 0.001 & 0.005 & & 303 & 0.000 & 0.000 & 0.000 & 0.001 \\
10000ms & 149 & 0.000 & 0.005 & 0.010 & 0.017 & & 84 & 0.000 & 0.007 & 0.005 & 0.010 \\[2ex]
\multicolumn{12}{l}{$\#_{bins}=5$} \\
100ms &  6450 & 0.000 & 0.001 & 0.002 & 0.004 & & 2503 & 0.000 & 0.000 & 0.000 & 0.000 \\
1000ms &  941 & 0.000 & 0.001 & 0.003 & 0.006 & & 404 & 0.000 & 0.001 & 0.002 & 0.003 \\
10000ms & 187 & 0.000 & 0.000 & 0.000 & 0.005 & & 103 & 0.000 & 0.000 & 0.001 & 0.009 \\[2ex]
& \multicolumn{5}{c}{\texttt{NTAP}} & & \multicolumn{5}{c}{\texttt{INTC}}\\
\multicolumn{12}{l}{$\#_{bins}=3$} \\
100ms & 1320 & 0.000 & 0.000 & 0.000 & 0.000 & & 2545 & 0.000 & 0.000 & 0.000 & 0.001 \\
1000ms & 237 & 0.000 & 0.000 & 0.000 & 0.000 & & 408 & 0.000 & 0.001 & 0.001 & 0.002 \\
10000ms & 72 & 0.000 & 0.006 & 0.003 & 0.007 & & 105 & 0.000 & 0.004 & 0.006 & 0.009 \\[2ex]
\multicolumn{12}{l}{$\#_{bins}=5$} \\
100ms & 1777 & 0.000 & 0.000 & 0.000 & 0.000 & &  3498 & 0.000 & 0.001 & 0.001 & 0.001 \\
1000ms & 308 & 0.000 & 0.001 & 0.000 & 0.001 & &  771 & 0.000 & 0.001 & 0.002 & 0.002 \\
10000ms & 87 & 0.000 & 0.000 & 0.002 & 0.010 & & 133 & 0.000 & 0.000 & 0.000 & 0.007 \\
\bottomrule
\end{tabular}
\end{adjustbox}
\caption[$p$-values for testing the time homogeneity hypothesis]{$\chi^2$-test $p$-values for testing the time homogeneity hypothesis. Tests were run for each ticker for each trading day of 2013, and averaged over the year. For calculating $n_{conv}$, the converge error threshold was $\epsilon = 1\times 10^{-10}$.}
\label{tbl:pvalues}
\end{table}
Considering the standard cutoff $p$-value of 0.05, the cross-validation results show a strong case for the rejection of the homogeneity hypothesis. However, using a non-homogeneous model falls outside of the scope of this research project, and instead suggests possible extensions to this research wherein the trading day is broken down into subintervals to better account for fluctuations and patterns in trading activity - perhaps early morning, mid-day, and final hour of trading. The severity of proceeding with the homogeneity hypothesis is not known \textit{a priori}, and may instead emerge with the backtesting results done later in this chapter and in Chapter 4.

\section{Predicting Future Price Changes}
It is crucial to note that the value $\Delta S(t)$ contains the price change from time $t$ over the \textit{future} $\Delta t_S$ seconds - hence in real-time one cannot know the state of the Markov Chain. However, the analytic results do prove enlightening: from the resulting timeseries we estimate a generator matrix $\mat{G}$, and transform it into a one-step transition probability matrix $\mat{P} = e^{\mat{G}\Delta t_I}$. The entries of $\mat{P}$ are the conditional probabilities 
\begin{align}
\mat{P}_{ij} & = \mathbb{P}\left[ \varphi( \rho_{[t-\Delta t_I, t]}, \Delta S_{[t,t+\Delta t_S]}) = j \; | \; \varphi( \rho_{[t-2\Delta t_I, t-\Delta t_I]}, \Delta S_{[t-\Delta t_I, t]} ) = i \right] \label{eq:POneStepUgly} \\
\intertext{which can be expressed semantically as}
& = \mathbb{P}\left[ \varphi( \rho_{curr}, \Delta S_{future}) = j \; | \; \varphi( \rho_{prev}, \Delta S_{curr} ) = i \right] \label{eq:POneStepNice} \\
\intertext{Since we can easily decode the 1-dimensional Markov state back into two dimensions, we can think of $\mat{P}$ as being four-dimensional and re-write its entries as}
& = \mathbb{P}\left[ \rho_{curr} = i,  \Delta S_{future} = j \; | \; \rho_{prev} = k, \Delta S_{curr} = m \right] \\
& = \mathbb{P}\left[ \rho_{curr} = i,  \Delta S_{future} = j \; | \; B \right]
\end{align}
where we're using the shorthand $B = (\rho_{prev} = k, \Delta S_{curr} = m)$ to represent the states in the previous timestep. Applying Bayes' Rule,
\begin{equation}\label{eq:POneStepBayes}
\mathbb{P}\left[ \Delta S_{future} = j \; | \; B, \rho_{curr} = i \right] = \dfrac{\mathbb{P}\left[ \rho_{curr} = i, \Delta S_{future} = j \; | \; B \right]}{\mathbb{P}\left[ \rho_{curr} = i \; | \; B \right]}
\end{equation}
where the right-hand-side numerator is each individual entry of the one-step probability matrix $\mat{P}$, and the denominator can be computed from $\mat{P}$ by
\begin{equation}\label{eq:POneStepBayesDenom}
\mathbb{P}\left[ \rho_{curr} = i \; | \; B \right] = \sum\limits_j \mathbb{P}\left[ \rho_{curr} = i,  \Delta S_{future} = j \; | \; B \right]
\end{equation}
The left-hand-side value in \eqref{eq:POneStepBayes} is the probability of seeing a given price change over the immediate future time interval conditional on past imbalances and the most recent price change, and therefore allows us to predict future price moves. We'll denote by $\mat{Q}$ the matrix containing all values given by \eqref{eq:POneStepBayes}.

The $\mat{Q}$ matrix in \autoref{tbl:Qmatrix} was obtained using data for \texttt{MMM} from 2013-05-15, averaging imbalance and price change timewindows $\Delta t_I = \Delta t_S = 1000\text{ms}$, and $K=3$ imbalance bins.
\begin{table}
\centering
\ra{1.2}
\begin{tabular}{@{} r@{\hskip 1cm} *{9}{r} @{}}
\toprule
& \multicolumn{3}{c}{$\Delta S_{curr} < 0$} & \multicolumn{3}{c}{$\Delta S_{curr} = 0$} & \multicolumn{3}{c}{$\Delta S_{curr} > 0$} \\
\cmidrule(lr){2-4} \cmidrule(lr){5-7} \cmidrule(lr){8-10}
\multicolumn{2}{r}{$\rho_{curr} = 1$} & 2 & 3 & 1 & 2 & 3 & 1 & 2 & 3 \\
\midrule
\multicolumn{10}{l}{$\Delta S_{future} < 0$} \\
$\hphantom{abcd}\rho_{prev} = 1$ & \bf 0.53 & 0.15 & 0.12 & 0.05 & 0.10 & 0.14 & 0.08 & 0.13 & 0.14 \\
$\rho_{prev} = 2$ & 0.10 & \bf 0.58 & 0.14 & 0.07 & 0.04 & 0.10 & 0.13 & 0.06 & 0.12 \\
$\rho_{prev} = 3$ & 0.08 & 0.12 & \bf 0.52 & 0.09 & 0.06 & 0.03 & 0.11 & 0.10 & 0.05 \\[0.6ex]
\multicolumn{10}{l}{$\Delta S_{future} = 0$} \\
$\rho_{prev} = 1$ & 0.41 & 0.75 & 0.78 & \bf 0.91 & 0.84 & 0.79 & 0.42 & 0.79 & 0.77 \\
$\rho_{prev} = 2$ & 0.79 & 0.36 & 0.71 & 0.83 & \bf 0.92 & 0.82 & 0.75 & 0.37 & 0.78 \\
$\rho_{prev} = 3$ & 0.79 & 0.74 & 0.40 & 0.81 & 0.83 & \bf 0.91 & 0.70 & 0.76 & 0.39 \\[0.6ex]
\multicolumn{10}{l}{$\Delta S_{future} > 0$} \\
$\rho_{prev} = 1$ & 0.06 & 0.10 & 0.09 & 0.04 & 0.06 & 0.07 & \bf 0.50 & 0.09 & 0.09 \\
$\rho_{prev} = 2$ & 0.10 & 0.06 & 0.15 & 0.10 & 0.04 & 0.08 & 0.12 & \bf 0.57 & 0.10 \\
$\rho_{prev} = 3$ & 0.13 & 0.14 & 0.08 & 0.10 & 0.11 & 0.05 & 0.19 & 0.14 & \bf 0.56 \\
\bottomrule
\end{tabular}
\caption[Probabilities of future price changes]{The $\mat{Q}$ matrix: conditional probabilities of future price changes, conditioned on current imbalance, current price change, and previous imbalance.}
\label{tbl:Qmatrix}
\end{table}

The three middle rows of \autoref{tbl:Qmatrix} contain the majority of values $>0.5$, showing that in most cases we are expecting no price change. The only other cases in which the probability of a price change is $>0.5$ show evidence of \textit{momentum}; for example, the value in row 1, column 1 can be interpreted as: if $\rho_{prev} = \rho_{curr} = 1$ and previously we saw a downward price change, then we expect to again see a downward price change. The bolded diagonal values in the table lend themselves to the empirical conclusion
\begin{equation}\label{eq:EDAKeyInsight}
\mathbb{P} \left[ \Delta S_{future} = \Delta S_{curr} \; | \; \rho_{curr} = \rho_{prev} \right] > 0.5
\end{equation}

\section{Naive Trading Strategies}
Using the key insight drawn from \eqref{eq:EDAKeyInsight}, we implemented several naive trading strategies, descriptions of which follow:

\paragraph{Naive Trading Strategy}  This strategy seeks to profit from expected price changes by using market orders. Using the conditional probabilities obtained from $\mat{Q}$, we will execute a buy (resp. sell) market order if the probability of an upward (resp. downward) price change is $> 0.5$. Pseudocode for this strategy is given in \autoref{algo:algo-naive}. In lines 4-6 we are forecasting a downward price move, and therefore sell one share with a market order at the best bid price. In lines 7-9, we are forecasting an upward price move, and buy one share with a market order at the best ask price. In lines 12-16 we have reached the end of the trading day, and liquidate our position at the at-the-touch price.
\begin{algorithm}
	\caption{Naive Trading Strategy}
	\begin{algorithmic}[1]
\State $cash = 0$
\State $asset = 0$
\For{$t=2 \; : \; \texttt{length}(timeseries)$}
	\If {$\mathbb{P} \left[ \Delta S_{future} < 0 \; | \; \rho_{curr}, \rho_{prev}, \Delta S_{curr} \right] > 0.5$}
		\State $cash \pluseq data.BuyPrice(\textit{t})$
		\State $asset \mineq 1$
	\ElsIf {$\mathbb{P} \left[ \Delta S_{future} > 0 \; | \; \rho_{curr}, \rho_{prev}, \Delta S_{curr} \right] > 0.5$}
		\State $cash \mineq data.SellPrice(\textit{t})$	
		\State $asset \pluseq 1$
	\EndIf
\EndFor
\If {$asset > 0$} 
	\State $cash \pluseq asset \times data.BuyPrice(\textit{t})$
\ElsIf {$asset < 0$} 
	\State $cash \pluseq asset \times data.SellPrice(\textit{t})$	
\EndIf
\end{algorithmic}

	\label{algo:algo-naive}
\end{algorithm}

\clearpage
\paragraph{Naive+ Trading Strategy} This strategy seeks to profit from no expected changes in price by using limit orders. If no change in midprice is expected then we'll post buy and sell limit orders at the touch. For every market order that arrives, we'll assume that our limit order is always filled, following which we will immediately repost whichever limit order was filled. Pseudocode for this strategy is given in \autoref{algo:algo-naive-plus}. In lines 11-13, since a sell market order is arriving, our buy limit order is being filled and we are paying the bid price. 
\begin{algorithm}
	\caption{Naive+ Trading Strategy}
	\begin{algorithmic}[1]
\State $cash = 0$
\State $asset = 0$
\For{$t=2 \; : \; \texttt{length}(timeseries)$}
	\If {$\mathbb{P} \left[ \Delta S_{curr} = 0 \; | \; \rho_{curr}, \rho_{prev}, \Delta S_{prev} \right] > 0.5$}
		\State $LO_{posted} = \texttt{True}$
	\Else
		\State $LO_{posted} = \texttt{False}$
	\EndIf
	\If {$LO_{posted}$}
		\For{$MO \in ArrivedMarketOrders(t,t+1)$}		
			\If {$MO == Sell$}
				\State $cash \mineq data.BuyPrice(\textit{t})$	
				\State $asset \pluseq 1$
			\ElsIf {$MO == Buy$}
				\State $cash \pluseq data.SellPrice(\textit{t})$
				\State $asset \mineq 1$
			\EndIf
		\EndFor
	\EndIf
\EndFor
\If {$asset > 0$} 
\State $cash \pluseq asset \times data.BuyPrice(\textit{t})$
\ElsIf {$asset < 0$} 
\State $cash \pluseq asset \times data.SellPrice(\textit{t})$	
\EndIf
\end{algorithmic}

	\label{algo:algo-naive-plus}
\end{algorithm}

\paragraph{Naive++ Trading Strategy} Like the Naive strategy, this strategy seeks to profit from expected price changes, but using limit orders (and therefore buying/selling at better prices). If we expect a downward (resp. upward) price change then we'll post an at-the-touch sell (resp. buy) limit order, which may be lifted by an agent who is executing a market order going against the price change momentum. 

\section{Calibration and Backtesting}
Backtesting these naive trading strategies required a choice of parameters for the price change observation period $\Delta t_S$, the imbalance averaging period $\Delta t_I$, and the number of imbalance bins $\#_{bins}$. We used a brute force calibration technique that, for each ticker and each day, traversed the potential parameter space searching for the highest number of timesteps at which \eqref{eq:EDAKeyInsight} could be used. We found that $\#_{bins} = 4$ provided the highest expected number of trades for most tickers. However, as we were using percentile bins symmetric around zero, we wanted to have $\#_{bins}$ as an odd number such that all behaviour around zero imbalance was treated equally; thus all backtesting was done with either $\#_{bins} = 3$ or $\#_{bins} = 5$. Additionally, we found empirically that calibration always yielded $\Delta t_S = \Delta t_I$, so this was taken as a given. The backtest for each ticker then consisted of first calibrating the value $\Delta t_I$ from the first day of data by maximizing the intra-day Sharpe ratio (average of returns divided by standard deviation of returns), then using the calibrated parameters to backtest the entire year.

\begin{figure}%
\centering%
\begin{subfigure}[b]{.4\linewidth}%
  %\centering
  \setlength\figureheight{\linewidth}%
  \setlength\figurewidth{\linewidth}%
  \tikzsetnextfilename{Ch2/FARO_naive_strat_comp}%
  % This file was created by matlab2tikz.
%
%The latest updates can be retrieved from
%  http://www.mathworks.com/matlabcentral/fileexchange/22022-matlab2tikz-matlab2tikz
%where you can also make suggestions and rate matlab2tikz.
%
\definecolor{mycolor1}{rgb}{0.00000,0.00000,0.00000}%
\definecolor{mycolor2}{rgb}{0.40000,0.40000,0.40000}%
\definecolor{mycolor3}{rgb}{0.70000,0.70000,0.70000}%
%
\begin{tikzpicture}[trim axis left, trim axis right]

\begin{axis}[%
width=\figurewidth,
height=\figureheight,
at={(0\figurewidth,0\figureheight)},
scale only axis,
every outer x axis line/.append style={black},
every x tick label/.append style={font=\color{black}},
xmin=9.5,
xmax=16,
every outer y axis line/.append style={black},
every y tick label/.append style={font=\color{black}},
ymin=-1,
ymax=0.2,
title={FARO},
axis background/.style={fill=white},
axis x line*=bottom,
axis y line*=left
]
\addplot [color=mycolor1,solid,line width=1.5pt,forget plot]
  table[row sep=crcr]{%
9.50027777777778	0\\
9.50583333333333	-0.0229983346119282\\
9.51138888888889	-0.037198704038164\\
9.51694444444444	-0.04191797618036\\
9.5225	-0.0472831079772788\\
9.52805555555556	-0.0493149724490774\\
9.53361111111111	-0.0520328986130521\\
9.53916666666667	-0.0560470232017754\\
9.54472222222222	-0.0586470071447678\\
9.55027777777778	-0.059791466334168\\
9.55583333333333	-0.0621255183525458\\
9.56138888888889	-0.0647583872653891\\
9.56694444444444	-0.0666701339343573\\
9.5725	-0.068705698440769\\
9.57805555555555	-0.0721958194818497\\
9.58361111111111	-0.0750134419531125\\
9.58916666666667	-0.0794696425270457\\
9.59472222222222	-0.0839165359350066\\
9.60027777777778	-0.0874583658829305\\
9.60583333333333	-0.0906790536494476\\
9.61138888888889	-0.0926326729372235\\
9.61694444444444	-0.093652876177913\\
9.6225	-0.0974835474757117\\
9.62805555555556	-0.101208020258433\\
9.63361111111111	-0.103605453254331\\
9.63916666666667	-0.105826359906983\\
9.64472222222222	-0.109206552359595\\
9.65027777777778	-0.110613563356721\\
9.65583333333333	-0.111576997617571\\
9.66138888888889	-0.113637672477417\\
9.66694444444444	-0.115488621408664\\
9.6725	-0.120445175316876\\
9.67805555555555	-0.122890606893941\\
9.68361111111111	-0.124589000865991\\
9.68916666666667	-0.124404465774063\\
9.69472222222222	-0.127702131828803\\
9.70027777777778	-0.130000157112447\\
9.70583333333333	-0.130759819354828\\
9.71138888888889	-0.134031650490121\\
9.71694444444444	-0.13594687320835\\
9.7225	-0.141105302925977\\
9.72805555555555	-0.144554512516046\\
9.73361111111111	-0.145190806464807\\
9.73916666666667	-0.147502640742201\\
9.74472222222222	-0.149486537214197\\
9.75027777777778	-0.150441626385252\\
9.75583333333333	-0.153917514022108\\
9.76138888888889	-0.156216458419853\\
9.76694444444444	-0.160525124470717\\
9.7725	-0.163170616460437\\
9.77805555555556	-0.166594841988628\\
9.78361111111111	-0.168276232361246\\
9.78916666666667	-0.17142395374086\\
9.79472222222222	-0.175235553986234\\
9.80027777777778	-0.176759286787734\\
9.80583333333333	-0.178684604818829\\
9.81138888888889	-0.181190919651135\\
9.81694444444444	-0.185392425419975\\
9.8225	-0.185425679142052\\
9.82805555555555	-0.186890394888519\\
9.83361111111111	-0.189134154570056\\
9.83916666666667	-0.191033805073258\\
9.84472222222222	-0.19291756607627\\
9.85027777777778	-0.196142666464392\\
9.85583333333333	-0.198860002218713\\
9.86138888888889	-0.198142324055353\\
9.86694444444444	-0.200270111924421\\
9.8725	-0.202225082087613\\
9.87805555555556	-0.202802913958332\\
9.88361111111111	-0.205300130230464\\
9.88916666666667	-0.206583598815069\\
9.89472222222222	-0.208997894972838\\
9.90027777777778	-0.209490739858473\\
9.90583333333333	-0.209711793634132\\
9.91138888888889	-0.212439481717762\\
9.91694444444444	-0.213137425059717\\
9.9225	-0.221404556342799\\
9.92805555555555	-0.218559608795498\\
9.93361111111111	-0.221833090218777\\
9.93916666666667	-0.221865499031365\\
9.94472222222222	-0.221189206662058\\
9.95027777777778	-0.223728743162025\\
9.95583333333333	-0.226195713208188\\
9.96138888888889	-0.225458841555265\\
9.96694444444444	-0.227693591405258\\
9.9725	-0.229968857411304\\
9.97805555555555	-0.232842373579213\\
9.98361111111111	-0.234079985773092\\
9.98916666666667	-0.236719890814237\\
9.99472222222222	-0.242242600782018\\
10.0002777777778	-0.250526991462242\\
10.0058333333333	-0.252228391958838\\
10.0113888888889	-0.252056688483037\\
10.0169444444444	-0.251338942853129\\
10.0225	-0.251502138023106\\
10.0280555555556	-0.253210450016836\\
10.0336111111111	-0.253873410960622\\
10.0391666666667	-0.256275374563062\\
10.0447222222222	-0.262341136844268\\
10.0502777777778	-0.260824253124689\\
10.0558333333333	-0.258916148717237\\
10.0613888888889	-0.261630997548185\\
10.0669444444444	-0.262477276667001\\
10.0725	-0.267036207688867\\
10.0780555555556	-0.268204199559602\\
10.0836111111111	-0.272864511794689\\
10.0891666666667	-0.275018076707627\\
10.0947222222222	-0.275329562055602\\
10.1002777777778	-0.27545093522718\\
10.1058333333333	-0.277688332752408\\
10.1113888888889	-0.278597492611467\\
10.1169444444444	-0.278124273723338\\
10.1225	-0.278751886278515\\
10.1280555555556	-0.278669486078873\\
10.1336111111111	-0.279521199583171\\
10.1391666666667	-0.281738616541576\\
10.1447222222222	-0.284755632836928\\
10.1502777777778	-0.287058030409526\\
10.1558333333333	-0.289049946271823\\
10.1613888888889	-0.287083498142268\\
10.1669444444444	-0.288014429889905\\
10.1725	-0.290212213868898\\
10.1780555555556	-0.291521170288517\\
10.1836111111111	-0.294613786624441\\
10.1891666666667	-0.296090216947284\\
10.1947222222222	-0.297738339219548\\
10.2002777777778	-0.297148896776102\\
10.2058333333333	-0.297422909043847\\
10.2113888888889	-0.296388817843021\\
10.2169444444444	-0.299606108054654\\
10.2225	-0.299447251922576\\
10.2280555555556	-0.30015582439155\\
10.2336111111111	-0.298234060504506\\
10.2391666666667	-0.300678914638395\\
10.2447222222222	-0.303159392845527\\
10.2502777777778	-0.30703629296771\\
10.2558333333333	-0.306044257450121\\
10.2613888888889	-0.3072394589305\\
10.2669444444444	-0.308400924924359\\
10.2725	-0.309507821935176\\
10.2780555555556	-0.312925502032218\\
10.2836111111111	-0.312837283713373\\
10.2891666666667	-0.313065281772778\\
10.2947222222222	-0.316139516836999\\
10.3002777777778	-0.318677449856058\\
10.3058333333333	-0.317413178981063\\
10.3113888888889	-0.319925624167555\\
10.3169444444444	-0.319744130563917\\
10.3225	-0.31927556338486\\
10.3280555555556	-0.319555932621803\\
10.3336111111111	-0.321779056723493\\
10.3391666666667	-0.322302708127\\
10.3447222222222	-0.321781811231468\\
10.3502777777778	-0.324046718925497\\
10.3558333333333	-0.324271758069199\\
10.3613888888889	-0.324279171728358\\
10.3669444444444	-0.325131300134782\\
10.3725	-0.324832169503375\\
10.3780555555556	-0.323828867794053\\
10.3836111111111	-0.32365925758399\\
10.3891666666667	-0.331217484779189\\
10.3947222222222	-0.330221297822973\\
10.4002777777778	-0.329884198314626\\
10.4058333333333	-0.329004100344726\\
10.4113888888889	-0.328234005525611\\
10.4169444444444	-0.324899919367325\\
10.4225	-0.325875385914077\\
10.4280555555556	-0.328816703116747\\
10.4336111111111	-0.328703287935595\\
10.4391666666667	-0.331090165170167\\
10.4447222222222	-0.332293538557373\\
10.4502777777778	-0.332249594352829\\
10.4558333333333	-0.335041943746742\\
10.4613888888889	-0.332335391873022\\
10.4669444444444	-0.331763169664432\\
10.4725	-0.334967470038853\\
10.4780555555556	-0.335233785623459\\
10.4836111111111	-0.335529227665336\\
10.4891666666667	-0.33669715128416\\
10.4947222222222	-0.338457168473819\\
10.5002777777778	-0.337772023978857\\
10.5058333333333	-0.338994503230696\\
10.5113888888889	-0.33962512239625\\
10.5169444444444	-0.340864084964683\\
10.5225	-0.342736378660375\\
10.5280555555556	-0.343059501184707\\
10.5336111111111	-0.344960239130184\\
10.5391666666667	-0.346388529030629\\
10.5447222222222	-0.345324722122909\\
10.5502777777778	-0.344890895088922\\
10.5558333333333	-0.346277537314824\\
10.5613888888889	-0.346892482445856\\
10.5669444444444	-0.350090047990126\\
10.5725	-0.351566900161166\\
10.5780555555556	-0.351139295177178\\
10.5836111111111	-0.35155160871422\\
10.5891666666667	-0.353933515552836\\
10.5947222222222	-0.355878019239486\\
10.6002777777778	-0.352655784641347\\
10.6058333333333	-0.353733354068353\\
10.6113888888889	-0.363395148176038\\
10.6169444444444	-0.359939747630297\\
10.6225	-0.36040546763364\\
10.6280555555556	-0.362630732972653\\
10.6336111111111	-0.364404008494635\\
10.6391666666667	-0.364572598575817\\
10.6447222222222	-0.364749083405409\\
10.6502777777778	-0.365961032668849\\
10.6558333333333	-0.365798997363369\\
10.6613888888889	-0.370948963660906\\
10.6669444444444	-0.371274890701952\\
10.6725	-0.370388182637959\\
10.6780555555556	-0.371077641451071\\
10.6836111111111	-0.371104008847464\\
10.6891666666667	-0.37043368779323\\
10.6947222222222	-0.369605114898361\\
10.7002777777778	-0.372124514837692\\
10.7058333333333	-0.371496686379981\\
10.7113888888889	-0.372644387991976\\
10.7169444444444	-0.372390062494101\\
10.7225	-0.373994725530584\\
10.7280555555556	-0.373112174771512\\
10.7336111111111	-0.3731985119718\\
10.7391666666667	-0.373979460395736\\
10.7447222222222	-0.377269559909143\\
10.7502777777778	-0.380506412048789\\
10.7558333333333	-0.382074838971334\\
10.7613888888889	-0.38218186005771\\
10.7669444444444	-0.383829137139557\\
10.7725	-0.385124773797932\\
10.7780555555556	-0.384841922163507\\
10.7836111111111	-0.387088503482378\\
10.7891666666667	-0.387955643812756\\
10.7947222222222	-0.389181083466794\\
10.8002777777778	-0.388673727974731\\
10.8058333333333	-0.391277104128828\\
10.8113888888889	-0.394606543246358\\
10.8169444444444	-0.394166553553304\\
10.8225	-0.393297376064967\\
10.8280555555556	-0.39321583945381\\
10.8336111111111	-0.395570289422739\\
10.8391666666667	-0.39652593628263\\
10.8447222222222	-0.397385060532065\\
10.8502777777778	-0.396465107167917\\
10.8558333333333	-0.395835210795096\\
10.8613888888889	-0.398167060862527\\
10.8669444444444	-0.397067239754717\\
10.8725	-0.397611527929219\\
10.8780555555556	-0.397128826238349\\
10.8836111111111	-0.397962927952979\\
10.8891666666667	-0.39813599942826\\
10.8947222222222	-0.398911903129479\\
10.9002777777778	-0.399250863912558\\
10.9058333333333	-0.40024538886454\\
10.9113888888889	-0.399936937787354\\
10.9169444444444	-0.399675559211102\\
10.9225	-0.399437575590116\\
10.9280555555556	-0.400601981999972\\
10.9336111111111	-0.402663039461193\\
10.9391666666667	-0.404268825060789\\
10.9447222222222	-0.406438735313458\\
10.9502777777778	-0.403933453670412\\
10.9558333333333	-0.403898547300867\\
10.9613888888889	-0.405120353081278\\
10.9669444444444	-0.406847745769655\\
10.9725	-0.40736215128677\\
10.9780555555556	-0.410438647012065\\
10.9836111111111	-0.409072172920645\\
10.9891666666667	-0.41033124139111\\
10.9947222222222	-0.412033774578778\\
11.0002777777778	-0.41357958314502\\
11.0058333333333	-0.412162778969361\\
11.0113888888889	-0.414411726397957\\
11.0169444444444	-0.416876201231992\\
11.0225	-0.418503195414093\\
11.0280555555556	-0.419767020181007\\
11.0336111111111	-0.422999670849334\\
11.0391666666667	-0.424457481867337\\
11.0447222222222	-0.425153562711827\\
11.0502777777778	-0.424812534081274\\
11.0558333333333	-0.427850453405745\\
11.0613888888889	-0.426698386393589\\
11.0669444444444	-0.428050102082956\\
11.0725	-0.428299101181483\\
11.0780555555556	-0.42994726648771\\
11.0836111111111	-0.431418316385652\\
11.0891666666667	-0.432139391159641\\
11.0947222222222	-0.433388938733232\\
11.1002777777778	-0.43416146096197\\
11.1058333333333	-0.435473435006016\\
11.1113888888889	-0.435502401100604\\
11.1169444444444	-0.436538252830166\\
11.1225	-0.436653307574155\\
11.1280555555556	-0.437629139096344\\
11.1336111111111	-0.437388091211308\\
11.1391666666667	-0.437092927082302\\
11.1447222222222	-0.439148018367502\\
11.1502777777778	-0.43910574030306\\
11.1558333333333	-0.439504394983493\\
11.1613888888889	-0.440447422976198\\
11.1669444444444	-0.445516683387397\\
11.1725	-0.448607814147722\\
11.1780555555556	-0.446951898867641\\
11.1836111111111	-0.446096249526242\\
11.1891666666667	-0.448354229610167\\
11.1947222222222	-0.447458552778328\\
11.2002777777778	-0.447246592693143\\
11.2058333333333	-0.447128946685897\\
11.2113888888889	-0.445906057821044\\
11.2169444444444	-0.444901628815627\\
11.2225	-0.444995048171816\\
11.2280555555556	-0.446596346674959\\
11.2336111111111	-0.445955046403402\\
11.2391666666667	-0.44838101164649\\
11.2447222222222	-0.448905148986385\\
11.2502777777778	-0.449683011604787\\
11.2558333333333	-0.449843047846189\\
11.2613888888889	-0.450259041724757\\
11.2669444444444	-0.448002259934868\\
11.2725	-0.452329867901367\\
11.2780555555556	-0.451195522185344\\
11.2836111111111	-0.450706983420812\\
11.2891666666667	-0.451718660415524\\
11.2947222222222	-0.449918822626432\\
11.3002777777778	-0.451072608974483\\
11.3058333333333	-0.450810083203618\\
11.3113888888889	-0.450184732285852\\
11.3169444444444	-0.451283509895902\\
11.3225	-0.451196191230069\\
11.3280555555556	-0.451222298681684\\
11.3336111111111	-0.451409181495138\\
11.3391666666667	-0.45281867064375\\
11.3447222222222	-0.452760006951922\\
11.3502777777778	-0.45419546853951\\
11.3558333333333	-0.453996561409477\\
11.3613888888889	-0.455473413287323\\
11.3669444444444	-0.457489560931169\\
11.3725	-0.455920063606175\\
11.3780555555556	-0.458456949458276\\
11.3836111111111	-0.458112587340249\\
11.3891666666667	-0.458047441549565\\
11.3947222222222	-0.458770644050726\\
11.4002777777778	-0.458952212318759\\
11.4058333333333	-0.46013900900359\\
11.4113888888889	-0.461786465740502\\
11.4169444444444	-0.461492076624798\\
11.4225	-0.461776136569709\\
11.4280555555556	-0.461960418014901\\
11.4336111111111	-0.463344511059269\\
11.4391666666667	-0.463600776412314\\
11.4447222222222	-0.463421017852987\\
11.4502777777778	-0.4620891606371\\
11.4558333333333	-0.463478174152806\\
11.4613888888889	-0.463821893660589\\
11.4669444444444	-0.466232231970444\\
11.4725	-0.468771954406819\\
11.4780555555556	-0.470706386000189\\
11.4836111111111	-0.471974534411122\\
11.4891666666667	-0.472395277161408\\
11.4947222222222	-0.473682268879246\\
11.5002777777778	-0.473956247647062\\
11.5058333333333	-0.473953501336819\\
11.5113888888889	-0.473763446698098\\
11.5169444444444	-0.475191806291252\\
11.5225	-0.47760747256977\\
11.5280555555556	-0.478190761994849\\
11.5336111111111	-0.483996107654051\\
11.5391666666667	-0.484592733259049\\
11.5447222222222	-0.484604752196052\\
11.5502777777778	-0.484399108779621\\
11.5558333333333	-0.484654121258275\\
11.5613888888889	-0.484467575979107\\
11.5669444444444	-0.483916195190301\\
11.5725	-0.4859560715722\\
11.5780555555556	-0.485655141493457\\
11.5836111111111	-0.482732946021208\\
11.5891666666667	-0.481652716979744\\
11.5947222222222	-0.481433334875683\\
11.6002777777778	-0.481843335366894\\
11.6058333333333	-0.483282953008979\\
11.6113888888889	-0.484277762487545\\
11.6169444444444	-0.485435162095674\\
11.6225	-0.484519229142364\\
11.6280555555556	-0.486103192236097\\
11.6336111111111	-0.486058115504494\\
11.6391666666667	-0.486916200743019\\
11.6447222222222	-0.488410500308105\\
11.6502777777778	-0.488917658402888\\
11.6558333333333	-0.487606679766428\\
11.6613888888889	-0.488017965219107\\
11.6669444444444	-0.485634132276809\\
11.6725	-0.48661047547811\\
11.6780555555556	-0.487982156778712\\
11.6836111111111	-0.488885260597259\\
11.6891666666667	-0.489828064480944\\
11.6947222222222	-0.492399336215375\\
11.7002777777778	-0.493618422062674\\
11.7058333333333	-0.493610401290909\\
11.7113888888889	-0.492785696780066\\
11.7169444444444	-0.493908078690399\\
11.7225	-0.494316307209122\\
11.7280555555556	-0.493466090263422\\
11.7336111111111	-0.494721601786214\\
11.7391666666667	-0.494546309272449\\
11.7447222222222	-0.492784404334823\\
11.7502777777778	-0.493246492224957\\
11.7558333333333	-0.493240968692437\\
11.7613888888889	-0.4942988308639\\
11.7669444444444	-0.495210697111893\\
11.7725	-0.494003343701209\\
11.7780555555556	-0.49547482244957\\
11.7836111111111	-0.495501412272129\\
11.7891666666667	-0.494981359538567\\
11.7947222222222	-0.497709506909277\\
11.8002777777778	-0.498902852133713\\
11.8058333333333	-0.498439268997737\\
11.8113888888889	-0.50105638249835\\
11.8169444444444	-0.501525611190522\\
11.8225	-0.504320791484934\\
11.8280555555556	-0.501374988209255\\
11.8336111111111	-0.502065693223779\\
11.8391666666667	-0.500682165088487\\
11.8447222222222	-0.499690172094844\\
11.8502777777778	-0.500475732022221\\
11.8558333333333	-0.505205860754126\\
11.8613888888889	-0.506644026029282\\
11.8669444444444	-0.50456902750707\\
11.8725	-0.505378896871529\\
11.8780555555556	-0.503954970854007\\
11.8836111111111	-0.507242824815796\\
11.8891666666667	-0.506963882187458\\
11.8947222222222	-0.506093488285238\\
11.9002777777778	-0.507199829277415\\
11.9058333333333	-0.505287228438443\\
11.9113888888889	-0.508402445395958\\
11.9169444444444	-0.508125769951531\\
11.9225	-0.506351572974852\\
11.9280555555556	-0.50747962406779\\
11.9336111111111	-0.508124866201215\\
11.9391666666667	-0.51078583944677\\
11.9447222222222	-0.511212491555791\\
11.9502777777778	-0.512826589096699\\
11.9558333333333	-0.514842086043366\\
11.9613888888889	-0.513370680695844\\
11.9669444444444	-0.514634596924129\\
11.9725	-0.514904277181283\\
11.9780555555556	-0.514035918731347\\
11.9836111111111	-0.514274008853131\\
11.9891666666667	-0.516668934868287\\
11.9947222222222	-0.51550775405422\\
12.0002777777778	-0.515664120439388\\
12.0058333333333	-0.517600684714311\\
12.0113888888889	-0.517529354713206\\
12.0169444444444	-0.520009662329018\\
12.0225	-0.519330161901589\\
12.0280555555556	-0.519472060961493\\
12.0336111111111	-0.518859982673857\\
12.0391666666667	-0.521716794733349\\
12.0447222222222	-0.521364034190718\\
12.0502777777778	-0.522652461428817\\
12.0558333333333	-0.523912012492375\\
12.0613888888889	-0.526976933481355\\
12.0669444444444	-0.526880030511132\\
12.0725	-0.527885242187889\\
12.0780555555556	-0.527252766821059\\
12.0836111111111	-0.528828308685027\\
12.0891666666667	-0.528062964823987\\
12.0947222222222	-0.529955041984683\\
12.1002777777778	-0.531383446025861\\
12.1058333333333	-0.53207149627988\\
12.1113888888889	-0.536217626937724\\
12.1169444444444	-0.535954953685123\\
12.1225	-0.535574172648326\\
12.1280555555556	-0.537038887602029\\
12.1336111111111	-0.538135210458971\\
12.1391666666667	-0.537944969211723\\
12.1447222222222	-0.536775411397741\\
12.1502777777778	-0.534447766654243\\
12.1558333333333	-0.535209664075057\\
12.1613888888889	-0.53550448072334\\
12.1669444444444	-0.534294999185289\\
12.1725	-0.534644550876862\\
12.1780555555556	-0.534504785298963\\
12.1836111111111	-0.534635113597635\\
12.1891666666667	-0.534569898043567\\
12.1947222222222	-0.53498157971738\\
12.2002777777778	-0.533858829358573\\
12.2058333333333	-0.535320791870643\\
12.2113888888889	-0.535903355830389\\
12.2169444444444	-0.536457064660523\\
12.2225	-0.537157047265586\\
12.2280555555556	-0.538976679089782\\
12.2336111111111	-0.537953305489382\\
12.2391666666667	-0.53785948405843\\
12.2447222222222	-0.537347626128381\\
12.2502777777778	-0.53744507481613\\
12.2558333333333	-0.537087579150738\\
12.2613888888889	-0.538336901859548\\
12.2669444444444	-0.537084901136988\\
12.2725	-0.537836943844704\\
12.2780555555556	-0.537115351073818\\
12.2836111111111	-0.540888449469406\\
12.2891666666667	-0.539420276546597\\
12.2947222222222	-0.538525179017648\\
12.3002777777778	-0.539591865921334\\
12.3058333333333	-0.540063646719775\\
12.3113888888889	-0.538200752581497\\
12.3169444444444	-0.538829475297829\\
12.3225	-0.541272870729623\\
12.3280555555556	-0.542677022158416\\
12.3336111111111	-0.543571010619986\\
12.3391666666667	-0.541775681270135\\
12.3447222222222	-0.543817753245204\\
12.3502777777778	-0.544496303061848\\
12.3558333333333	-0.547342220139007\\
12.3613888888889	-0.547075319312011\\
12.3669444444444	-0.54998140687173\\
12.3725	-0.550197013448606\\
12.3780555555556	-0.549629877543188\\
12.3836111111111	-0.550964140600108\\
12.3891666666667	-0.552810746562837\\
12.3947222222222	-0.55361944179318\\
12.4002777777778	-0.55465178849208\\
12.4058333333333	-0.556310214974785\\
12.4113888888889	-0.555844361452961\\
12.4169444444444	-0.55497578888758\\
12.4225	-0.558495380785977\\
12.4280555555556	-0.561075243648699\\
12.4336111111111	-0.561838498535801\\
12.4391666666667	-0.562689971435324\\
12.4447222222222	-0.564840134846993\\
12.4502777777778	-0.565819788756808\\
12.4558333333333	-0.566019476831761\\
12.4613888888889	-0.56738651442276\\
12.4669444444444	-0.56882927812636\\
12.4725	-0.569989922440231\\
12.4780555555556	-0.569006996913467\\
12.4836111111111	-0.56882891288959\\
12.4891666666667	-0.569841881258264\\
12.4947222222222	-0.570140441096029\\
12.5002777777778	-0.568663292884232\\
12.5058333333333	-0.572132875025143\\
12.5113888888889	-0.568578584573532\\
12.5169444444444	-0.570093897126871\\
12.5225	-0.572096618655736\\
12.5280555555556	-0.574564226377389\\
12.5336111111111	-0.575541254277437\\
12.5391666666667	-0.576767946070765\\
12.5447222222222	-0.578555768035065\\
12.5502777777778	-0.577351344517739\\
12.5558333333333	-0.577802555199736\\
12.5613888888889	-0.57674606339628\\
12.5669444444444	-0.577733711971859\\
12.5725	-0.578150503419982\\
12.5780555555556	-0.581802594938692\\
12.5836111111111	-0.578980014258536\\
12.5891666666667	-0.583060702316382\\
12.5947222222222	-0.584811631273153\\
12.6002777777778	-0.586457363309442\\
12.6058333333333	-0.585972358758286\\
12.6113888888889	-0.588268353228709\\
12.6169444444444	-0.587200080603408\\
12.6225	-0.58773412589794\\
12.6280555555556	-0.586129095838088\\
12.6336111111111	-0.587079523904849\\
12.6391666666667	-0.587767457797756\\
12.6447222222222	-0.588714456210384\\
12.6502777777778	-0.589830798050801\\
12.6558333333333	-0.591592064668231\\
12.6613888888889	-0.592555892497382\\
12.6669444444444	-0.591607613515137\\
12.6725	-0.591776587371017\\
12.6780555555556	-0.593032842865731\\
12.6836111111111	-0.594039764551677\\
12.6891666666667	-0.594026451168539\\
12.6947222222222	-0.59173444778537\\
12.7002777777778	-0.594033850473208\\
12.7058333333333	-0.597422862984461\\
12.7113888888889	-0.595996314762546\\
12.7169444444444	-0.597767727265755\\
12.7225	-0.598215134657463\\
12.7280555555556	-0.595991587409877\\
12.7336111111111	-0.595400660818848\\
12.7391666666667	-0.596260687589742\\
12.7447222222222	-0.596767700145899\\
12.7502777777778	-0.600697096525789\\
12.7558333333333	-0.599798912823143\\
12.7613888888889	-0.600268294698645\\
12.7669444444444	-0.601715311243491\\
12.7725	-0.604027505940956\\
12.7780555555556	-0.604244685506625\\
12.7836111111111	-0.606865946732948\\
12.7891666666667	-0.607188460960308\\
12.7947222222222	-0.607906477807337\\
12.8002777777778	-0.607392952205715\\
12.8058333333333	-0.60671682636941\\
12.8113888888889	-0.608186473160003\\
12.8169444444444	-0.610475587554572\\
12.8225	-0.608964636335922\\
12.8280555555556	-0.607283187630496\\
12.8336111111111	-0.60606133885585\\
12.8391666666667	-0.60570246317196\\
12.8447222222222	-0.605875828315185\\
12.8502777777778	-0.606797127087023\\
12.8558333333333	-0.609035863651239\\
12.8613888888889	-0.607741343934159\\
12.8669444444444	-0.607311944232779\\
12.8725	-0.60619381665401\\
12.8780555555556	-0.606166163959668\\
12.8836111111111	-0.607845615853896\\
12.8891666666667	-0.606833675409881\\
12.8947222222222	-0.607185769435871\\
12.9002777777778	-0.612224133302461\\
12.9058333333333	-0.615279627911497\\
12.9113888888889	-0.611322830929202\\
12.9169444444444	-0.609848726445008\\
12.9225	-0.611146428258279\\
12.9280555555556	-0.612386512316051\\
12.9336111111111	-0.609485397704615\\
12.9391666666667	-0.607287699905386\\
12.9447222222222	-0.608558868380805\\
12.9502777777778	-0.608164395600024\\
12.9558333333333	-0.607686142791098\\
12.9613888888889	-0.609061990774628\\
12.9669444444444	-0.609123106597579\\
12.9725	-0.609696520169478\\
12.9780555555556	-0.609333938591785\\
12.9836111111111	-0.61372423722695\\
12.9891666666667	-0.608907118259401\\
12.9947222222222	-0.610728016330386\\
13.0002777777778	-0.61246412933886\\
13.0058333333333	-0.612928247293337\\
13.0113888888889	-0.613413614576977\\
13.0169444444444	-0.613836066203041\\
13.0225	-0.610631733345624\\
13.0280555555556	-0.612667098004957\\
13.0336111111111	-0.613820845474745\\
13.0391666666667	-0.614885573678488\\
13.0447222222222	-0.615409543683181\\
13.0502777777778	-0.613415534043365\\
13.0558333333333	-0.614045449015688\\
13.0613888888889	-0.614240755725627\\
13.0669444444444	-0.614611540683379\\
13.0725	-0.616463030708046\\
13.0780555555556	-0.618536827765337\\
13.0836111111111	-0.613722020580712\\
13.0891666666667	-0.615952893871943\\
13.0947222222222	-0.613107822881747\\
13.1002777777778	-0.614508579359504\\
13.1058333333333	-0.614503777900047\\
13.1113888888889	-0.614594565163048\\
13.1169444444444	-0.617228507913558\\
13.1225	-0.616869307941918\\
13.1280555555556	-0.614917346834771\\
13.1336111111111	-0.61476582046064\\
13.1391666666667	-0.614804846315705\\
13.1447222222222	-0.615164864135623\\
13.1502777777778	-0.615639052403682\\
13.1558333333333	-0.619252114720957\\
13.1613888888889	-0.622249429426161\\
13.1669444444444	-0.624524273515264\\
13.1725	-0.620867546560727\\
13.1780555555556	-0.622080328564198\\
13.1836111111111	-0.624018355213058\\
13.1891666666667	-0.623305146818413\\
13.1947222222222	-0.628134433349742\\
13.2002777777778	-0.626567027332351\\
13.2058333333333	-0.623424704001822\\
13.2113888888889	-0.623496071133304\\
13.2169444444444	-0.623865748633242\\
13.2225	-0.625186098607316\\
13.2280555555556	-0.622735739573011\\
13.2336111111111	-0.623673737892679\\
13.2391666666667	-0.625229983441897\\
13.2447222222222	-0.624507386373969\\
13.2502777777778	-0.624128787226751\\
13.2558333333333	-0.62434475241085\\
13.2613888888889	-0.625802068700173\\
13.2669444444444	-0.624110809117361\\
13.2725	-0.624640047997441\\
13.2780555555556	-0.626152534145855\\
13.2836111111111	-0.629577245925415\\
13.2891666666667	-0.629362061839482\\
13.2947222222222	-0.630363765424769\\
13.3002777777778	-0.629497499144828\\
13.3058333333333	-0.629334435949084\\
13.3113888888889	-0.626676043687539\\
13.3169444444444	-0.62367310712633\\
13.3225	-0.623803406855647\\
13.3280555555556	-0.624346456782949\\
13.3336111111111	-0.623562651635266\\
13.3391666666667	-0.624627404435219\\
13.3447222222222	-0.625637958695758\\
13.3502777777778	-0.62833806976453\\
13.3558333333333	-0.628469065471588\\
13.3613888888889	-0.627975319490036\\
13.3669444444444	-0.627181622292224\\
13.3725	-0.629088143416802\\
13.3780555555556	-0.631041805219641\\
13.3836111111111	-0.62836687867583\\
13.3891666666667	-0.630828953243971\\
13.3947222222222	-0.630880259240058\\
13.4002777777778	-0.631761140023026\\
13.4058333333333	-0.630769244207041\\
13.4113888888889	-0.630884790407173\\
13.4169444444444	-0.632869753768766\\
13.4225	-0.633312858656047\\
13.4280555555556	-0.63429017694788\\
13.4336111111111	-0.63582906254743\\
13.4391666666667	-0.63613526999865\\
13.4447222222222	-0.639175602331859\\
13.4502777777778	-0.640031316821976\\
13.4558333333333	-0.642089928630285\\
13.4613888888889	-0.643165228692628\\
13.4669444444444	-0.645467005375935\\
13.4725	-0.646281894539954\\
13.4780555555556	-0.64832274436305\\
13.4836111111111	-0.647850455682422\\
13.4891666666667	-0.649340661007896\\
13.4947222222222	-0.651076060464911\\
13.5002777777778	-0.651338900505971\\
13.5058333333333	-0.65437169457499\\
13.5113888888889	-0.653099821843671\\
13.5169444444444	-0.653789866942296\\
13.5225	-0.655420099670803\\
13.5280555555556	-0.655576196431648\\
13.5336111111111	-0.656292363558263\\
13.5391666666667	-0.655379656753317\\
13.5447222222222	-0.654459242360029\\
13.5502777777778	-0.65207454723977\\
13.5558333333333	-0.651864758361568\\
13.5613888888889	-0.653227647311481\\
13.5669444444444	-0.650871747670076\\
13.5725	-0.651105219340026\\
13.5780555555556	-0.653471862689979\\
13.5836111111111	-0.658438934783377\\
13.5891666666667	-0.658592935627172\\
13.5947222222222	-0.65972815100681\\
13.6002777777778	-0.661248813082762\\
13.6058333333333	-0.662225792646533\\
13.6113888888889	-0.661613786786279\\
13.6169444444444	-0.661567380238076\\
13.6225	-0.663281206140762\\
13.6280555555556	-0.664535042899562\\
13.6336111111111	-0.665114458648488\\
13.6391666666667	-0.667632305006531\\
13.6447222222222	-0.669288525414332\\
13.6502777777778	-0.670158622139707\\
13.6558333333333	-0.670095870258933\\
13.6613888888889	-0.671525069922869\\
13.6669444444444	-0.671592604396184\\
13.6725	-0.673747209812754\\
13.6780555555556	-0.674972021886899\\
13.6836111111111	-0.673555053808931\\
13.6891666666667	-0.675196543860343\\
13.6947222222222	-0.677795576732433\\
13.7002777777778	-0.67576845786776\\
13.7058333333333	-0.676180240465552\\
13.7113888888889	-0.675917303349234\\
13.7169444444444	-0.678141177397116\\
13.7225	-0.676403443557136\\
13.7280555555556	-0.674978922404265\\
13.7336111111111	-0.675862939076451\\
13.7391666666667	-0.674767456018024\\
13.7447222222222	-0.678871434655237\\
13.7502777777778	-0.67971235097245\\
13.7558333333333	-0.679106485818267\\
13.7613888888889	-0.67894644380325\\
13.7669444444444	-0.676598290952352\\
13.7725	-0.674472248325446\\
13.7780555555556	-0.67665320524252\\
13.7836111111111	-0.676656299823756\\
13.7891666666667	-0.673778417532635\\
13.7947222222222	-0.675141911983776\\
13.8002777777778	-0.675176335825829\\
13.8058333333333	-0.678549816392662\\
13.8113888888889	-0.678479666657214\\
13.8169444444444	-0.677786302188091\\
13.8225	-0.681221761334059\\
13.8280555555556	-0.679985960741367\\
13.8336111111111	-0.685428126437229\\
13.8391666666667	-0.682700538927041\\
13.8447222222222	-0.683359149522806\\
13.8502777777778	-0.684226911787678\\
13.8558333333333	-0.683121429779282\\
13.8613888888889	-0.681137536980378\\
13.8669444444444	-0.681659657381398\\
13.8725	-0.681919416662969\\
13.8780555555556	-0.681747593977427\\
13.8836111111111	-0.682672857581186\\
13.8891666666667	-0.680603126950746\\
13.8947222222222	-0.683031698330096\\
13.9002777777778	-0.685448552187034\\
13.9058333333333	-0.685035579842984\\
13.9113888888889	-0.684100255242256\\
13.9169444444444	-0.682668699673027\\
13.9225	-0.688855275926027\\
13.9280555555556	-0.688334400947997\\
13.9336111111111	-0.691057331200811\\
13.9391666666667	-0.690495163940495\\
13.9447222222222	-0.689164395123815\\
13.9502777777778	-0.688772818637117\\
13.9558333333333	-0.692002958640992\\
13.9613888888889	-0.694279753198809\\
13.9669444444444	-0.697650986361309\\
13.9725	-0.69747524049396\\
13.9780555555556	-0.699066541121965\\
13.9836111111111	-0.698719099483346\\
13.9891666666667	-0.700442258944559\\
13.9947222222222	-0.702978760113109\\
14.0002777777778	-0.703192109634984\\
14.0058333333333	-0.703695119145751\\
14.0113888888889	-0.702112622143383\\
14.0169444444444	-0.698490599355465\\
14.0225	-0.698609486724775\\
14.0280555555556	-0.702574789865457\\
14.0336111111111	-0.705935377768456\\
14.0391666666667	-0.702787866004955\\
14.0447222222222	-0.704373559835223\\
14.0502777777778	-0.703903820084543\\
14.0558333333333	-0.703953344013283\\
14.0613888888889	-0.703693932965981\\
14.0669444444444	-0.705214324263532\\
14.0725	-0.708106055899456\\
14.0780555555556	-0.705230082609622\\
14.0836111111111	-0.7051425917066\\
14.0891666666667	-0.705193200234621\\
14.0947222222222	-0.705693552912185\\
14.1002777777778	-0.701667088992839\\
14.1058333333333	-0.698740930823159\\
14.1113888888889	-0.697941359281184\\
14.1169444444444	-0.70132205520001\\
14.1225	-0.703683662963016\\
14.1280555555556	-0.704001842066614\\
14.1336111111111	-0.707372221209914\\
14.1391666666667	-0.709694733779303\\
14.1447222222222	-0.709948281262958\\
14.1502777777778	-0.713406627658034\\
14.1558333333333	-0.715050584172497\\
14.1613888888889	-0.718525464811944\\
14.1669444444444	-0.71817927925316\\
14.1725	-0.718602434055494\\
14.1780555555556	-0.720938577176819\\
14.1836111111111	-0.723004920398543\\
14.1891666666667	-0.722613759167904\\
14.1947222222222	-0.723836496054032\\
14.2002777777778	-0.725978776552046\\
14.2058333333333	-0.726037769405142\\
14.2113888888889	-0.722663390805553\\
14.2169444444444	-0.719925802998046\\
14.2225	-0.721928879352077\\
14.2280555555556	-0.722048251609402\\
14.2336111111111	-0.72227208617593\\
14.2391666666667	-0.720540219166009\\
14.2447222222222	-0.723173141457593\\
14.2502777777778	-0.723851959841115\\
14.2558333333333	-0.723906026502651\\
14.2613888888889	-0.725791598845702\\
14.2669444444444	-0.726596869583041\\
14.2725	-0.721893145403675\\
14.2780555555556	-0.719542771304035\\
14.2836111111111	-0.72265277456521\\
14.2891666666667	-0.727141140859724\\
14.2947222222222	-0.726517715302939\\
14.3002777777778	-0.727873097538396\\
14.3058333333333	-0.727492398412717\\
14.3113888888889	-0.72380512525436\\
14.3169444444444	-0.726070960143247\\
14.3225	-0.722925313035285\\
14.3280555555556	-0.725245737499439\\
14.3336111111111	-0.72682898517443\\
14.3391666666667	-0.726888894706423\\
14.3447222222222	-0.728311867509323\\
14.3502777777778	-0.7316542423434\\
14.3558333333333	-0.730213051534883\\
14.3613888888889	-0.733926200828783\\
14.3669444444444	-0.737798191717852\\
14.3725	-0.739656489941077\\
14.3780555555556	-0.737296449510164\\
14.3836111111111	-0.740999087417716\\
14.3891666666667	-0.745746943423113\\
14.3947222222222	-0.748160001180235\\
14.4002777777778	-0.749802596224952\\
14.4058333333333	-0.749205906563197\\
14.4113888888889	-0.752814026731625\\
14.4169444444444	-0.752610579827891\\
14.4225	-0.753971855701993\\
14.4280555555556	-0.754310732590895\\
14.4336111111111	-0.756395096626034\\
14.4391666666667	-0.755581869947617\\
14.4447222222222	-0.755575473966096\\
14.4502777777778	-0.759705321428477\\
14.4558333333333	-0.76247096030708\\
14.4613888888889	-0.762617854089614\\
14.4669444444444	-0.761477533110243\\
14.4725	-0.760500667931167\\
14.4780555555556	-0.759687301006494\\
14.4836111111111	-0.755932444245403\\
14.4891666666667	-0.754941947770578\\
14.4947222222222	-0.7546985554027\\
14.5002777777778	-0.752677108197843\\
14.5058333333333	-0.754940805207442\\
14.5113888888889	-0.756028642364936\\
14.5169444444444	-0.758544095727483\\
14.5225	-0.753118419938762\\
14.5280555555556	-0.753767116481385\\
14.5336111111111	-0.755031947390872\\
14.5391666666667	-0.753117109221404\\
14.5447222222222	-0.754044777087732\\
14.5502777777778	-0.756912187532685\\
14.5558333333333	-0.755999536311602\\
14.5613888888889	-0.757126003657279\\
14.5669444444444	-0.762661448074216\\
14.5725	-0.75984469497918\\
14.5780555555556	-0.762731885879857\\
14.5836111111111	-0.763572568269344\\
14.5891666666667	-0.763585091547174\\
14.5947222222222	-0.758012101354822\\
14.6002777777778	-0.757788292406186\\
14.6058333333333	-0.754931371607808\\
14.6113888888889	-0.752926975683568\\
14.6169444444444	-0.758024801275077\\
14.6225	-0.756950048565418\\
14.6280555555556	-0.757665525913944\\
14.6336111111111	-0.760982508311873\\
14.6391666666667	-0.760299676166155\\
14.6447222222222	-0.762709273905188\\
14.6502777777778	-0.76482769404049\\
14.6558333333333	-0.765125373580713\\
14.6613888888889	-0.764053369839981\\
14.6669444444444	-0.75977360337917\\
14.6725	-0.757844577013448\\
14.6780555555556	-0.755327038701444\\
14.6836111111111	-0.75889068111089\\
14.6891666666667	-0.757762960151514\\
14.6947222222222	-0.760156410503112\\
14.7002777777778	-0.762129035614929\\
14.7058333333333	-0.763103472009179\\
14.7113888888889	-0.768914422069299\\
14.7169444444444	-0.770518376695738\\
14.7225	-0.770096695492413\\
14.7280555555556	-0.770090797596497\\
14.7336111111111	-0.764626396209216\\
14.7391666666667	-0.765498616281607\\
14.7447222222222	-0.765994018896794\\
14.7502777777778	-0.767287986842107\\
14.7558333333333	-0.764642570202838\\
14.7613888888889	-0.76466254898882\\
14.7669444444444	-0.770148482675015\\
14.7725	-0.774135076282522\\
14.7780555555556	-0.771809817309183\\
14.7836111111111	-0.772738522757328\\
14.7891666666667	-0.768634439807497\\
14.7947222222222	-0.769256682438001\\
14.8002777777778	-0.772272466617437\\
14.8058333333333	-0.770975684617384\\
14.8113888888889	-0.774566129225379\\
14.8169444444444	-0.773935938711943\\
14.8225	-0.770539122603994\\
14.8280555555556	-0.770372871822221\\
14.8336111111111	-0.770456711554539\\
14.8391666666667	-0.769457606539126\\
14.8447222222222	-0.769953148790957\\
14.8502777777778	-0.773271434508535\\
14.8558333333333	-0.77125093250701\\
14.8613888888889	-0.769441320776778\\
14.8669444444444	-0.772056081974288\\
14.8725	-0.771759396593585\\
14.8780555555556	-0.774466921031437\\
14.8836111111111	-0.770136542945107\\
14.8891666666667	-0.774020210759245\\
14.8947222222222	-0.778947081418154\\
14.9002777777778	-0.779564748319977\\
14.9058333333333	-0.778488496636313\\
14.9113888888889	-0.780430882287818\\
14.9169444444444	-0.780044919336313\\
14.9225	-0.779547560953892\\
14.9280555555556	-0.783587086702966\\
14.9336111111111	-0.785300408972481\\
14.9391666666667	-0.785891573429015\\
14.9447222222222	-0.789137392124207\\
14.9502777777778	-0.795433179715151\\
14.9558333333333	-0.795710351062901\\
14.9613888888889	-0.796583623123092\\
14.9669444444444	-0.802271613252517\\
14.9725	-0.799863540434677\\
14.9780555555556	-0.798769674822696\\
14.9836111111111	-0.801778665554128\\
14.9891666666667	-0.803988162673052\\
14.9947222222222	-0.804671322876472\\
15.0002777777778	-0.803171751304625\\
15.0058333333333	-0.805425424328911\\
15.0113888888889	-0.805210480941836\\
15.0169444444444	-0.806579411223401\\
15.0225	-0.803100522583137\\
15.0280555555556	-0.802102786860524\\
15.0336111111111	-0.80033499788531\\
15.0391666666667	-0.803534742537679\\
15.0447222222222	-0.80388826999095\\
15.0502777777778	-0.800154600253761\\
15.0558333333333	-0.805536028949251\\
15.0613888888889	-0.807932447764973\\
15.0669444444444	-0.807994058812677\\
15.0725	-0.81111923915715\\
15.0780555555556	-0.80670236744175\\
15.0836111111111	-0.80764086172925\\
15.0891666666667	-0.805956830279588\\
15.0947222222222	-0.801899818932875\\
15.1002777777778	-0.808253134578818\\
15.1058333333333	-0.805050940360093\\
15.1113888888889	-0.807036469023624\\
15.1169444444444	-0.807574463203464\\
15.1225	-0.809061414263216\\
15.1280555555556	-0.810833973813632\\
15.1336111111111	-0.815461505750313\\
15.1391666666667	-0.817789642275779\\
15.1447222222222	-0.821257527731952\\
15.1502777777778	-0.818027771224373\\
15.1558333333333	-0.815341674516322\\
15.1613888888889	-0.815235996912877\\
15.1669444444444	-0.813013327203019\\
15.1725	-0.812412471827924\\
15.1780555555556	-0.814629231322684\\
15.1836111111111	-0.814333608938366\\
15.1891666666667	-0.809575060539141\\
15.1947222222222	-0.813239826122132\\
15.2002777777778	-0.817728337954096\\
15.2058333333333	-0.818430347489331\\
15.2113888888889	-0.818773591476787\\
15.2169444444444	-0.816529743938061\\
15.2225	-0.816056965627495\\
15.2280555555556	-0.814387950911255\\
15.2336111111111	-0.816086653294471\\
15.2391666666667	-0.816685504711415\\
15.2447222222222	-0.817055932613165\\
15.2502777777778	-0.816748937976692\\
15.2558333333333	-0.808174726528042\\
15.2613888888889	-0.810466765712517\\
15.2669444444444	-0.805783713366952\\
15.2725	-0.80628315746463\\
15.2780555555556	-0.808995616619155\\
15.2836111111111	-0.810662404332951\\
15.2891666666667	-0.811286340323554\\
15.2947222222222	-0.815738443258842\\
15.3002777777778	-0.818081573560572\\
15.3058333333333	-0.819380029374976\\
15.3113888888889	-0.814751622345475\\
15.3169444444444	-0.814442610622907\\
15.3225	-0.820826096808109\\
15.3280555555556	-0.820268547201806\\
15.3336111111111	-0.81945528166134\\
15.3391666666667	-0.823535220660844\\
15.3447222222222	-0.821339763534496\\
15.3502777777778	-0.823783310738536\\
15.3558333333333	-0.822729360485044\\
15.3613888888889	-0.819020958439091\\
15.3669444444444	-0.821380779126777\\
15.3725	-0.818017990741433\\
15.3780555555556	-0.820334651324578\\
15.3836111111111	-0.817880630925516\\
15.3891666666667	-0.81201574322863\\
15.3947222222222	-0.812066440835601\\
15.4002777777778	-0.815402990844965\\
15.4058333333333	-0.816192258075127\\
15.4113888888889	-0.819011739712651\\
15.4169444444444	-0.820641736227495\\
15.4225	-0.820599683151728\\
15.4280555555556	-0.827513827747755\\
15.4336111111111	-0.825726489506258\\
15.4391666666667	-0.82494976626959\\
15.4447222222222	-0.826589598149266\\
15.4502777777778	-0.824029319619602\\
15.4558333333333	-0.826321234241524\\
15.4613888888889	-0.823797208491305\\
15.4669444444444	-0.826745853658597\\
15.4725	-0.828134822715008\\
15.4780555555556	-0.825289248854597\\
15.4836111111111	-0.82924164050237\\
15.4891666666667	-0.83357012738747\\
15.4947222222222	-0.832656151480139\\
15.5002777777778	-0.836972726712138\\
15.5058333333333	-0.838107764801558\\
15.5113888888889	-0.831286732071752\\
15.5169444444444	-0.835001355595182\\
15.5225	-0.84008039684805\\
15.5280555555556	-0.841063000998182\\
15.5336111111111	-0.840441473798881\\
15.5391666666667	-0.834671853069631\\
15.5447222222222	-0.833016445116237\\
15.5502777777778	-0.830350361044812\\
15.5558333333333	-0.824100294117744\\
15.5613888888889	-0.824931380752761\\
15.5669444444444	-0.825781424720184\\
15.5725	-0.825743500749489\\
15.5780555555556	-0.825045081295848\\
15.5836111111111	-0.817269960281409\\
15.5891666666667	-0.815495887638299\\
15.5947222222222	-0.814843451745877\\
15.6002777777778	-0.818340768503544\\
15.6058333333333	-0.818427827246335\\
15.6113888888889	-0.821528898591172\\
15.6169444444444	-0.820629785337292\\
15.6225	-0.821386586219743\\
15.6280555555556	-0.82819118269461\\
15.6336111111111	-0.831049946135609\\
15.6391666666667	-0.837079267924365\\
15.6447222222222	-0.836803497594793\\
15.6502777777778	-0.839165737566752\\
15.6558333333333	-0.835126767312025\\
15.6613888888889	-0.835795895097754\\
15.6669444444444	-0.838461045881894\\
15.6725	-0.838162083495872\\
15.6780555555556	-0.84216815385128\\
15.6836111111111	-0.849906017435262\\
15.6891666666667	-0.854508296407809\\
15.6947222222222	-0.853248285627061\\
15.7002777777778	-0.854457727801998\\
15.7058333333333	-0.854510952664139\\
15.7113888888889	-0.856270192935349\\
15.7169444444444	-0.855410161778512\\
15.7225	-0.857040725247309\\
15.7280555555556	-0.85926369091493\\
15.7336111111111	-0.851475701898829\\
15.7391666666667	-0.849653024231933\\
15.7447222222222	-0.854171245652937\\
15.7502777777778	-0.856702533208756\\
15.7558333333333	-0.865241264965839\\
15.7613888888889	-0.859101570065881\\
15.7669444444444	-0.861791891881759\\
15.7725	-0.864108716039518\\
15.7780555555556	-0.854588078271755\\
15.7836111111111	-0.86367280339019\\
15.7891666666667	-0.87773806015056\\
15.7947222222222	-0.8764456413684\\
15.8002777777778	-0.880812074923967\\
15.8058333333333	-0.881708403032997\\
15.8113888888889	-0.885319569165206\\
15.8169444444444	-0.88568033808487\\
15.8225	-0.881203102970459\\
15.8280555555556	-0.884735033157661\\
15.8336111111111	-0.919798462800304\\
15.8391666666667	-0.908984040041074\\
15.8447222222222	-0.906967544402984\\
15.8502777777778	-0.898591303267311\\
15.8558333333333	-0.900724103442832\\
15.8613888888889	-0.892394022915103\\
15.8669444444444	-0.904022795433502\\
15.8725	-0.898049943425861\\
15.8780555555556	-0.896838409654624\\
15.8836111111111	-0.908765631960822\\
15.8891666666667	-0.899476931984892\\
15.8947222222222	-0.906938638987731\\
15.9002777777778	-0.914784430444608\\
15.9058333333333	-0.909415909556316\\
15.9113888888889	-0.907346954624461\\
15.9169444444444	-0.903592259730431\\
15.9225	-0.907031150242735\\
15.9280555555556	-0.894708707537449\\
15.9336111111111	-0.892106635833594\\
15.9391666666667	-0.896293633681821\\
15.9447222222222	-0.898983039481799\\
15.9502777777778	-0.899787098898981\\
15.9558333333333	-0.896729727811109\\
15.9613888888889	-0.888273094355373\\
15.9669444444444	-0.885516859032661\\
15.9725	-0.889384681614325\\
15.9780555555556	-0.880242871334206\\
15.9836111111111	-0.874797373460251\\
15.9891666666667	-0.871530896651742\\
15.9947222222222	-0.882136651163474\\
};
\addplot [color=mycolor2,solid,line width=1.5pt,forget plot]
  table[row sep=crcr]{%
9.50027777777778	0\\
9.50583333333333	-0.00118690851220868\\
9.51138888888889	-0.00102092507742882\\
9.51694444444444	-0.00120404118379766\\
9.5225	-0.0010857217962683\\
9.52805555555556	-0.000935857133090005\\
9.53361111111111	-0.000926683504034984\\
9.53916666666667	-0.00121894921802729\\
9.54472222222222	-0.000244523286985722\\
9.55027777777778	-0.00143803741857625\\
9.55583333333333	-0.00162179197062479\\
9.56138888888889	-0.000900767925745425\\
9.56694444444444	0.000545911952860313\\
9.5725	0.000787118117557995\\
9.57805555555555	0.000550792964309129\\
9.58361111111111	0.000226846591321458\\
9.58916666666667	-0.000530995973028416\\
9.59472222222222	-0.000555416859615458\\
9.60027777777778	-0.00116314653568453\\
9.60583333333333	-0.000872177279574233\\
9.61138888888889	-0.00170230848492346\\
9.61694444444444	-0.00109595556131079\\
9.6225	-0.000541718606054697\\
9.62805555555556	-0.000729424049678731\\
9.63361111111111	-0.000713166391756397\\
9.63916666666667	-0.00061775649877082\\
9.64472222222222	-0.00183586678095483\\
9.65027777777778	-0.000927144720081744\\
9.65583333333333	-0.0024345600196896\\
9.66138888888889	-0.00323814950862362\\
9.66694444444444	-0.00220886907527787\\
9.6725	-0.00152448995789302\\
9.67805555555555	-0.0025716667858848\\
9.68361111111111	-0.00291639541980416\\
9.68916666666667	-0.0030233573200453\\
9.69472222222222	-0.00372835374312237\\
9.70027777777778	-0.00188651942876661\\
9.70583333333333	-0.0028294057454821\\
9.71138888888889	-0.00187385645305503\\
9.71694444444444	-0.00132908423786026\\
9.7225	-0.000192627883242276\\
9.72805555555555	0.00072823411767411\\
9.73361111111111	0.00133010637269195\\
9.73916666666667	0.00147044529999107\\
9.74472222222222	6.69855050173627e-05\\
9.75027777777778	-0.000311498502459185\\
9.75583333333333	0.000825597012421952\\
9.76138888888889	0.00143169484487149\\
9.76694444444444	0.000553658847475127\\
9.7725	0.000322155484883535\\
9.77805555555556	0.000575325847945978\\
9.78361111111111	0.00160924591974885\\
9.78916666666667	0.00219018258177845\\
9.79472222222222	0.0027290973378632\\
9.80027777777778	0.00254706976308442\\
9.80583333333333	0.00214213671158458\\
9.81138888888889	0.00270635533709811\\
9.81694444444444	0.00259531914385913\\
9.8225	0.00265255838119492\\
9.82805555555555	0.00119296777353006\\
9.83361111111111	0.00228049627107549\\
9.83916666666667	0.00296315366341586\\
9.84472222222222	0.00238900862553728\\
9.85027777777778	0.00468751486898732\\
9.85583333333333	0.00522640590970852\\
9.86138888888889	0.0029536638722036\\
9.86694444444444	0.00532585073799376\\
9.8725	0.00589466041303879\\
9.87805555555556	0.00525980678720498\\
9.88361111111111	0.00553137994374313\\
9.88916666666667	0.00591513559137735\\
9.89472222222222	0.00637819760661142\\
9.90027777777778	0.00501691799191735\\
9.90583333333333	0.00510102579826683\\
9.91138888888889	0.00550752173787104\\
9.91694444444444	0.00441963415570459\\
9.9225	0.00453361473904952\\
9.92805555555555	0.00537454543813722\\
9.93361111111111	0.00531921598621706\\
9.93916666666667	0.00543357799063687\\
9.94472222222222	0.00482911983562537\\
9.95027777777778	0.00484578003661077\\
9.95583333333333	0.00525835066040726\\
9.96138888888889	0.00396201999587995\\
9.96694444444444	0.00403059677444558\\
9.9725	0.00445687215712475\\
9.97805555555555	0.00452659009383529\\
9.98361111111111	0.0028020762027398\\
9.98916666666667	0.0026495983221484\\
9.99472222222222	0.00279484710225583\\
10.0002777777778	0.00430055256334943\\
10.0058333333333	0.00529363651812418\\
10.0113888888889	0.00494969786357445\\
10.0169444444444	0.00560915153185648\\
10.0225	0.00589000302716068\\
10.0280555555556	0.00589635468078102\\
10.0336111111111	0.00594363021995802\\
10.0391666666667	0.00434459678830989\\
10.0447222222222	0.00577217564854162\\
10.0502777777778	0.00617479441946959\\
10.0558333333333	0.00723821778153183\\
10.0613888888889	0.00599316915846778\\
10.0669444444444	0.0066637958654724\\
10.0725	0.00716759439908915\\
10.0780555555556	0.00717723115808679\\
10.0836111111111	0.00684636565292001\\
10.0891666666667	0.0064535282641456\\
10.0947222222222	0.00717621197109675\\
10.1002777777778	0.00701434049837339\\
10.1058333333333	0.00733091495854474\\
10.1113888888889	0.00837599026826034\\
10.1169444444444	0.0068447290486434\\
10.1225	0.00759259939041152\\
10.1280555555556	0.00793373601189489\\
10.1336111111111	0.00923072493637504\\
10.1391666666667	0.00903147694321682\\
10.1447222222222	0.00760576315767748\\
10.1502777777778	0.00845987897713434\\
10.1558333333333	0.00901631313546819\\
10.1613888888889	0.00932283248095165\\
10.1669444444444	0.0086689450904597\\
10.1725	0.00836561254237321\\
10.1780555555556	0.00900911703775887\\
10.1836111111111	0.00857402843335379\\
10.1891666666667	0.00591171994576832\\
10.1947222222222	0.00594708260258928\\
10.2002777777778	0.00730853233694995\\
10.2058333333333	0.007519509156017\\
10.2113888888889	0.0077928161667456\\
10.2169444444444	0.00833044370612251\\
10.2225	0.00894531466455785\\
10.2280555555556	0.00785194856698419\\
10.2336111111111	0.00837461327033701\\
10.2391666666667	0.00877276358648719\\
10.2447222222222	0.00810896687451768\\
10.2502777777778	0.00704393002322065\\
10.2558333333333	0.00730982893168046\\
10.2613888888889	0.00934219623999571\\
10.2669444444444	0.00955263695339134\\
10.2725	0.010051235386502\\
10.2780555555556	0.0079734330287139\\
10.2836111111111	0.00779230781660985\\
10.2891666666667	0.00685223696358575\\
10.2947222222222	0.00677657058363833\\
10.3002777777778	0.00558037373386346\\
10.3058333333333	0.00541345102678844\\
10.3113888888889	0.0061118471473948\\
10.3169444444444	0.00689139076228188\\
10.3225	0.00582243812741695\\
10.3280555555556	0.007061280244042\\
10.3336111111111	0.00820995580447963\\
10.3391666666667	0.00668004765049057\\
10.3447222222222	0.00702445083999393\\
10.3502777777778	0.00610331668390398\\
10.3558333333333	0.00567538917837953\\
10.3613888888889	0.00726201687173763\\
10.3669444444444	0.00371925348257963\\
10.3725	0.00455088135833447\\
10.3780555555556	0.00223181420603157\\
10.3836111111111	0.0042296524122714\\
10.3891666666667	0.00379544838059649\\
10.3947222222222	0.0040172917494264\\
10.4002777777778	0.00249292199326279\\
10.4058333333333	0.00197059430043678\\
10.4113888888889	0.00343716942713388\\
10.4169444444444	0.00555400068303252\\
10.4225	0.00615833716862259\\
10.4280555555556	0.00468112665281563\\
10.4336111111111	0.00276786867431769\\
10.4391666666667	0.000856107735260543\\
10.4447222222222	0.00664114993157647\\
10.4502777777778	0.00407987019406909\\
10.4558333333333	0.00438030941294732\\
10.4613888888889	0.00477833661786095\\
10.4669444444444	0.00473352612440943\\
10.4725	0.00512932852634805\\
10.4780555555556	0.004436979818971\\
10.4836111111111	0.00405105775667924\\
10.4891666666667	0.00351301191322229\\
10.4947222222222	0.00458829891526315\\
10.5002777777778	0.00556759944825407\\
10.5058333333333	0.00538722616419747\\
10.5113888888889	0.0047775798456615\\
10.5169444444444	0.00738523608492192\\
10.5225	0.00668833783319097\\
10.5280555555556	0.00358766738682613\\
10.5336111111111	0.00377873957350399\\
10.5391666666667	0.00686749962339217\\
10.5447222222222	0.00713450638751249\\
10.5502777777778	0.00733199523876612\\
10.5558333333333	0.00779218311592006\\
10.5613888888889	0.00715785355958501\\
10.5669444444444	0.00715434137363343\\
10.5725	0.00943973528644739\\
10.5780555555556	0.00943171449455328\\
10.5836111111111	0.0124477642573475\\
10.5891666666667	0.0106447301914825\\
10.5947222222222	0.0115194998585199\\
10.6002777777778	0.0118091035782377\\
10.6058333333333	0.0134306231931392\\
10.6113888888889	0.0129283046444189\\
10.6169444444444	0.0131965973151727\\
10.6225	0.0138897989224942\\
10.6280555555556	0.0143733062023359\\
10.6336111111111	0.0145355840388495\\
10.6391666666667	0.0159656347646721\\
10.6447222222222	0.0166056306686763\\
10.6502777777778	0.0177148487245439\\
10.6558333333333	0.0159875261128477\\
10.6613888888889	0.0155003985921635\\
10.6669444444444	0.0150464827574188\\
10.6725	0.015285547979166\\
10.6780555555556	0.0158441651717931\\
10.6836111111111	0.0161004359967972\\
10.6891666666667	0.01621060582772\\
10.6947222222222	0.0160073713583151\\
10.7002777777778	0.0157706628623229\\
10.7058333333333	0.0156256810845556\\
10.7113888888889	0.0147052587610857\\
10.7169444444444	0.0152805684166007\\
10.7225	0.0153268915708315\\
10.7280555555556	0.0144135913843107\\
10.7336111111111	0.0145604540206634\\
10.7391666666667	0.0151676300953374\\
10.7447222222222	0.0148702395088576\\
10.7502777777778	0.0150040300013093\\
10.7558333333333	0.0154314945851891\\
10.7613888888889	0.0163295483407431\\
10.7669444444444	0.0164741299703119\\
10.7725	0.0165251546734055\\
10.7780555555556	0.0157616458219627\\
10.7836111111111	0.0155436678318043\\
10.7891666666667	0.0158755388179688\\
10.7947222222222	0.0149100000596739\\
10.8002777777778	0.0157413575667173\\
10.8058333333333	0.0139312398718226\\
10.8113888888889	0.0133106276183274\\
10.8169444444444	0.014014972344497\\
10.8225	0.0154057390649502\\
10.8280555555556	0.0123774396050599\\
10.8336111111111	0.011359309293682\\
10.8391666666667	0.0107789233767613\\
10.8447222222222	0.0123036699150855\\
10.8502777777778	0.0124878927811983\\
10.8558333333333	0.0135074755412266\\
10.8613888888889	0.0134864506653418\\
10.8669444444444	0.0136381079082196\\
10.8725	0.0139613484311844\\
10.8780555555556	0.0142803888453323\\
10.8836111111111	0.0147434059632149\\
10.8891666666667	0.0152597474981215\\
10.8947222222222	0.0149434424087766\\
10.9002777777778	0.0150986933770883\\
10.9058333333333	0.0119333398109088\\
10.9113888888889	0.0119341320367295\\
10.9169444444444	0.0131102186144048\\
10.9225	0.0127228298299916\\
10.9280555555556	0.0125711715781906\\
10.9336111111111	0.0125900051545267\\
10.9391666666667	0.0126518986998055\\
10.9447222222222	0.011844925106144\\
10.9502777777778	0.0118823571155936\\
10.9558333333333	0.0117455566418016\\
10.9613888888889	0.0118700095432636\\
10.9669444444444	0.0123890478771331\\
10.9725	0.0131515080292142\\
10.9780555555556	0.0117126642698484\\
10.9836111111111	0.0117339247104234\\
10.9891666666667	0.0114807527224527\\
10.9947222222222	0.010783703707245\\
11.0002777777778	0.00968290058923234\\
11.0058333333333	0.0102670189020993\\
11.0113888888889	0.00661377194224985\\
11.0169444444444	0.00622924073544679\\
11.0225	0.00608216821491843\\
11.0280555555556	0.00813857960004969\\
11.0336111111111	0.00979658485863887\\
11.0391666666667	0.00933676433449039\\
11.0447222222222	0.0101272166646509\\
11.0502777777778	0.0112627454192351\\
11.0558333333333	0.011048928630229\\
11.0613888888889	0.0111606108670088\\
11.0669444444444	0.0111980942724837\\
11.0725	0.0122096443378154\\
11.0780555555556	0.0115675676930502\\
11.0836111111111	0.0127645911363903\\
11.0891666666667	0.0132970141556761\\
11.0947222222222	0.0136456142575767\\
11.1002777777778	0.0138105990921492\\
11.1058333333333	0.0137373597244699\\
11.1113888888889	0.0121880040263974\\
11.1169444444444	0.0128785978248432\\
11.1225	0.0134968249588412\\
11.1280555555556	0.0129823987132277\\
11.1336111111111	0.0137212913130168\\
11.1391666666667	0.0143323275195923\\
11.1447222222222	0.0128276257525159\\
11.1502777777778	0.01307094000712\\
11.1558333333333	0.0119295341955447\\
11.1613888888889	0.0120710486247321\\
11.1669444444444	0.00917505562911648\\
11.1725	0.00846687463947259\\
11.1780555555556	0.0102708980474306\\
11.1836111111111	0.0100922924699365\\
11.1891666666667	0.00983147236071864\\
11.1947222222222	0.00881522623030991\\
11.2002777777778	0.00856035714386721\\
11.2058333333333	0.00858646030325101\\
11.2113888888889	0.00808016368826501\\
11.2169444444444	0.00920113479723496\\
11.2225	0.00970574412062632\\
11.2280555555556	0.0100775633562998\\
11.2336111111111	0.010404476003391\\
11.2391666666667	0.00925931095065734\\
11.2447222222222	0.00861456045553955\\
11.2502777777778	0.00706377303591057\\
11.2558333333333	0.00508472477910918\\
11.2613888888889	0.00452904226300607\\
11.2669444444444	0.00512617207270533\\
11.2725	0.00392874933077105\\
11.2780555555556	0.00383231214662805\\
11.2836111111111	0.00406595180586141\\
11.2891666666667	0.00342006285833134\\
11.2947222222222	0.00350139748246581\\
11.3002777777778	-0.000445684045197531\\
11.3058333333333	-8.14815937075996e-06\\
11.3113888888889	0.000990380003251711\\
11.3169444444444	0.00574247073966494\\
11.3225	0.00517707198383168\\
11.3280555555556	0.00426840964539405\\
11.3336111111111	0.00519171716418963\\
11.3391666666667	0.00594736557272106\\
11.3447222222222	0.00611469565444928\\
11.3502777777778	0.0077198615906276\\
11.3558333333333	0.00845862826220492\\
11.3613888888889	0.00939582190538636\\
11.3669444444444	0.00813238290305363\\
11.3725	0.00933518597439814\\
11.3780555555556	0.00952804127787855\\
11.3836111111111	0.00868700970493806\\
11.3891666666667	0.00848852546007524\\
11.3947222222222	0.00930851761829602\\
11.4002777777778	0.00912603454952521\\
11.4058333333333	0.00972879927368512\\
11.4113888888889	0.00909770683711287\\
11.4169444444444	0.0101902981882867\\
11.4225	0.00983964299650572\\
11.4280555555556	0.00890928283048887\\
11.4336111111111	0.00905031786170257\\
11.4391666666667	0.00790492332511763\\
11.4447222222222	0.0074569682444655\\
11.4502777777778	0.00876143998391955\\
11.4558333333333	0.00939377762711558\\
11.4613888888889	0.0100475297363255\\
11.4669444444444	0.0112777174345455\\
11.4725	0.00965490257625077\\
11.4780555555556	0.0103648086902261\\
11.4836111111111	0.0111497834162257\\
11.4891666666667	0.0105319124091428\\
11.4947222222222	0.00930676787025046\\
11.5002777777778	0.00857026120978147\\
11.5058333333333	0.00912006535840596\\
11.5113888888889	0.0091212313348106\\
11.5169444444444	0.00919924937558534\\
11.5225	0.00771617854514689\\
11.5280555555556	0.0097548106935428\\
11.5336111111111	0.00779072530516183\\
11.5391666666667	0.00771511271361937\\
11.5447222222222	0.00850093884786082\\
11.5502777777778	0.00900094549316559\\
11.5558333333333	0.00961356284135624\\
11.5613888888889	0.00885899842018639\\
11.5669444444444	0.00940084516648876\\
11.5725	0.00843330877170197\\
11.5780555555556	0.00430460065426539\\
11.5836111111111	0.00319644251995204\\
11.5891666666667	0.00270144513660072\\
11.5947222222222	0.00376551605464277\\
11.6002777777778	0.00440107609023663\\
11.6058333333333	0.00363318577869335\\
11.6113888888889	0.0036460996082488\\
11.6169444444444	0.00411757625496464\\
11.6225	0.00341291085653561\\
11.6280555555556	0.00707208400169773\\
11.6336111111111	0.00722218616870107\\
11.6391666666667	0.00733597828110706\\
11.6447222222222	0.00732785510559013\\
11.6502777777778	0.00794176188507668\\
11.6558333333333	0.0104707003282149\\
11.6613888888889	0.0100357033960789\\
11.6669444444444	0.00909382765850099\\
11.6725	0.00786883838614928\\
11.6780555555556	0.00780517683867939\\
11.6836111111111	0.00939177519718659\\
11.6891666666667	0.00908884973918616\\
11.6947222222222	0.00913046001966231\\
11.7002777777778	0.00788162241563385\\
11.7058333333333	0.00716089539260838\\
11.7113888888889	0.00612368242086075\\
11.7169444444444	0.00615916448979733\\
11.7225	0.00753425766721842\\
11.7280555555556	0.00542667010228793\\
11.7336111111111	0.00607454171656201\\
11.7391666666667	0.00624960502337738\\
11.7447222222222	0.00675878945052275\\
11.7502777777778	0.0057033684814191\\
11.7558333333333	0.00653042000247914\\
11.7613888888889	0.00567225521508509\\
11.7669444444444	0.00690451987671984\\
11.7725	0.0074938574412684\\
11.7780555555556	0.00599992696840443\\
11.7836111111111	0.00450700064606734\\
11.7891666666667	0.00444160670625489\\
11.7947222222222	0.00388517811029182\\
11.8002777777778	0.00274037600175767\\
11.8058333333333	0.00311037452223868\\
11.8113888888889	0.00192675368318869\\
11.8169444444444	0.00163313944073314\\
11.8225	0.00188075149433308\\
11.8280555555556	0.00171921620914139\\
11.8336111111111	0.000845596502851545\\
11.8391666666667	0.00236880610164273\\
11.8447222222222	0.00375739978849155\\
11.8502777777778	0.00496948030047353\\
11.8558333333333	0.00259156372063031\\
11.8613888888889	0.00443496789507394\\
11.8669444444444	0.0066825142113228\\
11.8725	0.00572297987016859\\
11.8780555555556	0.00597895130973397\\
11.8836111111111	0.00667381306108522\\
11.8891666666667	0.009772649792528\\
11.8947222222222	0.00797751198840013\\
11.9002777777778	0.00582968536037986\\
11.9058333333333	0.00464039355074432\\
11.9113888888889	0.00697470584327298\\
11.9169444444444	0.00782971102236596\\
11.9225	0.00740089202296478\\
11.9280555555556	0.00642512765590751\\
11.9336111111111	0.0076936261715976\\
11.9391666666667	0.00736451778078157\\
11.9447222222222	0.010563113814503\\
11.9502777777778	0.00846790186465986\\
11.9558333333333	0.00768721836523769\\
11.9613888888889	0.00907136912292626\\
11.9669444444444	0.0084463398177538\\
11.9725	0.00880759457713004\\
11.9780555555556	0.0112455882151061\\
11.9836111111111	0.00682553570963579\\
11.9891666666667	0.00888432988937971\\
11.9947222222222	0.00770165439413377\\
12.0002777777778	0.00922362510760713\\
12.0058333333333	0.00734671752149755\\
12.0113888888889	0.00827685039494115\\
12.0169444444444	0.0071891280139355\\
12.0225	0.00954849887808282\\
12.0280555555556	0.00964987352475815\\
12.0336111111111	0.00995522238805097\\
12.0391666666667	0.00972848079007598\\
12.0447222222222	0.0101986731914261\\
12.0502777777778	0.00952682581729697\\
12.0558333333333	0.00887619568705862\\
12.0613888888889	0.0080179374960894\\
12.0669444444444	0.00745777972322456\\
12.0725	0.00734105565462872\\
12.0780555555556	0.00781618233321634\\
12.0836111111111	0.00807761245283833\\
12.0891666666667	0.00527389464668719\\
12.0947222222222	0.00513661202763371\\
12.1002777777778	0.005719894930188\\
12.1058333333333	0.00626143226213779\\
12.1113888888889	0.00602996054596115\\
12.1169444444444	0.00609817955670064\\
12.1225	0.00594693336545787\\
12.1280555555556	0.00511827185535772\\
12.1336111111111	0.00402705314037257\\
12.1391666666667	0.00495197547695608\\
12.1447222222222	0.00554592866470921\\
12.1502777777778	0.00782180721092256\\
12.1558333333333	0.00838076177872412\\
12.1613888888889	0.00880714390149257\\
12.1669444444444	0.00919291268492174\\
12.1725	0.00772776845062637\\
12.1780555555556	0.00836355264132025\\
12.1836111111111	0.00916865446430038\\
12.1891666666667	0.00879509831739307\\
12.1947222222222	0.00851695708943935\\
12.2002777777778	0.0102240390576451\\
12.2058333333333	0.00862468071022106\\
12.2113888888889	0.00946395181759056\\
12.2169444444444	0.0102187340093267\\
12.2225	0.0103752696239474\\
12.2280555555556	0.00828755316903015\\
12.2336111111111	0.00881403228232364\\
12.2391666666667	0.00798282801382319\\
12.2447222222222	0.00673520559346963\\
12.2502777777778	0.00567938518251096\\
12.2558333333333	0.00793103368593123\\
12.2613888888889	0.00839586389531141\\
12.2669444444444	0.0084440652126009\\
12.2725	0.0067248468786438\\
12.2780555555556	0.00420389983863976\\
12.2836111111111	0.00474879300121641\\
12.2891666666667	0.00505682439940396\\
12.2947222222222	0.00257682192643422\\
12.3002777777778	0.000466162792293882\\
12.3058333333333	0.00189025368927048\\
12.3113888888889	0.00041410737149233\\
12.3169444444444	0.00313606238928735\\
12.3225	0.00300459464698777\\
12.3280555555556	0.00500304924399355\\
12.3336111111111	0.00559547420905958\\
12.3391666666667	0.00137058403512107\\
12.3447222222222	0.00407785522232272\\
12.3502777777778	0.00389564338687195\\
12.3558333333333	-0.00184181899198053\\
12.3613888888889	0.0028212403887738\\
12.3669444444444	0.000640899756847607\\
12.3725	0.00126692624893089\\
12.3780555555556	-0.00100161179982024\\
12.3836111111111	0.00337075776816126\\
12.3891666666667	0.00290390299295474\\
12.3947222222222	0.00226606971626983\\
12.4002777777778	0.00284304379214373\\
12.4058333333333	0.00456778013038769\\
12.4113888888889	0.00441138409719789\\
12.4169444444444	0.00529725326125463\\
12.4225	0.00182348937856181\\
12.4280555555556	0.0013336449191367\\
12.4336111111111	0.000659344309019395\\
12.4391666666667	0.000320398337381899\\
12.4447222222222	0.00357280548775787\\
12.4502777777778	0.00634183972834582\\
12.4558333333333	0.00374297994768648\\
12.4613888888889	0.00628071277647406\\
12.4669444444444	0.00521891149219523\\
12.4725	0.00346203401893436\\
12.4780555555556	0.0025488742467169\\
12.4836111111111	0.00247017258941186\\
12.4891666666667	0.00341113198967745\\
12.4947222222222	0.00324238894650697\\
12.5002777777778	0.00531042568870536\\
12.5058333333333	0.00537051686647435\\
12.5113888888889	0.00701440443180293\\
12.5169444444444	0.00444837888160713\\
12.5225	0.0040719637194294\\
12.5280555555556	0.00414286691866417\\
12.5336111111111	0.00385353953029812\\
12.5391666666667	0.00407500120791167\\
12.5447222222222	0.00278608420835534\\
12.5502777777778	0.00539793030186006\\
12.5558333333333	0.00642952195696886\\
12.5613888888889	0.00713221529961519\\
12.5669444444444	0.00729918743631027\\
12.5725	0.00668530682567103\\
12.5780555555556	0.00788177020108867\\
12.5836111111111	0.00635478282010833\\
12.5891666666667	0.00488850920848006\\
12.5947222222222	0.00379969524330444\\
12.6002777777778	0.00514084378291189\\
12.6058333333333	0.00351659000243314\\
12.6113888888889	0.0035869044864442\\
12.6169444444444	0.0056763760735207\\
12.6225	0.00697306167499312\\
12.6280555555556	0.00582870615062133\\
12.6336111111111	0.00616308916909972\\
12.6391666666667	0.00537189221189495\\
12.6447222222222	0.00547943047673392\\
12.6502777777778	0.00587732474352274\\
12.6558333333333	0.00595841230392653\\
12.6613888888889	0.00535273450692266\\
12.6669444444444	0.00732077182986535\\
12.6725	0.00706935596690223\\
12.6780555555556	0.00723282407560466\\
12.6836111111111	0.00875179242552049\\
12.6891666666667	0.00879426279838098\\
12.6947222222222	0.00992904940646367\\
12.7002777777778	0.0102861462069171\\
12.7058333333333	0.00487177482683411\\
12.7113888888889	0.00678995456120041\\
12.7169444444444	0.00812809878581401\\
12.7225	0.00812450997185701\\
12.7280555555556	0.00707260283909075\\
12.7336111111111	0.00832900937373349\\
12.7391666666667	0.00885743157764845\\
12.7447222222222	0.00848384766558957\\
12.7502777777778	0.00703881450226487\\
12.7558333333333	0.00445168923926163\\
12.7613888888889	0.00683406512499344\\
12.7669444444444	0.00345540751897027\\
12.7725	0.00363792602009156\\
12.7780555555556	0.00327293687632565\\
12.7836111111111	0.00178736035383817\\
12.7891666666667	0.00904518745276423\\
12.7947222222222	0.00838777574989764\\
12.8002777777778	0.00920561476619339\\
12.8058333333333	0.00900167866774841\\
12.8113888888889	0.00853528245356508\\
12.8169444444444	0.00845183640826836\\
12.8225	0.00658973628169565\\
12.8280555555556	0.00686651839316866\\
12.8336111111111	0.00805659465898096\\
12.8391666666667	0.00708910510567205\\
12.8447222222222	0.00520362456296486\\
12.8502777777778	0.00605608603474416\\
12.8558333333333	0.00618850590120417\\
12.8613888888889	0.00551890641745844\\
12.8669444444444	0.00528568223334643\\
12.8725	0.00518580107466229\\
12.8780555555556	0.00536737427325283\\
12.8836111111111	0.00466294385105187\\
12.8891666666667	0.00460882672239751\\
12.8947222222222	0.00490648965145092\\
12.9002777777778	0.00540031121591981\\
12.9058333333333	0.00523702499493682\\
12.9113888888889	0.00606409875353313\\
12.9169444444444	0.00602540890659561\\
12.9225	0.0071957130398518\\
12.9280555555556	0.00656334450695088\\
12.9336111111111	0.00802681370149263\\
12.9391666666667	0.00842921641187736\\
12.9447222222222	0.00939634399309202\\
12.9502777777778	0.00857995596899969\\
12.9558333333333	0.00896337439913134\\
12.9613888888889	0.00822993897879785\\
12.9669444444444	0.0057994964443709\\
12.9725	0.0052664528876326\\
12.9780555555556	0.00509598936393998\\
12.9836111111111	0.00348762627998836\\
12.9891666666667	-0.0020265549658193\\
12.9947222222222	-0.000374781955822791\\
13.0002777777778	-0.000346890447008065\\
13.0058333333333	-0.00125377741098365\\
13.0113888888889	-0.00039773761776431\\
13.0169444444444	-3.27065937286036e-05\\
13.0225	-0.000406530129333778\\
13.0280555555556	-0.000849350613574379\\
13.0336111111111	-0.000903747751605463\\
13.0391666666667	-0.000804839366313121\\
13.0447222222222	-0.000681034649650984\\
13.0502777777778	0.00159210732676544\\
13.0558333333333	0.00245103832151098\\
13.0613888888889	0.00267513913450294\\
13.0669444444444	0.00189006928025976\\
13.0725	0.00410766333457819\\
13.0780555555556	0.00387616005815525\\
13.0836111111111	0.00342243809880322\\
13.0891666666667	0.00261481309802753\\
13.0947222222222	0.00435789154261409\\
13.1002777777778	0.00339035809110327\\
13.1058333333333	0.00434086588434362\\
13.1113888888889	0.00493458293146208\\
13.1169444444444	0.00447815288663508\\
13.1225	0.00311129100628628\\
13.1280555555556	0.00379220870929478\\
13.1336111111111	0.00352284034940706\\
13.1391666666667	0.00494831714067307\\
13.1447222222222	0.00487207291178346\\
13.1502777777778	0.00746297019361393\\
13.1558333333333	0.00573190492693352\\
13.1613888888889	0.00765819616446359\\
13.1669444444444	0.00868666089144676\\
13.1725	0.00803049627551531\\
13.1780555555556	0.00773303645661894\\
13.1836111111111	0.00813689638136275\\
13.1891666666667	0.00889573561939308\\
13.1947222222222	0.00516858004218222\\
13.2002777777778	0.00585134655349787\\
13.2058333333333	0.00411005391297599\\
13.2113888888889	0.00323933423744387\\
13.2169444444444	0.00488366344218239\\
13.2225	0.00367184451418562\\
13.2280555555556	0.00444858100092044\\
13.2336111111111	0.00454401237709055\\
13.2391666666667	0.0043986106682105\\
13.2447222222222	0.00436895045329374\\
13.2502777777778	0.00578281295199549\\
13.2558333333333	0.00704135383439691\\
13.2613888888889	0.00700692389181432\\
13.2669444444444	0.00915115350634056\\
13.2725	0.0068133834007184\\
13.2780555555556	0.00727654405956933\\
13.2836111111111	0.00795247397598771\\
13.2891666666667	0.00906128629637857\\
13.2947222222222	0.00867488840644853\\
13.3002777777778	0.00821685490888411\\
13.3058333333333	0.00949492848714113\\
13.3113888888889	0.0119248117085571\\
13.3169444444444	0.0121134122417903\\
13.3225	0.0115246360035592\\
13.3280555555556	0.0105649684283603\\
13.3336111111111	0.00999277600235167\\
13.3391666666667	0.0110340247101957\\
13.3447222222222	0.0106909568044387\\
13.3502777777778	0.010590934430998\\
13.3558333333333	0.0102866450041077\\
13.3613888888889	0.0114487510234987\\
13.3669444444444	0.0133299151929194\\
13.3725	0.0133022862967721\\
13.3780555555556	0.0127441171677203\\
13.3836111111111	0.0120446060773687\\
13.3891666666667	0.012320976929846\\
13.3947222222222	0.0114839088539121\\
13.4002777777778	0.0122606884573719\\
13.4058333333333	0.0148496118819243\\
13.4113888888889	0.0166798789996886\\
13.4169444444444	0.0161968758536681\\
13.4225	0.0152115762171314\\
13.4280555555556	0.0139651887209675\\
13.4336111111111	0.0136794594454038\\
13.4391666666667	0.0153894115414498\\
13.4447222222222	0.0156561778332868\\
13.4502777777778	0.0127787788759307\\
13.4558333333333	0.0123681055671747\\
13.4613888888889	0.0116224095500909\\
13.4669444444444	0.0128661987844814\\
13.4725	0.0136068475017976\\
13.4780555555556	0.012409882465388\\
13.4836111111111	0.0122231374628466\\
13.4891666666667	0.0115801229579607\\
13.4947222222222	0.0120692283914665\\
13.5002777777778	0.0130553404342841\\
13.5058333333333	0.0108637399894878\\
13.5113888888889	0.0114544789553314\\
13.5169444444444	0.0107445418342729\\
13.5225	0.01162974679812\\
13.5280555555556	0.0102172761136075\\
13.5336111111111	0.00965651438523455\\
13.5391666666667	0.0109204579908833\\
13.5447222222222	0.0130874927447726\\
13.5502777777778	0.0160276563897504\\
13.5558333333333	0.0186076932010576\\
13.5613888888889	0.0184970648611239\\
13.5669444444444	0.0200631090004127\\
13.5725	0.0202841086015577\\
13.5780555555556	0.0210519383877426\\
13.5836111111111	0.0175392029806612\\
13.5891666666667	0.0168300646387743\\
13.5947222222222	0.014196353786979\\
13.6002777777778	0.0151046720487997\\
13.6058333333333	0.014679587806791\\
13.6113888888889	0.0139347918801996\\
13.6169444444444	0.0141895488887828\\
13.6225	0.0144595468552188\\
13.6280555555556	0.0135946599274494\\
13.6336111111111	0.0149221161911486\\
13.6391666666667	0.0139531268294144\\
13.6447222222222	0.0152676350467655\\
13.6502777777778	0.0150153245054227\\
13.6558333333333	0.0155570851123597\\
13.6613888888889	0.0160377145669646\\
13.6669444444444	0.0166227685100645\\
13.6725	0.0149726935873517\\
13.6780555555556	0.0130206555622729\\
13.6836111111111	0.0157190269281637\\
13.6891666666667	0.0171419186399627\\
13.6947222222222	0.0168918634726939\\
13.7002777777778	0.0128135319113984\\
13.7058333333333	0.0125594068986096\\
13.7113888888889	0.0130297345644072\\
13.7169444444444	0.0144532856779959\\
13.7225	0.0162253730142979\\
13.7280555555556	0.0167327291245616\\
13.7336111111111	0.0156772099209833\\
13.7391666666667	0.015697337453192\\
13.7447222222222	0.0153590792315895\\
13.7502777777778	0.0182551707349627\\
13.7558333333333	0.0194493019318458\\
13.7613888888889	0.0203972333115155\\
13.7669444444444	0.0215019244416062\\
13.7725	0.0193562432388446\\
13.7780555555556	0.0205323944399071\\
13.7836111111111	0.0225872129168175\\
13.7891666666667	0.0210712052930924\\
13.7947222222222	0.0225906084782137\\
13.8002777777778	0.0222749499676333\\
13.8058333333333	0.0230375358484084\\
13.8113888888889	0.0231097786738671\\
13.8169444444444	0.023805655395718\\
13.8225	0.024151330433222\\
13.8280555555556	0.0241025835229619\\
13.8336111111111	0.0237082686228238\\
13.8391666666667	0.0236568772200813\\
13.8447222222222	0.0248943926409673\\
13.8502777777778	0.0252857375423667\\
13.8558333333333	0.0254947876944636\\
13.8613888888889	0.0272025691456673\\
13.8669444444444	0.0262312368843198\\
13.8725	0.0257640905278822\\
13.8780555555556	0.0279698942968106\\
13.8836111111111	0.0278327328009361\\
13.8891666666667	0.0290729863317272\\
13.8947222222222	0.0261422289482373\\
13.9002777777778	0.0253274203427447\\
13.9058333333333	0.0241894624904609\\
13.9113888888889	0.0263225207900822\\
13.9169444444444	0.0264197285679697\\
13.9225	0.0239393755648027\\
13.9280555555556	0.024387406143854\\
13.9336111111111	0.0250176882468975\\
13.9391666666667	0.0208574384864935\\
13.9447222222222	0.023539734187667\\
13.9502777777778	0.0262994609063134\\
13.9558333333333	0.0249281076516765\\
13.9613888888889	0.024678376136961\\
13.9669444444444	0.0231151072147555\\
13.9725	0.0245268108645456\\
13.9780555555556	0.0242939350124849\\
13.9836111111111	0.0211380777188475\\
13.9891666666667	0.0210509898515042\\
13.9947222222222	0.0208410926891576\\
14.0002777777778	0.0197654897696443\\
14.0058333333333	0.0178726300637849\\
14.0113888888889	0.0183285157847414\\
14.0169444444444	0.0213692178218011\\
14.0225	0.0203786327376978\\
14.0280555555556	0.0198747787848117\\
14.0336111111111	0.0210930412100357\\
14.0391666666667	0.0213715950989642\\
14.0447222222222	0.0199982914250978\\
14.0502777777778	0.0207940253595202\\
14.0558333333333	0.0228420783038822\\
14.0613888888889	0.0225558468856602\\
14.0669444444444	0.0245150256238619\\
14.0725	0.0208548563669384\\
14.0780555555556	0.0219242195923727\\
14.0836111111111	0.0228253295536391\\
14.0891666666667	0.0207247460778044\\
14.0947222222222	0.0198076446007377\\
14.1002777777778	0.0232020915967223\\
14.1058333333333	0.025397546857503\\
14.1113888888889	0.0265127019272548\\
14.1169444444444	0.0236986733433429\\
14.1225	0.0194697359562572\\
14.1280555555556	0.0215197752155888\\
14.1336111111111	0.0199687623111868\\
14.1391666666667	0.0208572930589862\\
14.1447222222222	0.0206977997425138\\
14.1502777777778	0.0190595835535453\\
14.1558333333333	0.0167514084648579\\
14.1613888888889	0.0167467576284682\\
14.1669444444444	0.0184656928402715\\
14.1725	0.0198984035461502\\
14.1780555555556	0.0205894170641555\\
14.1836111111111	0.0199158594500688\\
14.1891666666667	0.0188755802068242\\
14.1947222222222	0.0168028224898945\\
14.2002777777778	0.0164676624203033\\
14.2058333333333	0.0182572213063253\\
14.2113888888889	0.0181313934062824\\
14.2169444444444	0.017431629260507\\
14.2225	0.0178533361172591\\
14.2280555555556	0.0186068068381256\\
14.2336111111111	0.0202508286600584\\
14.2391666666667	0.0212248724104927\\
14.2447222222222	0.0209061074453013\\
14.2502777777778	0.0213842651528757\\
14.2558333333333	0.0235432550402699\\
14.2613888888889	0.0219766988029899\\
14.2669444444444	0.0213049964563655\\
14.2725	0.0212690910509311\\
14.2780555555556	0.0227835563652989\\
14.2836111111111	0.0253478192446473\\
14.2891666666667	0.0248769067181668\\
14.2947222222222	0.0254805416888654\\
14.3002777777778	0.024521903489995\\
14.3058333333333	0.0254928139003101\\
14.3113888888889	0.0262221299368101\\
14.3169444444444	0.0251262132013989\\
14.3225	0.0270071398234986\\
14.3280555555556	0.0264065834082337\\
14.3336111111111	0.024947404894406\\
14.3391666666667	0.0264435438421694\\
14.3447222222222	0.0265362577262421\\
14.3502777777778	0.0240324727968849\\
14.3558333333333	0.0254527090119597\\
14.3613888888889	0.0251553772526743\\
14.3669444444444	0.0262309900637133\\
14.3725	0.0258817179516381\\
14.3780555555556	0.0279495077989734\\
14.3836111111111	0.0272676794591469\\
14.3891666666667	0.0269051331924538\\
14.3947222222222	0.0256123156936237\\
14.4002777777778	0.0234901571017048\\
14.4058333333333	0.0271171423889937\\
14.4113888888889	0.026743643581604\\
14.4169444444444	0.0269151293872367\\
14.4225	0.0273829719726276\\
14.4280555555556	0.027188828299506\\
14.4336111111111	0.0274412171980148\\
14.4391666666667	0.0261972390217346\\
14.4447222222222	0.0250998456401567\\
14.4502777777778	0.0240612363010183\\
14.4558333333333	0.0236298677802241\\
14.4613888888889	0.0241297072702718\\
14.4669444444444	0.0236938746812492\\
14.4725	0.0246896093702908\\
14.4780555555556	0.0250447312507898\\
14.4836111111111	0.0274662168952934\\
14.4891666666667	0.0266169906320109\\
14.4947222222222	0.0258587502053944\\
14.5002777777778	0.0240152874451401\\
14.5058333333333	0.0246747934660796\\
14.5113888888889	0.0263177919452884\\
14.5169444444444	0.0255248977193419\\
14.5225	0.0253339295325304\\
14.5280555555556	0.0259744381903984\\
14.5336111111111	0.0259424196963903\\
14.5391666666667	0.0269452778591911\\
14.5447222222222	0.0268077868034288\\
14.5502777777778	0.0268976846356511\\
14.5558333333333	0.0249311218104935\\
14.5613888888889	0.0222178286318313\\
14.5669444444444	0.0222560022313908\\
14.5725	0.0224349528851695\\
14.5780555555556	0.0250044462418408\\
14.5836111111111	0.0239893663242078\\
14.5891666666667	0.022148082496868\\
14.5947222222222	0.0211690392194035\\
14.6002777777778	0.0210106871360717\\
14.6058333333333	0.0247060345944365\\
14.6113888888889	0.0209942417702359\\
14.6169444444444	0.0213229833567215\\
14.6225	0.0222386945013286\\
14.6280555555556	0.0238377218634932\\
14.6336111111111	0.0236024237136599\\
14.6391666666667	0.0283628143377437\\
14.6447222222222	0.0269411378349315\\
14.6502777777778	0.0258960305856274\\
14.6558333333333	0.0287327606743032\\
14.6613888888889	0.0278883952435619\\
14.6669444444444	0.0276368323604597\\
14.6725	0.0271717174987142\\
14.6780555555556	0.0286390416498953\\
14.6836111111111	0.0267440926371043\\
14.6891666666667	0.0269912344094793\\
14.6947222222222	0.0283185616789683\\
14.7002777777778	0.0290140378342915\\
14.7058333333333	0.0302294403333809\\
14.7113888888889	0.0318309559651728\\
14.7169444444444	0.0329960409751473\\
14.7225	0.0344656149885948\\
14.7280555555556	0.0354472298389163\\
14.7336111111111	0.0368125377396083\\
14.7391666666667	0.034622548908104\\
14.7447222222222	0.035356237931368\\
14.7502777777778	0.0356905696165271\\
14.7558333333333	0.0345575102991092\\
14.7613888888889	0.033010990366604\\
14.7669444444444	0.0317100722170767\\
14.7725	0.0320862316035579\\
14.7780555555556	0.0311349827266938\\
14.7836111111111	0.0327431764779849\\
14.7891666666667	0.0352227523679373\\
14.7947222222222	0.0330163074464892\\
14.8002777777778	0.0346168194744784\\
14.8058333333333	0.0356037947612738\\
14.8113888888889	0.0356001681936421\\
14.8169444444444	0.0365401890170762\\
14.8225	0.0351865671917737\\
14.8280555555556	0.0347914863331731\\
14.8336111111111	0.036087050153389\\
14.8391666666667	0.0358830139579582\\
14.8447222222222	0.0337965240167427\\
14.8502777777778	0.0316499686053678\\
14.8558333333333	0.0312645262542377\\
14.8613888888889	0.0300215968472571\\
14.8669444444444	0.0262818561504772\\
14.8725	0.028383058423172\\
14.8780555555556	0.0285683136532306\\
14.8836111111111	0.0314570963801589\\
14.8891666666667	0.0309706209840686\\
14.8947222222222	0.0300488837933692\\
14.9002777777778	0.0308968433447868\\
14.9058333333333	0.0294561030355534\\
14.9113888888889	0.0307312388752592\\
14.9169444444444	0.0307661966996743\\
14.9225	0.0308961930690303\\
14.9280555555556	0.0320313948172129\\
14.9336111111111	0.0340428189533821\\
14.9391666666667	0.0327114181047545\\
14.9447222222222	0.0317221003232441\\
14.9502777777778	0.0298022844504049\\
14.9558333333333	0.0282463404232963\\
14.9613888888889	0.0292333239098707\\
14.9669444444444	0.0280846364238689\\
14.9725	0.0230533664438187\\
14.9780555555556	0.0212239385784819\\
14.9836111111111	0.0219526295982246\\
14.9891666666667	0.0221732219667847\\
14.9947222222222	0.0231610523051494\\
15.0002777777778	0.0250270656025299\\
15.0058333333333	0.0294476844055423\\
15.0113888888889	0.0286201766816519\\
15.0169444444444	0.0266431000358578\\
15.0225	0.0239690311986574\\
15.0280555555556	0.0258851917305043\\
15.0336111111111	0.0286870773306487\\
15.0391666666667	0.030403810858762\\
15.0447222222222	0.0285704773324276\\
15.0502777777778	0.0313772319119043\\
15.0558333333333	0.0321868644522184\\
15.0613888888889	0.031839251034367\\
15.0669444444444	0.0313623034984624\\
15.0725	0.031170222184487\\
15.0780555555556	0.0344443524300633\\
15.0836111111111	0.029489923167348\\
15.0891666666667	0.0330610921447299\\
15.0947222222222	0.0369277114505085\\
15.1002777777778	0.0299614222926112\\
15.1058333333333	0.0285826468215933\\
15.1113888888889	0.0338698301476331\\
15.1169444444444	0.0332487816188263\\
15.1225	0.0324400632453218\\
15.1280555555556	0.0322790183768723\\
15.1336111111111	0.0276177563635906\\
15.1391666666667	0.0276090648780128\\
15.1447222222222	0.0310918885621273\\
15.1502777777778	0.0268953533176945\\
15.1558333333333	0.0295684307966181\\
15.1613888888889	0.028880066516304\\
15.1669444444444	0.0289851676120966\\
15.1725	0.0299141109598367\\
15.1780555555556	0.0320871731474677\\
15.1836111111111	0.0303991983862765\\
15.1891666666667	0.0332534658012859\\
15.1947222222222	0.0303017397729118\\
15.2002777777778	0.0291001771694004\\
15.2058333333333	0.0291331386529873\\
15.2113888888889	0.0265837929156722\\
15.2169444444444	0.0304131304421119\\
15.2225	0.0343418144757511\\
15.2280555555556	0.0375445715812903\\
15.2336111111111	0.0352691932058056\\
15.2391666666667	0.0337024555217568\\
15.2447222222222	0.034868561302054\\
15.2502777777778	0.0346763322745327\\
15.2558333333333	0.0395567595499325\\
15.2613888888889	0.0322422903621692\\
15.2669444444444	0.0321762021490331\\
15.2725	0.0336726594079319\\
15.2780555555556	0.035778298031417\\
15.2836111111111	0.0353851239891334\\
15.2891666666667	0.0360865955303715\\
15.2947222222222	0.036640612624971\\
15.3002777777778	0.0368522844112813\\
15.3058333333333	0.0368733856213463\\
15.3113888888889	0.0371095587835691\\
15.3169444444444	0.0388310396606889\\
15.3225	0.0384321646149264\\
15.3280555555556	0.0389143455727235\\
15.3336111111111	0.0402116765215663\\
15.3391666666667	0.0408443146280626\\
15.3447222222222	0.0425891029691682\\
15.3502777777778	0.0414608316416802\\
15.3558333333333	0.0399800267360296\\
15.3613888888889	0.0417086348762751\\
15.3669444444444	0.0392612303802365\\
15.3725	0.0403702963604167\\
15.3780555555556	0.0401707668348074\\
15.3836111111111	0.0418626579924849\\
15.3891666666667	0.0437044492515943\\
15.3947222222222	0.0464617321272493\\
15.4002777777778	0.0469774318592813\\
15.4058333333333	0.0491914920803585\\
15.4113888888889	0.0480048410844199\\
15.4169444444444	0.0491761341442377\\
15.4225	0.050148199806271\\
15.4280555555556	0.0510547339300541\\
15.4336111111111	0.0501124025298091\\
15.4391666666667	0.0525268613133583\\
15.4447222222222	0.0505317595685872\\
15.4502777777778	0.0445442015138142\\
15.4558333333333	0.0455407911170513\\
15.4613888888889	0.0477225847288411\\
15.4669444444444	0.0484559038892355\\
15.4725	0.042587510704549\\
15.4780555555556	0.0476750336826965\\
15.4836111111111	0.0465891716242375\\
15.4891666666667	0.0465054145357388\\
15.4947222222222	0.0471108788275044\\
15.5002777777778	0.0438141906715522\\
15.5058333333333	0.0443243813714864\\
15.5113888888889	0.0437931688178853\\
15.5169444444444	0.0432505801954629\\
15.5225	0.044361268200215\\
15.5280555555556	0.0421577087003692\\
15.5336111111111	0.0419645977383097\\
15.5391666666667	0.0457037482526499\\
15.5447222222222	0.0465971770375464\\
15.5502777777778	0.046226313096716\\
15.5558333333333	0.0485830421001078\\
15.5613888888889	0.0451134551621939\\
15.5669444444444	0.0461591434829476\\
15.5725	0.0470845469657579\\
15.5780555555556	0.0482228987430891\\
15.5836111111111	0.0492811973248882\\
15.5891666666667	0.0506526359796042\\
15.5947222222222	0.0490887439539014\\
15.6002777777778	0.0516319988100894\\
15.6058333333333	0.0526770212270635\\
15.6113888888889	0.0502532574383992\\
15.6169444444444	0.0525239052723439\\
15.6225	0.0493863779301073\\
15.6280555555556	0.0477655753916665\\
15.6336111111111	0.0474719828519393\\
15.6391666666667	0.0498991756987361\\
15.6447222222222	0.050664039934026\\
15.6502777777778	0.0489968229956398\\
15.6558333333333	0.0454377140455271\\
15.6613888888889	0.0457996417173165\\
15.6669444444444	0.0475638387115343\\
15.6725	0.0465875804513283\\
15.6780555555556	0.0457641310351589\\
15.6836111111111	0.0484030862912926\\
15.6891666666667	0.0476182674994607\\
15.6947222222222	0.0482618886851329\\
15.7002777777778	0.0445791149434585\\
15.7058333333333	0.0442000682209414\\
15.7113888888889	0.0439605091528166\\
15.7169444444444	0.0449901222014101\\
15.7225	0.0463856648530958\\
15.7280555555556	0.0457287474888263\\
15.7336111111111	0.0488079620256437\\
15.7391666666667	0.0429229636170942\\
15.7447222222222	0.0474435223001173\\
15.7502777777778	0.0495259464809787\\
15.7558333333333	0.0500266056865258\\
15.7613888888889	0.0552632084779861\\
15.7669444444444	0.0569430292814038\\
15.7725	0.061310992894618\\
15.7780555555556	0.0620823766325994\\
15.7836111111111	0.0555148809058106\\
15.7891666666667	0.0556119007733178\\
15.7947222222222	0.0586006630715365\\
15.8002777777778	0.0629320185280912\\
15.8058333333333	0.06832811281642\\
15.8113888888889	0.0672392136456711\\
15.8169444444444	0.0688970753984203\\
15.8225	0.0694485815715837\\
15.8280555555556	0.0670214383390141\\
15.8336111111111	0.0612204721696243\\
15.8391666666667	0.0650120403976796\\
15.8447222222222	0.0635912536648799\\
15.8502777777778	0.0676002674296895\\
15.8558333333333	0.0746725511908084\\
15.8613888888889	0.0756691678579427\\
15.8669444444444	0.0674380972294014\\
15.8725	0.0732336347650445\\
15.8780555555556	0.0713432785317632\\
15.8836111111111	0.0733114169838708\\
15.8891666666667	0.0718578129158426\\
15.8947222222222	0.0698315188311542\\
15.9002777777778	0.0725987427655653\\
15.9058333333333	0.0748176009500715\\
15.9113888888889	0.0768920537312472\\
15.9169444444444	0.0816102056017892\\
15.9225	0.0836912250135488\\
15.9280555555556	0.0842916292371016\\
15.9336111111111	0.0827708825088755\\
15.9391666666667	0.0798443993072999\\
15.9447222222222	0.0846707021652646\\
15.9502777777778	0.0833277384989921\\
15.9558333333333	0.0895018926873854\\
15.9613888888889	0.0964078166545377\\
15.9669444444444	0.0963794059753942\\
15.9725	0.0976468378981219\\
15.9780555555556	0.099794351744351\\
15.9836111111111	0.101122182695495\\
15.9891666666667	0.100107749245786\\
15.9947222222222	0.0951357980214773\\
};
\addplot [color=mycolor3,solid,line width=1.5pt,forget plot]
  table[row sep=crcr]{%
9.50027777777778	0\\
9.50583333333333	2.86965455098513e-05\\
9.51138888888889	3.32601396607555e-05\\
9.51694444444444	2.94339335927753e-05\\
9.5225	-7.87157557003169e-06\\
9.52805555555556	-0.000106076086431978\\
9.53361111111111	-9.44428267090603e-05\\
9.53916666666667	-8.66667013579199e-05\\
9.54472222222222	-7.03770625727153e-05\\
9.55027777777778	-6.45449685593596e-05\\
9.55583333333333	-5.87128745460039e-05\\
9.56138888888889	-5.68306998158092e-05\\
9.56694444444444	-5.48866684780244e-05\\
9.5725	-5.29426371402396e-05\\
9.57805555555555	1.45984942214801e-06\\
9.58361111111111	3.24975349001924e-05\\
9.58916666666667	9.92422325351202e-05\\
9.59472222222222	0.000132156254789603\\
9.60027777777778	0.000115891764878453\\
9.60583333333333	9.25696523258967e-05\\
9.61138888888889	0.000210270954545675\\
9.61694444444444	0.000220911154763096\\
9.6225	0.000240294763921222\\
9.62805555555556	0.000219836606825941\\
9.63361111111111	0.000222768229009072\\
9.63916666666667	0.000211008866527723\\
9.64472222222222	0.000324892189937595\\
9.65027777777778	0.000262970423755766\\
9.65583333333333	0.000206235529074029\\
9.66138888888889	0.000362558225848878\\
9.66694444444444	0.000313675854236733\\
9.6725	0.000136924231738131\\
9.67805555555555	0.000133461425166446\\
9.68361111111111	0.00010084569833778\\
9.68916666666667	-3.9831786715081e-05\\
9.69472222222222	-0.000199445576514025\\
9.70027777777778	-0.000238236313700226\\
9.70583333333333	-5.64488825491318e-05\\
9.71138888888889	-0.000131522971509794\\
9.71694444444444	-0.000178555905688997\\
9.7225	-0.000107610349372122\\
9.72805555555555	-0.000225607590404257\\
9.73361111111111	-0.000287363167120671\\
9.73916666666667	-2.11322386789503e-05\\
9.74472222222222	4.40449144812253e-05\\
9.75027777777778	0.000124182598696576\\
9.75583333333333	-3.1633097342972e-05\\
9.76138888888889	-0.000221881033815161\\
9.76694444444444	-0.00025311180424863\\
9.7725	-0.00028359143694769\\
9.77805555555556	-0.000242528481594559\\
9.78361111111111	-0.000142835414616487\\
9.78916666666667	-0.000133548803700836\\
9.79472222222222	-0.000106296163089614\\
9.80027777777778	-9.49085146247811e-05\\
9.80583333333333	-6.18497260377388e-05\\
9.81138888888889	1.90165012270091e-05\\
9.81694444444444	-6.32069012505078e-05\\
9.8225	-0.000147724397400441\\
9.82805555555555	-3.53743306309853e-05\\
9.83361111111111	4.16402827816338e-05\\
9.83916666666667	0.000121971532921123\\
9.84472222222222	9.47665551126637e-05\\
9.85027777777778	2.37636074703277e-05\\
9.85583333333333	7.56417196638156e-05\\
9.86138888888889	0.000145112515244581\\
9.86694444444444	0.000215611120336897\\
9.8725	0.000237424845337156\\
9.87805555555556	0.00021897152449487\\
9.88361111111111	0.00024664836531043\\
9.88916666666667	0.000282052812168719\\
9.89472222222222	0.000206006834054962\\
9.90027777777778	0.000238493664016705\\
9.90583333333333	0.000320350068903194\\
9.91138888888889	0.000245531246712118\\
9.91694444444444	0.000272836825581712\\
9.9225	-1.87611608766782e-05\\
9.92805555555555	4.04206102091022e-05\\
9.93361111111111	4.84913095899657e-05\\
9.93916666666667	3.91877662008647e-05\\
9.94472222222222	-2.03070195209392e-05\\
9.95027777777778	2.91802710894689e-05\\
9.95583333333333	5.58192791257008e-05\\
9.96138888888889	2.77683196514615e-05\\
9.96694444444444	2.99752081895132e-05\\
9.9725	6.03493240643524e-05\\
9.97805555555555	0.000142908453959537\\
9.98361111111111	0.000119265785798277\\
9.98916666666667	7.10938398563597e-05\\
9.99472222222222	4.34539585303636e-05\\
10.0002777777778	-2.86337087213765e-05\\
10.0058333333333	-0.00013939018282873\\
10.0113888888889	-7.90338617441912e-05\\
10.0169444444444	-2.55852488758473e-05\\
10.0225	-9.25112759319867e-05\\
10.0280555555556	-5.82915830627827e-05\\
10.0336111111111	-0.000108857193386661\\
10.0391666666667	-9.12026772336227e-05\\
10.0447222222222	-0.000341315153808146\\
10.0502777777778	-0.000299320410000451\\
10.0558333333333	-8.64054787317393e-05\\
10.0613888888889	-0.000105650159728638\\
10.0669444444444	5.63540514573241e-05\\
10.0725	-6.80184619682748e-05\\
10.0780555555556	8.33164019794047e-05\\
10.0836111111111	-0.000145601930682107\\
10.0891666666667	-0.000199726668057692\\
10.0947222222222	-0.000173530837722488\\
10.1002777777778	-0.000191280929000695\\
10.1058333333333	-0.000112750082960707\\
10.1113888888889	-6.58996781963372e-05\\
10.1169444444444	-0.00024291424082847\\
10.1225	-0.000316829250273133\\
10.1280555555556	-4.47418662697635e-05\\
10.1336111111111	-2.85279051357429e-05\\
10.1391666666667	-8.44475454479317e-05\\
10.1447222222222	-5.99453455255038e-05\\
10.1502777777778	-7.7233991245549e-05\\
10.1558333333333	-6.42738682298463e-05\\
10.1613888888889	2.54168444161785e-05\\
10.1669444444444	-1.35874912787369e-05\\
10.1725	-3.1931183822989e-05\\
10.1780555555556	-7.44925728498702e-05\\
10.1836111111111	-5.4823365474819e-05\\
10.1891666666667	-6.25203674547423e-05\\
10.1947222222222	-0.000115846760558047\\
10.2002777777778	-6.6633977259781e-05\\
10.2058333333333	7.83352026224271e-05\\
10.2113888888889	9.87531015960589e-05\\
10.2169444444444	6.86788195612836e-05\\
10.2225	-5.0457460543859e-05\\
10.2280555555556	1.1311968919983e-07\\
10.2336111111111	-2.8807927781001e-06\\
10.2391666666667	2.09664848542544e-05\\
10.2447222222222	-7.09175671228616e-05\\
10.2502777777778	-0.000135733485500412\\
10.2558333333333	-7.65285844486355e-05\\
10.2613888888889	9.18936736925269e-07\\
10.2669444444444	-1.15241854364688e-05\\
10.2725	0.000110241552145762\\
10.2780555555556	6.23231168182739e-05\\
10.2836111111111	7.94386168231549e-05\\
10.2891666666667	9.53788074602389e-05\\
10.2947222222222	-1.68863774728223e-05\\
10.3002777777778	-0.000221547360660613\\
10.3058333333333	-0.000255153672603678\\
10.3113888888889	-0.000197756627882826\\
10.3169444444444	-0.000173939151484392\\
10.3225	-0.000173169682482529\\
10.3280555555556	-6.24021427544113e-05\\
10.3336111111111	1.4073920915609e-06\\
10.3391666666667	-0.000253368451180946\\
10.3447222222222	-0.00019528294116043\\
10.3502777777778	-0.000241591786630479\\
10.3558333333333	-0.000178945435965933\\
10.3613888888889	-1.73843160265757e-05\\
10.3669444444444	-0.000287852719478024\\
10.3725	-0.00031514126602518\\
10.3780555555556	-0.000430897282244927\\
10.3836111111111	-0.00014798263838654\\
10.3891666666667	-0.000147508183384378\\
10.3947222222222	-0.000123634331741777\\
10.4002777777778	-0.000203017032748523\\
10.4058333333333	-0.000364979785612652\\
10.4113888888889	-0.000163418256566872\\
10.4169444444444	-5.33291384614529e-06\\
10.4225	3.23555047082549e-05\\
10.4280555555556	-0.00013305363914577\\
10.4336111111111	-0.000174219304724449\\
10.4391666666667	-0.000339896930499811\\
10.4447222222222	-4.28018075575342e-05\\
10.4502777777778	-0.000105131774001627\\
10.4558333333333	-0.000134583152299367\\
10.4613888888889	-1.7058789011228e-05\\
10.4669444444444	8.2038898869462e-05\\
10.4725	-2.37422015270841e-05\\
10.4780555555556	-4.05972852237042e-05\\
10.4836111111111	-8.69737056740738e-05\\
10.4891666666667	-0.000178840351692444\\
10.4947222222222	-3.43271926294821e-05\\
10.5002777777778	6.38360466373386e-05\\
10.5058333333333	0.000102101048266897\\
10.5113888888889	-4.35724359277254e-05\\
10.5169444444444	0.000184967542367741\\
10.5225	0.000128583590695607\\
10.5280555555556	-0.000120508000592323\\
10.5336111111111	-0.00013451719351154\\
10.5391666666667	-4.01330363044671e-05\\
10.5447222222222	4.42221355542055e-05\\
10.5502777777778	6.81086431654626e-06\\
10.5558333333333	6.59177637972443e-05\\
10.5613888888889	8.34849936371458e-05\\
10.5669444444444	9.21478737861067e-05\\
10.5725	0.000133270583387342\\
10.5780555555556	6.18388623527843e-05\\
10.5836111111111	0.00022257564115149\\
10.5891666666667	7.79188069023015e-05\\
10.5947222222222	6.59348950004232e-05\\
10.6002777777778	7.66202873693333e-05\\
10.6058333333333	0.000187995682330967\\
10.6113888888889	0.000217199736089158\\
10.6169444444444	0.00021454638899442\\
10.6225	0.0002445998729001\\
10.6280555555556	0.000424748652636465\\
10.6336111111111	0.000404702245214859\\
10.6391666666667	0.000490799602382502\\
10.6447222222222	0.000602058876942149\\
10.6502777777778	0.00063141007263601\\
10.6558333333333	0.000420922635210001\\
10.6613888888889	0.000445408958747501\\
10.6669444444444	0.000357645825672212\\
10.6725	0.000345161118181665\\
10.6780555555556	0.000428176957707005\\
10.6836111111111	0.000457968144108903\\
10.6891666666667	0.000453479409707729\\
10.6947222222222	0.000461721991272895\\
10.7002777777778	0.000338821913003189\\
10.7058333333333	0.000354183928481052\\
10.7113888888889	0.00021844069512752\\
10.7169444444444	0.000136329970695349\\
10.7225	0.000152235565691636\\
10.7280555555556	6.5832627082843e-05\\
10.7336111111111	3.93234194284124e-05\\
10.7391666666667	0.000137377264992486\\
10.7447222222222	7.89852960727675e-05\\
10.7502777777778	7.31551930543306e-05\\
10.7558333333333	7.90670574883002e-05\\
10.7613888888889	8.7565724472819e-05\\
10.7669444444444	0.000158980383461097\\
10.7725	0.000205916820195387\\
10.7780555555556	0.000121211973788241\\
10.7836111111111	-5.15488240832743e-05\\
10.7891666666667	-4.04896228577793e-05\\
10.7947222222222	-0.000140807734975594\\
10.8002777777778	-5.28301298486018e-05\\
10.8058333333333	-0.000242733384684614\\
10.8113888888889	-0.000291698711771608\\
10.8169444444444	-0.000196245618590453\\
10.8225	2.2258860307076e-05\\
10.8280555555556	-0.000263840520145725\\
10.8336111111111	-0.00035789625716486\\
10.8391666666667	-0.000391090396097\\
10.8447222222222	-0.00040368936954201\\
10.8502777777778	-0.000397227602013988\\
10.8558333333333	-0.000315553008888795\\
10.8613888888889	-0.000312811070828908\\
10.8669444444444	-0.000245838779601275\\
10.8725	-0.000194245044264096\\
10.8780555555556	-0.000226763503204977\\
10.8836111111111	-0.000120648120603223\\
10.8891666666667	-3.03193204219362e-05\\
10.8947222222222	-5.42298512920855e-05\\
10.9002777777778	-3.05360313969115e-05\\
10.9058333333333	-0.000225850749111609\\
10.9113888888889	-0.000207352618420465\\
10.9169444444444	-0.0001942511479374\\
10.9225	-0.000184895659983574\\
10.9280555555556	-0.000239382144487547\\
10.9336111111111	-0.000227530955252023\\
10.9391666666667	-0.000224320962124892\\
10.9447222222222	-0.000339529611787831\\
10.9502777777778	-0.000215437091910826\\
10.9558333333333	-0.000271154853870345\\
10.9613888888889	-0.00015742863773345\\
10.9669444444444	-0.00016399752432222\\
10.9725	2.38973577263089e-06\\
10.9780555555556	-0.000122183423474475\\
10.9836111111111	-0.000101605347539381\\
10.9891666666667	-9.19005728295792e-05\\
10.9947222222222	-0.000276015148612587\\
11.0002777777778	-0.00024157093492453\\
11.0058333333333	-0.000182552682607626\\
11.0113888888889	-0.0003883495629385\\
11.0169444444444	-0.000422854302726068\\
11.0225	-0.000449964684730888\\
11.0280555555556	-0.000421757201807732\\
11.0336111111111	-0.000332258248727146\\
11.0391666666667	-0.000302022940348055\\
11.0447222222222	-0.000175115581381981\\
11.0502777777778	-0.000116988181547107\\
11.0558333333333	4.69137887247929e-06\\
11.0613888888889	-0.000112134430693134\\
11.0669444444444	-0.000111952856734665\\
11.0725	-2.85854121692285e-05\\
11.0780555555556	-7.61256478172924e-05\\
11.0836111111111	-1.67666331840188e-05\\
11.0891666666667	9.50347813263326e-05\\
11.0947222222222	0.000104794614587818\\
11.1002777777778	-3.27753051988473e-05\\
11.1058333333333	-3.12354939607577e-05\\
11.1113888888889	-8.22558008627916e-05\\
11.1169444444444	-3.82425794394992e-05\\
11.1225	-2.99032380722034e-05\\
11.1280555555556	-0.000131530369174523\\
11.1336111111111	-0.000152227314832665\\
11.1391666666667	-0.000120200569670232\\
11.1447222222222	-8.90513675474679e-05\\
11.1502777777778	-8.63210620152282e-05\\
11.1558333333333	-0.000223589391202584\\
11.1613888888889	-0.000247295229070967\\
11.1669444444444	-0.000483670429411092\\
11.1725	-0.000598845848203551\\
11.1780555555556	-0.000468237262154707\\
11.1836111111111	-0.000488274619529109\\
11.1891666666667	-0.000509520199458081\\
11.1947222222222	-0.00058266045678108\\
11.2002777777778	-0.000615921177826475\\
11.2058333333333	-0.000625459009835983\\
11.2113888888889	-0.000687297274671295\\
11.2169444444444	-0.000581775989118328\\
11.2225	-0.000548521841601306\\
11.2280555555556	-0.000538847207224315\\
11.2336111111111	-0.000488947867290758\\
11.2391666666667	-0.000621494295243742\\
11.2447222222222	-0.000630668553619105\\
11.2502777777778	-0.000624494630016061\\
11.2558333333333	-0.000733241295529543\\
11.2613888888889	-0.000909807110415026\\
11.2669444444444	-0.000768634899105753\\
11.2725	-0.000861025175053048\\
11.2780555555556	-0.00079855157826762\\
11.2836111111111	-0.000813126912411829\\
11.2891666666667	-0.000775973438320333\\
11.2947222222222	-0.00084759109478259\\
11.3002777777778	-0.00116247450274095\\
11.3058333333333	-0.00117928348769644\\
11.3113888888889	-0.00116991459461019\\
11.3169444444444	-0.000894300001167605\\
11.3225	-0.000947456367076829\\
11.3280555555556	-0.000927558065915451\\
11.3336111111111	-0.000846126278798463\\
11.3391666666667	-0.000825016284720853\\
11.3447222222222	-0.000791209364867998\\
11.3502777777778	-0.000725066677074734\\
11.3558333333333	-0.000631519555909889\\
11.3613888888889	-0.000608031958894178\\
11.3669444444444	-0.000687507556665082\\
11.3725	-0.000583788857748832\\
11.3780555555556	-0.000605343361511307\\
11.3836111111111	-0.00060612131229097\\
11.3891666666667	-0.000745048459438316\\
11.3947222222222	-0.000629567244280109\\
11.4002777777778	-0.00059320137776492\\
11.4058333333333	-0.000594007962701185\\
11.4113888888889	-0.00058285128109708\\
11.4169444444444	-0.000522744169634939\\
11.4225	-0.000496708799497513\\
11.4280555555556	-0.000458797124180646\\
11.4336111111111	-0.000466529963230975\\
11.4391666666667	-0.000643751720812588\\
11.4447222222222	-0.000649893326693658\\
11.4502777777778	-0.000600275670152415\\
11.4558333333333	-0.000583581023203286\\
11.4613888888889	-0.000515659893120394\\
11.4669444444444	-0.000573728241992614\\
11.4725	-0.000724171597049417\\
11.4780555555556	-0.000651945296341855\\
11.4836111111111	-0.000669896548862287\\
11.4891666666667	-0.000590823026083389\\
11.4947222222222	-0.000584137540241235\\
11.5002777777778	-0.000584000177191348\\
11.5058333333333	-0.000501979346449117\\
11.5113888888889	-0.000515052003562993\\
11.5169444444444	-0.000595330032567988\\
11.5225	-0.000583687196519518\\
11.5280555555556	-0.00056597397927322\\
11.5336111111111	-0.000550314213790051\\
11.5391666666667	-0.000504722101148782\\
11.5447222222222	-0.000574354561475843\\
11.5502777777778	-0.000546269647604437\\
11.5558333333333	-0.000551952297608624\\
11.5613888888889	-0.000700818264025828\\
11.5669444444444	-0.00070469085163081\\
11.5725	-0.000722378283074935\\
11.5780555555556	-0.000910412585091958\\
11.5836111111111	-0.000848916959713712\\
11.5891666666667	-0.000785627723362466\\
11.5947222222222	-0.000798799393471902\\
11.6002777777778	-0.000782142414531614\\
11.6058333333333	-0.000798324074836821\\
11.6113888888889	-0.000744465027221775\\
11.6169444444444	-0.000763307351294665\\
11.6225	-0.000814554943732022\\
11.6280555555556	-0.000620911375732693\\
11.6336111111111	-0.000586550299223279\\
11.6391666666667	-0.000506093094160564\\
11.6447222222222	-0.000638765429490311\\
11.6502777777778	-0.000639596561433889\\
11.6558333333333	-0.000532107815546752\\
11.6613888888889	-0.000501178832465125\\
11.6669444444444	-0.000605360370373618\\
11.6725	-0.000626324391497326\\
11.6780555555556	-0.000643264257118426\\
11.6836111111111	-0.000602182479288549\\
11.6891666666667	-0.000558267847145\\
11.6947222222222	-0.000690329414563916\\
11.7002777777778	-0.000673113067654788\\
11.7058333333333	-0.000680323793050439\\
11.7113888888889	-0.000742227604781195\\
11.7169444444444	-0.000731968964920399\\
11.7225	-0.000759599810528414\\
11.7280555555556	-0.000930423049214718\\
11.7336111111111	-0.000851109404032804\\
11.7391666666667	-0.000883317555389075\\
11.7447222222222	-0.000827027981040877\\
11.7502777777778	-0.000857874261910808\\
11.7558333333333	-0.000797349091187463\\
11.7613888888889	-0.0010275471716757\\
11.7669444444444	-0.000953484201873079\\
11.7725	-0.00091710853782763\\
11.7780555555556	-0.001053462336842\\
11.7836111111111	-0.00121944736099555\\
11.7891666666667	-0.00131012888522747\\
11.7947222222222	-0.00129999536468172\\
11.8002777777778	-0.00121791445475325\\
11.8058333333333	-0.00119369725216247\\
11.8113888888889	-0.00125986156726285\\
11.8169444444444	-0.00130993969324689\\
11.8225	-0.00122793489370943\\
11.8280555555556	-0.00130423985238794\\
11.8336111111111	-0.00136339568644404\\
11.8391666666667	-0.00124011565588663\\
11.8447222222222	-0.00114838803848424\\
11.8502777777778	-0.00113062613941814\\
11.8558333333333	-0.00138469596129677\\
11.8613888888889	-0.00127670213386986\\
11.8669444444444	-0.00111869048618566\\
11.8725	-0.00114900458396117\\
11.8780555555556	-0.00114391995768625\\
11.8836111111111	-0.00110807343165314\\
11.8891666666667	-0.00108365958304306\\
11.8947222222222	-0.00118620670621874\\
11.9002777777778	-0.00118944645328331\\
11.9058333333333	-0.0010328779965896\\
11.9113888888889	-0.0009583160329304\\
11.9169444444444	-0.000927131398762827\\
11.9225	-0.000972025203688385\\
11.9280555555556	-0.00091341737806219\\
11.9336111111111	-0.000844940834547092\\
11.9391666666667	-0.000836041886027721\\
11.9447222222222	-0.000852787074273641\\
11.9502777777778	-0.000803641940557046\\
11.9558333333333	-0.000737126237268257\\
11.9613888888889	-0.000696852800197984\\
11.9669444444444	-0.000712488365182821\\
11.9725	-0.00067530897481346\\
11.9780555555556	-0.000728166334055621\\
11.9836111111111	-0.000914716222566107\\
11.9891666666667	-0.000711100684073881\\
11.9947222222222	-0.00064766299781485\\
12.0002777777778	-0.000537427760864486\\
12.0058333333333	-0.000594358272987106\\
12.0113888888889	-0.000676099227589474\\
12.0169444444444	-0.000630723634889191\\
12.0225	-0.00053152981113686\\
12.0280555555556	-0.000561123689068467\\
12.0336111111111	-0.000577871764858602\\
12.0391666666667	-0.000490445786876816\\
12.0447222222222	-0.000478482379772176\\
12.0502777777778	-0.000589255949305172\\
12.0558333333333	-0.000591220083654999\\
12.0613888888889	-0.000468376841952418\\
12.0669444444444	-0.000436997367852534\\
12.0725	-0.000462007097092859\\
12.0780555555556	-0.000456978645305675\\
12.0836111111111	-0.000479599172831046\\
12.0891666666667	-0.000763869149168232\\
12.0947222222222	-0.00075159610306738\\
12.1002777777778	-0.000702701779592257\\
12.1058333333333	-0.000553703145532445\\
12.1113888888889	-0.000585932850820847\\
12.1169444444444	-0.000565154160848081\\
12.1225	-0.000540808211796317\\
12.1280555555556	-0.000472221004533628\\
12.1336111111111	-0.000493869075702711\\
12.1391666666667	-0.000466121385409695\\
12.1447222222222	-0.000458740496737981\\
12.1502777777778	-0.000237139832604356\\
12.1558333333333	-9.90287771370385e-05\\
12.1613888888889	-0.00010340622827031\\
12.1669444444444	-5.97169817989774e-05\\
12.1725	-0.000252747170205601\\
12.1780555555556	-0.000280887777233134\\
12.1836111111111	-0.000272375845258115\\
12.1891666666667	-0.000432433974795318\\
12.1947222222222	-0.000535312087885424\\
12.2002777777778	-0.000451407996805452\\
12.2058333333333	-0.000729052242314158\\
12.2113888888889	-0.000787457607998274\\
12.2169444444444	-0.000544124951929126\\
12.2225	-0.000552457357068901\\
12.2280555555556	-0.000720431658435715\\
12.2336111111111	-0.000856516299737241\\
12.2391666666667	-0.000908988309602384\\
12.2447222222222	-0.00123350287869791\\
12.2502777777778	-0.0011814909125479\\
12.2558333333333	-0.00114292270013696\\
12.2613888888889	-0.00104455299695137\\
12.2669444444444	-0.00109931698450937\\
12.2725	-0.00109713717801889\\
12.2780555555556	-0.00119190029839499\\
12.2836111111111	-0.00114890415729354\\
12.2891666666667	-0.0010764183153637\\
12.2947222222222	-0.00128946542328292\\
12.3002777777778	-0.00133845763551641\\
12.3058333333333	-0.00120636294021651\\
12.3113888888889	-0.00132845801426739\\
12.3169444444444	-0.00124236163693691\\
12.3225	-0.00117778435900045\\
12.3280555555556	-0.000968240237617692\\
12.3336111111111	-0.000969486691437243\\
12.3391666666667	-0.00116839512147841\\
12.3447222222222	-0.00113640940660094\\
12.3502777777778	-0.00113192464871299\\
12.3558333333333	-0.00153500138498281\\
12.3613888888889	-0.00123867243443403\\
12.3669444444444	-0.00129869310070277\\
12.3725	-0.00115367248603658\\
12.3780555555556	-0.00124887161451743\\
12.3836111111111	-0.00115517831099358\\
12.3891666666667	-0.00122173426566818\\
12.3947222222222	-0.00128196181860462\\
12.4002777777778	-0.00118310814969514\\
12.4058333333333	-0.00101663313334115\\
12.4113888888889	-0.0010640743746609\\
12.4169444444444	-0.00109884571912279\\
12.4225	-0.00124887935024304\\
12.4280555555556	-0.00122684101537494\\
12.4336111111111	-0.00138263811696329\\
12.4391666666667	-0.00128898280917457\\
12.4447222222222	-0.00107448865731457\\
12.4502777777778	-0.000921217061301113\\
12.4558333333333	-0.00107431824675682\\
12.4613888888889	-0.000935338418620058\\
12.4669444444444	-0.00108709660904\\
12.4725	-0.00120056681169766\\
12.4780555555556	-0.00116063715234758\\
12.4836111111111	-0.00113230541792194\\
12.4891666666667	-0.00100869527759837\\
12.4947222222222	-0.000943907618689856\\
12.5002777777778	-0.000925474179098468\\
12.5058333333333	-0.00100935584820233\\
12.5113888888889	-0.0010576791349849\\
12.5169444444444	-0.00124795458959564\\
12.5225	-0.00113715246658762\\
12.5280555555556	-0.00103916053919508\\
12.5336111111111	-0.000974196701933344\\
12.5391666666667	-0.00106827512901228\\
12.5447222222222	-0.00127032398806739\\
12.5502777777778	-0.00115505602986904\\
12.5558333333333	-0.00111060049787725\\
12.5613888888889	-0.00101697936034664\\
12.5669444444444	-0.00109993740540874\\
12.5725	-0.00112898750826835\\
12.5780555555556	-0.00116731013020794\\
12.5836111111111	-0.00121657863756621\\
12.5891666666667	-0.00139902085747974\\
12.5947222222222	-0.00151742710564901\\
12.6002777777778	-0.00159365559680579\\
12.6058333333333	-0.00158942820062091\\
12.6113888888889	-0.00164462722421201\\
12.6169444444444	-0.00147094750620696\\
12.6225	-0.00144297860826665\\
12.6280555555556	-0.00162871201253512\\
12.6336111111111	-0.00158815201892757\\
12.6391666666667	-0.00162781285359516\\
12.6447222222222	-0.00172209854883438\\
12.6502777777778	-0.00170762025036921\\
12.6558333333333	-0.00172156502036849\\
12.6613888888889	-0.00170236563823946\\
12.6669444444444	-0.00162399615620818\\
12.6725	-0.00168898491720073\\
12.6780555555556	-0.00157868191521992\\
12.6836111111111	-0.0015244608587131\\
12.6891666666667	-0.00148932497238883\\
12.6947222222222	-0.00142926122400924\\
12.7002777777778	-0.00128736358595375\\
12.7058333333333	-0.00160331675797884\\
12.7113888888889	-0.00146538369033985\\
12.7169444444444	-0.00142510360000601\\
12.7225	-0.00143780729994477\\
12.7280555555556	-0.00147601025284263\\
12.7336111111111	-0.00139650073669534\\
12.7391666666667	-0.00139033300915587\\
12.7447222222222	-0.00145084765589073\\
12.7502777777778	-0.00152064552838568\\
12.7558333333333	-0.00159998494030904\\
12.7613888888889	-0.0014798422863986\\
12.7669444444444	-0.00167473535803338\\
12.7725	-0.00170615529007077\\
12.7780555555556	-0.00155862062932626\\
12.7836111111111	-0.0017541978722444\\
12.7891666666667	-0.00137542445500539\\
12.7947222222222	-0.00143958850300164\\
12.8002777777778	-0.00142435061156031\\
12.8058333333333	-0.0014296656635491\\
12.8113888888889	-0.00135499498480179\\
12.8169444444444	-0.00124389472331569\\
12.8225	-0.00143594444560365\\
12.8280555555556	-0.00143743603894499\\
12.8336111111111	-0.00139390761834272\\
12.8391666666667	-0.00140055568062427\\
12.8447222222222	-0.00139079828269567\\
12.8502777777778	-0.00140130909124741\\
12.8558333333333	-0.00144282770527249\\
12.8613888888889	-0.00142876030310611\\
12.8669444444444	-0.00152150519693919\\
12.8725	-0.00149863516199858\\
12.8780555555556	-0.00153405247424526\\
12.8836111111111	-0.00144294525204749\\
12.8891666666667	-0.00139670985719372\\
12.8947222222222	-0.00144380553132619\\
12.9002777777778	-0.00134420326635528\\
12.9058333333333	-0.00123816836852768\\
12.9113888888889	-0.00118353352898198\\
12.9169444444444	-0.00125626071530138\\
12.9225	-0.00126784323333021\\
12.9280555555556	-0.00131573889768667\\
12.9336111111111	-0.00132970703484127\\
12.9391666666667	-0.00123099664139112\\
12.9447222222222	-0.00128741951574419\\
12.9502777777778	-0.00124014345267989\\
12.9558333333333	-0.00116508923481001\\
12.9613888888889	-0.00121899834471201\\
12.9669444444444	-0.00134665610535013\\
12.9725	-0.00130265312343152\\
12.9780555555556	-0.0013961559132076\\
12.9836111111111	-0.00154953477784232\\
12.9891666666667	-0.0019829108351193\\
12.9947222222222	-0.00183650231956855\\
13.0002777777778	-0.00173977457640317\\
13.0058333333333	-0.00184369514761069\\
13.0113888888889	-0.0018253369556217\\
13.0169444444444	-0.00180188983889227\\
13.0225	-0.00184195025370057\\
13.0280555555556	-0.00186975179147955\\
13.0336111111111	-0.00182541289624713\\
13.0391666666667	-0.00186913404464663\\
13.0447222222222	-0.00189939655473198\\
13.0502777777778	-0.00181174767630116\\
13.0558333333333	-0.00168485039498863\\
13.0613888888889	-0.00166392042167607\\
13.0669444444444	-0.00176716152136753\\
13.0725	-0.00166568832559307\\
13.0780555555556	-0.00164860964390891\\
13.0836111111111	-0.00162913864560792\\
13.0891666666667	-0.00173444206480408\\
13.0947222222222	-0.00167730711845871\\
13.1002777777778	-0.00162501872523302\\
13.1058333333333	-0.00163028777736814\\
13.1113888888889	-0.00159390778394309\\
13.1169444444444	-0.00153546598586476\\
13.1225	-0.00153482183853543\\
13.1280555555556	-0.00151596790177928\\
13.1336111111111	-0.00168829922009128\\
13.1391666666667	-0.00155050061638873\\
13.1447222222222	-0.00164025846123272\\
13.1502777777778	-0.0014929013914104\\
13.1558333333333	-0.00146556024780254\\
13.1613888888889	-0.00124181787806143\\
13.1669444444444	-0.00127974489072017\\
13.1725	-0.00120599819978235\\
13.1780555555556	-0.00121200762684317\\
13.1836111111111	-0.00127506928070055\\
13.1891666666667	-0.0011734885964188\\
13.1947222222222	-0.00138298870135097\\
13.2002777777778	-0.00139680414713731\\
13.2058333333333	-0.00145840823823003\\
13.2113888888889	-0.00164952139718008\\
13.2169444444444	-0.00155989082812379\\
13.2225	-0.00158229444916857\\
13.2280555555556	-0.00149078974012291\\
13.2336111111111	-0.00145014030439537\\
13.2391666666667	-0.00147101861737947\\
13.2447222222222	-0.00154769918460032\\
13.2502777777778	-0.00152497797739268\\
13.2558333333333	-0.00150519814341501\\
13.2613888888889	-0.00144711143433023\\
13.2669444444444	-0.00140898995373333\\
13.2725	-0.00144605513528981\\
13.2780555555556	-0.00154146194685568\\
13.2836111111111	-0.00147618541550859\\
13.2891666666667	-0.00134013639604103\\
13.2947222222222	-0.00138553204677994\\
13.3002777777778	-0.00153159124588639\\
13.3058333333333	-0.0014217057703018\\
13.3113888888889	-0.00131100104832792\\
13.3169444444444	-0.00125694350971809\\
13.3225	-0.00132443406420052\\
13.3280555555556	-0.00133282566794301\\
13.3336111111111	-0.00134067655291981\\
13.3391666666667	-0.00125553446259047\\
13.3447222222222	-0.00128304914748248\\
13.3502777777778	-0.00115129243490547\\
13.3558333333333	-0.00114373422965655\\
13.3613888888889	-0.0010265333198143\\
13.3669444444444	-0.000941316781882298\\
13.3725	-0.000933160989022755\\
13.3780555555556	-0.000887050425508819\\
13.3836111111111	-0.000978022440593834\\
13.3891666666667	-0.000951868889746774\\
13.3947222222222	-0.000859876577836009\\
13.4002777777778	-0.000808521341373437\\
13.4058333333333	-0.000696203319972747\\
13.4113888888889	-0.000575601802900488\\
13.4169444444444	-0.000539056737911985\\
13.4225	-0.000506603834815855\\
13.4280555555556	-0.000619567852414352\\
13.4336111111111	-0.000525273406596061\\
13.4391666666667	-0.000385800359032438\\
13.4447222222222	-0.00037807959380014\\
13.4502777777778	-0.000487659651581808\\
13.4558333333333	-0.000399253042929257\\
13.4613888888889	-0.000336256815648052\\
13.4669444444444	-0.000314562486900906\\
13.4725	-0.000231918933846679\\
13.4780555555556	-0.000362724474844314\\
13.4836111111111	-0.000388009593613422\\
13.4891666666667	-0.000437652653795661\\
13.4947222222222	-0.000346807743729023\\
13.5002777777778	-0.000355225794495248\\
13.5058333333333	-0.000340782745653277\\
13.5113888888889	-0.000310716816906091\\
13.5169444444444	-0.000317812583924019\\
13.5225	-0.00024453278557475\\
13.5280555555556	-0.000224651045598948\\
13.5336111111111	-0.000520678800719253\\
13.5391666666667	-0.000373841355848834\\
13.5447222222222	-0.000192068945114578\\
13.5502777777778	-9.69230820938411e-05\\
13.5558333333333	-9.41104203223462e-05\\
13.5613888888889	-0.000278068673953399\\
13.5669444444444	-0.000200896936954629\\
13.5725	-0.000172063249421911\\
13.5780555555556	-0.00025853654168679\\
13.5836111111111	-0.000298000921839618\\
13.5891666666667	-0.000242861710426892\\
13.5947222222222	-0.000898848161760887\\
13.6002777777778	-0.000458943158990701\\
13.6058333333333	-0.000477387164203834\\
13.6113888888889	-0.000564415672683637\\
13.6169444444444	-0.000552792710139444\\
13.6225	-0.000495157545245941\\
13.6280555555556	-0.000744095264845016\\
13.6336111111111	-0.000597510123317344\\
13.6391666666667	-0.000517380026793115\\
13.6447222222222	-0.000560756511426547\\
13.6502777777778	-0.000587186530837195\\
13.6558333333333	-0.000537590933928812\\
13.6613888888889	-0.00045548577497408\\
13.6669444444444	-0.000416138322906643\\
13.6725	-0.000554361052793472\\
13.6780555555556	-0.00084853285055066\\
13.6836111111111	-0.000711880928102716\\
13.6891666666667	-0.000631708890349609\\
13.6947222222222	-0.000590345363949053\\
13.7002777777778	-0.000749280438179654\\
13.7058333333333	-0.000704797997616658\\
13.7113888888889	-0.00075574438655039\\
13.7169444444444	-0.000712192983441257\\
13.7225	-0.000635727275859772\\
13.7280555555556	-0.000634832368098055\\
13.7336111111111	-0.000637277831031372\\
13.7391666666667	-0.000616994190145334\\
13.7447222222222	-0.000577493839266017\\
13.7502777777778	-0.000299538018605362\\
13.7558333333333	-0.000418833728589275\\
13.7613888888889	-0.000390629455574365\\
13.7669444444444	-0.000407997560734738\\
13.7725	-0.00062337161084013\\
13.7780555555556	-0.000630363175735819\\
13.7836111111111	-0.000454885662089657\\
13.7891666666667	-0.000549107888966087\\
13.7947222222222	-0.0004852206440206\\
13.8002777777778	-0.000423500841746544\\
13.8058333333333	-0.000479659102958873\\
13.8113888888889	-0.000470827173548774\\
13.8169444444444	-0.000308877951337742\\
13.8225	-0.000387646150460925\\
13.8280555555556	-0.000375615504864864\\
13.8336111111111	-0.00053750210807327\\
13.8391666666667	-0.00037717166548414\\
13.8447222222222	-0.000334191839147108\\
13.8502777777778	-0.000371544627512273\\
13.8558333333333	-0.000377860198613876\\
13.8613888888889	-0.000296791724136236\\
13.8669444444444	-0.000335666971405817\\
13.8725	-0.00029723854741743\\
13.8780555555556	-0.000214099755065462\\
13.8836111111111	-0.000104499562106553\\
13.8891666666667	5.73888845992347e-05\\
13.8947222222222	-0.000162899839923506\\
13.9002777777778	-0.000143258020323542\\
13.9058333333333	-0.000194069371305771\\
13.9113888888889	-7.42948688070421e-05\\
13.9169444444444	-2.22527623026455e-05\\
13.9225	-8.53119955778552e-05\\
13.9280555555556	-4.6716760315792e-05\\
13.9336111111111	-0.000153544981498785\\
13.9391666666667	-0.000210745706848482\\
13.9447222222222	-0.000265752592404589\\
13.9502777777778	-0.000212260539342775\\
13.9558333333333	-0.000267539399882481\\
13.9613888888889	-0.000347467889190715\\
13.9669444444444	-0.00037483607677055\\
13.9725	-0.000321884323240371\\
13.9780555555556	-0.000295395411812655\\
13.9836111111111	-0.000508586480514821\\
13.9891666666667	-0.000549022429600745\\
13.9947222222222	-0.000593683375379526\\
14.0002777777778	-0.000705878313046974\\
14.0058333333333	-0.000770092883151216\\
14.0113888888889	-0.000702874607382728\\
14.0169444444444	-0.000529872288759118\\
14.0225	-0.000601972950964163\\
14.0280555555556	-0.000606909877529373\\
14.0336111111111	-0.00058196122632357\\
14.0391666666667	-0.000431059412618052\\
14.0447222222222	-0.000397811638419873\\
14.0502777777778	-0.000449658199765534\\
14.0558333333333	-0.000379810676791439\\
14.0613888888889	-0.000441502831680979\\
14.0669444444444	-0.000237497856650353\\
14.0725	-0.000429244051550286\\
14.0780555555556	-0.000521938577312612\\
14.0836111111111	-0.000488168463648446\\
14.0891666666667	-0.000547109209675508\\
14.0947222222222	-0.000559588608733791\\
14.1002777777778	-0.000562117305804544\\
14.1058333333333	-0.000460747911741236\\
14.1113888888889	-0.000576947656828449\\
14.1169444444444	-0.000547341523399303\\
14.1225	-0.000753109011828885\\
14.1280555555556	-0.000646247265491697\\
14.1336111111111	-0.000669130477755924\\
14.1391666666667	-0.000603380214325229\\
14.1447222222222	-0.000643385722579799\\
14.1502777777778	-0.000630114584431994\\
14.1558333333333	-0.000719421811348654\\
14.1613888888889	-0.000657274984406056\\
14.1669444444444	-0.000602535927922047\\
14.1725	-0.000530341899075459\\
14.1780555555556	-0.000631949469902557\\
14.1836111111111	-0.000670805285433903\\
14.1891666666667	-0.000623489097410813\\
14.1947222222222	-0.000917392352823986\\
14.2002777777778	-0.000978925032343921\\
14.2058333333333	-0.000910616086980544\\
14.2113888888889	-0.000959007599903354\\
14.2169444444444	-0.000916841274262649\\
14.2225	-0.000955969501708054\\
14.2280555555556	-0.0012242466877694\\
14.2336111111111	-0.00107512715182683\\
14.2391666666667	-0.000917747691422502\\
14.2447222222222	-0.000917057265975376\\
14.2502777777778	-0.000801405513951229\\
14.2558333333333	-0.00060139139868793\\
14.2613888888889	-0.000789860134958264\\
14.2669444444444	-0.000703888275774632\\
14.2725	-0.000621524704178468\\
14.2780555555556	-0.000627918945345651\\
14.2836111111111	-0.00049677597418216\\
14.2891666666667	-0.000470105784651167\\
14.2947222222222	-0.000588960145540649\\
14.3002777777778	-0.000585502420764849\\
14.3058333333333	-0.000530543390620261\\
14.3113888888889	-0.000472831122363829\\
14.3169444444444	-0.000639288034694554\\
14.3225	-0.000567549928322975\\
14.3280555555556	-0.000445294126530161\\
14.3336111111111	-0.000547442359101669\\
14.3391666666667	-0.000484918155018071\\
14.3447222222222	-0.000656010338126826\\
14.3502777777778	-0.000729418277089042\\
14.3558333333333	-0.000730815193126352\\
14.3613888888889	-0.000749240603824755\\
14.3669444444444	-0.000699617349629331\\
14.3725	-0.00079205964951598\\
14.3780555555556	-0.000787261832617004\\
14.3836111111111	-0.000750097165261277\\
14.3891666666667	-0.000848569627098937\\
14.3947222222222	-0.000952860426780578\\
14.4002777777778	-0.0007794244856425\\
14.4058333333333	-0.000723912485575253\\
14.4113888888889	-0.000624807322157522\\
14.4169444444444	-0.000690105660168226\\
14.4225	-0.000667688227168087\\
14.4280555555556	-0.00067269948507019\\
14.4336111111111	-0.000842688204225449\\
14.4391666666667	-0.000886343404136576\\
14.4447222222222	-0.000952343628359719\\
14.4502777777778	-0.00102711407685873\\
14.4558333333333	-0.00114166447492316\\
14.4613888888889	-0.00114485892482093\\
14.4669444444444	-0.00120335912525326\\
14.4725	-0.00111528902126883\\
14.4780555555556	-0.00105338712920564\\
14.4836111111111	-0.00114327606562392\\
14.4891666666667	-0.00106362646616995\\
14.4947222222222	-0.00108700023169896\\
14.5002777777778	-0.001285376400996\\
14.5058333333333	-0.0012456846906904\\
14.5113888888889	-0.00116655036817769\\
14.5169444444444	-0.00113794106011525\\
14.5225	-0.00114363940830352\\
14.5280555555556	-0.00116495024797541\\
14.5336111111111	-0.00108361597998821\\
14.5391666666667	-0.00103490812578441\\
14.5447222222222	-0.00112135177041072\\
14.5502777777778	-0.00103212490905379\\
14.5558333333333	-0.00129679547084062\\
14.5613888888889	-0.00149312269103157\\
14.5669444444444	-0.00124151043057676\\
14.5725	-0.00127883953416733\\
14.5780555555556	-0.00127751673729522\\
14.5836111111111	-0.00133839166851086\\
14.5891666666667	-0.00150197786319609\\
14.5947222222222	-0.00140423783295529\\
14.6002777777778	-0.0015560104675366\\
14.6058333333333	-0.00126734241309554\\
14.6113888888889	-0.00165652122683054\\
14.6169444444444	-0.0016971202956878\\
14.6225	-0.00156745655828101\\
14.6280555555556	-0.00158659799490964\\
14.6336111111111	-0.00169813779829592\\
14.6391666666667	-0.00143233277063011\\
14.6447222222222	-0.00151006684429676\\
14.6502777777778	-0.00147508595148087\\
14.6558333333333	-0.00126166413889411\\
14.6613888888889	-0.00128576889065336\\
14.6669444444444	-0.00112127944118388\\
14.6725	-0.00129159783708866\\
14.6780555555556	-0.00124944353340612\\
14.6836111111111	-0.00115768428605995\\
14.6891666666667	-0.00129699290013003\\
14.6947222222222	-0.0012527955296736\\
14.7002777777778	-0.00109681552103566\\
14.7058333333333	-0.000909567655136153\\
14.7113888888889	-0.0007868978011622\\
14.7169444444444	-0.00078045474700693\\
14.7225	-0.000696546202150832\\
14.7280555555556	-0.000713804021958728\\
14.7336111111111	-0.000578505976051568\\
14.7391666666667	-0.000583228356725046\\
14.7447222222222	-0.000619766999356529\\
14.7502777777778	-0.000722066243640465\\
14.7558333333333	-0.00068319494093911\\
14.7613888888889	-0.000832062224725974\\
14.7669444444444	-0.000883383328617552\\
14.7725	-0.000887313259826688\\
14.7780555555556	-0.000863135457321446\\
14.7836111111111	-0.000721822481095191\\
14.7891666666667	-0.000743222027243466\\
14.7947222222222	-0.000915354171277627\\
14.8002777777778	-0.000851663593745628\\
14.8058333333333	-0.00069499687045626\\
14.8113888888889	-0.000839319195108838\\
14.8169444444444	-0.000862717163370646\\
14.8225	-0.000819882207741821\\
14.8280555555556	-0.000791903452169167\\
14.8336111111111	-0.000751345465680396\\
14.8391666666667	-0.000712899725643195\\
14.8447222222222	-0.000871312253601696\\
14.8502777777778	-0.0012745508187838\\
14.8558333333333	-0.00127601466426468\\
14.8613888888889	-0.00117932290675613\\
14.8669444444444	-0.00125505418419363\\
14.8725	-0.00118319448255856\\
14.8780555555556	-0.00107949667619138\\
14.8836111111111	-0.00102995468187221\\
14.8891666666667	-0.00113805229447092\\
14.8947222222222	-0.000930449936349133\\
14.9002777777778	-0.000903493341874499\\
14.9058333333333	-0.00106942055864637\\
14.9113888888889	-0.00105790725520857\\
14.9169444444444	-0.00078112211937119\\
14.9225	-0.000869431506484582\\
14.9280555555556	-0.000855063567180387\\
14.9336111111111	-0.000843822644798119\\
14.9391666666667	-0.000851883314591752\\
14.9447222222222	-0.000941636816462768\\
14.9502777777778	-0.000958261338802247\\
14.9558333333333	-0.000935609197658799\\
14.9613888888889	-0.000945664760259127\\
14.9669444444444	-0.000886317823435409\\
14.9725	-0.00122942824057867\\
14.9780555555556	-0.00119645236277237\\
14.9836111111111	-0.00127170105957572\\
14.9891666666667	-0.00126106888700308\\
14.9947222222222	-0.00120748735609319\\
15.0002777777778	-0.00126961206582143\\
15.0058333333333	-0.000921068470606471\\
15.0113888888889	-0.000813200524280294\\
15.0169444444444	-0.000839539908075834\\
15.0225	-0.000890991911370496\\
15.0280555555556	-0.000736276448169999\\
15.0336111111111	-0.000581943593414697\\
15.0391666666667	-0.000537820351843206\\
15.0447222222222	-0.000785716545988079\\
15.0502777777778	-0.000623145554880524\\
15.0558333333333	-0.000676537306963797\\
15.0613888888889	-0.000779046509015247\\
15.0669444444444	-0.00101166513134459\\
15.0725	-0.0011759308386981\\
15.0780555555556	-0.000780346724208158\\
15.0836111111111	-0.00134026978966838\\
15.0891666666667	-0.00106640505089504\\
15.0947222222222	-0.000896143308849632\\
15.1002777777778	-0.00153508005710342\\
15.1058333333333	-0.00153823310602854\\
15.1113888888889	-0.00119033806146499\\
15.1169444444444	-0.0012373033636099\\
15.1225	-0.00120594407310013\\
15.1280555555556	-0.00120471733031721\\
15.1336111111111	-0.00154865477822872\\
15.1391666666667	-0.00160949658299823\\
15.1447222222222	-0.00107254233222695\\
15.1502777777778	-0.00123783149931805\\
15.1558333333333	-0.00116615379981551\\
15.1613888888889	-0.00117358251006572\\
15.1669444444444	-0.00125722176100756\\
15.1725	-0.001318850795145\\
15.1780555555556	-0.00118661819421851\\
15.1836111111111	-0.00117756934374977\\
15.1891666666667	-0.00116589592231917\\
15.1947222222222	-0.0013914522917875\\
15.2002777777778	-0.00130144302122198\\
15.2058333333333	-0.00122876721059062\\
15.2113888888889	-0.00125012881594995\\
15.2169444444444	-0.00112847858102469\\
15.2225	-0.00100647390107912\\
15.2280555555556	-0.000869981058441256\\
15.2336111111111	-0.000979057561657117\\
15.2391666666667	-0.00120344549703644\\
15.2447222222222	-0.00106740908092534\\
15.2502777777778	-0.00115430302178213\\
15.2558333333333	-0.000948401904138284\\
15.2613888888889	-0.00157525847073684\\
15.2669444444444	-0.00139953263059345\\
15.2725	-0.00135196600894047\\
15.2780555555556	-0.00122328995947655\\
15.2836111111111	-0.00136128282855138\\
15.2891666666667	-0.00132463318286437\\
15.2947222222222	-0.0012564823539525\\
15.3002777777778	-0.00126865599180296\\
15.3058333333333	-0.00131098078552417\\
15.3113888888889	-0.00122733960016024\\
15.3169444444444	-0.00117015656782704\\
15.3225	-0.00139840008858285\\
15.3280555555556	-0.00137935014635024\\
15.3336111111111	-0.00129655005011194\\
15.3391666666667	-0.00137790759353497\\
15.3447222222222	-0.00142332560625904\\
15.3502777777778	-0.00149755430294273\\
15.3558333333333	-0.0015832508851108\\
15.3613888888889	-0.00141525012698683\\
15.3669444444444	-0.00146935597654729\\
15.3725	-0.00138535466515072\\
15.3780555555556	-0.00138376696289457\\
15.3836111111111	-0.00142557226469609\\
15.3891666666667	-0.00136136375740968\\
15.3947222222222	-0.00127002477916318\\
15.4002777777778	-0.00117840335644738\\
15.4058333333333	-0.00108892053691065\\
15.4113888888889	-0.00114960326379129\\
15.4169444444444	-0.00109858913526516\\
15.4225	-0.00111230534504163\\
15.4280555555556	-0.000917293476864487\\
15.4336111111111	-0.000892959509515828\\
15.4391666666667	-0.000777079973369751\\
15.4447222222222	-0.000732842558385653\\
15.4502777777778	-0.000995210067704333\\
15.4558333333333	-0.000969830306533641\\
15.4613888888889	-0.00102581566235831\\
15.4669444444444	-0.000898019379374531\\
15.4725	-0.00103266557167928\\
15.4780555555556	-0.000781905247917136\\
15.4836111111111	-0.000769903713577026\\
15.4891666666667	-0.000830776889004374\\
15.4947222222222	-0.000638865890636924\\
15.5002777777778	-0.000761480877731172\\
15.5058333333333	-0.000598480272279471\\
15.5113888888889	-0.000646680287458837\\
15.5169444444444	-0.00067660356210882\\
15.5225	-0.000722570582525674\\
15.5280555555556	-0.000734957467166267\\
15.5336111111111	-0.000858868124844109\\
15.5391666666667	-0.000555377572444148\\
15.5447222222222	-0.000519644460631142\\
15.5502777777778	-0.000852528862521798\\
15.5558333333333	-0.000677387238212686\\
15.5613888888889	-0.000861755685493175\\
15.5669444444444	-0.000981450872289259\\
15.5725	-0.000781772811690468\\
15.5780555555556	-0.000784013248019909\\
15.5836111111111	-0.000719007212177333\\
15.5891666666667	-0.000614413784562433\\
15.5947222222222	-0.000582710542225348\\
15.6002777777778	-0.000478682958535431\\
15.6058333333333	-0.000355400360065396\\
15.6113888888889	-0.000533337938103142\\
15.6169444444444	-0.000427969738747531\\
15.6225	-0.000340947750610434\\
15.6280555555556	-0.000488265355474887\\
15.6336111111111	-0.000580105373449178\\
15.6391666666667	-0.000363515898307188\\
15.6447222222222	-0.000595297725078962\\
15.6502777777778	-0.000790351195468587\\
15.6558333333333	-0.000664429545156003\\
15.6613888888889	-0.000698142164111671\\
15.6669444444444	-0.000696503410545458\\
15.6725	-0.000682575855016985\\
15.6780555555556	-0.000712565743389686\\
15.6836111111111	-0.000664782858606428\\
15.6891666666667	-0.000709739422820169\\
15.6947222222222	-0.000845048556223537\\
15.7002777777778	-0.000933633467706047\\
15.7058333333333	-0.000885029766202255\\
15.7113888888889	-0.00100393407905016\\
15.7169444444444	-0.00118584253057802\\
15.7225	-0.00105340665563347\\
15.7280555555556	-0.00108239942439075\\
15.7336111111111	-0.000812275503754875\\
15.7391666666667	-0.000869529469209384\\
15.7447222222222	-0.000754559267428368\\
15.7502777777778	-0.000803460582153257\\
15.7558333333333	-0.000809685817188731\\
15.7613888888889	-0.000669136605416397\\
15.7669444444444	-0.000521610267377188\\
15.7725	-0.000399816052325202\\
15.7780555555556	-0.00031485740468733\\
15.7836111111111	-0.00055117977191337\\
15.7891666666667	-0.000498889787372233\\
15.7947222222222	-0.000474154925798391\\
15.8002777777778	-0.000571077842744468\\
15.8058333333333	-0.000476051679670023\\
15.8113888888889	-0.000597266703122178\\
15.8169444444444	-0.00026318442523247\\
15.8225	-0.000113130144454621\\
15.8280555555556	-0.000276272959330753\\
15.8336111111111	-0.000705416322888009\\
15.8391666666667	-0.000548461904256147\\
15.8447222222222	-0.000682004911855587\\
15.8502777777778	-0.000187594356305634\\
15.8558333333333	-0.00026243893677283\\
15.8613888888889	0.000142062983016787\\
15.8669444444444	-0.000392595388607378\\
15.8725	3.03235211463119e-05\\
15.8780555555556	-0.000113387322405005\\
15.8836111111111	7.71377994460504e-05\\
15.8891666666667	-0.000156104706234462\\
15.8947222222222	-0.000336933218735723\\
15.9002777777778	-0.000150488761096942\\
15.9058333333333	7.8091089029774e-06\\
15.9113888888889	0.000107279463119896\\
15.9169444444444	0.000273424953343104\\
15.9225	0.000381245352952507\\
15.9280555555556	0.000419894346790361\\
15.9336111111111	0.000167923048835567\\
15.9391666666667	0.000159674651368252\\
15.9447222222222	0.000240484978432645\\
15.9502777777778	-0.000104551328617213\\
15.9558333333333	0.000295273948804992\\
15.9613888888889	0.000670996493506632\\
15.9669444444444	0.00079312530975321\\
15.9725	0.000617701730567474\\
15.9780555555556	0.000828614543997956\\
15.9836111111111	0.00106015451821448\\
15.9891666666667	0.00126513930563257\\
15.9947222222222	0.00113145196738207\\
};
\end{axis}
\end{tikzpicture}%
%
\end{subfigure}%
\hspace{1.5cm}%
\begin{subfigure}[b]{.4\linewidth}%
  %\centering
  \setlength\figureheight{\linewidth}%
  \setlength\figurewidth{\linewidth}%
  \tikzsetnextfilename{Ch2/NTAP_naive_strat_comp}%
  % This file was created by matlab2tikz.
%
%The latest updates can be retrieved from
%  http://www.mathworks.com/matlabcentral/fileexchange/22022-matlab2tikz-matlab2tikz
%where you can also make suggestions and rate matlab2tikz.
%
\definecolor{mycolor1}{rgb}{0.00000,0.00000,0.00000}%
\definecolor{mycolor2}{rgb}{0.40000,0.40000,0.40000}%
\definecolor{mycolor3}{rgb}{0.70000,0.70000,0.70000}%
%
\begin{tikzpicture}[trim axis left, trim axis right]

\begin{axis}[%
width=\figurewidth,
height=\figureheight,
at={(0\figurewidth,0\figureheight)},
scale only axis,
every outer x axis line/.append style={black},
every x tick label/.append style={font=\color{black}},
xmin=9.5,
xmax=16,
every outer y axis line/.append style={black},
every y tick label/.append style={font=\color{black}},
ymin=-0.25,
ymax=0.5,
title={NTAP},
axis background/.style={fill=white},
axis x line*=bottom,
axis y line*=left
]
\addplot [color=mycolor1,solid,line width=1.5pt,forget plot]
  table[row sep=crcr]{%
9.50027777777778	0\\
9.50583333333333	-0.00502268582941208\\
9.51138888888889	-0.00691967951917791\\
9.51694444444444	-0.00926668634702683\\
9.5225	-0.0116351913896457\\
9.52805555555556	-0.0142733183030538\\
9.53361111111111	-0.0161783926393745\\
9.53916666666667	-0.0177788075053063\\
9.54472222222222	-0.0182962547101569\\
9.55027777777778	-0.0205803533153953\\
9.55583333333333	-0.0227376741894775\\
9.56138888888889	-0.0238215107421463\\
9.56694444444444	-0.0245163780819641\\
9.5725	-0.0252051661326391\\
9.57805555555555	-0.0251800478311632\\
9.58361111111111	-0.0262792548373514\\
9.58916666666667	-0.0280984557726248\\
9.59472222222222	-0.0298832446083843\\
9.60027777777778	-0.0290249829385948\\
9.60583333333333	-0.0301580123394497\\
9.61138888888889	-0.031506267130757\\
9.61694444444444	-0.0323813402780362\\
9.6225	-0.0328737860625159\\
9.62805555555556	-0.0343327605559936\\
9.63361111111111	-0.0350542314027246\\
9.63916666666667	-0.0348111733884213\\
9.64472222222222	-0.0355238887042569\\
9.65027777777778	-0.0370648119713244\\
9.65583333333333	-0.0379066086039422\\
9.66138888888889	-0.0392449332565995\\
9.66694444444444	-0.0402063745231215\\
9.6725	-0.0413097328565058\\
9.67805555555555	-0.0424497249362485\\
9.68361111111111	-0.0437837547522824\\
9.68916666666667	-0.044756484958714\\
9.69472222222222	-0.0457974082911496\\
9.70027777777778	-0.046856143655606\\
9.70583333333333	-0.0490943023090688\\
9.71138888888889	-0.0494381532328733\\
9.71694444444444	-0.0491573748666364\\
9.7225	-0.0496711345711329\\
9.72805555555555	-0.0507954690536888\\
9.73361111111111	-0.0510804297648781\\
9.73916666666667	-0.0518414679789816\\
9.74472222222222	-0.0520925333605142\\
9.75027777777778	-0.0520904441541501\\
9.75583333333333	-0.0525784615269224\\
9.76138888888889	-0.0526417122909287\\
9.76694444444444	-0.0532549178287082\\
9.7725	-0.0544988321761572\\
9.77805555555556	-0.0550373957915459\\
9.78361111111111	-0.0557860900716676\\
9.78916666666667	-0.0564848322917526\\
9.79472222222222	-0.0569573202686613\\
9.80027777777778	-0.0581361023303229\\
9.80583333333333	-0.059926282647512\\
9.81138888888889	-0.0597649332241851\\
9.81694444444444	-0.060003610377814\\
9.8225	-0.0612943032458012\\
9.82805555555555	-0.0633504183117416\\
9.83361111111111	-0.0634326484002769\\
9.83916666666667	-0.0648481156498929\\
9.84472222222222	-0.0650938236475491\\
9.85027777777778	-0.0650509632233873\\
9.85583333333333	-0.0659063778096838\\
9.86138888888889	-0.0660892237236504\\
9.86694444444444	-0.0654940575689737\\
9.8725	-0.0649357327358978\\
9.87805555555556	-0.0655578775801907\\
9.88361111111111	-0.0665896237248169\\
9.88916666666667	-0.0657168599866305\\
9.89472222222222	-0.0671343630500972\\
9.90027777777778	-0.067261481055247\\
9.90583333333333	-0.0675194232552959\\
9.91138888888889	-0.0679521673069546\\
9.91694444444444	-0.0685130394627655\\
9.9225	-0.0709403093296872\\
9.92805555555555	-0.0692828238827042\\
9.93361111111111	-0.0703651864658131\\
9.93916666666667	-0.0702048273111548\\
9.94472222222222	-0.0699358426535726\\
9.95027777777778	-0.0720979359832932\\
9.95583333333333	-0.0716356299371102\\
9.96138888888889	-0.0722008472378786\\
9.96694444444444	-0.0716666186499422\\
9.9725	-0.0718377717761696\\
9.97805555555555	-0.0729834652627774\\
9.98361111111111	-0.0732674155314051\\
9.98916666666667	-0.0734318893564807\\
9.99472222222222	-0.0743362763945472\\
10.0002777777778	-0.0749303622045134\\
10.0058333333333	-0.0756314666200951\\
10.0113888888889	-0.0753825177824155\\
10.0169444444444	-0.0754494348322014\\
10.0225	-0.074378725719472\\
10.0280555555556	-0.0755455757431942\\
10.0336111111111	-0.0769780855316285\\
10.0391666666667	-0.0769369781786548\\
10.0447222222222	-0.0780233597342436\\
10.0502777777778	-0.0784472867032396\\
10.0558333333333	-0.0802899607262151\\
10.0613888888889	-0.0808830862532304\\
10.0669444444444	-0.0801171765194816\\
10.0725	-0.0793246680570998\\
10.0780555555556	-0.0793814917831989\\
10.0836111111111	-0.0808877134483259\\
10.0891666666667	-0.0804785764018981\\
10.0947222222222	-0.0802142493281436\\
10.1002777777778	-0.0809156347314003\\
10.1058333333333	-0.0810895489226039\\
10.1113888888889	-0.0799946117731337\\
10.1169444444444	-0.0808304782270345\\
10.1225	-0.0801600849224397\\
10.1280555555556	-0.0794837045265761\\
10.1336111111111	-0.0808821048561179\\
10.1391666666667	-0.0805322168421013\\
10.1447222222222	-0.0832013755050574\\
10.1502777777778	-0.0840888115724031\\
10.1558333333333	-0.083941597016464\\
10.1613888888889	-0.0837051289595631\\
10.1669444444444	-0.085012575564696\\
10.1725	-0.0853536595649209\\
10.1780555555556	-0.0869946966055677\\
10.1836111111111	-0.0880132133996236\\
10.1891666666667	-0.0877562848422994\\
10.1947222222222	-0.089282383822974\\
10.2002777777778	-0.090405685949718\\
10.2058333333333	-0.089090784663096\\
10.2113888888889	-0.0897911799153483\\
10.2169444444444	-0.0906911952867592\\
10.2225	-0.0918009536118413\\
10.2280555555556	-0.0913189019227132\\
10.2336111111111	-0.0911435662963385\\
10.2391666666667	-0.0920125324247725\\
10.2447222222222	-0.0942886619035264\\
10.2502777777778	-0.0930222503015981\\
10.2558333333333	-0.0922851146687683\\
10.2613888888889	-0.0938335298572359\\
10.2669444444444	-0.0963805584576468\\
10.2725	-0.0954079774010898\\
10.2780555555556	-0.0974147945744323\\
10.2836111111111	-0.0979156732816011\\
10.2891666666667	-0.0989872582851961\\
10.2947222222222	-0.0996909180942037\\
10.3002777777778	-0.0999689587679293\\
10.3058333333333	-0.0999102819270796\\
10.3113888888889	-0.100452041966299\\
10.3169444444444	-0.102887074623201\\
10.3225	-0.10344723619277\\
10.3280555555556	-0.104292778360955\\
10.3336111111111	-0.104539302715236\\
10.3391666666667	-0.105704656537977\\
10.3447222222222	-0.106430327947515\\
10.3502777777778	-0.10645627467582\\
10.3558333333333	-0.107061934458917\\
10.3613888888889	-0.10763915797209\\
10.3669444444444	-0.108283356444876\\
10.3725	-0.107866503937906\\
10.3780555555556	-0.108330627531753\\
10.3836111111111	-0.107879896481595\\
10.3891666666667	-0.108238872229302\\
10.3947222222222	-0.108839263307283\\
10.4002777777778	-0.109052358123656\\
10.4058333333333	-0.109024479291762\\
10.4113888888889	-0.109307843145036\\
10.4169444444444	-0.10814339648577\\
10.4225	-0.10966227016967\\
10.4280555555556	-0.108057811145422\\
10.4336111111111	-0.109181385704108\\
10.4391666666667	-0.110723480949811\\
10.4447222222222	-0.110211380072091\\
10.4502777777778	-0.111252867028557\\
10.4558333333333	-0.112918260173219\\
10.4613888888889	-0.112766641038333\\
10.4669444444444	-0.113386770346657\\
10.4725	-0.114251974102153\\
10.4780555555556	-0.114100116654638\\
10.4836111111111	-0.112797830062155\\
10.4891666666667	-0.112487019918355\\
10.4947222222222	-0.11302645658634\\
10.5002777777778	-0.112652759370558\\
10.5058333333333	-0.111404049514361\\
10.5113888888889	-0.11145498402943\\
10.5169444444444	-0.110809602620695\\
10.5225	-0.111616872752698\\
10.5280555555556	-0.113101915390212\\
10.5336111111111	-0.113203614140286\\
10.5391666666667	-0.115865671547985\\
10.5447222222222	-0.11511951178194\\
10.5502777777778	-0.114651300935995\\
10.5558333333333	-0.114735005856521\\
10.5613888888889	-0.114606292956472\\
10.5669444444444	-0.114470103486447\\
10.5725	-0.113762178945286\\
10.5780555555556	-0.115200928433748\\
10.5836111111111	-0.117435837607991\\
10.5891666666667	-0.11899037434329\\
10.5947222222222	-0.119956615063445\\
10.6002777777778	-0.119519265786246\\
10.6058333333333	-0.117129595573039\\
10.6113888888889	-0.11748496753074\\
10.6169444444444	-0.119011223697414\\
10.6225	-0.119067645430019\\
10.6280555555556	-0.118993043940803\\
10.6336111111111	-0.118580230931824\\
10.6391666666667	-0.119297882308396\\
10.6447222222222	-0.119718040449163\\
10.6502777777778	-0.11927402321317\\
10.6558333333333	-0.121194962405912\\
10.6613888888889	-0.120240937003029\\
10.6669444444444	-0.121501861758233\\
10.6725	-0.120414572891388\\
10.6780555555556	-0.120607581175547\\
10.6836111111111	-0.121844147012258\\
10.6891666666667	-0.120840794368592\\
10.6947222222222	-0.121841326860269\\
10.7002777777778	-0.12274118760576\\
10.7058333333333	-0.122726938218547\\
10.7113888888889	-0.122146665471365\\
10.7169444444444	-0.122487954057203\\
10.7225	-0.124073764007186\\
10.7280555555556	-0.126062654869375\\
10.7336111111111	-0.127541868538846\\
10.7391666666667	-0.128166248711188\\
10.7447222222222	-0.127121699136223\\
10.7502777777778	-0.126538592843176\\
10.7558333333333	-0.126722190282933\\
10.7613888888889	-0.126761124075724\\
10.7669444444444	-0.125746555668251\\
10.7725	-0.125726515624407\\
10.7780555555556	-0.124212048406504\\
10.7836111111111	-0.128015133640778\\
10.7891666666667	-0.129564994745206\\
10.7947222222222	-0.130967082941854\\
10.8002777777778	-0.130525840147019\\
10.8058333333333	-0.129149435577627\\
10.8113888888889	-0.13047261734072\\
10.8169444444444	-0.129675694773274\\
10.8225	-0.132324407095032\\
10.8280555555556	-0.133193332395182\\
10.8336111111111	-0.135072628606436\\
10.8391666666667	-0.136137258655721\\
10.8447222222222	-0.134910374168478\\
10.8502777777778	-0.135480398664271\\
10.8558333333333	-0.136004267250494\\
10.8613888888889	-0.135699895997058\\
10.8669444444444	-0.134550089635025\\
10.8725	-0.135986242644483\\
10.8780555555556	-0.137019310259267\\
10.8836111111111	-0.137680024712056\\
10.8891666666667	-0.136426750095076\\
10.8947222222222	-0.135745051134384\\
10.9002777777778	-0.137430831050801\\
10.9058333333333	-0.136322879969061\\
10.9113888888889	-0.134989940442964\\
10.9169444444444	-0.135459763324484\\
10.9225	-0.135179881765046\\
10.9280555555556	-0.134923779884061\\
10.9336111111111	-0.133132511268906\\
10.9391666666667	-0.133816471542238\\
10.9447222222222	-0.133283566241334\\
10.9502777777778	-0.13492054285138\\
10.9558333333333	-0.135635013250896\\
10.9613888888889	-0.134423380989713\\
10.9669444444444	-0.135335712592811\\
10.9725	-0.136699058256335\\
10.9780555555556	-0.138024979311577\\
10.9836111111111	-0.136816004865292\\
10.9891666666667	-0.136621943643451\\
10.9947222222222	-0.137743100899393\\
11.0002777777778	-0.136683611076919\\
11.0058333333333	-0.136973466338564\\
11.0113888888889	-0.138150457121631\\
11.0169444444444	-0.13713731997627\\
11.0225	-0.136181211639206\\
11.0280555555556	-0.135803884356156\\
11.0336111111111	-0.136864641272502\\
11.0391666666667	-0.136740072815783\\
11.0447222222222	-0.136555513656908\\
11.0502777777778	-0.136148084272069\\
11.0558333333333	-0.133995767565217\\
11.0613888888889	-0.133569990389427\\
11.0669444444444	-0.135947283445192\\
11.0725	-0.135167711415101\\
11.0780555555556	-0.134531440847735\\
11.0836111111111	-0.135093445224138\\
11.0891666666667	-0.132608724607818\\
11.0947222222222	-0.132846574055169\\
11.1002777777778	-0.133781248992974\\
11.1058333333333	-0.13370827379475\\
11.1113888888889	-0.132653087209291\\
11.1169444444444	-0.132472267673102\\
11.1225	-0.132884146954175\\
11.1280555555556	-0.131990223694703\\
11.1336111111111	-0.130331730876575\\
11.1391666666667	-0.13099708788426\\
11.1447222222222	-0.131674480083835\\
11.1502777777778	-0.129520596879006\\
11.1558333333333	-0.129455076445807\\
11.1613888888889	-0.130436470929781\\
11.1669444444444	-0.132287125506649\\
11.1725	-0.133029351399258\\
11.1780555555556	-0.131583407685365\\
11.1836111111111	-0.131486728994841\\
11.1891666666667	-0.132486388924415\\
11.1947222222222	-0.132347870661629\\
11.2002777777778	-0.132799688387314\\
11.2058333333333	-0.133599637099863\\
11.2113888888889	-0.132107918373452\\
11.2169444444444	-0.131733992094576\\
11.2225	-0.133840567212638\\
11.2280555555556	-0.134424142633799\\
11.2336111111111	-0.136390818981119\\
11.2391666666667	-0.136094382586573\\
11.2447222222222	-0.136195326684273\\
11.2502777777778	-0.137596606629678\\
11.2558333333333	-0.139282171672792\\
11.2613888888889	-0.139547917202834\\
11.2669444444444	-0.139626479311003\\
11.2725	-0.138403050873319\\
11.2780555555556	-0.137509716605253\\
11.2836111111111	-0.138759610834384\\
11.2891666666667	-0.139355052029538\\
11.2947222222222	-0.140664561545674\\
11.3002777777778	-0.14051674558998\\
11.3058333333333	-0.142027455354652\\
11.3113888888889	-0.142982879387759\\
11.3169444444444	-0.141465030017941\\
11.3225	-0.140274196888445\\
11.3280555555556	-0.14049374737561\\
11.3336111111111	-0.140401949454533\\
11.3391666666667	-0.139679796282112\\
11.3447222222222	-0.139481872191583\\
11.3502777777778	-0.141277472892007\\
11.3558333333333	-0.140029839769109\\
11.3613888888889	-0.139967192463027\\
11.3669444444444	-0.139126959536\\
11.3725	-0.139822803598705\\
11.3780555555556	-0.138275017132673\\
11.3836111111111	-0.137308158588908\\
11.3891666666667	-0.138940866360991\\
11.3947222222222	-0.140827322428967\\
11.4002777777778	-0.139891167284172\\
11.4058333333333	-0.139859204228488\\
11.4113888888889	-0.140787871503319\\
11.4169444444444	-0.140761655716544\\
11.4225	-0.142218251310957\\
11.4280555555556	-0.141993478292442\\
11.4336111111111	-0.141818565251706\\
11.4391666666667	-0.143742352990248\\
11.4447222222222	-0.144702010882228\\
11.4502777777778	-0.145578623840675\\
11.4558333333333	-0.144834457356998\\
11.4613888888889	-0.144742432879437\\
11.4669444444444	-0.143308220899326\\
11.4725	-0.144247840685112\\
11.4780555555556	-0.143812098257834\\
11.4836111111111	-0.146032199193886\\
11.4891666666667	-0.146812524973823\\
11.4947222222222	-0.145627659554246\\
11.5002777777778	-0.145935132813704\\
11.5058333333333	-0.143781269630939\\
11.5113888888889	-0.14554319091193\\
11.5169444444444	-0.146879382564871\\
11.5225	-0.147030908444405\\
11.5280555555556	-0.148374136765434\\
11.5336111111111	-0.149566643429945\\
11.5391666666667	-0.149651881426773\\
11.5447222222222	-0.150583618660129\\
11.5502777777778	-0.149702593681319\\
11.5558333333333	-0.149729192671968\\
11.5613888888889	-0.150188618891299\\
11.5669444444444	-0.149484225626817\\
11.5725	-0.149919887206266\\
11.5780555555556	-0.149536719898256\\
11.5836111111111	-0.150731230340385\\
11.5891666666667	-0.149923003724876\\
11.5947222222222	-0.150438447232428\\
11.6002777777778	-0.150250245298897\\
11.6058333333333	-0.150093975005287\\
11.6113888888889	-0.147789660657205\\
11.6169444444444	-0.146275236099857\\
11.6225	-0.146947499824216\\
11.6280555555556	-0.146983293758335\\
11.6336111111111	-0.148645390797165\\
11.6391666666667	-0.15140756472899\\
11.6447222222222	-0.151536728288376\\
11.6502777777778	-0.150840295218541\\
11.6558333333333	-0.151776268890812\\
11.6613888888889	-0.150738138531954\\
11.6669444444444	-0.153907838323919\\
11.6725	-0.152081084129736\\
11.6780555555556	-0.152098416046608\\
11.6836111111111	-0.152015810722268\\
11.6891666666667	-0.151833315318593\\
11.6947222222222	-0.152069876562363\\
11.7002777777778	-0.154524956281513\\
11.7058333333333	-0.154944639005605\\
11.7113888888889	-0.154455274771963\\
11.7169444444444	-0.153062061726692\\
11.7225	-0.152654874952703\\
11.7280555555556	-0.150645074960614\\
11.7336111111111	-0.15080664650915\\
11.7391666666667	-0.150971082806504\\
11.7447222222222	-0.152493785330133\\
11.7502777777778	-0.152388416127467\\
11.7558333333333	-0.154635214786666\\
11.7613888888889	-0.154948979355293\\
11.7669444444444	-0.156383161951068\\
11.7725	-0.15707934279144\\
11.7780555555556	-0.157111975999534\\
11.7836111111111	-0.156781171392286\\
11.7891666666667	-0.156878551173396\\
11.7947222222222	-0.157024222844265\\
11.8002777777778	-0.157409744283294\\
11.8058333333333	-0.159346697971438\\
11.8113888888889	-0.159442883632868\\
11.8169444444444	-0.157510814638346\\
11.8225	-0.160715406563078\\
11.8280555555556	-0.162190554282827\\
11.8336111111111	-0.162590063516094\\
11.8391666666667	-0.160516222229564\\
11.8447222222222	-0.15866135375488\\
11.8502777777778	-0.157600486332845\\
11.8558333333333	-0.15755040099659\\
11.8613888888889	-0.156850842293959\\
11.8669444444444	-0.156308256024176\\
11.8725	-0.158480510102787\\
11.8780555555556	-0.158149417201673\\
11.8836111111111	-0.155942091463399\\
11.8891666666667	-0.157956475138319\\
11.8947222222222	-0.156644751103777\\
11.9002777777778	-0.157836530557388\\
11.9058333333333	-0.158769029837441\\
11.9113888888889	-0.160069722007096\\
11.9169444444444	-0.160160488316555\\
11.9225	-0.15947412373432\\
11.9280555555556	-0.159290127526497\\
11.9336111111111	-0.159202236528139\\
11.9391666666667	-0.160403597348566\\
11.9447222222222	-0.159040065374483\\
11.9502777777778	-0.158238594501468\\
11.9558333333333	-0.161425743403898\\
11.9613888888889	-0.160805644103136\\
11.9669444444444	-0.161927169726932\\
11.9725	-0.161779770086636\\
11.9780555555556	-0.161188751170077\\
11.9836111111111	-0.160973925129576\\
11.9891666666667	-0.161333631344114\\
11.9947222222222	-0.162725093166799\\
12.0002777777778	-0.162998319785833\\
12.0058333333333	-0.161275126536038\\
12.0113888888889	-0.162273833683468\\
12.0169444444444	-0.161599480829848\\
12.0225	-0.162266668961678\\
12.0280555555556	-0.162583981401006\\
12.0336111111111	-0.160421353376007\\
12.0391666666667	-0.161234220231234\\
12.0447222222222	-0.16138601149518\\
12.0502777777778	-0.161281406164482\\
12.0558333333333	-0.161587223026034\\
12.0613888888889	-0.164501810731551\\
12.0669444444444	-0.167595427802232\\
12.0725	-0.167447183333552\\
12.0780555555556	-0.167937320209265\\
12.0836111111111	-0.16675032564384\\
12.0891666666667	-0.164920406411505\\
12.0947222222222	-0.16588885379537\\
12.1002777777778	-0.166884438739336\\
12.1058333333333	-0.165900659206932\\
12.1113888888889	-0.165564205804142\\
12.1169444444444	-0.166011830371131\\
12.1225	-0.166126974061055\\
12.1280555555556	-0.167384241428232\\
12.1336111111111	-0.170377617030637\\
12.1391666666667	-0.170500723710055\\
12.1447222222222	-0.169787669594324\\
12.1502777777778	-0.170355337355193\\
12.1558333333333	-0.169729595452076\\
12.1613888888889	-0.171310768146698\\
12.1669444444444	-0.17272697297682\\
12.1725	-0.172501747673139\\
12.1780555555556	-0.173515035260253\\
12.1836111111111	-0.171475307856672\\
12.1891666666667	-0.169417918625019\\
12.1947222222222	-0.17022317760013\\
12.2002777777778	-0.171065166353975\\
12.2058333333333	-0.170125475473267\\
12.2113888888889	-0.172000239440218\\
12.2169444444444	-0.171790885390603\\
12.2225	-0.172343802378211\\
12.2280555555556	-0.171898546559828\\
12.2336111111111	-0.171439273608036\\
12.2391666666667	-0.172529684392608\\
12.2447222222222	-0.172517448647149\\
12.2502777777778	-0.172186398863583\\
12.2558333333333	-0.171508193768998\\
12.2613888888889	-0.1726252927029\\
12.2669444444444	-0.172134997083478\\
12.2725	-0.171415274006229\\
12.2780555555556	-0.170070479167042\\
12.2836111111111	-0.172203478585542\\
12.2891666666667	-0.170414566680937\\
12.2947222222222	-0.168299972881171\\
12.3002777777778	-0.169142114268513\\
12.3058333333333	-0.168762617676079\\
12.3113888888889	-0.169299383211487\\
12.3169444444444	-0.168792109684325\\
12.3225	-0.168468901505235\\
12.3280555555556	-0.167966661319936\\
12.3336111111111	-0.167212437736136\\
12.3391666666667	-0.167966076105741\\
12.3447222222222	-0.166354316190116\\
12.3502777777778	-0.164940214520741\\
12.3558333333333	-0.166990786214222\\
12.3613888888889	-0.166235016227293\\
12.3669444444444	-0.16522985221822\\
12.3725	-0.166499708266443\\
12.3780555555556	-0.16730695933598\\
12.3836111111111	-0.164591232877656\\
12.3891666666667	-0.164222093355465\\
12.3947222222222	-0.163181992544535\\
12.4002777777778	-0.165479564046478\\
12.4058333333333	-0.164109918581062\\
12.4113888888889	-0.165513993517276\\
12.4169444444444	-0.167951800199883\\
12.4225	-0.168536210577744\\
12.4280555555556	-0.168447702603203\\
12.4336111111111	-0.168706221864511\\
12.4391666666667	-0.170474549344854\\
12.4447222222222	-0.171776963682201\\
12.4502777777778	-0.172083449258643\\
12.4558333333333	-0.171548950675336\\
12.4613888888889	-0.170850680562564\\
12.4669444444444	-0.169948580709667\\
12.4725	-0.170008392825424\\
12.4780555555556	-0.169950685564679\\
12.4836111111111	-0.171955755755234\\
12.4891666666667	-0.173557684034053\\
12.4947222222222	-0.173995781792024\\
12.5002777777778	-0.174418367729858\\
12.5058333333333	-0.174636045799423\\
12.5113888888889	-0.174302315358657\\
12.5169444444444	-0.174056423445763\\
12.5225	-0.172775739745878\\
12.5280555555556	-0.170541736747945\\
12.5336111111111	-0.171417763843463\\
12.5391666666667	-0.171900333849267\\
12.5447222222222	-0.172401241618767\\
12.5502777777778	-0.170902385153383\\
12.5558333333333	-0.169399465571793\\
12.5613888888889	-0.171073234284644\\
12.5669444444444	-0.170394942114795\\
12.5725	-0.169751431667362\\
12.5780555555556	-0.171113922757756\\
12.5836111111111	-0.170133506604096\\
12.5891666666667	-0.17052494065122\\
12.5947222222222	-0.171149779578755\\
12.6002777777778	-0.172738029794138\\
12.6058333333333	-0.173705527082417\\
12.6113888888889	-0.172225999956682\\
12.6169444444444	-0.168923353155472\\
12.6225	-0.168808327033644\\
12.6280555555556	-0.170418459515088\\
12.6336111111111	-0.169116658872723\\
12.6391666666667	-0.169000186482536\\
12.6447222222222	-0.1697998925744\\
12.6502777777778	-0.167228307856175\\
12.6558333333333	-0.16656790530738\\
12.6613888888889	-0.166296136248476\\
12.6669444444444	-0.16671416600294\\
12.6725	-0.164315838766071\\
12.6780555555556	-0.166477389108581\\
12.6836111111111	-0.166549077652132\\
12.6891666666667	-0.167786566958507\\
12.6947222222222	-0.168576904579281\\
12.7002777777778	-0.16969355547431\\
12.7058333333333	-0.16972281066056\\
12.7113888888889	-0.167577597685033\\
12.7169444444444	-0.168368877358048\\
12.7225	-0.167317397025529\\
12.7280555555556	-0.164563128954608\\
12.7336111111111	-0.16592461293522\\
12.7391666666667	-0.165282615139183\\
12.7447222222222	-0.165213766928408\\
12.7502777777778	-0.164726678716574\\
12.7558333333333	-0.164237177181114\\
12.7613888888889	-0.165006651331941\\
12.7669444444444	-0.166076470354442\\
12.7725	-0.165661770117309\\
12.7780555555556	-0.164904003913933\\
12.7836111111111	-0.166649536339659\\
12.7891666666667	-0.167629843525098\\
12.7947222222222	-0.168524536681853\\
12.8002777777778	-0.169130806730636\\
12.8058333333333	-0.169052086181219\\
12.8113888888889	-0.170783580823003\\
12.8169444444444	-0.17029146077324\\
12.8225	-0.170723902947841\\
12.8280555555556	-0.171480903215555\\
12.8336111111111	-0.170715802910646\\
12.8391666666667	-0.169487175020868\\
12.8447222222222	-0.16915745632075\\
12.8502777777778	-0.170307192098546\\
12.8558333333333	-0.17063878761372\\
12.8613888888889	-0.170677053560528\\
12.8669444444444	-0.171624880416079\\
12.8725	-0.171721702361705\\
12.8780555555556	-0.170080103048456\\
12.8836111111111	-0.171391409748272\\
12.8891666666667	-0.172783698892987\\
12.8947222222222	-0.172779117884418\\
12.9002777777778	-0.171741178855633\\
12.9058333333333	-0.171553459769955\\
12.9113888888889	-0.172132013615629\\
12.9169444444444	-0.173656249313393\\
12.9225	-0.175478554677428\\
12.9280555555556	-0.177854532008346\\
12.9336111111111	-0.17704865459052\\
12.9391666666667	-0.17721491116051\\
12.9447222222222	-0.178144157114939\\
12.9502777777778	-0.17888273880304\\
12.9558333333333	-0.178120438472544\\
12.9613888888889	-0.179040529807743\\
12.9669444444444	-0.180034412371363\\
12.9725	-0.179084060133017\\
12.9780555555556	-0.178321567979752\\
12.9836111111111	-0.178266283667584\\
12.9891666666667	-0.178852602870536\\
12.9947222222222	-0.17930649252114\\
13.0002777777778	-0.178464650891945\\
13.0058333333333	-0.177981719417146\\
13.0113888888889	-0.179410634753535\\
13.0169444444444	-0.178832549532025\\
13.0225	-0.180609227860427\\
13.0280555555556	-0.180052903817294\\
13.0336111111111	-0.180512379265987\\
13.0391666666667	-0.179832736768435\\
13.0447222222222	-0.177610256467492\\
13.0502777777778	-0.178181982374409\\
13.0558333333333	-0.177416448256896\\
13.0613888888889	-0.178196737658368\\
13.0669444444444	-0.178873036344574\\
13.0725	-0.180465356352418\\
13.0780555555556	-0.178844879301648\\
13.0836111111111	-0.178923013738115\\
13.0891666666667	-0.180172474835419\\
13.0947222222222	-0.179571607749934\\
13.1002777777778	-0.179934319578019\\
13.1058333333333	-0.181062303309061\\
13.1113888888889	-0.180245936225391\\
13.1169444444444	-0.181084331486349\\
13.1225	-0.181577232633712\\
13.1280555555556	-0.181998272901231\\
13.1336111111111	-0.181305455548133\\
13.1391666666667	-0.17800431427368\\
13.1447222222222	-0.178272699013511\\
13.1502777777778	-0.176877351851576\\
13.1558333333333	-0.175460887349869\\
13.1613888888889	-0.177533021273672\\
13.1669444444444	-0.178364506705992\\
13.1725	-0.179532110835743\\
13.1780555555556	-0.178392926071161\\
13.1836111111111	-0.180060040338177\\
13.1891666666667	-0.178623929098136\\
13.1947222222222	-0.180398280051156\\
13.2002777777778	-0.180685415587805\\
13.2058333333333	-0.181740281049423\\
13.2113888888889	-0.181120083004699\\
13.2169444444444	-0.18136100364141\\
13.2225	-0.181486157799821\\
13.2280555555556	-0.183602104223151\\
13.2336111111111	-0.183893391817749\\
13.2391666666667	-0.182206372615203\\
13.2447222222222	-0.181698304581697\\
13.2502777777778	-0.180523743046009\\
13.2558333333333	-0.181492495502447\\
13.2613888888889	-0.180499491474187\\
13.2669444444444	-0.181068941544207\\
13.2725	-0.182104651224937\\
13.2780555555556	-0.180829506360973\\
13.2836111111111	-0.181090840932163\\
13.2891666666667	-0.18176571407466\\
13.2947222222222	-0.181595154342129\\
13.3002777777778	-0.182526252110863\\
13.3058333333333	-0.184542042628534\\
13.3113888888889	-0.185142300169542\\
13.3169444444444	-0.185097981289061\\
13.3225	-0.185250277790867\\
13.3280555555556	-0.185230789014194\\
13.3336111111111	-0.183535199824136\\
13.3391666666667	-0.183581411853684\\
13.3447222222222	-0.182095201399199\\
13.3502777777778	-0.179619818461206\\
13.3558333333333	-0.177430879454975\\
13.3613888888889	-0.17439703930022\\
13.3669444444444	-0.172705286107167\\
13.3725	-0.173692799047576\\
13.3780555555556	-0.175545631205749\\
13.3836111111111	-0.174394098395107\\
13.3891666666667	-0.175539499658868\\
13.3947222222222	-0.176985996432451\\
13.4002777777778	-0.177911101637111\\
13.4058333333333	-0.178892045938182\\
13.4113888888889	-0.180731781522774\\
13.4169444444444	-0.180600553308227\\
13.4225	-0.181788380394449\\
13.4280555555556	-0.179595371749906\\
13.4336111111111	-0.180558018512647\\
13.4391666666667	-0.174542500241026\\
13.4447222222222	-0.173742797454931\\
13.4502777777778	-0.173840175657293\\
13.4558333333333	-0.172744248470701\\
13.4613888888889	-0.174123609086211\\
13.4669444444444	-0.173820698236889\\
13.4725	-0.171713316307967\\
13.4780555555556	-0.172167105126567\\
13.4836111111111	-0.171430814242225\\
13.4891666666667	-0.171317556543099\\
13.4947222222222	-0.172175713218002\\
13.5002777777778	-0.171754638388626\\
13.5058333333333	-0.170925282374901\\
13.5113888888889	-0.173041981291366\\
13.5169444444444	-0.174612341059122\\
13.5225	-0.174718548200656\\
13.5280555555556	-0.173198019062585\\
13.5336111111111	-0.173355132330719\\
13.5391666666667	-0.173277009928227\\
13.5447222222222	-0.171853446879084\\
13.5502777777778	-0.1718908963539\\
13.5558333333333	-0.170778445844271\\
13.5613888888889	-0.171958210183223\\
13.5669444444444	-0.171626541829401\\
13.5725	-0.172576694035909\\
13.5780555555556	-0.173070792957806\\
13.5836111111111	-0.172798502904047\\
13.5891666666667	-0.170533553820241\\
13.5947222222222	-0.170940806752804\\
13.6002777777778	-0.16930082798981\\
13.6058333333333	-0.16723021076725\\
13.6113888888889	-0.164032107880337\\
13.6169444444444	-0.162200928235482\\
13.6225	-0.162505705778979\\
13.6280555555556	-0.16062050732753\\
13.6336111111111	-0.162994136547514\\
13.6391666666667	-0.163945544630349\\
13.6447222222222	-0.162632307286955\\
13.6502777777778	-0.164457923855221\\
13.6558333333333	-0.162825236138483\\
13.6613888888889	-0.160246524495377\\
13.6669444444444	-0.159140583624826\\
13.6725	-0.160678828808501\\
13.6780555555556	-0.162290679331446\\
13.6836111111111	-0.161461515880071\\
13.6891666666667	-0.157985446082797\\
13.6947222222222	-0.159756250185379\\
13.7002777777778	-0.157866226861034\\
13.7058333333333	-0.157787954706311\\
13.7113888888889	-0.158310176975708\\
13.7169444444444	-0.15939971439803\\
13.7225	-0.161265179631283\\
13.7280555555556	-0.158641216469756\\
13.7336111111111	-0.157028018883089\\
13.7391666666667	-0.154444973160904\\
13.7447222222222	-0.153195337963245\\
13.7502777777778	-0.15457110429672\\
13.7558333333333	-0.154220606782921\\
13.7613888888889	-0.15596391899871\\
13.7669444444444	-0.156941671061928\\
13.7725	-0.156363062230761\\
13.7780555555556	-0.157886442926867\\
13.7836111111111	-0.159202523754086\\
13.7891666666667	-0.159603388414024\\
13.7947222222222	-0.159838572223547\\
13.8002777777778	-0.158888020920645\\
13.8058333333333	-0.155597588040171\\
13.8113888888889	-0.155026568571759\\
13.8169444444444	-0.15667734037023\\
13.8225	-0.152237616719064\\
13.8280555555556	-0.148514347372844\\
13.8336111111111	-0.150177461745029\\
13.8391666666667	-0.151984452916304\\
13.8447222222222	-0.15217348226185\\
13.8502777777778	-0.153796927218167\\
13.8558333333333	-0.154646066088432\\
13.8613888888889	-0.153335556499433\\
13.8669444444444	-0.152734613092994\\
13.8725	-0.150623670180302\\
13.8780555555556	-0.151189413313782\\
13.8836111111111	-0.150312702015814\\
13.8891666666667	-0.150721977622695\\
13.8947222222222	-0.147940368229879\\
13.9002777777778	-0.151835857134937\\
13.9058333333333	-0.151204562427669\\
13.9113888888889	-0.151580060192666\\
13.9169444444444	-0.150928367837625\\
13.9225	-0.146144308456521\\
13.9280555555556	-0.146008681977751\\
13.9336111111111	-0.147558659656682\\
13.9391666666667	-0.145859471993514\\
13.9447222222222	-0.145599714919657\\
13.9502777777778	-0.145799951852815\\
13.9558333333333	-0.149670366637128\\
13.9613888888889	-0.150222198951735\\
13.9669444444444	-0.151622710604413\\
13.9725	-0.152111510968037\\
13.9780555555556	-0.149489888216822\\
13.9836111111111	-0.15067940191629\\
13.9891666666667	-0.150222987456912\\
13.9947222222222	-0.149574253283747\\
14.0002777777778	-0.148239785866195\\
14.0058333333333	-0.148227774957582\\
14.0113888888889	-0.15023794550536\\
14.0169444444444	-0.148285881030627\\
14.0225	-0.154460232980095\\
14.0280555555556	-0.155838461265219\\
14.0336111111111	-0.159345335590193\\
14.0391666666667	-0.162394485759326\\
14.0447222222222	-0.160999833497982\\
14.0502777777778	-0.158778512764936\\
14.0558333333333	-0.161405915737731\\
14.0613888888889	-0.162138043131431\\
14.0669444444444	-0.160406198011316\\
14.0725	-0.161376875865283\\
14.0780555555556	-0.160889877808485\\
14.0836111111111	-0.162143180071023\\
14.0891666666667	-0.161291545507868\\
14.0947222222222	-0.160568370537323\\
14.1002777777778	-0.159522468024666\\
14.1058333333333	-0.157602431585714\\
14.1113888888889	-0.159492884942732\\
14.1169444444444	-0.157373307102088\\
14.1225	-0.15848528014214\\
14.1280555555556	-0.160573266216164\\
14.1336111111111	-0.159239270846917\\
14.1391666666667	-0.159867569117355\\
14.1447222222222	-0.160661686554866\\
14.1502777777778	-0.159023895709305\\
14.1558333333333	-0.157378991397026\\
14.1613888888889	-0.155006559238285\\
14.1669444444444	-0.155204887614849\\
14.1725	-0.153300031669377\\
14.1780555555556	-0.153466316981736\\
14.1836111111111	-0.154291394272534\\
14.1891666666667	-0.154662433703827\\
14.1947222222222	-0.150805708416715\\
14.2002777777778	-0.149237042823998\\
14.2058333333333	-0.149804919941427\\
14.2113888888889	-0.14910784251587\\
14.2169444444444	-0.149293679160963\\
14.2225	-0.14772908977235\\
14.2280555555556	-0.149171040948553\\
14.2336111111111	-0.150944442173363\\
14.2391666666667	-0.153852188845453\\
14.2447222222222	-0.151498620742684\\
14.2502777777778	-0.15035626424874\\
14.2558333333333	-0.149781406019633\\
14.2613888888889	-0.148568152903093\\
14.2669444444444	-0.148524522592587\\
14.2725	-0.149999184229724\\
14.2780555555556	-0.149759249982133\\
14.2836111111111	-0.151200658566489\\
14.2891666666667	-0.153806954928957\\
14.2947222222222	-0.154322955778905\\
14.3002777777778	-0.156273995723232\\
14.3058333333333	-0.15575044201747\\
14.3113888888889	-0.155632770149478\\
14.3169444444444	-0.156490273159661\\
14.3225	-0.158095408980748\\
14.3280555555556	-0.160070287516188\\
14.3336111111111	-0.161101071041724\\
14.3391666666667	-0.158249945470079\\
14.3447222222222	-0.158712741154286\\
14.3502777777778	-0.157138058039962\\
14.3558333333333	-0.15809700622387\\
14.3613888888889	-0.159449176872294\\
14.3669444444444	-0.16029125249979\\
14.3725	-0.158474647227542\\
14.3780555555556	-0.157404588336435\\
14.3836111111111	-0.159494404211078\\
14.3891666666667	-0.160794807161261\\
14.3947222222222	-0.163099349913155\\
14.4002777777778	-0.162960492831834\\
14.4058333333333	-0.161248487566896\\
14.4113888888889	-0.16271521708075\\
14.4169444444444	-0.161359870018579\\
14.4225	-0.161435332949893\\
14.4280555555556	-0.159939830785154\\
14.4336111111111	-0.160008211300861\\
14.4391666666667	-0.159596073964239\\
14.4447222222222	-0.161894334944683\\
14.4502777777778	-0.165586848439394\\
14.4558333333333	-0.166163091145424\\
14.4613888888889	-0.167912059029226\\
14.4669444444444	-0.168239956539033\\
14.4725	-0.167768633553016\\
14.4780555555556	-0.167940330031699\\
14.4836111111111	-0.169004182500616\\
14.4891666666667	-0.171535010794443\\
14.4947222222222	-0.171713076173872\\
14.5002777777778	-0.169635770051238\\
14.5058333333333	-0.169663043653406\\
14.5113888888889	-0.168251707534511\\
14.5169444444444	-0.168289314839182\\
14.5225	-0.167908010208899\\
14.5280555555556	-0.168641015288375\\
14.5336111111111	-0.168364368721416\\
14.5391666666667	-0.171033357540556\\
14.5447222222222	-0.172149531563461\\
14.5502777777778	-0.173723827862547\\
14.5558333333333	-0.175091909990964\\
14.5613888888889	-0.176474858698756\\
14.5669444444444	-0.177939226256317\\
14.5725	-0.176259213102806\\
14.5780555555556	-0.178832377254485\\
14.5836111111111	-0.178818014899946\\
14.5891666666667	-0.17958235276431\\
14.5947222222222	-0.180400739802587\\
14.6002777777778	-0.176607080977322\\
14.6058333333333	-0.173862290732665\\
14.6113888888889	-0.178010204765109\\
14.6169444444444	-0.173799489923733\\
14.6225	-0.17362502209778\\
14.6280555555556	-0.173043662663117\\
14.6336111111111	-0.172964564307133\\
14.6391666666667	-0.173916827325811\\
14.6447222222222	-0.175394488790379\\
14.6502777777778	-0.17674594466643\\
14.6558333333333	-0.174499053265281\\
14.6613888888889	-0.171643751090755\\
14.6669444444444	-0.171218648899038\\
14.6725	-0.169301844462502\\
14.6780555555556	-0.171094814447441\\
14.6836111111111	-0.170286014089364\\
14.6891666666667	-0.172606879373977\\
14.6947222222222	-0.174693783349955\\
14.7002777777778	-0.175051177966466\\
14.7058333333333	-0.172924807088759\\
14.7113888888889	-0.173593889560062\\
14.7169444444444	-0.17373833362334\\
14.7225	-0.174903826451452\\
14.7280555555556	-0.174030968397789\\
14.7336111111111	-0.173081039342109\\
14.7391666666667	-0.17604708716454\\
14.7447222222222	-0.17563813091112\\
14.7502777777778	-0.176261749471364\\
14.7558333333333	-0.171778295963193\\
14.7613888888889	-0.168664358974006\\
14.7669444444444	-0.166121052931058\\
14.7725	-0.166568603931849\\
14.7780555555556	-0.164788829369158\\
14.7836111111111	-0.165624682294733\\
14.7891666666667	-0.1644066529096\\
14.7947222222222	-0.16391345064098\\
14.8002777777778	-0.164203028394467\\
14.8058333333333	-0.165454353787923\\
14.8113888888889	-0.164834422174705\\
14.8169444444444	-0.164456903123103\\
14.8225	-0.164128282794803\\
14.8280555555556	-0.168390386901256\\
14.8336111111111	-0.16842057101768\\
14.8391666666667	-0.169635221384351\\
14.8447222222222	-0.171344114443625\\
14.8502777777778	-0.173103497734731\\
14.8558333333333	-0.174745173923675\\
14.8613888888889	-0.176701327552562\\
14.8669444444444	-0.175627360826081\\
14.8725	-0.174503418943873\\
14.8780555555556	-0.178207516948893\\
14.8836111111111	-0.178355213747593\\
14.8891666666667	-0.174866945582856\\
14.8947222222222	-0.179513170443241\\
14.9002777777778	-0.180314395456609\\
14.9058333333333	-0.181038948620689\\
14.9113888888889	-0.181721849181482\\
14.9169444444444	-0.179828417368482\\
14.9225	-0.176429170308682\\
14.9280555555556	-0.173595640146443\\
14.9336111111111	-0.171344040807709\\
14.9391666666667	-0.170583282185394\\
14.9447222222222	-0.169080480622234\\
14.9502777777778	-0.17062514885651\\
14.9558333333333	-0.172759571481964\\
14.9613888888889	-0.172487156378154\\
14.9669444444444	-0.172801440578899\\
14.9725	-0.175597289268916\\
14.9780555555556	-0.172347671220028\\
14.9836111111111	-0.173413574767271\\
14.9891666666667	-0.17718991286377\\
14.9947222222222	-0.180140695938207\\
15.0002777777778	-0.183153020440875\\
15.0058333333333	-0.181483687848137\\
15.0113888888889	-0.182303309994694\\
15.0169444444444	-0.182808032176007\\
15.0225	-0.183325451996357\\
15.0280555555556	-0.184237361666382\\
15.0336111111111	-0.18529120363337\\
15.0391666666667	-0.186841546465294\\
15.0447222222222	-0.189118267609255\\
15.0502777777778	-0.188613920736752\\
15.0558333333333	-0.187686163177451\\
15.0613888888889	-0.190094521804349\\
15.0669444444444	-0.190938665110683\\
15.0725	-0.192167510776452\\
15.0780555555556	-0.194457379342335\\
15.0836111111111	-0.194014566396938\\
15.0891666666667	-0.19530894845469\\
15.0947222222222	-0.195917165073449\\
15.1002777777778	-0.196423769214672\\
15.1058333333333	-0.196268735936369\\
15.1113888888889	-0.196437257370036\\
15.1169444444444	-0.198454697849014\\
15.1225	-0.198738992932031\\
15.1280555555556	-0.199619764727711\\
15.1336111111111	-0.20078759638909\\
15.1391666666667	-0.20354650927369\\
15.1447222222222	-0.206204335876604\\
15.1502777777778	-0.206900938570452\\
15.1558333333333	-0.211725499955004\\
15.1613888888889	-0.207204824074233\\
15.1669444444444	-0.206092224925429\\
15.1725	-0.206500331937849\\
15.1780555555556	-0.200310649812674\\
15.1836111111111	-0.201981043095278\\
15.1891666666667	-0.199908941881634\\
15.1947222222222	-0.199849936806755\\
15.2002777777778	-0.198446579539425\\
15.2058333333333	-0.198078860704843\\
15.2113888888889	-0.199104798174268\\
15.2169444444444	-0.201063320937603\\
15.2225	-0.1994028054906\\
15.2280555555556	-0.197249591358023\\
15.2336111111111	-0.197451542282779\\
15.2391666666667	-0.19490112851561\\
15.2447222222222	-0.193982620848816\\
15.2502777777778	-0.19182547550869\\
15.2558333333333	-0.188133246760277\\
15.2613888888889	-0.186322276953689\\
15.2669444444444	-0.188909583767761\\
15.2725	-0.188670410320484\\
15.2780555555556	-0.187524229847074\\
15.2836111111111	-0.188210384118699\\
15.2891666666667	-0.187270395534714\\
15.2947222222222	-0.188985390418382\\
15.3002777777778	-0.188072735827703\\
15.3058333333333	-0.190227390071534\\
15.3113888888889	-0.194347953503252\\
15.3169444444444	-0.192531441740897\\
15.3225	-0.190408649122822\\
15.3280555555556	-0.194396341614163\\
15.3336111111111	-0.195161176638121\\
15.3391666666667	-0.194427774494864\\
15.3447222222222	-0.191778768392761\\
15.3502777777778	-0.192144640564959\\
15.3558333333333	-0.190632990465392\\
15.3613888888889	-0.189407302946611\\
15.3669444444444	-0.190222574360723\\
15.3725	-0.18927677190594\\
15.3780555555556	-0.193199387877576\\
15.3836111111111	-0.192844915769851\\
15.3891666666667	-0.194921049879454\\
15.3947222222222	-0.196386085743487\\
15.4002777777778	-0.194210009059984\\
15.4058333333333	-0.193533991000536\\
15.4113888888889	-0.194903510274284\\
15.4169444444444	-0.193882612058337\\
15.4225	-0.195003325361074\\
15.4280555555556	-0.193692955200649\\
15.4336111111111	-0.192530503231873\\
15.4391666666667	-0.192286641543225\\
15.4447222222222	-0.191601324214172\\
15.4502777777778	-0.193244779303189\\
15.4558333333333	-0.197616373230751\\
15.4613888888889	-0.197418587292125\\
15.4669444444444	-0.19799779803367\\
15.4725	-0.198390290253803\\
15.4780555555556	-0.196445644054316\\
15.4836111111111	-0.194629261955509\\
15.4891666666667	-0.191179705388779\\
15.4947222222222	-0.19082111942662\\
15.5002777777778	-0.19137095457087\\
15.5058333333333	-0.193066181911304\\
15.5113888888889	-0.192848295800224\\
15.5169444444444	-0.187904692598861\\
15.5225	-0.187977182444586\\
15.5280555555556	-0.190200322049715\\
15.5336111111111	-0.193552951267746\\
15.5391666666667	-0.19196446534505\\
15.5447222222222	-0.192744388882421\\
15.5502777777778	-0.193901517113607\\
15.5558333333333	-0.190903696415371\\
15.5613888888889	-0.189225781346109\\
15.5669444444444	-0.189955511578426\\
15.5725	-0.190894525411696\\
15.5780555555556	-0.186380825019212\\
15.5836111111111	-0.186671869270863\\
15.5891666666667	-0.185413494802134\\
15.5947222222222	-0.184890125219703\\
15.6002777777778	-0.188224264362089\\
15.6058333333333	-0.190863477515436\\
15.6113888888889	-0.193444248416727\\
15.6169444444444	-0.19301565612411\\
15.6225	-0.193518200533224\\
15.6280555555556	-0.194420968428683\\
15.6336111111111	-0.195375057167821\\
15.6391666666667	-0.197641917493632\\
15.6447222222222	-0.198889911452819\\
15.6502777777778	-0.197558093944269\\
15.6558333333333	-0.201924428643857\\
15.6613888888889	-0.202787356030967\\
15.6669444444444	-0.202784793954155\\
15.6725	-0.199669617669989\\
15.6780555555556	-0.202165790046143\\
15.6836111111111	-0.205102650330324\\
15.6891666666667	-0.206332081011665\\
15.6947222222222	-0.210921124674126\\
15.7002777777778	-0.207069064767005\\
15.7058333333333	-0.207902539692438\\
15.7113888888889	-0.205631220735976\\
15.7169444444444	-0.204239223946097\\
15.7225	-0.204442557670955\\
15.7280555555556	-0.202294945979579\\
15.7336111111111	-0.201473141863368\\
15.7391666666667	-0.201003024662546\\
15.7447222222222	-0.20072899241301\\
15.7502777777778	-0.207765897371868\\
15.7558333333333	-0.202933356264896\\
15.7613888888889	-0.200347628995905\\
15.7669444444444	-0.19800463034755\\
15.7725	-0.195174773812824\\
15.7780555555556	-0.194087400999442\\
15.7836111111111	-0.19459451432522\\
15.7891666666667	-0.19275057373427\\
15.7947222222222	-0.19046503417938\\
15.8002777777778	-0.193588714285267\\
15.8058333333333	-0.190890188327144\\
15.8113888888889	-0.190656449815612\\
15.8169444444444	-0.193307825090433\\
15.8225	-0.197102810955097\\
15.8280555555556	-0.192807178448234\\
15.8336111111111	-0.187732918836586\\
15.8391666666667	-0.185014970246499\\
15.8447222222222	-0.185290008348212\\
15.8502777777778	-0.185850814699448\\
15.8558333333333	-0.185081872373909\\
15.8613888888889	-0.186547976377608\\
15.8669444444444	-0.184537104014166\\
15.8725	-0.184470446078751\\
15.8780555555556	-0.184372661293074\\
15.8836111111111	-0.186348255752722\\
15.8891666666667	-0.187853953257029\\
15.8947222222222	-0.188779167070109\\
15.9002777777778	-0.191827499935502\\
15.9058333333333	-0.189767280183738\\
15.9113888888889	-0.191591710787292\\
15.9169444444444	-0.190123854472384\\
15.9225	-0.187673558924666\\
15.9280555555556	-0.18899030240876\\
15.9336111111111	-0.185946317008632\\
15.9391666666667	-0.187702745436098\\
15.9447222222222	-0.190848139087879\\
15.9502777777778	-0.193325611789995\\
15.9558333333333	-0.194936524893615\\
15.9613888888889	-0.192277302349609\\
15.9669444444444	-0.191723735001976\\
15.9725	-0.192274400821446\\
15.9780555555556	-0.192835766239952\\
15.9836111111111	-0.192982231315416\\
15.9891666666667	-0.193893485818713\\
15.9947222222222	-0.186789440864256\\
};
\addplot [color=mycolor2,solid,line width=1.5pt,forget plot]
  table[row sep=crcr]{%
9.50027777777778	0\\
9.50583333333333	0.000329015641255178\\
9.51138888888889	0.00257017442617689\\
9.51694444444444	0.00578667431124095\\
9.5225	0.00842433490172277\\
9.52805555555556	0.0106036082334963\\
9.53361111111111	0.0142444890713512\\
9.53916666666667	0.0164719269116188\\
9.54472222222222	0.0175176519890628\\
9.55027777777778	0.0176408784833626\\
9.55583333333333	0.02056588109653\\
9.56138888888889	0.0247391543879242\\
9.56694444444444	0.0254266066362544\\
9.5725	0.025301045031543\\
9.57805555555555	0.0266778910444559\\
9.58361111111111	0.0273945171206416\\
9.58916666666667	0.0299695044785133\\
9.59472222222222	0.0318234949337429\\
9.60027777777778	0.0340041075679457\\
9.60583333333333	0.0344385949370029\\
9.61138888888889	0.0354452841074528\\
9.61694444444444	0.0371720604595089\\
9.6225	0.0391300474751823\\
9.62805555555556	0.0388992839557059\\
9.63361111111111	0.0353025092998824\\
9.63916666666667	0.0345967984127139\\
9.64472222222222	0.0348726159177081\\
9.65027777777778	0.0372630353902093\\
9.65583333333333	0.0395246689715001\\
9.66138888888889	0.0388207833127991\\
9.66694444444444	0.0385395851322645\\
9.6725	0.0398413608004226\\
9.67805555555555	0.0415990993392467\\
9.68361111111111	0.0467961274169751\\
9.68916666666667	0.0490127141638141\\
9.69472222222222	0.051069868622809\\
9.70027777777778	0.0486114913718941\\
9.70583333333333	0.0484533773516236\\
9.71138888888889	0.051624178071207\\
9.71694444444444	0.0547122190629005\\
9.7225	0.0562039198083961\\
9.72805555555555	0.0588902904854832\\
9.73361111111111	0.0625875018445171\\
9.73916666666667	0.0608869494483412\\
9.74472222222222	0.0628717740013179\\
9.75027777777778	0.0637840746257529\\
9.75583333333333	0.0638479039097664\\
9.76138888888889	0.0618417715158735\\
9.76694444444444	0.0656182498794643\\
9.7725	0.069997702611185\\
9.77805555555556	0.0685478878249143\\
9.78361111111111	0.0670977224093061\\
9.78916666666667	0.0668831640660315\\
9.79472222222222	0.0660288602624843\\
9.80027777777778	0.0646647854706413\\
9.80583333333333	0.064906650090441\\
9.81138888888889	0.067532468449622\\
9.81694444444444	0.0678136303504785\\
9.8225	0.0720420018505097\\
9.82805555555555	0.0738830722493517\\
9.83361111111111	0.0714965969208032\\
9.83916666666667	0.0747930826313531\\
9.84472222222222	0.0730710898508179\\
9.85027777777778	0.0726860670576064\\
9.85583333333333	0.072350874028252\\
9.86138888888889	0.0763252075666825\\
9.86694444444444	0.0759363212701392\\
9.8725	0.078539951665071\\
9.87805555555556	0.0787902962003642\\
9.88361111111111	0.0821060479928378\\
9.88916666666667	0.0858949990075402\\
9.89472222222222	0.0861697055171553\\
9.90027777777778	0.0822828959805295\\
9.90583333333333	0.0843776771841574\\
9.91138888888889	0.0863795474807867\\
9.91694444444444	0.0915060153816252\\
9.9225	0.0928504151446766\\
9.92805555555555	0.0949043787177153\\
9.93361111111111	0.0961110469375589\\
9.93916666666667	0.0973079619088287\\
9.94472222222222	0.0977313711044731\\
9.95027777777778	0.0985294344399043\\
9.95583333333333	0.0997835818337746\\
9.96138888888889	0.104877641925109\\
9.96694444444444	0.109620759467202\\
9.9725	0.109163269293429\\
9.97805555555555	0.113265497921399\\
9.98361111111111	0.115173925359867\\
9.98916666666667	0.117288266578753\\
9.99472222222222	0.117297821273647\\
10.0002777777778	0.117161914187772\\
10.0058333333333	0.119214401987172\\
10.0113888888889	0.112204994629047\\
10.0169444444444	0.118737080920806\\
10.0225	0.115778345212171\\
10.0280555555556	0.117808247207768\\
10.0336111111111	0.117869249572399\\
10.0391666666667	0.116393035004137\\
10.0447222222222	0.115787937592874\\
10.0502777777778	0.123137539331783\\
10.0558333333333	0.128236438333944\\
10.0613888888889	0.127935758361724\\
10.0669444444444	0.130089651367775\\
10.0725	0.130435615302979\\
10.0780555555556	0.125598238006075\\
10.0836111111111	0.127585455770583\\
10.0891666666667	0.129753030120384\\
10.0947222222222	0.132108339444936\\
10.1002777777778	0.128881762410743\\
10.1058333333333	0.128156397816546\\
10.1113888888889	0.128472897444344\\
10.1169444444444	0.12851409091176\\
10.1225	0.134932142393741\\
10.1280555555556	0.134651269680387\\
10.1336111111111	0.143225409447082\\
10.1391666666667	0.147603266942041\\
10.1447222222222	0.145764616543534\\
10.1502777777778	0.14759047080017\\
10.1558333333333	0.151087722956162\\
10.1613888888889	0.152799984897974\\
10.1669444444444	0.144073297593961\\
10.1725	0.143410275298855\\
10.1780555555556	0.14077057892279\\
10.1836111111111	0.133435084010206\\
10.1891666666667	0.138874738479301\\
10.1947222222222	0.139451288944275\\
10.2002777777778	0.143713517706848\\
10.2058333333333	0.145129775595011\\
10.2113888888889	0.144107316897042\\
10.2169444444444	0.143881739784181\\
10.2225	0.141842536335266\\
10.2280555555556	0.144350324133859\\
10.2336111111111	0.139614463187846\\
10.2391666666667	0.135146958754013\\
10.2447222222222	0.138412252926843\\
10.2502777777778	0.140049138551101\\
10.2558333333333	0.1419589202511\\
10.2613888888889	0.148516635321235\\
10.2669444444444	0.14489496520892\\
10.2725	0.14747230500918\\
10.2780555555556	0.146593193487318\\
10.2836111111111	0.145828660386757\\
10.2891666666667	0.14194024463799\\
10.2947222222222	0.142375102725479\\
10.3002777777778	0.140225365510105\\
10.3058333333333	0.140931570950832\\
10.3113888888889	0.142381554640307\\
10.3169444444444	0.143880154329895\\
10.3225	0.144475074609363\\
10.3280555555556	0.143640766369198\\
10.3336111111111	0.146640328384987\\
10.3391666666667	0.145260500518236\\
10.3447222222222	0.146423484515063\\
10.3502777777778	0.150247381391947\\
10.3558333333333	0.150518918366009\\
10.3613888888889	0.154144418787511\\
10.3669444444444	0.154821781095593\\
10.3725	0.155332493465281\\
10.3780555555556	0.152895908381357\\
10.3836111111111	0.153463331280502\\
10.3891666666667	0.1533148997648\\
10.3947222222222	0.14981672017621\\
10.4002777777778	0.144936518168968\\
10.4058333333333	0.146297892187059\\
10.4113888888889	0.146030268642182\\
10.4169444444444	0.146273516351326\\
10.4225	0.142478437103178\\
10.4280555555556	0.140537745985164\\
10.4336111111111	0.146866590801174\\
10.4391666666667	0.150341506023604\\
10.4447222222222	0.147161258751988\\
10.4502777777778	0.151105260640095\\
10.4558333333333	0.147507308757913\\
10.4613888888889	0.151071208706896\\
10.4669444444444	0.15600225373398\\
10.4725	0.156116047895291\\
10.4780555555556	0.161158467442147\\
10.4836111111111	0.166228741407988\\
10.4891666666667	0.164555555104411\\
10.4947222222222	0.167265658912683\\
10.5002777777778	0.168929241789412\\
10.5058333333333	0.164614256916008\\
10.5113888888889	0.170393132404755\\
10.5169444444444	0.163617170225956\\
10.5225	0.163048787220773\\
10.5280555555556	0.165959552159431\\
10.5336111111111	0.171106829992674\\
10.5391666666667	0.173547023374039\\
10.5447222222222	0.17337381454536\\
10.5502777777778	0.16849261427427\\
10.5558333333333	0.169933520520561\\
10.5613888888889	0.176089396801411\\
10.5669444444444	0.170658330062949\\
10.5725	0.171758182463661\\
10.5780555555556	0.171477056066538\\
10.5836111111111	0.169640558247021\\
10.5891666666667	0.16631230530875\\
10.5947222222222	0.170787645892277\\
10.6002777777778	0.175582905504465\\
10.6058333333333	0.173345868091685\\
10.6113888888889	0.174895430212131\\
10.6169444444444	0.17274113211152\\
10.6225	0.17840182177907\\
10.6280555555556	0.185371018073053\\
10.6336111111111	0.183333839188842\\
10.6391666666667	0.187430769587738\\
10.6447222222222	0.191202314109511\\
10.6502777777778	0.190435002443516\\
10.6558333333333	0.19438799077188\\
10.6613888888889	0.191198829477415\\
10.6669444444444	0.191917089889609\\
10.6725	0.194642842487585\\
10.6780555555556	0.194221331578392\\
10.6836111111111	0.202380026007658\\
10.6891666666667	0.196481882994903\\
10.6947222222222	0.19403569536358\\
10.7002777777778	0.196229212885095\\
10.7058333333333	0.197691608659656\\
10.7113888888889	0.199568985604901\\
10.7169444444444	0.206164174433208\\
10.7225	0.208287351238898\\
10.7280555555556	0.20391796678835\\
10.7336111111111	0.200978956653008\\
10.7391666666667	0.198491239862056\\
10.7447222222222	0.199852368460781\\
10.7502777777778	0.206498391229128\\
10.7558333333333	0.203989990576249\\
10.7613888888889	0.207808244223111\\
10.7669444444444	0.207717094105458\\
10.7725	0.211368796434469\\
10.7780555555556	0.209764352236948\\
10.7836111111111	0.21303673081975\\
10.7891666666667	0.212593337041259\\
10.7947222222222	0.213799646218419\\
10.8002777777778	0.205726081002085\\
10.8058333333333	0.210971456365835\\
10.8113888888889	0.211162821529736\\
10.8169444444444	0.21427039556591\\
10.8225	0.214247521760283\\
10.8280555555556	0.218255867323064\\
10.8336111111111	0.214604424320036\\
10.8391666666667	0.217121034046099\\
10.8447222222222	0.216467170579268\\
10.8502777777778	0.211300655414735\\
10.8558333333333	0.217119613617305\\
10.8613888888889	0.217997450384208\\
10.8669444444444	0.221725131345959\\
10.8725	0.21828123688801\\
10.8780555555556	0.220021866051959\\
10.8836111111111	0.217649030587346\\
10.8891666666667	0.22138173727451\\
10.8947222222222	0.220538077239227\\
10.9002777777778	0.222204629766237\\
10.9058333333333	0.220780636570172\\
10.9113888888889	0.22066080783213\\
10.9169444444444	0.219722979939683\\
10.9225	0.21637141579057\\
10.9280555555556	0.217166377375963\\
10.9336111111111	0.216699766199988\\
10.9391666666667	0.219740331169905\\
10.9447222222222	0.218303960908542\\
10.9502777777778	0.217542836416348\\
10.9558333333333	0.21772838857515\\
10.9613888888889	0.214617139159251\\
10.9669444444444	0.219508499484327\\
10.9725	0.222902615979744\\
10.9780555555556	0.222721682586927\\
10.9836111111111	0.226783571179106\\
10.9891666666667	0.230698229822984\\
10.9947222222222	0.229521898386693\\
11.0002777777778	0.229676071075408\\
11.0058333333333	0.227141476595301\\
11.0113888888889	0.225040762651659\\
11.0169444444444	0.228053796030124\\
11.0225	0.22715305746426\\
11.0280555555556	0.229612508433242\\
11.0336111111111	0.230133980742574\\
11.0391666666667	0.225779259829644\\
11.0447222222222	0.226398152348876\\
11.0502777777778	0.231004617433645\\
11.0558333333333	0.231084675713322\\
11.0613888888889	0.231811857439694\\
11.0669444444444	0.233579294597419\\
11.0725	0.232674497375087\\
11.0780555555556	0.238000687108168\\
11.0836111111111	0.242325652640972\\
11.0891666666667	0.237425508877632\\
11.0947222222222	0.236903050530853\\
11.1002777777778	0.235957221078242\\
11.1058333333333	0.232073414377574\\
11.1113888888889	0.234092505604139\\
11.1169444444444	0.241437711003819\\
11.1225	0.246326443127865\\
11.1280555555556	0.249381164355223\\
11.1336111111111	0.244855177858412\\
11.1391666666667	0.244751424480828\\
11.1447222222222	0.247057821322306\\
11.1502777777778	0.245320957729843\\
11.1558333333333	0.245528362281431\\
11.1613888888889	0.245653303975004\\
11.1669444444444	0.243086252946185\\
11.1725	0.245879291927186\\
11.1780555555556	0.239241247703714\\
11.1836111111111	0.237134332953469\\
11.1891666666667	0.239959973257611\\
11.1947222222222	0.240689769651963\\
11.2002777777778	0.240839412031142\\
11.2058333333333	0.243014813684169\\
11.2113888888889	0.244254090659321\\
11.2169444444444	0.245809419184291\\
11.2225	0.248644361769525\\
11.2280555555556	0.246153330722286\\
11.2336111111111	0.2475370403336\\
11.2391666666667	0.248386881377203\\
11.2447222222222	0.249069524172466\\
11.2502777777778	0.254438580189018\\
11.2558333333333	0.257510368659352\\
11.2613888888889	0.258982384361699\\
11.2669444444444	0.26371149629867\\
11.2725	0.233779775845602\\
11.2780555555556	0.215766762604722\\
11.2836111111111	0.229441072621399\\
11.2891666666667	0.220245142215405\\
11.2947222222222	0.228126018451128\\
11.3002777777778	0.222776352460394\\
11.3058333333333	0.246957519286486\\
11.3113888888889	0.249034423510464\\
11.3169444444444	0.243931080866859\\
11.3225	0.236869942917714\\
11.3280555555556	0.243221171648257\\
11.3336111111111	0.241979005175439\\
11.3391666666667	0.234115082434695\\
11.3447222222222	0.227172423979573\\
11.3502777777778	0.216017827861943\\
11.3558333333333	0.214362599261012\\
11.3613888888889	0.217016704978014\\
11.3669444444444	0.212881780218419\\
11.3725	0.206934242163542\\
11.3780555555556	0.206862712755677\\
11.3836111111111	0.199963263561382\\
11.3891666666667	0.200024402711787\\
11.3947222222222	0.205057872424666\\
11.4002777777778	0.205863790693608\\
11.4058333333333	0.205402525996948\\
11.4113888888889	0.205257356728394\\
11.4169444444444	0.20157839670827\\
11.4225	0.205681612936214\\
11.4280555555556	0.2123737027711\\
11.4336111111111	0.205403892034112\\
11.4391666666667	0.203063192039555\\
11.4447222222222	0.20537532473029\\
11.4502777777778	0.2056288953574\\
11.4558333333333	0.21132297667133\\
11.4613888888889	0.202155486581463\\
11.4669444444444	0.20527806942923\\
11.4725	0.199810827426694\\
11.4780555555556	0.19070408306388\\
11.4836111111111	0.18624675367087\\
11.4891666666667	0.185422300419279\\
11.4947222222222	0.185018363143793\\
11.5002777777778	0.182796831915972\\
11.5058333333333	0.177781796232543\\
11.5113888888889	0.176465525771423\\
11.5169444444444	0.176100432519729\\
11.5225	0.174129326004792\\
11.5280555555556	0.166969420324771\\
11.5336111111111	0.168443385929502\\
11.5391666666667	0.169420863004532\\
11.5447222222222	0.158520231984141\\
11.5502777777778	0.164226074320826\\
11.5558333333333	0.162591707403101\\
11.5613888888889	0.162552767541645\\
11.5669444444444	0.15336975046709\\
11.5725	0.151334185146526\\
11.5780555555556	0.142674075024434\\
11.5836111111111	0.153602680703869\\
11.5891666666667	0.1429755737876\\
11.5947222222222	0.135434625096621\\
11.6002777777778	0.145588322476972\\
11.6058333333333	0.14308070441934\\
11.6113888888889	0.127497123119162\\
11.6169444444444	0.110692682700013\\
11.6225	0.121065310199995\\
11.6280555555556	0.122770743778601\\
11.6336111111111	0.116279824810829\\
11.6391666666667	0.130970340963388\\
11.6447222222222	0.135716219576179\\
11.6502777777778	0.150393600627334\\
11.6558333333333	0.159368571243931\\
11.6613888888889	0.167772398196009\\
11.6669444444444	0.170918165657496\\
11.6725	0.179901626636573\\
11.6780555555556	0.181738243365078\\
11.6836111111111	0.190605455197793\\
11.6891666666667	0.191977179382759\\
11.6947222222222	0.197308886758299\\
11.7002777777778	0.196159430535474\\
11.7058333333333	0.190813443526352\\
11.7113888888889	0.186634142381954\\
11.7169444444444	0.18289612062199\\
11.7225	0.1866102704014\\
11.7280555555556	0.18480212209896\\
11.7336111111111	0.181827927049782\\
11.7391666666667	0.181736383819461\\
11.7447222222222	0.175540354566608\\
11.7502777777778	0.172647389858363\\
11.7558333333333	0.157812228005086\\
11.7613888888889	0.162581046173084\\
11.7669444444444	0.161039767054618\\
11.7725	0.164556750942648\\
11.7780555555556	0.154489846221173\\
11.7836111111111	0.154734121187192\\
11.7891666666667	0.153796339438576\\
11.7947222222222	0.143237099263055\\
11.8002777777778	0.147459730879917\\
11.8058333333333	0.150444366780278\\
11.8113888888889	0.15251677599999\\
11.8169444444444	0.147034323900019\\
11.8225	0.147040950224255\\
11.8280555555556	0.13981554935758\\
11.8336111111111	0.143516714064869\\
11.8391666666667	0.142085592789979\\
11.8447222222222	0.151698584157664\\
11.8502777777778	0.154823374561508\\
11.8558333333333	0.146785462854487\\
11.8613888888889	0.141607476600872\\
11.8669444444444	0.144474688013303\\
11.8725	0.144329473973922\\
11.8780555555556	0.140689709450001\\
11.8836111111111	0.137066102727846\\
11.8891666666667	0.134626398899536\\
11.8947222222222	0.134087118230735\\
11.9002777777778	0.133319089747682\\
11.9058333333333	0.137576286502976\\
11.9113888888889	0.141139743062094\\
11.9169444444444	0.133585464936206\\
11.9225	0.130972958288327\\
11.9280555555556	0.128149193209574\\
11.9336111111111	0.130281575271577\\
11.9391666666667	0.130953734833863\\
11.9447222222222	0.122731165220281\\
11.9502777777778	0.120637542636976\\
11.9558333333333	0.119793987100505\\
11.9613888888889	0.127163698694042\\
11.9669444444444	0.126214420866876\\
11.9725	0.128390906233147\\
11.9780555555556	0.1182027248398\\
11.9836111111111	0.136387950965549\\
11.9891666666667	0.138753140752839\\
11.9947222222222	0.136690976306416\\
12.0002777777778	0.145519964860534\\
12.0058333333333	0.145649192801929\\
12.0113888888889	0.146066264425944\\
12.0169444444444	0.145450571175158\\
12.0225	0.146409178075233\\
12.0280555555556	0.155370213787532\\
12.0336111111111	0.154484049808907\\
12.0391666666667	0.151942853028101\\
12.0447222222222	0.153776815964823\\
12.0502777777778	0.154150624915162\\
12.0558333333333	0.164353367639397\\
12.0613888888889	0.158729966104183\\
12.0669444444444	0.160843689940226\\
12.0725	0.161906301094198\\
12.0780555555556	0.169988370518392\\
12.0836111111111	0.169927272680826\\
12.0891666666667	0.173339069598543\\
12.0947222222222	0.176930905621874\\
12.1002777777778	0.174865604836542\\
12.1058333333333	0.173612908165673\\
12.1113888888889	0.168470633060649\\
12.1169444444444	0.160715594371516\\
12.1225	0.163635720467232\\
12.1280555555556	0.163167175406199\\
12.1336111111111	0.163598894324922\\
12.1391666666667	0.155930540920085\\
12.1447222222222	0.154822412595308\\
12.1502777777778	0.152410642695033\\
12.1558333333333	0.155816167843885\\
12.1613888888889	0.156886774516902\\
12.1669444444444	0.163918344210609\\
12.1725	0.167016312141561\\
12.1780555555556	0.170976188083135\\
12.1836111111111	0.183842675725847\\
12.1891666666667	0.192070469965786\\
12.1947222222222	0.184021649474921\\
12.2002777777778	0.182991631821238\\
12.2058333333333	0.185308272772761\\
12.2113888888889	0.184683677245171\\
12.2169444444444	0.178417808317674\\
12.2225	0.1795543206513\\
12.2280555555556	0.179911021333571\\
12.2336111111111	0.176187259607239\\
12.2391666666667	0.179570970512472\\
12.2447222222222	0.180687858400605\\
12.2502777777778	0.178395078878723\\
12.2558333333333	0.17630478354121\\
12.2613888888889	0.173200255470211\\
12.2669444444444	0.173136444057883\\
12.2725	0.175216935160257\\
12.2780555555556	0.17926692951973\\
12.2836111111111	0.18388271253324\\
12.2891666666667	0.182860302617521\\
12.2947222222222	0.182626242496533\\
12.3002777777778	0.175756859606795\\
12.3058333333333	0.175729972808105\\
12.3113888888889	0.17650093679498\\
12.3169444444444	0.172677178897322\\
12.3225	0.177680003576647\\
12.3280555555556	0.177491474041848\\
12.3336111111111	0.180829564492068\\
12.3391666666667	0.177257232520107\\
12.3447222222222	0.175559039868481\\
12.3502777777778	0.184703895425262\\
12.3558333333333	0.187430011278871\\
12.3613888888889	0.188849291381043\\
12.3669444444444	0.190407729546855\\
12.3725	0.186570907876243\\
12.3780555555556	0.193438764467988\\
12.3836111111111	0.192729723904357\\
12.3891666666667	0.193197589230237\\
12.3947222222222	0.189574825160608\\
12.4002777777778	0.183953962148486\\
12.4058333333333	0.17686638581864\\
12.4113888888889	0.183389451443826\\
12.4169444444444	0.180794018479247\\
12.4225	0.178396384996956\\
12.4280555555556	0.184668757288361\\
12.4336111111111	0.188967228794175\\
12.4391666666667	0.189763374173193\\
12.4447222222222	0.19898465413214\\
12.4502777777778	0.196336569940469\\
12.4558333333333	0.190021618562929\\
12.4613888888889	0.188495028942806\\
12.4669444444444	0.190787872682123\\
12.4725	0.190475712960592\\
12.4780555555556	0.195309472366488\\
12.4836111111111	0.194968552800064\\
12.4891666666667	0.185915814611057\\
12.4947222222222	0.191604811134541\\
12.5002777777778	0.183821295060531\\
12.5058333333333	0.181072817073923\\
12.5113888888889	0.181627099858904\\
12.5169444444444	0.18016411912825\\
12.5225	0.177065159681061\\
12.5280555555556	0.177166408837292\\
12.5336111111111	0.169071988919268\\
12.5391666666667	0.175935493098352\\
12.5447222222222	0.17273446705826\\
12.5502777777778	0.170206382713232\\
12.5558333333333	0.168915014938409\\
12.5613888888889	0.166373373778796\\
12.5669444444444	0.161209886604387\\
12.5725	0.162457672408236\\
12.5780555555556	0.164682630195825\\
12.5836111111111	0.163424064933153\\
12.5891666666667	0.17087252099185\\
12.5947222222222	0.167523683313038\\
12.6002777777778	0.177005924742125\\
12.6058333333333	0.172660216792451\\
12.6113888888889	0.179631814321731\\
12.6169444444444	0.186135716485144\\
12.6225	0.187017387637773\\
12.6280555555556	0.181463150557624\\
12.6336111111111	0.192625466910551\\
12.6391666666667	0.189406986493952\\
12.6447222222222	0.189497015525938\\
12.6502777777778	0.183619751607652\\
12.6558333333333	0.186849984253499\\
12.6613888888889	0.188111059894946\\
12.6669444444444	0.187634956184313\\
12.6725	0.187814824633936\\
12.6780555555556	0.183673674795764\\
12.6836111111111	0.186363731097036\\
12.6891666666667	0.182442743465402\\
12.6947222222222	0.172434208714101\\
12.7002777777778	0.177662727626359\\
12.7058333333333	0.173442627022036\\
12.7113888888889	0.173551532597269\\
12.7169444444444	0.174927750810227\\
12.7225	0.171311254342331\\
12.7280555555556	0.179151236817\\
12.7336111111111	0.179285384578574\\
12.7391666666667	0.179505685862668\\
12.7447222222222	0.179307075555682\\
12.7502777777778	0.187940076286936\\
12.7558333333333	0.187092497900142\\
12.7613888888889	0.187231348040115\\
12.7669444444444	0.190691017100644\\
12.7725	0.196504587725244\\
12.7780555555556	0.199617548234457\\
12.7836111111111	0.203775739864228\\
12.7891666666667	0.201962297977236\\
12.7947222222222	0.20351765589819\\
12.8002777777778	0.198520273496966\\
12.8058333333333	0.196971684716003\\
12.8113888888889	0.203840731625922\\
12.8169444444444	0.204724287687201\\
12.8225	0.203981373997578\\
12.8280555555556	0.20826964074213\\
12.8336111111111	0.202210716593375\\
12.8391666666667	0.204675381987175\\
12.8447222222222	0.201647221587663\\
12.8502777777778	0.205295882859121\\
12.8558333333333	0.200244355133937\\
12.8613888888889	0.195588051934917\\
12.8669444444444	0.199131993853847\\
12.8725	0.200696171325569\\
12.8780555555556	0.208279085968268\\
12.8836111111111	0.198530408875823\\
12.8891666666667	0.204158741704292\\
12.8947222222222	0.204411418011767\\
12.9002777777778	0.212067879235702\\
12.9058333333333	0.213332426284386\\
12.9113888888889	0.213897839260391\\
12.9169444444444	0.213570291316278\\
12.9225	0.21413955999382\\
12.9280555555556	0.211509207190917\\
12.9336111111111	0.211798745318307\\
12.9391666666667	0.213363119193653\\
12.9447222222222	0.214953285171955\\
12.9502777777778	0.216033062896514\\
12.9558333333333	0.213032566330813\\
12.9613888888889	0.214026118548535\\
12.9669444444444	0.21291442393205\\
12.9725	0.215648491958174\\
12.9780555555556	0.212117865502566\\
12.9836111111111	0.212733017940344\\
12.9891666666667	0.207081624911205\\
12.9947222222222	0.210513614800973\\
13.0002777777778	0.205878557072401\\
13.0058333333333	0.205006188717295\\
13.0113888888889	0.202054045843911\\
13.0169444444444	0.205833421205935\\
13.0225	0.208093805665752\\
13.0280555555556	0.211331698536198\\
13.0336111111111	0.213132689465\\
13.0391666666667	0.215405844634736\\
13.0447222222222	0.217501335386462\\
13.0502777777778	0.218859518106268\\
13.0558333333333	0.219082482427442\\
13.0613888888889	0.20873149849858\\
13.0669444444444	0.204393180053046\\
13.0725	0.200244779724135\\
13.0780555555556	0.218627159591464\\
13.0836111111111	0.216037230947358\\
13.0891666666667	0.213877412784818\\
13.0947222222222	0.213714742215255\\
13.1002777777778	0.214013088434267\\
13.1058333333333	0.217260411077212\\
13.1113888888889	0.219332803577843\\
13.1169444444444	0.22378706483448\\
13.1225	0.220397547839103\\
13.1280555555556	0.220204296787918\\
13.1336111111111	0.217843264414962\\
13.1391666666667	0.214643614118842\\
13.1447222222222	0.212275144607391\\
13.1502777777778	0.217165525759439\\
13.1558333333333	0.211753943511799\\
13.1613888888889	0.212514532858047\\
13.1669444444444	0.207005931199427\\
13.1725	0.200571769023785\\
13.1780555555556	0.199215681942498\\
13.1836111111111	0.203036920729405\\
13.1891666666667	0.204562478175031\\
13.1947222222222	0.206692089574343\\
13.2002777777778	0.204533775660366\\
13.2058333333333	0.208659236622293\\
13.2113888888889	0.208715570799436\\
13.2169444444444	0.205357420814088\\
13.2225	0.208775383666159\\
13.2280555555556	0.21514803730088\\
13.2336111111111	0.21785141374422\\
13.2391666666667	0.215855876524087\\
13.2447222222222	0.214487627616665\\
13.2502777777778	0.20885155069582\\
13.2558333333333	0.210756290203768\\
13.2613888888889	0.219646126132169\\
13.2669444444444	0.219497953051046\\
13.2725	0.223588931204435\\
13.2780555555556	0.21874467298809\\
13.2836111111111	0.21808157261468\\
13.2891666666667	0.220415636849508\\
13.2947222222222	0.217925298868578\\
13.3002777777778	0.214778284329061\\
13.3058333333333	0.213910184303117\\
13.3113888888889	0.210252867419995\\
13.3169444444444	0.21443072962283\\
13.3225	0.215207588379344\\
13.3280555555556	0.214893514837798\\
13.3336111111111	0.211706542602873\\
13.3391666666667	0.21740593071007\\
13.3447222222222	0.213436415613716\\
13.3502777777778	0.215636115764601\\
13.3558333333333	0.215666121960189\\
13.3613888888889	0.216295991069725\\
13.3669444444444	0.224319241958878\\
13.3725	0.234276022722042\\
13.3780555555556	0.231383306169721\\
13.3836111111111	0.23060079463188\\
13.3891666666667	0.234780780428468\\
13.3947222222222	0.232296359248174\\
13.4002777777778	0.233650448213532\\
13.4058333333333	0.234093390910826\\
13.4113888888889	0.234746997095801\\
13.4169444444444	0.237111403126371\\
13.4225	0.239699237837951\\
13.4280555555556	0.241852793344936\\
13.4336111111111	0.24239392301495\\
13.4391666666667	0.245736503484405\\
13.4447222222222	0.24869125540115\\
13.4502777777778	0.253682188129194\\
13.4558333333333	0.260669969731098\\
13.4613888888889	0.257200111885751\\
13.4669444444444	0.255726477362187\\
13.4725	0.256333905148868\\
13.4780555555556	0.257354788131088\\
13.4836111111111	0.251519229821248\\
13.4891666666667	0.251803418963675\\
13.4947222222222	0.246640458384435\\
13.5002777777778	0.25613738359753\\
13.5058333333333	0.254428108839413\\
13.5113888888889	0.251133299227741\\
13.5169444444444	0.247089577646223\\
13.5225	0.248053597716628\\
13.5280555555556	0.247410603171128\\
13.5336111111111	0.248540489423799\\
13.5391666666667	0.245157151279607\\
13.5447222222222	0.244671017042186\\
13.5502777777778	0.2378606456986\\
13.5558333333333	0.237132208084607\\
13.5613888888889	0.23771185448227\\
13.5669444444444	0.24225832007797\\
13.5725	0.244364465724306\\
13.5780555555556	0.249853307943355\\
13.5836111111111	0.247414836920256\\
13.5891666666667	0.245532803049587\\
13.5947222222222	0.249903134164346\\
13.6002777777778	0.252828594900655\\
13.6058333333333	0.249488019792352\\
13.6113888888889	0.252246909375174\\
13.6169444444444	0.250719253192831\\
13.6225	0.248105215805864\\
13.6280555555556	0.255109816417091\\
13.6336111111111	0.252337746738019\\
13.6391666666667	0.259338499636312\\
13.6447222222222	0.257667010140775\\
13.6502777777778	0.262315934673174\\
13.6558333333333	0.263400328250979\\
13.6613888888889	0.260425968608378\\
13.6669444444444	0.265120480234271\\
13.6725	0.263836950299433\\
13.6780555555556	0.266706217237627\\
13.6836111111111	0.270552242770415\\
13.6891666666667	0.268497609670967\\
13.6947222222222	0.272701127813606\\
13.7002777777778	0.273290949503731\\
13.7058333333333	0.271395599061009\\
13.7113888888889	0.272110126536675\\
13.7169444444444	0.276886463229045\\
13.7225	0.283166473273145\\
13.7280555555556	0.28306397885775\\
13.7336111111111	0.280123467997208\\
13.7391666666667	0.279469610683677\\
13.7447222222222	0.287616245866847\\
13.7502777777778	0.288786755950538\\
13.7558333333333	0.291620193577516\\
13.7613888888889	0.29718791455696\\
13.7669444444444	0.293673337704841\\
13.7725	0.290981686949483\\
13.7780555555556	0.285177176011531\\
13.7836111111111	0.288223582067156\\
13.7891666666667	0.28326952857711\\
13.7947222222222	0.281799889485033\\
13.8002777777778	0.281978486694631\\
13.8058333333333	0.270703489414492\\
13.8113888888889	0.266216354238957\\
13.8169444444444	0.253806463768361\\
13.8225	0.253698666326817\\
13.8280555555556	0.254633472084195\\
13.8336111111111	0.257813619456099\\
13.8391666666667	0.26745051764886\\
13.8447222222222	0.265373871259817\\
13.8502777777778	0.261251368513885\\
13.8558333333333	0.25953313994993\\
13.8613888888889	0.25452207433836\\
13.8669444444444	0.259268774945147\\
13.8725	0.254053432989199\\
13.8780555555556	0.254695580471121\\
13.8836111111111	0.254429033084238\\
13.8891666666667	0.247577023147859\\
13.8947222222222	0.245906792731723\\
13.9002777777778	0.24788267104495\\
13.9058333333333	0.247290091454717\\
13.9113888888889	0.253203940330238\\
13.9169444444444	0.253373612078134\\
13.9225	0.262301767263412\\
13.9280555555556	0.263026770396863\\
13.9336111111111	0.257286855942482\\
13.9391666666667	0.251720438006689\\
13.9447222222222	0.252620026115316\\
13.9502777777778	0.246469593713877\\
13.9558333333333	0.248446293829332\\
13.9613888888889	0.246044807963278\\
13.9669444444444	0.244404524534381\\
13.9725	0.241926321512052\\
13.9780555555556	0.229073489368029\\
13.9836111111111	0.224419580167607\\
13.9891666666667	0.221166005982183\\
13.9947222222222	0.214445438143396\\
14.0002777777778	0.226270033308124\\
14.0058333333333	0.23888615239915\\
14.0113888888889	0.251993559737911\\
14.0169444444444	0.258022885922463\\
14.0225	0.260615531926687\\
14.0280555555556	0.277053312985967\\
14.0336111111111	0.286562991734348\\
14.0391666666667	0.295195661254772\\
14.0447222222222	0.298301370218966\\
14.0502777777778	0.290368820279099\\
14.0558333333333	0.284194853562014\\
14.0613888888889	0.281986549409127\\
14.0669444444444	0.27831058607582\\
14.0725	0.285578696488021\\
14.0780555555556	0.292309180008651\\
14.0836111111111	0.279701664086916\\
14.0891666666667	0.266393814859112\\
14.0947222222222	0.267535612449985\\
14.1002777777778	0.264065245954443\\
14.1058333333333	0.267279621236528\\
14.1113888888889	0.261571703530114\\
14.1169444444444	0.263602597902482\\
14.1225	0.263896497786599\\
14.1280555555556	0.26697362457547\\
14.1336111111111	0.259173976757361\\
14.1391666666667	0.260276747994998\\
14.1447222222222	0.261159386975771\\
14.1502777777778	0.262451420623884\\
14.1558333333333	0.249437679884649\\
14.1613888888889	0.249140028106621\\
14.1669444444444	0.252179647860042\\
14.1725	0.260660317532519\\
14.1780555555556	0.257356001470246\\
14.1836111111111	0.252364325764712\\
14.1891666666667	0.254980242301113\\
14.1947222222222	0.256069847838166\\
14.2002777777778	0.259956060292783\\
14.2058333333333	0.248316176035411\\
14.2113888888889	0.255935073150875\\
14.2169444444444	0.253287438867607\\
14.2225	0.255000057286879\\
14.2280555555556	0.250500163888916\\
14.2336111111111	0.257334984373255\\
14.2391666666667	0.263522600339008\\
14.2447222222222	0.267704725516037\\
14.2502777777778	0.260309594636302\\
14.2558333333333	0.265627074558736\\
14.2613888888889	0.274971740085456\\
14.2669444444444	0.280544346273528\\
14.2725	0.285405805149249\\
14.2780555555556	0.281156894813807\\
14.2836111111111	0.280843053144621\\
14.2891666666667	0.287718961418613\\
14.2947222222222	0.290300514026044\\
14.3002777777778	0.30002932650335\\
14.3058333333333	0.297251824109863\\
14.3113888888889	0.298797561068838\\
14.3169444444444	0.299957340725809\\
14.3225	0.301392241536795\\
14.3280555555556	0.297480239219171\\
14.3336111111111	0.291325021906796\\
14.3391666666667	0.277138953554736\\
14.3447222222222	0.285991044492506\\
14.3502777777778	0.279063532697321\\
14.3558333333333	0.285114148426628\\
14.3613888888889	0.283905359867985\\
14.3669444444444	0.284453465495852\\
14.3725	0.287501413669981\\
14.3780555555556	0.291016631435474\\
14.3836111111111	0.298084500388233\\
14.3891666666667	0.296207237431849\\
14.3947222222222	0.292388694078727\\
14.4002777777778	0.296657023200303\\
14.4058333333333	0.302903280849426\\
14.4113888888889	0.299277908113841\\
14.4169444444444	0.303115516630601\\
14.4225	0.310083646835131\\
14.4280555555556	0.305964771320537\\
14.4336111111111	0.307086164924972\\
14.4391666666667	0.297134117834323\\
14.4447222222222	0.303500682243884\\
14.4502777777778	0.306510941489189\\
14.4558333333333	0.307040343212855\\
14.4613888888889	0.307420617053598\\
14.4669444444444	0.310333933162297\\
14.4725	0.314270349469542\\
14.4780555555556	0.313836686050984\\
14.4836111111111	0.326324135734376\\
14.4891666666667	0.326424313923122\\
14.4947222222222	0.32316856994385\\
14.5002777777778	0.321950392848813\\
14.5058333333333	0.319619262836939\\
14.5113888888889	0.331748586542722\\
14.5169444444444	0.329465328675581\\
14.5225	0.330931391861233\\
14.5280555555556	0.332610079019075\\
14.5336111111111	0.334043186566754\\
14.5391666666667	0.332961982904101\\
14.5447222222222	0.327364313522975\\
14.5502777777778	0.318909207359261\\
14.5558333333333	0.319648223249652\\
14.5613888888889	0.320036133019606\\
14.5669444444444	0.318321519124783\\
14.5725	0.322248108743106\\
14.5780555555556	0.328071470524935\\
14.5836111111111	0.334808804555891\\
14.5891666666667	0.332799796485453\\
14.5947222222222	0.334322956202041\\
14.6002777777778	0.32688669576709\\
14.6058333333333	0.321802264015507\\
14.6113888888889	0.315050095256115\\
14.6169444444444	0.310438341558662\\
14.6225	0.310422344329747\\
14.6280555555556	0.300196710818314\\
14.6336111111111	0.301231204334935\\
14.6391666666667	0.294131833038251\\
14.6447222222222	0.290965834443882\\
14.6502777777778	0.295030948104327\\
14.6558333333333	0.301091151472321\\
14.6613888888889	0.302144936112031\\
14.6669444444444	0.302382683947245\\
14.6725	0.296095392100838\\
14.6780555555556	0.294110797893372\\
14.6836111111111	0.291996509063568\\
14.6891666666667	0.291636387951052\\
14.6947222222222	0.286146509113468\\
14.7002777777778	0.291134498701657\\
14.7058333333333	0.292987744688895\\
14.7113888888889	0.295341942859239\\
14.7169444444444	0.299502708816039\\
14.7225	0.299576851716945\\
14.7280555555556	0.297370408647641\\
14.7336111111111	0.303634271603663\\
14.7391666666667	0.302798350162292\\
14.7447222222222	0.30820155246973\\
14.7502777777778	0.310745514954628\\
14.7558333333333	0.314234816946183\\
14.7613888888889	0.309316132006741\\
14.7669444444444	0.305038064572084\\
14.7725	0.30499788445045\\
14.7780555555556	0.312255156083756\\
14.7836111111111	0.30893531116935\\
14.7891666666667	0.29805808191595\\
14.7947222222222	0.301700052298588\\
14.8002777777778	0.301709024270795\\
14.8058333333333	0.302245143972325\\
14.8113888888889	0.300171751436581\\
14.8169444444444	0.309537755837862\\
14.8225	0.30722755499837\\
14.8280555555556	0.315651391689353\\
14.8336111111111	0.321077262451727\\
14.8391666666667	0.321860323967734\\
14.8447222222222	0.319252047519731\\
14.8502777777778	0.312305392272939\\
14.8558333333333	0.316216018617541\\
14.8613888888889	0.319614303838974\\
14.8669444444444	0.327833891013707\\
14.8725	0.329025691676326\\
14.8780555555556	0.333608048423855\\
14.8836111111111	0.334113199832168\\
14.8891666666667	0.328064427230926\\
14.8947222222222	0.328837816882816\\
14.9002777777778	0.32680753752774\\
14.9058333333333	0.326997262891402\\
14.9113888888889	0.332536466099984\\
14.9169444444444	0.335306580416292\\
14.9225	0.333203161870051\\
14.9280555555556	0.333855532909958\\
14.9336111111111	0.344032494313602\\
14.9391666666667	0.341888286752982\\
14.9447222222222	0.342595545737416\\
14.9502777777778	0.352782558241721\\
14.9558333333333	0.363777445280251\\
14.9613888888889	0.363297153140012\\
14.9669444444444	0.361380047617893\\
14.9725	0.36049947724211\\
14.9780555555556	0.353756159813793\\
14.9836111111111	0.350089759482269\\
14.9891666666667	0.351488491076446\\
14.9947222222222	0.346537728010392\\
15.0002777777778	0.342909153048849\\
15.0058333333333	0.342718103932034\\
15.0113888888889	0.343668418760501\\
15.0169444444444	0.347694411278777\\
15.0225	0.362463521113206\\
15.0280555555556	0.360693476729925\\
15.0336111111111	0.363504336608595\\
15.0391666666667	0.365051945182563\\
15.0447222222222	0.368003722381543\\
15.0502777777778	0.366426563952893\\
15.0558333333333	0.368069225549174\\
15.0613888888889	0.371158505699998\\
15.0669444444444	0.370033062235302\\
15.0725	0.368739393745805\\
15.0780555555556	0.37986860302138\\
15.0836111111111	0.380369140702015\\
15.0891666666667	0.376711652859683\\
15.0947222222222	0.377504474090038\\
15.1002777777778	0.374787517731874\\
15.1058333333333	0.373967248806237\\
15.1113888888889	0.374629847747403\\
15.1169444444444	0.368866121892842\\
15.1225	0.363603691224199\\
15.1280555555556	0.357901719849931\\
15.1336111111111	0.359536156926754\\
15.1391666666667	0.359149343675653\\
15.1447222222222	0.353866391626172\\
15.1502777777778	0.355079963722506\\
15.1558333333333	0.349938693931804\\
15.1613888888889	0.353144538695648\\
15.1669444444444	0.3568969752644\\
15.1725	0.356701407837459\\
15.1780555555556	0.354761985459531\\
15.1836111111111	0.36121495736088\\
15.1891666666667	0.37733413863124\\
15.1947222222222	0.379540775807793\\
15.2002777777778	0.377659192359455\\
15.2058333333333	0.381033648154584\\
15.2113888888889	0.371432662885671\\
15.2169444444444	0.370380332804638\\
15.2225	0.375551480973722\\
15.2280555555556	0.37585453065567\\
15.2336111111111	0.3730080213835\\
15.2391666666667	0.372684953055932\\
15.2447222222222	0.374085683224483\\
15.2502777777778	0.377966380682229\\
15.2558333333333	0.377275327483298\\
15.2613888888889	0.37615170213434\\
15.2669444444444	0.371777535580043\\
15.2725	0.371894751519415\\
15.2780555555556	0.369406514573659\\
15.2836111111111	0.364728310390574\\
15.2891666666667	0.365589297778386\\
15.2947222222222	0.36562089020657\\
15.3002777777778	0.361338173469966\\
15.3058333333333	0.348119079207975\\
15.3113888888889	0.348674534600045\\
15.3169444444444	0.338466305732953\\
15.3225	0.339307635126023\\
15.3280555555556	0.342263076991907\\
15.3336111111111	0.344920593259389\\
15.3391666666667	0.330950555153778\\
15.3447222222222	0.333146177346479\\
15.3502777777778	0.330759615012044\\
15.3558333333333	0.33237525737794\\
15.3613888888889	0.335990132278647\\
15.3669444444444	0.331922998011996\\
15.3725	0.334763245286374\\
15.3780555555556	0.339938889298509\\
15.3836111111111	0.337040560414422\\
15.3891666666667	0.339337291800455\\
15.3947222222222	0.347241563476251\\
15.4002777777778	0.346915563232153\\
15.4058333333333	0.351065337069579\\
15.4113888888889	0.347638969671251\\
15.4169444444444	0.345453616574882\\
15.4225	0.34053375897933\\
15.4280555555556	0.335905872669435\\
15.4336111111111	0.333725488687574\\
15.4391666666667	0.350581304397968\\
15.4447222222222	0.356115159081696\\
15.4502777777778	0.349041050581169\\
15.4558333333333	0.346414636581735\\
15.4613888888889	0.348932597480775\\
15.4669444444444	0.341591724639823\\
15.4725	0.344396578630116\\
15.4780555555556	0.343764147388399\\
15.4836111111111	0.339197758517881\\
15.4891666666667	0.33271751470235\\
15.4947222222222	0.332509578919722\\
15.5002777777778	0.33430191369295\\
15.5058333333333	0.343961742163677\\
15.5113888888889	0.343330203147673\\
15.5169444444444	0.344273720543942\\
15.5225	0.351893364229652\\
15.5280555555556	0.352919394986972\\
15.5336111111111	0.360605525611976\\
15.5391666666667	0.356582559106185\\
15.5447222222222	0.358056120550361\\
15.5502777777778	0.346886943714043\\
15.5558333333333	0.341769969891537\\
15.5613888888889	0.342653667067135\\
15.5669444444444	0.33430808094512\\
15.5725	0.343810623459711\\
15.5780555555556	0.343165133647171\\
15.5836111111111	0.345639965164281\\
15.5891666666667	0.335425197617421\\
15.5947222222222	0.330417936774056\\
15.6002777777778	0.33321392553341\\
15.6058333333333	0.345855414260513\\
15.6113888888889	0.303029803476725\\
15.6169444444444	0.308237249010805\\
15.6225	0.316263825311321\\
15.6280555555556	0.321581068310572\\
15.6336111111111	0.317433560529449\\
15.6391666666667	0.329162204482653\\
15.6447222222222	0.335885918579519\\
15.6502777777778	0.350495619125967\\
15.6558333333333	0.35475605562241\\
15.6613888888889	0.351317478180192\\
15.6669444444444	0.337587554739638\\
15.6725	0.352153468829225\\
15.6780555555556	0.352519844728073\\
15.6836111111111	0.353876428858566\\
15.6891666666667	0.349771533379786\\
15.6947222222222	0.351353029638784\\
15.7002777777778	0.351163880853143\\
15.7058333333333	0.354137569448324\\
15.7113888888889	0.357128861362644\\
15.7169444444444	0.36155631032755\\
15.7225	0.352171606079211\\
15.7280555555556	0.355080902453936\\
15.7336111111111	0.353812070324922\\
15.7391666666667	0.344887911668371\\
15.7447222222222	0.349351848422749\\
15.7502777777778	0.352834036394338\\
15.7558333333333	0.356888120672669\\
15.7613888888889	0.349298328880021\\
15.7669444444444	0.345590763160369\\
15.7725	0.348445391712288\\
15.7780555555556	0.351698792282236\\
15.7836111111111	0.351612739556822\\
15.7891666666667	0.355196418118995\\
15.7947222222222	0.349860230324939\\
15.8002777777778	0.343140332968898\\
15.8058333333333	0.348262527000746\\
15.8113888888889	0.345207599535356\\
15.8169444444444	0.34498371400842\\
15.8225	0.351768561229618\\
15.8280555555556	0.359765509782321\\
15.8336111111111	0.342572474093882\\
15.8391666666667	0.346637307122523\\
15.8447222222222	0.354919569438578\\
15.8502777777778	0.359088334956077\\
15.8558333333333	0.370118776949637\\
15.8613888888889	0.373256504027333\\
15.8669444444444	0.380335280589819\\
15.8725	0.376358781898745\\
15.8780555555556	0.379548639126007\\
15.8836111111111	0.371432859448584\\
15.8891666666667	0.37792110995389\\
15.8947222222222	0.384869998000046\\
15.9002777777778	0.389788057457531\\
15.9058333333333	0.392296577272432\\
15.9113888888889	0.38687064300067\\
15.9169444444444	0.386107607705148\\
15.9225	0.375753932398958\\
15.9280555555556	0.367523969508165\\
15.9336111111111	0.365443758308474\\
15.9391666666667	0.371780398204429\\
15.9447222222222	0.371136894204681\\
15.9502777777778	0.364999544492088\\
15.9558333333333	0.367478845865816\\
15.9613888888889	0.363071002942044\\
15.9669444444444	0.36650919420364\\
15.9725	0.369342661303882\\
15.9780555555556	0.377782321784467\\
15.9836111111111	0.378918823691729\\
15.9891666666667	0.381443072238011\\
15.9947222222222	0.396086672868292\\
};
\addplot [color=mycolor3,solid,line width=1.5pt,forget plot]
  table[row sep=crcr]{%
9.50027777777778	0\\
9.50583333333333	0.000230361606064374\\
9.51138888888889	0.000860827234408127\\
9.51694444444444	0.00106652080766038\\
9.5225	0.00134807637275717\\
9.52805555555556	0.0012974092213834\\
9.53361111111111	0.00202996743249181\\
9.53916666666667	0.00185353067229981\\
9.54472222222222	0.00207631956493986\\
9.55027777777778	0.000971573597444947\\
9.55583333333333	0.000825726126327008\\
9.56138888888889	0.0013675983197539\\
9.56694444444444	0.00133645310799876\\
9.5725	0.00145109892113502\\
9.57805555555555	0.00196357962247388\\
9.58361111111111	0.00257262407710027\\
9.58916666666667	0.00332854053821612\\
9.59472222222222	0.00393751935143996\\
9.60027777777778	0.00497835552360936\\
9.60583333333333	0.00544146550233822\\
9.61138888888889	0.00544495582858168\\
9.61694444444444	0.00491065190854416\\
9.6225	0.00530785944971938\\
9.62805555555556	0.00540633898282816\\
9.63361111111111	0.0058484905648068\\
9.63916666666667	0.00674586042301849\\
9.64472222222222	0.00775238395321017\\
9.65027777777778	0.00836553615282432\\
9.65583333333333	0.00785782838879183\\
9.66138888888889	0.00746968926667738\\
9.66694444444444	0.00609877503018592\\
9.6725	0.00587167144220827\\
9.67805555555555	0.00663168256222223\\
9.68361111111111	0.0069885991971425\\
9.68916666666667	0.00714228567954762\\
9.69472222222222	0.00674520066568665\\
9.70027777777778	0.00700933692259775\\
9.70583333333333	0.00602018731785136\\
9.71138888888889	0.00647124386474627\\
9.71694444444444	0.00642892346504945\\
9.7225	0.00698786628622715\\
9.72805555555555	0.00682498661337657\\
9.73361111111111	0.00740187163043122\\
9.73916666666667	0.00706240301610813\\
9.74472222222222	0.00692252692391461\\
9.75027777777778	0.00700546380313503\\
9.75583333333333	0.00798634181255167\\
9.76138888888889	0.00872173768756209\\
9.76694444444444	0.00877241594959755\\
9.7725	0.00811486987981889\\
9.77805555555556	0.00715565391176458\\
9.78361111111111	0.00540716169313199\\
9.78916666666667	0.00479639720913088\\
9.79472222222222	0.00526573077513164\\
9.80027777777778	0.00324127212793544\\
9.80583333333333	0.00262246127411124\\
9.81138888888889	0.00237063774966746\\
9.81694444444444	0.00128255290349705\\
9.8225	0.0020887689977686\\
9.82805555555555	0.00165802293677576\\
9.83361111111111	0.00123044618231844\\
9.83916666666667	6.94725906753074e-05\\
9.84472222222222	0.000494441797907009\\
9.85027777777778	-0.000301062724466915\\
9.85583333333333	-0.00159881846256882\\
9.86138888888889	-0.000294142813092793\\
9.86694444444444	-0.00111840271131036\\
9.8725	-0.000614926952930385\\
9.87805555555556	-0.00159486201458158\\
9.88361111111111	-0.00130777128996671\\
9.88916666666667	-0.000874557810175332\\
9.89472222222222	-0.00269619098784249\\
9.90027777777778	-0.00416688996604486\\
9.90583333333333	-0.00452990856394024\\
9.91138888888889	-0.00459139351799389\\
9.91694444444444	-0.00392890650656934\\
9.9225	-0.00425186938144926\\
9.92805555555555	-0.00305140892562288\\
9.93361111111111	-0.00295961371310928\\
9.93916666666667	-0.00353334914758817\\
9.94472222222222	-0.00391332752604753\\
9.95027777777778	-0.00399801267726632\\
9.95583333333333	-0.000751931750820508\\
9.96138888888889	-0.00109601162963629\\
9.96694444444444	0.00102827374372331\\
9.9725	-0.000950814932096069\\
9.97805555555555	-0.000689855404179413\\
9.98361111111111	-0.00130326742345897\\
9.98916666666667	-0.00128116432456705\\
9.99472222222222	-0.00130927843391746\\
10.0002777777778	-0.000841321773158446\\
10.0058333333333	-0.0017433066187663\\
10.0113888888889	-0.00119721787432087\\
10.0169444444444	-0.000513087359257575\\
10.0225	0.00100983211971997\\
10.0280555555556	0.000908126880469023\\
10.0336111111111	-0.000119801446365649\\
10.0391666666667	-0.000469075976642428\\
10.0447222222222	0.000619759935724787\\
10.0502777777778	0.0016298986625085\\
10.0558333333333	0.00434325691708696\\
10.0613888888889	0.00400142311467352\\
10.0669444444444	0.00522919334784825\\
10.0725	0.00461043411256668\\
10.0780555555556	0.00364985847969759\\
10.0836111111111	0.0050553069269997\\
10.0891666666667	0.00574134346869688\\
10.0947222222222	0.00610213537735773\\
10.1002777777778	0.00555793249579754\\
10.1058333333333	0.00519028003049703\\
10.1113888888889	0.00703646551158257\\
10.1169444444444	0.00808755508524803\\
10.1225	0.00943776165011336\\
10.1280555555556	0.00950283028113898\\
10.1336111111111	0.00969591250207692\\
10.1391666666667	0.00958031032949173\\
10.1447222222222	0.00827871440141407\\
10.1502777777778	0.00660985924070746\\
10.1558333333333	0.00692283028844175\\
10.1613888888889	0.00699595041043622\\
10.1669444444444	0.00644964428251805\\
10.1725	0.00778899385452488\\
10.1780555555556	0.00488597716546136\\
10.1836111111111	0.00479287134672791\\
10.1891666666667	0.00499744515114174\\
10.1947222222222	0.00426204208475131\\
10.2002777777778	0.0047779364158555\\
10.2058333333333	0.0054684934098052\\
10.2113888888889	0.00508136049811541\\
10.2169444444444	0.00450260306861442\\
10.2225	0.00351210817115203\\
10.2280555555556	0.00447183453829226\\
10.2336111111111	0.00320779251217462\\
10.2391666666667	0.00296981883868243\\
10.2447222222222	0.00241451062685252\\
10.2502777777778	0.00250402164202692\\
10.2558333333333	0.00274795316149311\\
10.2613888888889	0.00343554101727388\\
10.2669444444444	0.00200757725626726\\
10.2725	0.00321347849947754\\
10.2780555555556	0.00291837446573118\\
10.2836111111111	0.0034097058328802\\
10.2891666666667	0.00194576359028115\\
10.2947222222222	0.00307237470008412\\
10.3002777777778	0.0025011130352364\\
10.3058333333333	0.0023395116271176\\
10.3113888888889	0.002419985562115\\
10.3169444444444	0.00135075343153873\\
10.3225	0.00205083001583253\\
10.3280555555556	0.00320235384454531\\
10.3336111111111	0.00184418969005129\\
10.3391666666667	0.000825928906626734\\
10.3447222222222	0.000428023279564501\\
10.3502777777778	0.0010152189308846\\
10.3558333333333	-0.000515514877649898\\
10.3613888888889	-7.90114473406805e-05\\
10.3669444444444	-0.00104004530725909\\
10.3725	-0.00143105562770469\\
10.3780555555556	-0.0017557198292952\\
10.3836111111111	-0.000380759674723302\\
10.3891666666667	5.96570772834964e-05\\
10.3947222222222	-0.000666284200473683\\
10.4002777777778	-0.000819491965993076\\
10.4058333333333	-0.00094024512478373\\
10.4113888888889	-0.000845313846569287\\
10.4169444444444	8.84055031396634e-05\\
10.4225	-0.00118001757106977\\
10.4280555555556	-0.00107604976378693\\
10.4336111111111	-0.00102792289257133\\
10.4391666666667	-0.00133587093481235\\
10.4447222222222	-0.000382070888373417\\
10.4502777777778	0.000477581417885833\\
10.4558333333333	0.000368666421175151\\
10.4613888888889	0.00133483767994605\\
10.4669444444444	0.00102325269740119\\
10.4725	0.000794240261456498\\
10.4780555555556	0.000964945871505058\\
10.4836111111111	0.00240284848955132\\
10.4891666666667	0.00229908686555865\\
10.4947222222222	0.00309132290626728\\
10.5002777777778	0.00346189838869404\\
10.5058333333333	0.00428402664611652\\
10.5113888888889	0.00472929007795392\\
10.5169444444444	0.00461741470527969\\
10.5225	0.00572621301957371\\
10.5280555555556	0.00452436339040968\\
10.5336111111111	0.00555089650133745\\
10.5391666666667	0.00528204604286601\\
10.5447222222222	0.00518229571527266\\
10.5502777777778	0.00459810426506263\\
10.5558333333333	0.00496158877195586\\
10.5613888888889	0.00509505000624981\\
10.5669444444444	0.00524406107946279\\
10.5725	0.00508183376277766\\
10.5780555555556	0.00562357841051414\\
10.5836111111111	0.00396554405530975\\
10.5891666666667	0.00410476064100167\\
10.5947222222222	0.00319237162310225\\
10.6002777777778	0.00371095157749438\\
10.6058333333333	0.00501550582722433\\
10.6113888888889	0.00417913763288258\\
10.6169444444444	0.00475857653388613\\
10.6225	0.0053621092210607\\
10.6280555555556	0.00577805287065399\\
10.6336111111111	0.00502579892376296\\
10.6391666666667	0.00385307288014382\\
10.6447222222222	0.00442636497752215\\
10.6502777777778	0.00470083271148823\\
10.6558333333333	0.00518434397094825\\
10.6613888888889	0.00557259398082486\\
10.6669444444444	0.00469531418548011\\
10.6725	0.00530113673493144\\
10.6780555555556	0.00506403461505524\\
10.6836111111111	0.00535647948247307\\
10.6891666666667	0.00482594026614513\\
10.6947222222222	0.0029998572499368\\
10.7002777777778	0.00262915045245665\\
10.7058333333333	0.00238721581465826\\
10.7113888888889	0.00349514939724124\\
10.7169444444444	0.00384934118149512\\
10.7225	0.00512874613009069\\
10.7280555555556	0.00338144948375815\\
10.7336111111111	0.00189397214371312\\
10.7391666666667	0.000773341187982358\\
10.7447222222222	0.000929495867867303\\
10.7502777777778	0.00173496458568765\\
10.7558333333333	0.00149178859062282\\
10.7613888888889	0.00127267255999015\\
10.7669444444444	0.00157952072396912\\
10.7725	0.00306367431900605\\
10.7780555555556	0.00359663905524846\\
10.7836111111111	0.00233667149859094\\
10.7891666666667	0.00279365830183984\\
10.7947222222222	0.00284498139122013\\
10.8002777777778	0.00294964585576963\\
10.8058333333333	0.00377610980522574\\
10.8113888888889	0.00399751870327429\\
10.8169444444444	0.00474781660480199\\
10.8225	0.00441799916389627\\
10.8280555555556	0.00327038278630076\\
10.8336111111111	0.00349811002766536\\
10.8391666666667	0.00347988847777164\\
10.8447222222222	0.00294160571219616\\
10.8502777777778	0.00240753374515651\\
10.8558333333333	0.00248771221522779\\
10.8613888888889	0.00268835110874819\\
10.8669444444444	0.0029066213115095\\
10.8725	0.0021232195634498\\
10.8780555555556	0.00208915730837353\\
10.8836111111111	0.00176103760181348\\
10.8891666666667	0.00149955437297116\\
10.8947222222222	0.00162806622688669\\
10.9002777777778	2.9178627561375e-05\\
10.9058333333333	-6.19153509993893e-05\\
10.9113888888889	-0.000259117513012825\\
10.9169444444444	-0.000250145597047533\\
10.9225	0.00039313122834666\\
10.9280555555556	0.000181955514053937\\
10.9336111111111	0.00104524216743576\\
10.9391666666667	0.00191054012087388\\
10.9447222222222	0.00200358505730222\\
10.9502777777778	0.00134354948779612\\
10.9558333333333	0.00044166701219316\\
10.9613888888889	-8.94573852620424e-05\\
10.9669444444444	-7.58657515946893e-05\\
10.9725	-0.00151216318031868\\
10.9780555555556	-0.00057238265133812\\
10.9836111111111	-0.00044660821608948\\
10.9891666666667	-0.00148222093740052\\
10.9947222222222	-0.0014219372842689\\
11.0002777777778	-0.000523347163635578\\
11.0058333333333	-0.000271287372519826\\
11.0113888888889	6.08562115459679e-06\\
11.0169444444444	0.00145026905692374\\
11.0225	0.00261984998636481\\
11.0280555555556	0.00238878591225989\\
11.0336111111111	0.00301529960689042\\
11.0391666666667	0.00312568999734897\\
11.0447222222222	0.00365913661452393\\
11.0502777777778	0.00457348876267437\\
11.0558333333333	0.0062671168191829\\
11.0613888888889	0.00598386607242417\\
11.0669444444444	0.00564672005509635\\
11.0725	0.00618313311659286\\
11.0780555555556	0.00657161206415262\\
11.0836111111111	0.00591611863409991\\
11.0891666666667	0.00563919852952507\\
11.0947222222222	0.00510095210315955\\
11.1002777777778	0.00561866384236467\\
11.1058333333333	0.00488244463919618\\
11.1113888888889	0.00429740068862118\\
11.1169444444444	0.0045990808413354\\
11.1225	0.00500418300669477\\
11.1280555555556	0.00489351314115142\\
11.1336111111111	0.00537876696072342\\
11.1391666666667	0.00445967369529546\\
11.1447222222222	0.0046548456467339\\
11.1502777777778	0.00542184970975491\\
11.1558333333333	0.00544097318475104\\
11.1613888888889	0.00480339880999818\\
11.1669444444444	0.00477910722274972\\
11.1725	0.00411457777994003\\
11.1780555555556	0.00314588314730898\\
11.1836111111111	0.00292003047686214\\
11.1891666666667	0.00411257935552994\\
11.1947222222222	0.00504961845611438\\
11.2002777777778	0.00602137223551923\\
11.2058333333333	0.00591879958750622\\
11.2113888888889	0.00690478948225247\\
11.2169444444444	0.00730538340681996\\
11.2225	0.00679999770137347\\
11.2280555555556	0.00600261408716999\\
11.2336111111111	0.00579309678451375\\
11.2391666666667	0.00560709984193049\\
11.2447222222222	0.00526691271313078\\
11.2502777777778	0.00374396428508731\\
11.2558333333333	0.00389401267616937\\
11.2613888888889	0.00338032583898077\\
11.2669444444444	0.00313137992980036\\
11.2725	0.00334357774131777\\
11.2780555555556	0.0041003902879739\\
11.2836111111111	0.00352789797248366\\
11.2891666666667	0.00205521035686044\\
11.2947222222222	0.00213097572379635\\
11.3002777777778	0.00102397105468977\\
11.3058333333333	0.0021132183683838\\
11.3113888888889	0.00214796995602745\\
11.3169444444444	0.00216867675873019\\
11.3225	0.00330230244720871\\
11.3280555555556	0.0041546144067164\\
11.3336111111111	0.00404848210343818\\
11.3391666666667	0.00345244474457276\\
11.3447222222222	0.00277452943071004\\
11.3502777777778	0.0024664522467967\\
11.3558333333333	0.00242379748456573\\
11.3613888888889	0.00309822315724595\\
11.3669444444444	0.0031161295679536\\
11.3725	0.00224369685729774\\
11.3780555555556	0.00279032562593844\\
11.3836111111111	0.00280629537581699\\
11.3891666666667	0.00271178111142749\\
11.3947222222222	0.00222550430349507\\
11.4002777777778	0.00209462681654583\\
11.4058333333333	0.00144842518734685\\
11.4113888888889	0.00184650227475503\\
11.4169444444444	0.000764622569416582\\
11.4225	0.00042121943332059\\
11.4280555555556	0.00214463805055588\\
11.4336111111111	0.00222606382855518\\
11.4391666666667	0.00110788602846516\\
11.4447222222222	0.00184529833929893\\
11.4502777777778	0.0023852334351693\\
11.4558333333333	0.00268117061150784\\
11.4613888888889	0.000444288199832657\\
11.4669444444444	0.00133877578630416\\
11.4725	0.00141788758701943\\
11.4780555555556	0.000810865659022552\\
11.4836111111111	0.000490525187010042\\
11.4891666666667	0.000111094000560527\\
11.4947222222222	0.000321043700827529\\
11.5002777777778	5.06739146001211e-05\\
11.5058333333333	-4.6280750681492e-05\\
11.5113888888889	-0.000215037977666075\\
11.5169444444444	-0.000661415234521913\\
11.5225	-0.00108309784034426\\
11.5280555555556	-0.00274319116447853\\
11.5336111111111	-0.00278366791277757\\
11.5391666666667	-0.00233926271690483\\
11.5447222222222	-0.0040628725010872\\
11.5502777777778	-0.00358868735868545\\
11.5558333333333	-0.00380559564378732\\
11.5613888888889	-0.00448559994932391\\
11.5669444444444	-0.00584436975175029\\
11.5725	-0.00688235672175868\\
11.5780555555556	-0.00878361067997538\\
11.5836111111111	-0.00661376141721425\\
11.5891666666667	-0.00753531039466097\\
11.5947222222222	-0.00905289177067338\\
11.6002777777778	-0.00726148472654734\\
11.6058333333333	-0.00847673454046786\\
11.6113888888889	-0.00914218104902525\\
11.6169444444444	-0.0108861356624002\\
11.6225	-0.0111840182025698\\
11.6280555555556	-0.0119565515806571\\
11.6336111111111	-0.0128261905547603\\
11.6391666666667	-0.0100455627456966\\
11.6447222222222	-0.0102827050351415\\
11.6502777777778	-0.00710858715615025\\
11.6558333333333	-0.00581748839832879\\
11.6613888888889	-0.0040623509380888\\
11.6669444444444	-0.00408485050091388\\
11.6725	-0.00247081841070291\\
11.6780555555556	-0.00337683081510756\\
11.6836111111111	-0.00242664451398191\\
11.6891666666667	-0.00331041411496932\\
11.6947222222222	-0.00347353500976481\\
11.7002777777778	-0.00462429785927552\\
11.7058333333333	-0.0054614875984238\\
11.7113888888889	-0.00505321676341298\\
11.7169444444444	-0.0056095162190411\\
11.7225	-0.00439916247378189\\
11.7280555555556	-0.00340493879433915\\
11.7336111111111	-0.00475836003910281\\
11.7391666666667	-0.00505847894532737\\
11.7447222222222	-0.0079537312002759\\
11.7502777777778	-0.00771697901906089\\
11.7558333333333	-0.0108549635238623\\
11.7613888888889	-0.00958249650854332\\
11.7669444444444	-0.010395417563624\\
11.7725	-0.0108008290491643\\
11.7780555555556	-0.0121064127563993\\
11.7836111111111	-0.0119261142371718\\
11.7891666666667	-0.0118465539924039\\
11.7947222222222	-0.0137917442867912\\
11.8002777777778	-0.0142082331822134\\
11.8058333333333	-0.0137197415300987\\
11.8113888888889	-0.0139086923751084\\
11.8169444444444	-0.0141119031816698\\
11.8225	-0.0147065292960987\\
11.8280555555556	-0.0157922772052939\\
11.8336111111111	-0.0156484618437916\\
11.8391666666667	-0.0161227165982594\\
11.8447222222222	-0.015081815693571\\
11.8502777777778	-0.0148577324366763\\
11.8558333333333	-0.0165314416358977\\
11.8613888888889	-0.016283278945797\\
11.8669444444444	-0.0160134690436193\\
11.8725	-0.0168978825466258\\
11.8780555555556	-0.0167325699998199\\
11.8836111111111	-0.0166827465478918\\
11.8891666666667	-0.0179979041252945\\
11.8947222222222	-0.0172943619115542\\
11.9002777777778	-0.0189671292929416\\
11.9058333333333	-0.0184204585964246\\
11.9113888888889	-0.0183633432678962\\
11.9169444444444	-0.0190044542159074\\
11.9225	-0.0192528012981271\\
11.9280555555556	-0.0191622023697406\\
11.9336111111111	-0.0183257493624671\\
11.9391666666667	-0.0191017097374627\\
11.9447222222222	-0.0190462771108042\\
11.9502777777778	-0.0187556950778566\\
11.9558333333333	-0.0192545546396194\\
11.9613888888889	-0.0193690159334076\\
11.9669444444444	-0.0194271712912997\\
11.9725	-0.0194725862528037\\
11.9780555555556	-0.0212236581458992\\
11.9836111111111	-0.0191063034683122\\
11.9891666666667	-0.0195995229587353\\
11.9947222222222	-0.0202307219139517\\
12.0002777777778	-0.0198992927389462\\
12.0058333333333	-0.0190284153564783\\
12.0113888888889	-0.0194393866633161\\
12.0169444444444	-0.0193462840702963\\
12.0225	-0.0189001693825308\\
12.0280555555556	-0.0175707848949565\\
12.0336111111111	-0.0172721751651936\\
12.0391666666667	-0.0179955238026464\\
12.0447222222222	-0.0180898079559107\\
12.0502777777778	-0.017864310966881\\
12.0558333333333	-0.0168096195128705\\
12.0613888888889	-0.0176038401954918\\
12.0669444444444	-0.0181986718983346\\
12.0725	-0.0182852720922168\\
12.0780555555556	-0.0166168093913578\\
12.0836111111111	-0.0163091297273856\\
12.0891666666667	-0.0156419815379736\\
12.0947222222222	-0.0157017459239867\\
12.1002777777778	-0.016077468998035\\
12.1058333333333	-0.0150815901769516\\
12.1113888888889	-0.0150583503389517\\
12.1169444444444	-0.0171411743203707\\
12.1225	-0.0170912468167137\\
12.1280555555556	-0.0171740897282598\\
12.1336111111111	-0.0185999086777536\\
12.1391666666667	-0.0188501389954634\\
12.1447222222222	-0.0192590595377876\\
12.1502777777778	-0.0196175017996559\\
12.1558333333333	-0.0201309356501792\\
12.1613888888889	-0.0204322572664561\\
12.1669444444444	-0.0199553141440022\\
12.1725	-0.0198963335799768\\
12.1780555555556	-0.0200470043085886\\
12.1836111111111	-0.0177766690544365\\
12.1891666666667	-0.0173416373551278\\
12.1947222222222	-0.0186716625547804\\
12.2002777777778	-0.0186629645953435\\
12.2058333333333	-0.0180276131334315\\
12.2113888888889	-0.0184130208843952\\
12.2169444444444	-0.0184678038868582\\
12.2225	-0.0176622107861049\\
12.2280555555556	-0.0173471462797348\\
12.2336111111111	-0.0177127149357156\\
12.2391666666667	-0.0174766537871373\\
12.2447222222222	-0.0176906688653095\\
12.2502777777778	-0.0176232843712621\\
12.2558333333333	-0.0165002755888577\\
12.2613888888889	-0.0168543723703487\\
12.2669444444444	-0.0170758532894222\\
12.2725	-0.0164915301249127\\
12.2780555555556	-0.0159621089610368\\
12.2836111111111	-0.0156484831787077\\
12.2891666666667	-0.0152642912563671\\
12.2947222222222	-0.015114947389627\\
12.3002777777778	-0.0168613126212464\\
12.3058333333333	-0.0169939123811787\\
12.3113888888889	-0.0173812502434077\\
12.3169444444444	-0.0173842695240338\\
12.3225	-0.0164357004832944\\
12.3280555555556	-0.0157696402836204\\
12.3336111111111	-0.0149025720857286\\
12.3391666666667	-0.0152360691612356\\
12.3447222222222	-0.0140649304020318\\
12.3502777777778	-0.0122291445475932\\
12.3558333333333	-0.0122924790417963\\
12.3613888888889	-0.0123518195754754\\
12.3669444444444	-0.0122032900521661\\
12.3725	-0.0126212106657148\\
12.3780555555556	-0.0113299097779902\\
12.3836111111111	-0.0117266241988773\\
12.3891666666667	-0.0115175270739704\\
12.3947222222222	-0.0118838929195837\\
12.4002777777778	-0.0120440290842856\\
12.4058333333333	-0.0130891442633757\\
12.4113888888889	-0.0131076590951468\\
12.4169444444444	-0.0134565715108717\\
12.4225	-0.013356333123174\\
12.4280555555556	-0.0125080691418338\\
12.4336111111111	-0.0133375574103907\\
12.4391666666667	-0.0136497984131679\\
12.4447222222222	-0.0128696882996912\\
12.4502777777778	-0.0130641501915579\\
12.4558333333333	-0.0131364789238074\\
12.4613888888889	-0.0128094319825984\\
12.4669444444444	-0.0124903353985175\\
12.4725	-0.0121814519927981\\
12.4780555555556	-0.0119608706667754\\
12.4836111111111	-0.0123199331531268\\
12.4891666666667	-0.0133737404624737\\
12.4947222222222	-0.0125870160787039\\
12.5002777777778	-0.0116671451372892\\
12.5058333333333	-0.0121452475979798\\
12.5113888888889	-0.0114023747256039\\
12.5169444444444	-0.012164197428086\\
12.5225	-0.0123838959604457\\
12.5280555555556	-0.0136503744818198\\
12.5336111111111	-0.0149038955917102\\
12.5391666666667	-0.0142623630559114\\
12.5447222222222	-0.0145882214427373\\
12.5502777777778	-0.0139866410758059\\
12.5558333333333	-0.0141480691156752\\
12.5613888888889	-0.0149051318332283\\
12.5669444444444	-0.0147243952176638\\
12.5725	-0.0151558403343144\\
12.5780555555556	-0.0152816094579446\\
12.5836111111111	-0.016144351384504\\
12.5891666666667	-0.0151835784905129\\
12.5947222222222	-0.0153875827357111\\
12.6002777777778	-0.0147876596703844\\
12.6058333333333	-0.0151786728385189\\
12.6113888888889	-0.0142098079362621\\
12.6169444444444	-0.0131097096884058\\
12.6225	-0.0122493939165086\\
12.6280555555556	-0.0122029276442218\\
12.6336111111111	-0.0108281587978788\\
12.6391666666667	-0.011617282441443\\
12.6447222222222	-0.0115376993589369\\
12.6502777777778	-0.0112742756848269\\
12.6558333333333	-0.0109623901491514\\
12.6613888888889	-0.0103677716680938\\
12.6669444444444	-0.0108222324230359\\
12.6725	-0.00991521821424249\\
12.6780555555556	-0.0103813404248378\\
12.6836111111111	-0.00970518274242447\\
12.6891666666667	-0.0103574564826722\\
12.6947222222222	-0.0113027322505119\\
12.7002777777778	-0.01100554991064\\
12.7058333333333	-0.010771824855822\\
12.7113888888889	-0.010104095814541\\
12.7169444444444	-0.0101202166306996\\
12.7225	-0.010380135925031\\
12.7280555555556	-0.00894475457834054\\
12.7336111111111	-0.00888740735292283\\
12.7391666666667	-0.00980436412587349\\
12.7447222222222	-0.0102289431486201\\
12.7502777777778	-0.00975949309027599\\
12.7558333333333	-0.00971148141591196\\
12.7613888888889	-0.00993461690981457\\
12.7669444444444	-0.0104405468481295\\
12.7725	-0.01099591644336\\
12.7780555555556	-0.0105380016738243\\
12.7836111111111	-0.0114366341670169\\
12.7891666666667	-0.0120117412759711\\
12.7947222222222	-0.0121629437537819\\
12.8002777777778	-0.0128603929482181\\
12.8058333333333	-0.0131116214279641\\
12.8113888888889	-0.0126051148420795\\
12.8169444444444	-0.0122230602673258\\
12.8225	-0.0132609169722035\\
12.8280555555556	-0.0131222031449115\\
12.8336111111111	-0.01320738041556\\
12.8391666666667	-0.0127158554021349\\
12.8447222222222	-0.0122727945170373\\
12.8502777777778	-0.0123450626578772\\
12.8558333333333	-0.0127516892762717\\
12.8613888888889	-0.0127211172830787\\
12.8669444444444	-0.0126589618811212\\
12.8725	-0.0122572863776397\\
12.8780555555556	-0.0113401508091407\\
12.8836111111111	-0.0124804333453186\\
12.8891666666667	-0.0124071429002876\\
12.8947222222222	-0.0120098251540725\\
12.9002777777778	-0.0108949268653015\\
12.9058333333333	-0.0105349369104773\\
12.9113888888889	-0.0108019458699444\\
12.9169444444444	-0.0111302922913575\\
12.9225	-0.0111626789618404\\
12.9280555555556	-0.0118624182302365\\
12.9336111111111	-0.0121608025612732\\
12.9391666666667	-0.0114309890013561\\
12.9447222222222	-0.0108820660892261\\
12.9502777777778	-0.0107497022074889\\
12.9558333333333	-0.0111101351630805\\
12.9613888888889	-0.0114285109898395\\
12.9669444444444	-0.0114423632103742\\
12.9725	-0.0113847620546784\\
12.9780555555556	-0.0114302883981198\\
12.9836111111111	-0.0111530095796313\\
12.9891666666667	-0.0115626226988326\\
12.9947222222222	-0.0102526438548463\\
13.0002777777778	-0.00967993488565722\\
13.0058333333333	-0.00954003402136711\\
13.0113888888889	-0.00953165753413897\\
13.0169444444444	-0.00898268917821461\\
13.0225	-0.00880233631892425\\
13.0280555555556	-0.00794736378958009\\
13.0336111111111	-0.00777090862825711\\
13.0391666666667	-0.00768159382661972\\
13.0447222222222	-0.00744520178597646\\
13.0502777777778	-0.00785053800271804\\
13.0558333333333	-0.00788152636378372\\
13.0613888888889	-0.00881173600015533\\
13.0669444444444	-0.00927098686354302\\
13.0725	-0.00958291121536011\\
13.0780555555556	-0.00800272252458316\\
13.0836111111111	-0.00816118969261893\\
13.0891666666667	-0.00921801801136793\\
13.0947222222222	-0.00892137561855307\\
13.1002777777778	-0.00901598099815506\\
13.1058333333333	-0.00860622168854856\\
13.1113888888889	-0.00737016157851616\\
13.1169444444444	-0.00648141390493314\\
13.1225	-0.00651503723976463\\
13.1280555555556	-0.00571039890129245\\
13.1336111111111	-0.00546111950008004\\
13.1391666666667	-0.00498739848234979\\
13.1447222222222	-0.00557932257255014\\
13.1502777777778	-0.00472561829516062\\
13.1558333333333	-0.00564467688827918\\
13.1613888888889	-0.00611014658875857\\
13.1669444444444	-0.00617839767991014\\
13.1725	-0.00654428737862323\\
13.1780555555556	-0.00659005982799215\\
13.1836111111111	-0.00699004612199733\\
13.1891666666667	-0.00592476435686198\\
13.1947222222222	-0.00620150131845155\\
13.2002777777778	-0.00602063867368489\\
13.2058333333333	-0.00589620903426411\\
13.2113888888889	-0.00623946557626625\\
13.2169444444444	-0.0069538155251724\\
13.2225	-0.00701639663471657\\
13.2280555555556	-0.0071699054346324\\
13.2336111111111	-0.00756649370019885\\
13.2391666666667	-0.00705852675794507\\
13.2447222222222	-0.00687536204284158\\
13.2502777777778	-0.00611176587581643\\
13.2558333333333	-0.00645948394206999\\
13.2613888888889	-0.00571204619748197\\
13.2669444444444	-0.00583458461782249\\
13.2725	-0.00615991297542907\\
13.2780555555556	-0.00657873469654192\\
13.2836111111111	-0.00654146574666396\\
13.2891666666667	-0.00664816214871051\\
13.2947222222222	-0.00583583179824346\\
13.3002777777778	-0.0067391948692839\\
13.3058333333333	-0.00728320747891226\\
13.3113888888889	-0.00771716842840752\\
13.3169444444444	-0.0082967317860627\\
13.3225	-0.00825090702516105\\
13.3280555555556	-0.00762177661167994\\
13.3336111111111	-0.00717313496516618\\
13.3391666666667	-0.00686180170996065\\
13.3447222222222	-0.00645788705813822\\
13.3502777777778	-0.00559088841611006\\
13.3558333333333	-0.00521904914109706\\
13.3613888888889	-0.00503927344608149\\
13.3669444444444	-0.00406321162328104\\
13.3725	-0.0042713203226014\\
13.3780555555556	-0.00499865055911627\\
13.3836111111111	-0.00542155985273085\\
13.3891666666667	-0.00606946244752256\\
13.3947222222222	-0.006508392716333\\
13.4002777777778	-0.00670721182352686\\
13.4058333333333	-0.00771764441056513\\
13.4113888888889	-0.00817085833568332\\
13.4169444444444	-0.00774271253264197\\
13.4225	-0.00776929060420862\\
13.4280555555556	-0.00814039340603479\\
13.4336111111111	-0.00846913105097403\\
13.4391666666667	-0.00739785466851431\\
13.4447222222222	-0.00713611185753558\\
13.4502777777778	-0.00662228804210381\\
13.4558333333333	-0.00654756944758266\\
13.4613888888889	-0.00659308601256046\\
13.4669444444444	-0.00624186450818494\\
13.4725	-0.00614575344764797\\
13.4780555555556	-0.00674688484442274\\
13.4836111111111	-0.00678957068481508\\
13.4891666666667	-0.00649753291932683\\
13.4947222222222	-0.00683080618529531\\
13.5002777777778	-0.00697340768670355\\
13.5058333333333	-0.00663564399536122\\
13.5113888888889	-0.00796889789180878\\
13.5169444444444	-0.00955473981258121\\
13.5225	-0.0101997699855409\\
13.5280555555556	-0.0101616528489127\\
13.5336111111111	-0.0106918348171619\\
13.5391666666667	-0.0101934063695176\\
13.5447222222222	-0.01032133991892\\
13.5502777777778	-0.0097538516505669\\
13.5558333333333	-0.00911094886514986\\
13.5613888888889	-0.00876436003643747\\
13.5669444444444	-0.00821293209980741\\
13.5725	-0.00835204611038573\\
13.5780555555556	-0.00804029644791126\\
13.5836111111111	-0.00768546266089829\\
13.5891666666667	-0.00776129303452839\\
13.5947222222222	-0.00786099149571988\\
13.6002777777778	-0.00746001174433892\\
13.6058333333333	-0.00775948362932698\\
13.6113888888889	-0.00685614980763643\\
13.6169444444444	-0.00671143871451323\\
13.6225	-0.00653796629105868\\
13.6280555555556	-0.00531120571298835\\
13.6336111111111	-0.00572028341998512\\
13.6391666666667	-0.00538729262442913\\
13.6447222222222	-0.00502714240578628\\
13.6502777777778	-0.00552574776609775\\
13.6558333333333	-0.00512777021418943\\
13.6613888888889	-0.00553218512363181\\
13.6669444444444	-0.00508722455050351\\
13.6725	-0.00494980197527906\\
13.6780555555556	-0.00580524065487332\\
13.6836111111111	-0.00528557337536061\\
13.6891666666667	-0.0053217971794841\\
13.6947222222222	-0.00534471714818828\\
13.7002777777778	-0.00599671620779005\\
13.7058333333333	-0.00663130139885809\\
13.7113888888889	-0.00661148083951763\\
13.7169444444444	-0.00701239900777122\\
13.7225	-0.00672132634018766\\
13.7280555555556	-0.00538820495570163\\
13.7336111111111	-0.00484102374652252\\
13.7391666666667	-0.00357813728787782\\
13.7447222222222	-0.00239131220769622\\
13.7502777777778	-0.00275167540117678\\
13.7558333333333	-0.00252527197522017\\
13.7613888888889	-0.00268722105436327\\
13.7669444444444	-0.00328533523010807\\
13.7725	-0.0046205656978102\\
13.7780555555556	-0.00560168567066331\\
13.7836111111111	-0.00600859348037624\\
13.7891666666667	-0.00695650941042456\\
13.7947222222222	-0.00645434730313306\\
13.8002777777778	-0.00653126891046084\\
13.8058333333333	-0.00626625397665484\\
13.8113888888889	-0.00608739929748007\\
13.8169444444444	-0.00747433841172336\\
13.8225	-0.00613354802734125\\
13.8280555555556	-0.0053898499389069\\
13.8336111111111	-0.00480900714800762\\
13.8391666666667	-0.00485779800442948\\
13.8447222222222	-0.00461050770058146\\
13.8502777777778	-0.00481816652337504\\
13.8558333333333	-0.00507650476382441\\
13.8613888888889	-0.00542546510547726\\
13.8669444444444	-0.00521698689971106\\
13.8725	-0.00525022844440476\\
13.8780555555556	-0.005081954478155\\
13.8836111111111	-0.00510802977091056\\
13.8891666666667	-0.00477467622810594\\
13.8947222222222	-0.00467701074887938\\
13.9002777777778	-0.00505706608641275\\
13.9058333333333	-0.00512754231321934\\
13.9113888888889	-0.00442198592850514\\
13.9169444444444	-0.00471621245253138\\
13.9225	-0.00342022766932721\\
13.9280555555556	-0.00387412772939249\\
13.9336111111111	-0.00398416532400632\\
13.9391666666667	-0.00372949610730202\\
13.9447222222222	-0.00318985191750241\\
13.9502777777778	-0.00257410225072352\\
13.9558333333333	-0.00415020355085574\\
13.9613888888889	-0.00402115149929561\\
13.9669444444444	-0.00492645972888966\\
13.9725	-0.0063083714176939\\
13.9780555555556	-0.00562591721921379\\
13.9836111111111	-0.00548064446752273\\
13.9891666666667	-0.00434730681599563\\
13.9947222222222	-0.00454809414881479\\
14.0002777777778	-0.00598000499128748\\
14.0058333333333	-0.00459185805773152\\
14.0113888888889	-0.00481739248529005\\
14.0169444444444	-0.00402188911820054\\
14.0225	-0.00604696497104835\\
14.0280555555556	-0.00591651171227123\\
14.0336111111111	-0.00709465271975586\\
14.0391666666667	-0.00833526414776549\\
14.0447222222222	-0.00771806377566428\\
14.0502777777778	-0.00804126461285485\\
14.0558333333333	-0.00804810067972529\\
14.0613888888889	-0.00783677162437905\\
14.0669444444444	-0.00773225184539134\\
14.0725	-0.00962019484409095\\
14.0780555555556	-0.00822629593524091\\
14.0836111111111	-0.00953026275160404\\
14.0891666666667	-0.0110472655932699\\
14.0947222222222	-0.0105305283011586\\
14.1002777777778	-0.0115147259590457\\
14.1058333333333	-0.0112091831763572\\
14.1113888888889	-0.0131408373710742\\
14.1169444444444	-0.0129264827857131\\
14.1225	-0.0128487075576798\\
14.1280555555556	-0.0130940761835452\\
14.1336111111111	-0.0137096328002743\\
14.1391666666667	-0.0132394078764364\\
14.1447222222222	-0.0137850149082609\\
14.1502777777778	-0.0141323365797537\\
14.1558333333333	-0.0151185195665404\\
14.1613888888889	-0.0144792641227554\\
14.1669444444444	-0.0144981376506179\\
14.1725	-0.0141235963038678\\
14.1780555555556	-0.0141831161612199\\
14.1836111111111	-0.0152469454474\\
14.1891666666667	-0.0152568406272452\\
14.1947222222222	-0.0142395156696734\\
14.2002777777778	-0.0136515681735984\\
14.2058333333333	-0.0145027225217814\\
14.2113888888889	-0.0137811070333727\\
14.2169444444444	-0.0132899573134508\\
14.2225	-0.0127501745739204\\
14.2280555555556	-0.0138815415322978\\
14.2336111111111	-0.0145382919794732\\
14.2391666666667	-0.0150939863972047\\
14.2447222222222	-0.0146997316928911\\
14.2502777777778	-0.0148268824521888\\
14.2558333333333	-0.0142360950104363\\
14.2613888888889	-0.0129127407336788\\
14.2669444444444	-0.0127224857564499\\
14.2725	-0.0126788353231947\\
14.2780555555556	-0.0131590874251076\\
14.2836111111111	-0.0137553053332603\\
14.2891666666667	-0.0136428729992233\\
14.2947222222222	-0.0132513459904474\\
14.3002777777778	-0.0122932121904305\\
14.3058333333333	-0.0125766137221021\\
14.3113888888889	-0.0123464528721993\\
14.3169444444444	-0.0123807021946748\\
14.3225	-0.0126305205925444\\
14.3280555555556	-0.0136495069764995\\
14.3336111111111	-0.0143395922979931\\
14.3391666666667	-0.0139252567285785\\
14.3447222222222	-0.0135287562194336\\
14.3502777777778	-0.0130274017431903\\
14.3558333333333	-0.0129858290532531\\
14.3613888888889	-0.0124823676427177\\
14.3669444444444	-0.0119646286149094\\
14.3725	-0.0115196528442153\\
14.3780555555556	-0.0105676097126483\\
14.3836111111111	-0.00957308016651788\\
14.3891666666667	-0.0103735229018722\\
14.3947222222222	-0.0111424375991493\\
14.4002777777778	-0.0102359220750557\\
14.4058333333333	-0.00880742733309068\\
14.4113888888889	-0.00985146935775221\\
14.4169444444444	-0.0103707962195886\\
14.4225	-0.00923015096588768\\
14.4280555555556	-0.00920451029190733\\
14.4336111111111	-0.00851832051178475\\
14.4391666666667	-0.00837722289851363\\
14.4447222222222	-0.00792715402306173\\
14.4502777777778	-0.00843064673318617\\
14.4558333333333	-0.00839064294144125\\
14.4613888888889	-0.00786538036106642\\
14.4669444444444	-0.00758688208540681\\
14.4725	-0.00737702560582276\\
14.4780555555556	-0.00725925665447203\\
14.4836111111111	-0.00710273328765602\\
14.4891666666667	-0.00710051976358569\\
14.4947222222222	-0.00841449825696933\\
14.5002777777778	-0.00787101530644596\\
14.5058333333333	-0.0070998280261986\\
14.5113888888889	-0.00633625202867297\\
14.5169444444444	-0.00588156518305138\\
14.5225	-0.00568755868082941\\
14.5280555555556	-0.00542214579053203\\
14.5336111111111	-0.00420495188977127\\
14.5391666666667	-0.00493553481267108\\
14.5447222222222	-0.00590304615704301\\
14.5502777777778	-0.00640205294281199\\
14.5558333333333	-0.00603838601149599\\
14.5613888888889	-0.00623313018913625\\
14.5669444444444	-0.00666262654615994\\
14.5725	-0.00594965658836991\\
14.5780555555556	-0.00548672646122148\\
14.5836111111111	-0.00500440809331768\\
14.5891666666667	-0.00623723315601617\\
14.5947222222222	-0.00634568839119547\\
14.6002777777778	-0.00543700213309166\\
14.6058333333333	-0.00576152831099657\\
14.6113888888889	-0.00764756931163963\\
14.6169444444444	-0.00722983435321311\\
14.6225	-0.00798263550710804\\
14.6280555555556	-0.00906081393151\\
14.6336111111111	-0.00910528343622248\\
14.6391666666667	-0.0101012398605579\\
14.6447222222222	-0.0108526368485759\\
14.6502777777778	-0.0101390926100786\\
14.6558333333333	-0.00853242472155498\\
14.6613888888889	-0.00772243105213874\\
14.6669444444444	-0.00835118995098731\\
14.6725	-0.00860638558307315\\
14.6780555555556	-0.0100816123575268\\
14.6836111111111	-0.00908745132695281\\
14.6891666666667	-0.008487282169996\\
14.6947222222222	-0.00858842133564964\\
14.7002777777778	-0.00812615308434891\\
14.7058333333333	-0.0070296804796532\\
14.7113888888889	-0.00780980385991558\\
14.7169444444444	-0.00651163215252177\\
14.7225	-0.00675412866492301\\
14.7280555555556	-0.00772988911250241\\
14.7336111111111	-0.00618505688245075\\
14.7391666666667	-0.00602507970489329\\
14.7447222222222	-0.00649000364442071\\
14.7502777777778	-0.00609260949189814\\
14.7558333333333	-0.00451794678752546\\
14.7613888888889	-0.00421263179243\\
14.7669444444444	-0.00370013379906204\\
14.7725	-0.00365883487316564\\
14.7780555555556	-0.00311957484919281\\
14.7836111111111	-0.00254447618834468\\
14.7891666666667	-0.00357028693230231\\
14.7947222222222	-0.00274969071403118\\
14.8002777777778	-0.00244575587176605\\
14.8058333333333	-0.00276343080561754\\
14.8113888888889	-0.00333758925157561\\
14.8169444444444	-0.00327523445256975\\
14.8225	-0.00347548547382244\\
14.8280555555556	-0.00217933475349104\\
14.8336111111111	-0.00240042673062404\\
14.8391666666667	-0.00353888808401782\\
14.8447222222222	-0.00313654797935807\\
14.8502777777778	-0.00435953642935036\\
14.8558333333333	-0.00398012552961673\\
14.8613888888889	-0.00417015520506458\\
14.8669444444444	-0.00254356938484367\\
14.8725	-0.00163524274042692\\
14.8780555555556	-0.00264181359608739\\
14.8836111111111	-0.00207754131291663\\
14.8891666666667	-0.000652680838115768\\
14.8947222222222	-0.00287365871089355\\
14.9002777777778	-0.00333805387142057\\
14.9058333333333	-0.00306716366097269\\
14.9113888888889	-0.00315667554669947\\
14.9169444444444	-0.00291827197614096\\
14.9225	-0.00248694473909764\\
14.9280555555556	-0.00120133847148782\\
14.9336111111111	-0.000788254925377721\\
14.9391666666667	0.000555575909861689\\
14.9447222222222	0.000361736658706593\\
14.9502777777778	-0.00101739862403415\\
14.9558333333333	-0.000456282296572344\\
14.9613888888889	0.000388309906454094\\
14.9669444444444	0.0011138118229512\\
14.9725	0.000657885061085926\\
14.9780555555556	0.0016009329538823\\
14.9836111111111	0.00107396569049708\\
14.9891666666667	0.000658315911111941\\
14.9947222222222	8.09804710761907e-05\\
15.0002777777778	-0.000409581709339677\\
15.0058333333333	-0.000690690698581774\\
15.0113888888889	-0.00125707387600945\\
15.0169444444444	-0.00117520213362488\\
15.0225	-0.000725420786017831\\
15.0280555555556	-0.00113444421062391\\
15.0336111111111	-0.000959123863608823\\
15.0391666666667	-0.000922185868086566\\
15.0447222222222	-0.000323652382227865\\
15.0502777777778	-0.000470030671273902\\
15.0558333333333	-0.000331812402205373\\
15.0613888888889	-0.00110147221300267\\
15.0669444444444	-0.00161192369240523\\
15.0725	-0.00144625814104129\\
15.0780555555556	-0.00150264028958876\\
15.0836111111111	-0.0013655743767919\\
15.0891666666667	-0.00215798867133396\\
15.0947222222222	-0.00253966674996826\\
15.1002777777778	-0.00298695948137349\\
15.1058333333333	-0.00343622362914545\\
15.1113888888889	-0.00318694170169471\\
15.1169444444444	-0.00375953021257213\\
15.1225	-0.00345519863241495\\
15.1280555555556	-0.00294535868322626\\
15.1336111111111	-0.00270680030306348\\
15.1391666666667	-0.00354523874429662\\
15.1447222222222	-0.0038794553796168\\
15.1502777777778	-0.00454055175215921\\
15.1558333333333	-0.00507515761630944\\
15.1613888888889	-0.00485153619055781\\
15.1669444444444	-0.00481237375091377\\
15.1725	-0.00536402773339604\\
15.1780555555556	-0.00375442713127117\\
15.1836111111111	-0.00368237945770743\\
15.1891666666667	-0.00276559638022167\\
15.1947222222222	-0.00261425247324741\\
15.2002777777778	-0.00246592611027273\\
15.2058333333333	-0.00306010897917127\\
15.2113888888889	-0.0038625411888806\\
15.2169444444444	-0.00399657720521298\\
15.2225	-0.00345175653036046\\
15.2280555555556	-0.00240207034839039\\
15.2336111111111	-0.00444916427161933\\
15.2391666666667	-0.00335934865258356\\
15.2447222222222	-0.00363657519914574\\
15.2502777777778	-0.00394441221293007\\
15.2558333333333	-0.0028100228220727\\
15.2613888888889	-0.00224076091563122\\
15.2669444444444	-0.00275628680384937\\
15.2725	-0.00317505217874557\\
15.2780555555556	-0.00390482906684855\\
15.2836111111111	-0.00376485230598025\\
15.2891666666667	-0.00458698381187091\\
15.2947222222222	-0.00452773359711813\\
15.3002777777778	-0.00387578814516842\\
15.3058333333333	-0.00633606478377011\\
15.3113888888889	-0.00751786844655707\\
15.3169444444444	-0.00832334846904011\\
15.3225	-0.00853815147160667\\
15.3280555555556	-0.0084420345544545\\
15.3336111111111	-0.00804481122887219\\
15.3391666666667	-0.00904970573521\\
15.3447222222222	-0.00739616151376268\\
15.3502777777778	-0.00826835521824595\\
15.3558333333333	-0.00829909966083354\\
15.3613888888889	-0.00805882563807117\\
15.3669444444444	-0.00926045302436979\\
15.3725	-0.00906301850271348\\
15.3780555555556	-0.00837648877382736\\
15.3836111111111	-0.00792624514378125\\
15.3891666666667	-0.00797288912034757\\
15.3947222222222	-0.00793794602626256\\
15.4002777777778	-0.00761809150426281\\
15.4058333333333	-0.00718018645392928\\
15.4113888888889	-0.00784566883369138\\
15.4169444444444	-0.00834619771849569\\
15.4225	-0.00814827350663206\\
15.4280555555556	-0.00824803478033772\\
15.4336111111111	-0.00876516564504224\\
15.4391666666667	-0.00841913217483499\\
15.4447222222222	-0.0083196935011529\\
15.4502777777778	-0.00957695206171771\\
15.4558333333333	-0.0101181508993023\\
15.4613888888889	-0.00984575789407323\\
15.4669444444444	-0.0117614040220872\\
15.4725	-0.0118955149864457\\
15.4780555555556	-0.0124065669596446\\
15.4836111111111	-0.0134370352416702\\
15.4891666666667	-0.0135151539909905\\
15.4947222222222	-0.0143329828499895\\
15.5002777777778	-0.0141145067223378\\
15.5058333333333	-0.013826847969419\\
15.5113888888889	-0.0138385736418172\\
15.5169444444444	-0.0133507167235435\\
15.5225	-0.0126057344730874\\
15.5280555555556	-0.0139110694751884\\
15.5336111111111	-0.0135926733130742\\
15.5391666666667	-0.0144230143111523\\
15.5447222222222	-0.013842196658374\\
15.5502777777778	-0.0146306281047359\\
15.5558333333333	-0.0139896466893955\\
15.5613888888889	-0.0131790623275268\\
15.5669444444444	-0.0137430691145898\\
15.5725	-0.0141418819114314\\
15.5780555555556	-0.0135590284151066\\
15.5836111111111	-0.0133709705626107\\
15.5891666666667	-0.0122054589944673\\
15.5947222222222	-0.0120531551126168\\
15.6002777777778	-0.0131425643241552\\
15.6058333333333	-0.0121658790621896\\
15.6113888888889	-0.0143249568774837\\
15.6169444444444	-0.0139276886214034\\
15.6225	-0.0135619764958239\\
15.6280555555556	-0.0129433482319623\\
15.6336111111111	-0.0151593692627004\\
15.6391666666667	-0.0155641047500658\\
15.6447222222222	-0.0151907047342282\\
15.6502777777778	-0.014099891729207\\
15.6558333333333	-0.015244113285566\\
15.6613888888889	-0.0162130633921167\\
15.6669444444444	-0.0180874871306779\\
15.6725	-0.0157076141542893\\
15.6780555555556	-0.0154602409646335\\
15.6836111111111	-0.0156724664741987\\
15.6891666666667	-0.0157660508832735\\
15.6947222222222	-0.0182146962545835\\
15.7002777777778	-0.0157404939944859\\
15.7058333333333	-0.0159227399208004\\
15.7113888888889	-0.014737242705222\\
15.7169444444444	-0.0139518047103942\\
15.7225	-0.0144233421951046\\
15.7280555555556	-0.0142919924314975\\
15.7336111111111	-0.0144778908569434\\
15.7391666666667	-0.015381903997072\\
15.7447222222222	-0.014626593975499\\
15.7502777777778	-0.0179909139404739\\
15.7558333333333	-0.0166677393052304\\
15.7613888888889	-0.0159806039104449\\
15.7669444444444	-0.0159736540329543\\
15.7725	-0.0159954262242296\\
15.7780555555556	-0.0150980675857258\\
15.7836111111111	-0.0151441421590557\\
15.7891666666667	-0.0143694542230159\\
15.7947222222222	-0.0139690750097726\\
15.8002777777778	-0.0150297521925222\\
15.8058333333333	-0.0139827205128389\\
15.8113888888889	-0.0131725001356074\\
15.8169444444444	-0.0135874562170061\\
15.8225	-0.013662454481162\\
15.8280555555556	-0.0124869555708878\\
15.8336111111111	-0.0105241899907609\\
15.8391666666667	-0.0101359876233995\\
15.8447222222222	-0.00861071966368249\\
15.8502777777778	-0.00844816848051772\\
15.8558333333333	-0.00753593350994935\\
15.8613888888889	-0.00776341196462378\\
15.8669444444444	-0.00717360015018841\\
15.8725	-0.00658068222205901\\
15.8780555555556	-0.00624225035407267\\
15.8836111111111	-0.00661927237776876\\
15.8891666666667	-0.00664814711956104\\
15.8947222222222	-0.00649633432240005\\
15.9002777777778	-0.00539410999121325\\
15.9058333333333	-0.00452246782663139\\
15.9113888888889	-0.00584502436054512\\
15.9169444444444	-0.00560154504636914\\
15.9225	-0.00480440013942322\\
15.9280555555556	-0.00583370341564927\\
15.9336111111111	-0.00553339356785002\\
15.9391666666667	-0.00630705308207743\\
15.9447222222222	-0.00661363458782192\\
15.9502777777778	-0.00835833246016834\\
15.9558333333333	-0.00827292209200582\\
15.9613888888889	-0.00777376113790102\\
15.9669444444444	-0.00715687612726455\\
15.9725	-0.00736378433169585\\
15.9780555555556	-0.0069434245053543\\
15.9836111111111	-0.00714243103227136\\
15.9891666666667	-0.00773977450296018\\
15.9947222222222	-0.00645423925970033\\
};
\end{axis}
\end{tikzpicture}%
%
\end{subfigure}\\%
\vspace{1cm}%
\begin{subfigure}[b]{.4\linewidth}%
  %\centering
  \setlength\figureheight{\linewidth}%
  \setlength\figurewidth{\linewidth}%
  \tikzsetnextfilename{Ch2/ORCL_naive_strat_comp}%
  % This file was created by matlab2tikz.
%
%The latest updates can be retrieved from
%  http://www.mathworks.com/matlabcentral/fileexchange/22022-matlab2tikz-matlab2tikz
%where you can also make suggestions and rate matlab2tikz.
%
\definecolor{mycolor1}{rgb}{0.00000,0.00000,0.00000}%
\definecolor{mycolor2}{rgb}{0.40000,0.40000,0.40000}%
\definecolor{mycolor3}{rgb}{0.70000,0.70000,0.70000}%
%
\begin{tikzpicture}[trim axis left, trim axis right]

\begin{axis}[%
width=\figurewidth,
height=\figureheight,
at={(0\figurewidth,0\figureheight)},
scale only axis,
every outer x axis line/.append style={black},
every x tick label/.append style={font=\color{black}},
xmin=9.5,
xmax=16,
every outer y axis line/.append style={black},
every y tick label/.append style={font=\color{black}},
ymin=-0.25,
ymax=0.5,
title={ORCL},
axis background/.style={fill=white},
axis x line*=bottom,
axis y line*=left
]
\addplot [color=mycolor1,solid,line width=1.5pt,forget plot]
  table[row sep=crcr]{%
9.50027777777778	0\\
9.50583333333333	-0.00251866868840465\\
9.51138888888889	-0.00416320489159229\\
9.51694444444444	-0.00569768514295216\\
9.5225	-0.0074220094042891\\
9.52805555555556	-0.00729951352298086\\
9.53361111111111	-0.00827140207556348\\
9.53916666666667	-0.00907384955459683\\
9.54472222222222	-0.00941345998716128\\
9.55027777777778	-0.00995046457141032\\
9.55583333333333	-0.0100481207070692\\
9.56138888888889	-0.0101235870817901\\
9.56694444444444	-0.0100761311735677\\
9.5725	-0.0107078368918143\\
9.57805555555555	-0.0112449725954927\\
9.58361111111111	-0.0117240888034633\\
9.58916666666667	-0.0124112673304792\\
9.59472222222222	-0.012855712458319\\
9.60027777777778	-0.0122024551485652\\
9.60583333333333	-0.0126248207708646\\
9.61138888888889	-0.0127907431304789\\
9.61694444444444	-0.0130152441706339\\
9.6225	-0.0129703789779459\\
9.62805555555556	-0.0126316670979517\\
9.63361111111111	-0.0132683831475815\\
9.63916666666667	-0.0131148727810126\\
9.64472222222222	-0.0138007711138083\\
9.65027777777778	-0.014078816864429\\
9.65583333333333	-0.0141166699402395\\
9.66138888888889	-0.0144192476307414\\
9.66694444444444	-0.0148972388356188\\
9.6725	-0.0147975851645547\\
9.67805555555555	-0.0151773395089842\\
9.68361111111111	-0.0155392819804846\\
9.68916666666667	-0.0152418818589089\\
9.69472222222222	-0.0159438198312221\\
9.70027777777778	-0.0150698770078797\\
9.70583333333333	-0.0146963102208897\\
9.71138888888889	-0.015414902506762\\
9.71694444444444	-0.0157430033701571\\
9.7225	-0.0162396476852457\\
9.72805555555555	-0.0175748675031368\\
9.73361111111111	-0.0182707924228703\\
9.73916666666667	-0.0180909976690031\\
9.74472222222222	-0.0175659387335111\\
9.75027777777778	-0.0188889937087382\\
9.75583333333333	-0.0194003485381888\\
9.76138888888889	-0.0199877130406889\\
9.76694444444444	-0.020446360877141\\
9.7725	-0.0217948630309615\\
9.77805555555556	-0.0219152027717397\\
9.78361111111111	-0.0219049656768217\\
9.78916666666667	-0.0215328272059266\\
9.79472222222222	-0.0216439765953865\\
9.80027777777778	-0.0214539737328235\\
9.80583333333333	-0.0215013533950004\\
9.81138888888889	-0.0219831470135609\\
9.81694444444444	-0.0213489026024317\\
9.8225	-0.0221466807472572\\
9.82805555555555	-0.0228038531831655\\
9.83361111111111	-0.0230450133039184\\
9.83916666666667	-0.0236008511918116\\
9.84472222222222	-0.0238815044534093\\
9.85027777777778	-0.0235627340378199\\
9.85583333333333	-0.0245697961785512\\
9.86138888888889	-0.0257010982784749\\
9.86694444444444	-0.0257158546386733\\
9.8725	-0.0257531921635968\\
9.87805555555556	-0.0268646309096931\\
9.88361111111111	-0.0270512634442396\\
9.88916666666667	-0.0273177844068347\\
9.89472222222222	-0.0280359451196672\\
9.90027777777778	-0.0284849821359688\\
9.90583333333333	-0.027633233514096\\
9.91138888888889	-0.0275038214008793\\
9.91694444444444	-0.0270025668203362\\
9.9225	-0.0278580114484786\\
9.92805555555555	-0.0291114170665098\\
9.93361111111111	-0.0280037684987452\\
9.93916666666667	-0.0298971128153998\\
9.94472222222222	-0.0309200699638076\\
9.95027777777778	-0.0311975665695818\\
9.95583333333333	-0.0308240500037536\\
9.96138888888889	-0.03112261877765\\
9.96694444444444	-0.0313588654821426\\
9.9725	-0.0308935165770913\\
9.97805555555555	-0.0306355526825182\\
9.98361111111111	-0.0299505458863203\\
9.98916666666667	-0.0308758632448606\\
9.99472222222222	-0.0311723544352085\\
10.0002777777778	-0.031300565452067\\
10.0058333333333	-0.0321583108784538\\
10.0113888888889	-0.0331138869854326\\
10.0169444444444	-0.0314981364997293\\
10.0225	-0.0322496452102265\\
10.0280555555556	-0.0329448229857544\\
10.0336111111111	-0.0327411749959374\\
10.0391666666667	-0.0323507834093694\\
10.0447222222222	-0.0319118540352493\\
10.0502777777778	-0.0321604469386132\\
10.0558333333333	-0.0318918031444823\\
10.0613888888889	-0.0326256649157676\\
10.0669444444444	-0.0331956802916445\\
10.0725	-0.0327759395968464\\
10.0780555555556	-0.03225314341663\\
10.0836111111111	-0.03365868000235\\
10.0891666666667	-0.0341469754222182\\
10.0947222222222	-0.0348989551288841\\
10.1002777777778	-0.0357893187734246\\
10.1058333333333	-0.0349232741997106\\
10.1113888888889	-0.0351186678652549\\
10.1169444444444	-0.0348772368714428\\
10.1225	-0.0357396346565513\\
10.1280555555556	-0.0364585971475087\\
10.1336111111111	-0.037471735480518\\
10.1391666666667	-0.0371485795731262\\
10.1447222222222	-0.0384059289416052\\
10.1502777777778	-0.0376686615929433\\
10.1558333333333	-0.03729210513732\\
10.1613888888889	-0.037541653969403\\
10.1669444444444	-0.0375492341366605\\
10.1725	-0.037481252307358\\
10.1780555555556	-0.0374277712398248\\
10.1836111111111	-0.0373631137677867\\
10.1891666666667	-0.0377169310218142\\
10.1947222222222	-0.0385946364285898\\
10.2002777777778	-0.0369756083959431\\
10.2058333333333	-0.0368837555450883\\
10.2113888888889	-0.0371947147280481\\
10.2169444444444	-0.0361680005475584\\
10.2225	-0.0362922871505211\\
10.2280555555556	-0.0368957040683037\\
10.2336111111111	-0.0367285060196415\\
10.2391666666667	-0.0359326071089023\\
10.2447222222222	-0.034665111266651\\
10.2502777777778	-0.0337355580057602\\
10.2558333333333	-0.0343092553651142\\
10.2613888888889	-0.0342144061343623\\
10.2669444444444	-0.035971492851545\\
10.2725	-0.0356243584297724\\
10.2780555555556	-0.0349868959326017\\
10.2836111111111	-0.0356376618278302\\
10.2891666666667	-0.0357974353675859\\
10.2947222222222	-0.0360805717901715\\
10.3002777777778	-0.0357389046932461\\
10.3058333333333	-0.0360990067814422\\
10.3113888888889	-0.0353348701963132\\
10.3169444444444	-0.034566773739204\\
10.3225	-0.0339885788614852\\
10.3280555555556	-0.0348844909347718\\
10.3336111111111	-0.0358485958521415\\
10.3391666666667	-0.0354842914556083\\
10.3447222222222	-0.0355883096275688\\
10.3502777777778	-0.0360443842053464\\
10.3558333333333	-0.0363093279406341\\
10.3613888888889	-0.0348841838566046\\
10.3669444444444	-0.0349033165254642\\
10.3725	-0.0350539270278144\\
10.3780555555556	-0.0358592440121094\\
10.3836111111111	-0.0367625936350675\\
10.3891666666667	-0.0377828623269915\\
10.3947222222222	-0.0380430658438595\\
10.4002777777778	-0.0374619152822945\\
10.4058333333333	-0.0377454136029192\\
10.4113888888889	-0.0375762036946894\\
10.4169444444444	-0.037022848049276\\
10.4225	-0.0363189880172462\\
10.4280555555556	-0.0358580252863999\\
10.4336111111111	-0.0359271707681366\\
10.4391666666667	-0.0358701236087959\\
10.4447222222222	-0.0360384415956202\\
10.4502777777778	-0.0357132401041654\\
10.4558333333333	-0.0365334351233844\\
10.4613888888889	-0.0364857213257379\\
10.4669444444444	-0.0364217247543081\\
10.4725	-0.036492475643579\\
10.4780555555556	-0.0370277433302848\\
10.4836111111111	-0.0363891232568622\\
10.4891666666667	-0.0359395331629914\\
10.4947222222222	-0.036126593510254\\
10.5002777777778	-0.0357467138277366\\
10.5058333333333	-0.036846303919952\\
10.5113888888889	-0.0366468498748477\\
10.5169444444444	-0.0375218334741339\\
10.5225	-0.0371343630799786\\
10.5280555555556	-0.0372536336186196\\
10.5336111111111	-0.0372148078592389\\
10.5391666666667	-0.0394926218993975\\
10.5447222222222	-0.0410156102085565\\
10.5502777777778	-0.0415291949550318\\
10.5558333333333	-0.0415049568765901\\
10.5613888888889	-0.0408505582441173\\
10.5669444444444	-0.0409303987966059\\
10.5725	-0.0419736404770296\\
10.5780555555556	-0.0416634178391633\\
10.5836111111111	-0.0435789424393542\\
10.5891666666667	-0.043103797486981\\
10.5947222222222	-0.0433678540497325\\
10.6002777777778	-0.0430985511337294\\
10.6058333333333	-0.04365629703294\\
10.6113888888889	-0.0419934682652415\\
10.6169444444444	-0.0422445544525267\\
10.6225	-0.0431426853749906\\
10.6280555555556	-0.0436464407423261\\
10.6336111111111	-0.044122456283219\\
10.6391666666667	-0.0452510641883139\\
10.6447222222222	-0.0461945955582101\\
10.6502777777778	-0.0460877520053538\\
10.6558333333333	-0.0463470174270398\\
10.6613888888889	-0.0466056501400716\\
10.6669444444444	-0.0469704454147594\\
10.6725	-0.0465900268877497\\
10.6780555555556	-0.0466755270104422\\
10.6836111111111	-0.0462587004454538\\
10.6891666666667	-0.0470637558195193\\
10.6947222222222	-0.0470497227128174\\
10.7002777777778	-0.0458737481957881\\
10.7058333333333	-0.0462831215492671\\
10.7113888888889	-0.0460115062976672\\
10.7169444444444	-0.0463257932501963\\
10.7225	-0.0463467404789048\\
10.7280555555556	-0.046514527737766\\
10.7336111111111	-0.0464601318154394\\
10.7391666666667	-0.0461478329867858\\
10.7447222222222	-0.0458637336321072\\
10.7502777777778	-0.0449930044498402\\
10.7558333333333	-0.0461963530892491\\
10.7613888888889	-0.046266610041629\\
10.7669444444444	-0.0466887945128187\\
10.7725	-0.0458973975023969\\
10.7780555555556	-0.0460833807287468\\
10.7836111111111	-0.0459467292011157\\
10.7891666666667	-0.0468894276779238\\
10.7947222222222	-0.0474232709461586\\
10.8002777777778	-0.0472929127190717\\
10.8058333333333	-0.0468760123513193\\
10.8113888888889	-0.0464539803971429\\
10.8169444444444	-0.0467336372836038\\
10.8225	-0.0464486090032615\\
10.8280555555556	-0.0449119388462347\\
10.8336111111111	-0.044918744808474\\
10.8391666666667	-0.0455748839569626\\
10.8447222222222	-0.0457141832475908\\
10.8502777777778	-0.045316186305337\\
10.8558333333333	-0.045670900714829\\
10.8613888888889	-0.0451728087449168\\
10.8669444444444	-0.0459465503294097\\
10.8725	-0.0461747730510091\\
10.8780555555556	-0.0458919971394274\\
10.8836111111111	-0.0455845822663804\\
10.8891666666667	-0.0452540075979513\\
10.8947222222222	-0.0449118285621867\\
10.9002777777778	-0.0442276683980379\\
10.9058333333333	-0.0441608078639872\\
10.9113888888889	-0.0448177812727378\\
10.9169444444444	-0.0445073036827456\\
10.9225	-0.0442922146563184\\
10.9280555555556	-0.0435947820018691\\
10.9336111111111	-0.0442859768649693\\
10.9391666666667	-0.0456605249270522\\
10.9447222222222	-0.0461943271228112\\
10.9502777777778	-0.0455781963476752\\
10.9558333333333	-0.0451184988593536\\
10.9613888888889	-0.0450755815631698\\
10.9669444444444	-0.0467385207232017\\
10.9725	-0.0458301542939984\\
10.9780555555556	-0.047393160432172\\
10.9836111111111	-0.0478963841951889\\
10.9891666666667	-0.0479197209920431\\
10.9947222222222	-0.0483465475407552\\
11.0002777777778	-0.0496351159489455\\
11.0058333333333	-0.049184132562286\\
11.0113888888889	-0.0484240101899484\\
11.0169444444444	-0.048254704049724\\
11.0225	-0.0484433686719424\\
11.0280555555556	-0.0482444437823538\\
11.0336111111111	-0.049123335156841\\
11.0391666666667	-0.0493050113017226\\
11.0447222222222	-0.050402942502443\\
11.0502777777778	-0.0515968561888308\\
11.0558333333333	-0.0519882911065256\\
11.0613888888889	-0.0529517511232055\\
11.0669444444444	-0.0538724369024097\\
11.0725	-0.0533106042777667\\
11.0780555555556	-0.0537143673362166\\
11.0836111111111	-0.0547563586188247\\
11.0891666666667	-0.0548934843821108\\
11.0947222222222	-0.0554867437293101\\
11.1002777777778	-0.0556145255443678\\
11.1058333333333	-0.0535023829903946\\
11.1113888888889	-0.0538391475149978\\
11.1169444444444	-0.0536909215278567\\
11.1225	-0.0530609263786053\\
11.1280555555556	-0.0517978843865621\\
11.1336111111111	-0.0532419650625771\\
11.1391666666667	-0.0544515407676806\\
11.1447222222222	-0.053978483374903\\
11.1502777777778	-0.0534843219449055\\
11.1558333333333	-0.052702161904176\\
11.1613888888889	-0.0535028163538422\\
11.1669444444444	-0.053500772067925\\
11.1725	-0.0523272535761696\\
11.1780555555556	-0.0521325364693233\\
11.1836111111111	-0.0519171189802437\\
11.1891666666667	-0.0513398994518512\\
11.1947222222222	-0.0516373890354431\\
11.2002777777778	-0.0528311545217756\\
11.2058333333333	-0.0519532144151772\\
11.2113888888889	-0.0516970303299584\\
11.2169444444444	-0.0525740348977108\\
11.2225	-0.0530409756102832\\
11.2280555555556	-0.0532309101762679\\
11.2336111111111	-0.0527554298453215\\
11.2391666666667	-0.0525372737744132\\
11.2447222222222	-0.0531432400342154\\
11.2502777777778	-0.0535055637788933\\
11.2558333333333	-0.0528450304800313\\
11.2613888888889	-0.0534482564383007\\
11.2669444444444	-0.0531269416458805\\
11.2725	-0.0528592210042259\\
11.2780555555556	-0.0535003528196217\\
11.2836111111111	-0.0535985436978982\\
11.2891666666667	-0.0535827741328365\\
11.2947222222222	-0.0539604516289882\\
11.3002777777778	-0.0541321106337488\\
11.3058333333333	-0.0545551405919456\\
11.3113888888889	-0.054137506859646\\
11.3169444444444	-0.0538025688673844\\
11.3225	-0.0532392192920789\\
11.3280555555556	-0.0529529193094153\\
11.3336111111111	-0.0537433904584193\\
11.3391666666667	-0.0533561799827769\\
11.3447222222222	-0.0532843591208038\\
11.3502777777778	-0.0536194977265293\\
11.3558333333333	-0.0539364404432569\\
11.3613888888889	-0.0547281060171527\\
11.3669444444444	-0.0541379225671345\\
11.3725	-0.0542008470595515\\
11.3780555555556	-0.0549123469196423\\
11.3836111111111	-0.0550631013683881\\
11.3891666666667	-0.0553924826991943\\
11.3947222222222	-0.0550967939524358\\
11.4002777777778	-0.0549951597411768\\
11.4058333333333	-0.0547723985802137\\
11.4113888888889	-0.0548180494996876\\
11.4169444444444	-0.0551082717237604\\
11.4225	-0.0555068761261411\\
11.4280555555556	-0.0559650582214392\\
11.4336111111111	-0.0565061115807938\\
11.4391666666667	-0.0571053818186359\\
11.4447222222222	-0.0572888009300059\\
11.4502777777778	-0.0572062356865447\\
11.4558333333333	-0.057535777832338\\
11.4613888888889	-0.0571877768184235\\
11.4669444444444	-0.0576681975753085\\
11.4725	-0.058546251182901\\
11.4780555555556	-0.0592384673081985\\
11.4836111111111	-0.0582049466125668\\
11.4891666666667	-0.0591401998290366\\
11.4947222222222	-0.0579295028482124\\
11.5002777777778	-0.0581918645593011\\
11.5058333333333	-0.0584355087626488\\
11.5113888888889	-0.0585323497764411\\
11.5169444444444	-0.0584433818869067\\
11.5225	-0.0577931310012069\\
11.5280555555556	-0.0576946942522592\\
11.5336111111111	-0.0581926577091962\\
11.5391666666667	-0.0584129802494833\\
11.5447222222222	-0.0587170891800698\\
11.5502777777778	-0.0576681607109381\\
11.5558333333333	-0.057749855686682\\
11.5613888888889	-0.0570660309167444\\
11.5669444444444	-0.0560991776822232\\
11.5725	-0.0558471501881155\\
11.5780555555556	-0.055853987178223\\
11.5836111111111	-0.0556451340580556\\
11.5891666666667	-0.056595191329723\\
11.5947222222222	-0.0559480621282486\\
11.6002777777778	-0.0570386346550886\\
11.6058333333333	-0.0578352329337689\\
11.6113888888889	-0.0584736180087915\\
11.6169444444444	-0.0582135300332695\\
11.6225	-0.0588743471605483\\
11.6280555555556	-0.0592390426396446\\
11.6336111111111	-0.0589301299110879\\
11.6391666666667	-0.0584619980329878\\
11.6447222222222	-0.0583436462641626\\
11.6502777777778	-0.0580436541608826\\
11.6558333333333	-0.0576627067207125\\
11.6613888888889	-0.056713296942128\\
11.6669444444444	-0.0574406355082685\\
11.6725	-0.0578403928960405\\
11.6780555555556	-0.0589102519597309\\
11.6836111111111	-0.0592058954188532\\
11.6891666666667	-0.0593420730550852\\
11.6947222222222	-0.0598250308232701\\
11.7002777777778	-0.0590900152682458\\
11.7058333333333	-0.0589836895329908\\
11.7113888888889	-0.0585252385546756\\
11.7169444444444	-0.0577663655318181\\
11.7225	-0.0574884561763197\\
11.7280555555556	-0.0582075680583659\\
11.7336111111111	-0.0578906304658426\\
11.7391666666667	-0.0582746818752459\\
11.7447222222222	-0.058113575951384\\
11.7502777777778	-0.0580685464066746\\
11.7558333333333	-0.0580954473202174\\
11.7613888888889	-0.0576496255185493\\
11.7669444444444	-0.0581286309974995\\
11.7725	-0.0591600067139845\\
11.7780555555556	-0.058446389449559\\
11.7836111111111	-0.0590977939987737\\
11.7891666666667	-0.0593888817988683\\
11.7947222222222	-0.0590936069117686\\
11.8002777777778	-0.0587504207458762\\
11.8058333333333	-0.0595076060395124\\
11.8113888888889	-0.0597332652087623\\
11.8169444444444	-0.0589541323703715\\
11.8225	-0.0593414915707987\\
11.8280555555556	-0.0598669225945596\\
11.8336111111111	-0.0604280874790347\\
11.8391666666667	-0.0608461613344905\\
11.8447222222222	-0.0617380127580984\\
11.8502777777778	-0.0614651457911753\\
11.8558333333333	-0.0621822136236743\\
11.8613888888889	-0.0623128753328126\\
11.8669444444444	-0.061510456742857\\
11.8725	-0.0625631439109253\\
11.8780555555556	-0.062785052067946\\
11.8836111111111	-0.0636818269918126\\
11.8891666666667	-0.0623612111320731\\
11.8947222222222	-0.0617236848034054\\
11.9002777777778	-0.0615361840501133\\
11.9058333333333	-0.061608484311756\\
11.9113888888889	-0.0619886622734021\\
11.9169444444444	-0.0622596467560848\\
11.9225	-0.0624776863877243\\
11.9280555555556	-0.0637919606702826\\
11.9336111111111	-0.0644305522808085\\
11.9391666666667	-0.0638878168437694\\
11.9447222222222	-0.0638872598633224\\
11.9502777777778	-0.065103018523817\\
11.9558333333333	-0.064765513168125\\
11.9613888888889	-0.0635969449148571\\
11.9669444444444	-0.0635353355633509\\
11.9725	-0.0629607024959181\\
11.9780555555556	-0.0621245256005621\\
11.9836111111111	-0.0623619476996155\\
11.9891666666667	-0.0626136868278868\\
11.9947222222222	-0.0628539180231127\\
12.0002777777778	-0.0637085283653682\\
12.0058333333333	-0.0634783642545506\\
12.0113888888889	-0.0632331872793582\\
12.0169444444444	-0.063483250278486\\
12.0225	-0.0632728171209387\\
12.0280555555556	-0.0625483786662335\\
12.0336111111111	-0.062492437411522\\
12.0391666666667	-0.0631970388117079\\
12.0447222222222	-0.0646686016284712\\
12.0502777777778	-0.0655791051267038\\
12.0558333333333	-0.064790111451146\\
12.0613888888889	-0.0652703878764958\\
12.0669444444444	-0.0632564244516173\\
12.0725	-0.0636458008326878\\
12.0780555555556	-0.0628654846546931\\
12.0836111111111	-0.0623249420065833\\
12.0891666666667	-0.0607178614446831\\
12.0947222222222	-0.0605720498994327\\
12.1002777777778	-0.0615921802360498\\
12.1058333333333	-0.0617099808812625\\
12.1113888888889	-0.0628131678261048\\
12.1169444444444	-0.0633892153118737\\
12.1225	-0.0630598479288368\\
12.1280555555556	-0.0624482590671182\\
12.1336111111111	-0.0634467512936762\\
12.1391666666667	-0.0633627355839357\\
12.1447222222222	-0.0637907518883662\\
12.1502777777778	-0.0643930449537644\\
12.1558333333333	-0.0652110923682575\\
12.1613888888889	-0.0654961678457931\\
12.1669444444444	-0.0653773549258824\\
12.1725	-0.064536119147063\\
12.1780555555556	-0.0648765925628538\\
12.1836111111111	-0.0641734264131848\\
12.1891666666667	-0.0643163520651343\\
12.1947222222222	-0.0641949388333305\\
12.2002777777778	-0.0644268102573625\\
12.2058333333333	-0.0644701683425692\\
12.2113888888889	-0.063932825622121\\
12.2169444444444	-0.0637254571154087\\
12.2225	-0.0638540284544894\\
12.2280555555556	-0.0639653391362591\\
12.2336111111111	-0.063735928528805\\
12.2391666666667	-0.063760862650193\\
12.2447222222222	-0.0638942019490427\\
12.2502777777778	-0.0637864387120422\\
12.2558333333333	-0.062738241673934\\
12.2613888888889	-0.0637085173713688\\
12.2669444444444	-0.0639266265763936\\
12.2725	-0.06368575715825\\
12.2780555555556	-0.0639080130580176\\
12.2836111111111	-0.0636422510093715\\
12.2891666666667	-0.0644121750729798\\
12.2947222222222	-0.0644724841933679\\
12.3002777777778	-0.064146641516878\\
12.3058333333333	-0.0641720693430753\\
12.3113888888889	-0.0638273488648626\\
12.3169444444444	-0.0637476255519984\\
12.3225	-0.0634136560749044\\
12.3280555555556	-0.0642345512247377\\
12.3336111111111	-0.0636440193453761\\
12.3391666666667	-0.0643815292639165\\
12.3447222222222	-0.0641872487656899\\
12.3502777777778	-0.064265653030454\\
12.3558333333333	-0.0639178098818937\\
12.3613888888889	-0.0638340669811926\\
12.3669444444444	-0.0633264166211391\\
12.3725	-0.0633918770422199\\
12.3780555555556	-0.064233587762983\\
12.3836111111111	-0.0659148492633326\\
12.3891666666667	-0.066100021736031\\
12.3947222222222	-0.0660506874437009\\
12.4002777777778	-0.0659545675015561\\
12.4058333333333	-0.0653596811425668\\
12.4113888888889	-0.0654794320440981\\
12.4169444444444	-0.06621316989814\\
12.4225	-0.0664343906859457\\
12.4280555555556	-0.0668314481427883\\
12.4336111111111	-0.0672331215800132\\
12.4391666666667	-0.0681930070900196\\
12.4447222222222	-0.0682119668608468\\
12.4502777777778	-0.0677658456904735\\
12.4558333333333	-0.0683862751286037\\
12.4613888888889	-0.0684742753559971\\
12.4669444444444	-0.0683341871473037\\
12.4725	-0.0686204840659816\\
12.4780555555556	-0.0694813044318845\\
12.4836111111111	-0.0700558511508143\\
12.4891666666667	-0.0702155936849503\\
12.4947222222222	-0.069239087356335\\
12.5002777777778	-0.0688751263921074\\
12.5058333333333	-0.0689859535463654\\
12.5113888888889	-0.0697861925675911\\
12.5169444444444	-0.0684472043698532\\
12.5225	-0.0679730753570509\\
12.5280555555556	-0.0691088600735892\\
12.5336111111111	-0.0689906370197755\\
12.5391666666667	-0.0692780903042874\\
12.5447222222222	-0.069392162766285\\
12.5502777777778	-0.0691698924612624\\
12.5558333333333	-0.068969489312635\\
12.5613888888889	-0.0691871617467522\\
12.5669444444444	-0.0697392491374908\\
12.5725	-0.0696466165266971\\
12.5780555555556	-0.0699773868530759\\
12.5836111111111	-0.0707344752419352\\
12.5891666666667	-0.0700960240309238\\
12.5947222222222	-0.0697771863434638\\
12.6002777777778	-0.0694914016417771\\
12.6058333333333	-0.0689366707588377\\
12.6113888888889	-0.068697662297832\\
12.6169444444444	-0.0688466052895974\\
12.6225	-0.0689587115043592\\
12.6280555555556	-0.0692283992382564\\
12.6336111111111	-0.0686314153011632\\
12.6391666666667	-0.0695282199287196\\
12.6447222222222	-0.070445505148021\\
12.6502777777778	-0.0692323868067357\\
12.6558333333333	-0.0679827535052856\\
12.6613888888889	-0.0685390518342916\\
12.6669444444444	-0.0683334748314689\\
12.6725	-0.068724681460064\\
12.6780555555556	-0.0689460742530821\\
12.6836111111111	-0.0695739882620954\\
12.6891666666667	-0.0695508599362886\\
12.6947222222222	-0.070035123804873\\
12.7002777777778	-0.0697758372621707\\
12.7058333333333	-0.0707632613340488\\
12.7113888888889	-0.0699886789258933\\
12.7169444444444	-0.0709015314857906\\
12.7225	-0.0716558365977214\\
12.7280555555556	-0.0709646413639881\\
12.7336111111111	-0.0720070879339145\\
12.7391666666667	-0.0708475217535108\\
12.7447222222222	-0.0709462424884107\\
12.7502777777778	-0.0692789179213509\\
12.7558333333333	-0.0678295047818962\\
12.7613888888889	-0.0682725945241056\\
12.7669444444444	-0.06835644145195\\
12.7725	-0.0687660476227011\\
12.7780555555556	-0.0694361369853321\\
12.7836111111111	-0.0690621308423384\\
12.7891666666667	-0.0674002611055317\\
12.7947222222222	-0.0666093934285541\\
12.8002777777778	-0.0670655011493398\\
12.8058333333333	-0.0658448457589304\\
12.8113888888889	-0.0654708350429466\\
12.8169444444444	-0.0645080045677019\\
12.8225	-0.0648852143332896\\
12.8280555555556	-0.0644413893706939\\
12.8336111111111	-0.0645611917192063\\
12.8391666666667	-0.0634775661243296\\
12.8447222222222	-0.0634977421581194\\
12.8502777777778	-0.0638859736598407\\
12.8558333333333	-0.0632314396235057\\
12.8613888888889	-0.0638297461319871\\
12.8669444444444	-0.0629092821234597\\
12.8725	-0.062247304163103\\
12.8780555555556	-0.0621404392132389\\
12.8836111111111	-0.0620584241557999\\
12.8891666666667	-0.0614939253293717\\
12.8947222222222	-0.0616834540884783\\
12.9002777777778	-0.0598900755227054\\
12.9058333333333	-0.0595176520964\\
12.9113888888889	-0.061100646216735\\
12.9169444444444	-0.0606300702614948\\
12.9225	-0.061033361448624\\
12.9280555555556	-0.0615883006212221\\
12.9336111111111	-0.0620089129505687\\
12.9391666666667	-0.0635806226684704\\
12.9447222222222	-0.0643618364919451\\
12.9502777777778	-0.0650220797716716\\
12.9558333333333	-0.0651903345409399\\
12.9613888888889	-0.0656072829132736\\
12.9669444444444	-0.065007359788257\\
12.9725	-0.0653035670100604\\
12.9780555555556	-0.0641633729519237\\
12.9836111111111	-0.0641221129124905\\
12.9891666666667	-0.0634938310952476\\
12.9947222222222	-0.0638549734463542\\
13.0002777777778	-0.0660182000865831\\
13.0058333333333	-0.0651325611947792\\
13.0113888888889	-0.065505867460093\\
13.0169444444444	-0.0674355582653507\\
13.0225	-0.0665269302606238\\
13.0280555555556	-0.0672345406632143\\
13.0336111111111	-0.0683413461743549\\
13.0391666666667	-0.0686163838156665\\
13.0447222222222	-0.0684789066757444\\
13.0502777777778	-0.0686191227268987\\
13.0558333333333	-0.0689921927820487\\
13.0613888888889	-0.0685892964068211\\
13.0669444444444	-0.0691537789223049\\
13.0725	-0.069530624840093\\
13.0780555555556	-0.0678961456320187\\
13.0836111111111	-0.0665722904836328\\
13.0891666666667	-0.0675117531969874\\
13.0947222222222	-0.0677374632255277\\
13.1002777777778	-0.0671300579631336\\
13.1058333333333	-0.0683371039622952\\
13.1113888888889	-0.0673726589572069\\
13.1169444444444	-0.0675085564519323\\
13.1225	-0.0676726193898492\\
13.1280555555556	-0.0682915295017374\\
13.1336111111111	-0.0672045199340039\\
13.1391666666667	-0.0681415136530183\\
13.1447222222222	-0.0681345155914728\\
13.1502777777778	-0.0698963183317884\\
13.1558333333333	-0.0689325158361092\\
13.1613888888889	-0.0693216002350385\\
13.1669444444444	-0.0674698591880779\\
13.1725	-0.0689199580207764\\
13.1780555555556	-0.0686105669574179\\
13.1836111111111	-0.0697220735244599\\
13.1891666666667	-0.0716298755864188\\
13.1947222222222	-0.0716961566181594\\
13.2002777777778	-0.0728192701480636\\
13.2058333333333	-0.0735965979850421\\
13.2113888888889	-0.07222496667037\\
13.2169444444444	-0.0725535117412333\\
13.2225	-0.0721998183798021\\
13.2280555555556	-0.0720741764424732\\
13.2336111111111	-0.0717513156400639\\
13.2391666666667	-0.072875454471861\\
13.2447222222222	-0.0731803269173441\\
13.2502777777778	-0.0730342054811034\\
13.2558333333333	-0.0718059787194068\\
13.2613888888889	-0.0723965189806506\\
13.2669444444444	-0.0722185076315224\\
13.2725	-0.072701679124902\\
13.2780555555556	-0.0743610195861822\\
13.2836111111111	-0.0740905381293862\\
13.2891666666667	-0.0746806810177618\\
13.2947222222222	-0.0749988577850566\\
13.3002777777778	-0.0754102645663259\\
13.3058333333333	-0.0738183556787349\\
13.3113888888889	-0.0728921018461691\\
13.3169444444444	-0.0732026811090462\\
13.3225	-0.0735315698965469\\
13.3280555555556	-0.0721703628350206\\
13.3336111111111	-0.0723936010529667\\
13.3391666666667	-0.0728259795207366\\
13.3447222222222	-0.0731903188490403\\
13.3502777777778	-0.0724392577092385\\
13.3558333333333	-0.0720215249364714\\
13.3613888888889	-0.0715361887705979\\
13.3669444444444	-0.0707828387153651\\
13.3725	-0.0704053834803282\\
13.3780555555556	-0.0711403943867911\\
13.3836111111111	-0.0714131993106594\\
13.3891666666667	-0.0716985988371752\\
13.3947222222222	-0.0710837495812422\\
13.4002777777778	-0.0712415287962077\\
13.4058333333333	-0.0705286900998139\\
13.4113888888889	-0.0709860004789664\\
13.4169444444444	-0.0703172103295977\\
13.4225	-0.0701197406343192\\
13.4280555555556	-0.0708060027924313\\
13.4336111111111	-0.0727525054728905\\
13.4391666666667	-0.0732524605185407\\
13.4447222222222	-0.0735198384112861\\
13.4502777777778	-0.0723685814194713\\
13.4558333333333	-0.0719831581095892\\
13.4613888888889	-0.0718142962043488\\
13.4669444444444	-0.0721104750753255\\
13.4725	-0.0725780375038573\\
13.4780555555556	-0.0735787659221279\\
13.4836111111111	-0.0741183406701638\\
13.4891666666667	-0.0731350657003601\\
13.4947222222222	-0.0727858829571947\\
13.5002777777778	-0.07461959347022\\
13.5058333333333	-0.073228008730576\\
13.5113888888889	-0.0723396263690475\\
13.5169444444444	-0.0739365024648472\\
13.5225	-0.0754528373562203\\
13.5280555555556	-0.0758225363951101\\
13.5336111111111	-0.0756818046412535\\
13.5391666666667	-0.0756679862722873\\
13.5447222222222	-0.0747448408150355\\
13.5502777777778	-0.0753626563217979\\
13.5558333333333	-0.076674063814562\\
13.5613888888889	-0.0764860381170291\\
13.5669444444444	-0.0774246258537626\\
13.5725	-0.0762771691581394\\
13.5780555555556	-0.0764989098357024\\
13.5836111111111	-0.0772021530233924\\
13.5891666666667	-0.0771542184844484\\
13.5947222222222	-0.07622486633491\\
13.6002777777778	-0.0758993438694029\\
13.6058333333333	-0.075946659780669\\
13.6113888888889	-0.0752953756745858\\
13.6169444444444	-0.0744549938550137\\
13.6225	-0.0749608049537528\\
13.6280555555556	-0.0747794756230171\\
13.6336111111111	-0.0754568467450019\\
13.6391666666667	-0.0751930507665022\\
13.6447222222222	-0.0745866069492284\\
13.6502777777778	-0.0742182037931127\\
13.6558333333333	-0.0746789253206497\\
13.6613888888889	-0.0743448704070702\\
13.6669444444444	-0.0728694120521384\\
13.6725	-0.0726038128697557\\
13.6780555555556	-0.0721997161369663\\
13.6836111111111	-0.0721687813796317\\
13.6891666666667	-0.0731766048566252\\
13.6947222222222	-0.0739521062436857\\
13.7002777777778	-0.074806899001259\\
13.7058333333333	-0.0740546860113526\\
13.7113888888889	-0.0715389358587047\\
13.7169444444444	-0.0713487404606267\\
13.7225	-0.0709724493946687\\
13.7280555555556	-0.0706141534084659\\
13.7336111111111	-0.0702358266298741\\
13.7391666666667	-0.0685309997800924\\
13.7447222222222	-0.0683340377807787\\
13.7502777777778	-0.0680874578165031\\
13.7558333333333	-0.0678200973010413\\
13.7613888888889	-0.0673261894324674\\
13.7669444444444	-0.0669151200391067\\
13.7725	-0.0670683452590104\\
13.7780555555556	-0.0671513596804587\\
13.7836111111111	-0.0663624243857121\\
13.7891666666667	-0.0658996179231038\\
13.7947222222222	-0.0667857870788711\\
13.8002777777778	-0.067449460045248\\
13.8058333333333	-0.0667493785907456\\
13.8113888888889	-0.0672352890795974\\
13.8169444444444	-0.0680243149667839\\
13.8225	-0.0673018118828341\\
13.8280555555556	-0.0683340219236182\\
13.8336111111111	-0.0695202549725546\\
13.8391666666667	-0.0697605925353138\\
13.8447222222222	-0.0688811047503586\\
13.8502777777778	-0.0701220772782912\\
13.8558333333333	-0.070211502912058\\
13.8613888888889	-0.0700891892509668\\
13.8669444444444	-0.0701044521975438\\
13.8725	-0.0705984696867364\\
13.8780555555556	-0.0696621241143957\\
13.8836111111111	-0.0696149663251121\\
13.8891666666667	-0.0692032219182014\\
13.8947222222222	-0.0675479279528533\\
13.9002777777778	-0.0674273792238488\\
13.9058333333333	-0.0661973268970952\\
13.9113888888889	-0.0653033635670953\\
13.9169444444444	-0.0652099978783527\\
13.9225	-0.064552399074553\\
13.9280555555556	-0.0638678233348221\\
13.9336111111111	-0.0633499694097048\\
13.9391666666667	-0.062866879618405\\
13.9447222222222	-0.0623828179915418\\
13.9502777777778	-0.062043119247678\\
13.9558333333333	-0.0619815402863925\\
13.9613888888889	-0.0614395087332478\\
13.9669444444444	-0.0612727886467152\\
13.9725	-0.0608710444530521\\
13.9780555555556	-0.061244009916931\\
13.9836111111111	-0.0611362013748489\\
13.9891666666667	-0.0608842104773669\\
13.9947222222222	-0.0604960382627258\\
14.0002777777778	-0.0613394434246124\\
14.0058333333333	-0.0627686470401233\\
14.0113888888889	-0.0617036114686338\\
14.0169444444444	-0.061257550549\\
14.0225	-0.062259145068145\\
14.0280555555556	-0.0617172587382772\\
14.0336111111111	-0.0608566305974533\\
14.0391666666667	-0.0615053860808313\\
14.0447222222222	-0.0611451652370109\\
14.0502777777778	-0.0607038101583132\\
14.0558333333333	-0.0612594658419435\\
14.0613888888889	-0.0612879433136865\\
14.0669444444444	-0.0624466217113794\\
14.0725	-0.0627151308844631\\
14.0780555555556	-0.0629297607314379\\
14.0836111111111	-0.0627068028290443\\
14.0891666666667	-0.0621766083303344\\
14.0947222222222	-0.0611819681802262\\
14.1002777777778	-0.0623967981578618\\
14.1058333333333	-0.0610920927224494\\
14.1113888888889	-0.0620703599472671\\
14.1169444444444	-0.0615089207838944\\
14.1225	-0.0606368687717122\\
14.1280555555556	-0.0609083990640153\\
14.1336111111111	-0.0609798498793743\\
14.1391666666667	-0.0612105975227691\\
14.1447222222222	-0.0598603702838986\\
14.1502777777778	-0.0589190015072518\\
14.1558333333333	-0.0582669729557108\\
14.1613888888889	-0.0599481036654737\\
14.1669444444444	-0.0602147834910535\\
14.1725	-0.0604162635527598\\
14.1780555555556	-0.0607676256325488\\
14.1836111111111	-0.0607158450794666\\
14.1891666666667	-0.0617916300865534\\
14.1947222222222	-0.0621066490033909\\
14.2002777777778	-0.0636327631604632\\
14.2058333333333	-0.0626460212152882\\
14.2113888888889	-0.0626685658765436\\
14.2169444444444	-0.0645944772846486\\
14.2225	-0.0650787574581594\\
14.2280555555556	-0.0661616001064637\\
14.2336111111111	-0.0657623463900405\\
14.2391666666667	-0.0666937349493926\\
14.2447222222222	-0.0659341612680331\\
14.2502777777778	-0.0664834819547608\\
14.2558333333333	-0.0663489839362656\\
14.2613888888889	-0.0665340401346114\\
14.2669444444444	-0.0673076457651127\\
14.2725	-0.0666523882561866\\
14.2780555555556	-0.0660684735828388\\
14.2836111111111	-0.0657616388036779\\
14.2891666666667	-0.0647435806971204\\
14.2947222222222	-0.0647298752156737\\
14.3002777777778	-0.06408913190108\\
14.3058333333333	-0.065062261092891\\
14.3113888888889	-0.0647426854469131\\
14.3169444444444	-0.0650185531630137\\
14.3225	-0.0652205827649804\\
14.3280555555556	-0.0663145826776392\\
14.3336111111111	-0.065077030563028\\
14.3391666666667	-0.0641077246224421\\
14.3447222222222	-0.0648209980503669\\
14.3502777777778	-0.0648521522419405\\
14.3558333333333	-0.0649895130872296\\
14.3613888888889	-0.0632235163983158\\
14.3669444444444	-0.0642281428303262\\
14.3725	-0.0636078869965802\\
14.3780555555556	-0.0644376612205881\\
14.3836111111111	-0.0652414783285626\\
14.3891666666667	-0.063466468928428\\
14.3947222222222	-0.062766437044286\\
14.4002777777778	-0.0641421326853142\\
14.4058333333333	-0.0648079264218986\\
14.4113888888889	-0.0648280943448936\\
14.4169444444444	-0.0662408425784507\\
14.4225	-0.0655054176441091\\
14.4280555555556	-0.0651942287369431\\
14.4336111111111	-0.0656806668138698\\
14.4391666666667	-0.0647587414728388\\
14.4447222222222	-0.065344820988238\\
14.4502777777778	-0.0660779957967214\\
14.4558333333333	-0.066038642547056\\
14.4613888888889	-0.0654496850724434\\
14.4669444444444	-0.0664652811410294\\
14.4725	-0.0679337592596381\\
14.4780555555556	-0.0671057619509467\\
14.4836111111111	-0.0673324903011319\\
14.4891666666667	-0.0674060186158976\\
14.4947222222222	-0.0673323215117534\\
14.5002777777778	-0.067832463719412\\
14.5058333333333	-0.0673044545309151\\
14.5113888888889	-0.0684126419658605\\
14.5169444444444	-0.0690769275549345\\
14.5225	-0.0693919600881082\\
14.5280555555556	-0.0703635128703301\\
14.5336111111111	-0.0710914540242212\\
14.5391666666667	-0.0711804896092766\\
14.5447222222222	-0.0713947839692986\\
14.5502777777778	-0.0710436539732126\\
14.5558333333333	-0.0701932668469838\\
14.5613888888889	-0.0705304460248374\\
14.5669444444444	-0.070901237345851\\
14.5725	-0.0707955726891858\\
14.5780555555556	-0.0714459167127239\\
14.5836111111111	-0.0730670828439303\\
14.5891666666667	-0.0733350181929429\\
14.5947222222222	-0.0745754037635578\\
14.6002777777778	-0.0748778474968513\\
14.6058333333333	-0.0733972657575251\\
14.6113888888889	-0.0732299857850316\\
14.6169444444444	-0.0717316603270446\\
14.6225	-0.0734501291542758\\
14.6280555555556	-0.0734592397237968\\
14.6336111111111	-0.0733348232289223\\
14.6391666666667	-0.0749448290687767\\
14.6447222222222	-0.0751917814183084\\
14.6502777777778	-0.0767614609275527\\
14.6558333333333	-0.0760803986122194\\
14.6613888888889	-0.0744331615101199\\
14.6669444444444	-0.0756617302436859\\
14.6725	-0.0765974876867253\\
14.6780555555556	-0.0770753567201115\\
14.6836111111111	-0.0766676585413628\\
14.6891666666667	-0.0775214324667632\\
14.6947222222222	-0.0779929583644729\\
14.7002777777778	-0.0790350844941691\\
14.7058333333333	-0.0797237141475276\\
14.7113888888889	-0.0804136571544673\\
14.7169444444444	-0.0794706560058537\\
14.7225	-0.0788245373155958\\
14.7280555555556	-0.0782197129123499\\
14.7336111111111	-0.0777179015860579\\
14.7391666666667	-0.0779632691713359\\
14.7447222222222	-0.0776469523471385\\
14.7502777777778	-0.0767138406578074\\
14.7558333333333	-0.0766252081319487\\
14.7613888888889	-0.0773038746163118\\
14.7669444444444	-0.0780323978493188\\
14.7725	-0.0797913171230056\\
14.7780555555556	-0.0789394507801705\\
14.7836111111111	-0.0787367876271445\\
14.7891666666667	-0.0787039295782204\\
14.7947222222222	-0.0783315591665982\\
14.8002777777778	-0.0779978705465149\\
14.8058333333333	-0.0789912584657632\\
14.8113888888889	-0.078958832347505\\
14.8169444444444	-0.0795962818357559\\
14.8225	-0.0797869273873544\\
14.8280555555556	-0.0805043753413888\\
14.8336111111111	-0.079671065291214\\
14.8391666666667	-0.0803198546587274\\
14.8447222222222	-0.0805468770092243\\
14.8502777777778	-0.0816461223863616\\
14.8558333333333	-0.0814793228666571\\
14.8613888888889	-0.0814372049193186\\
14.8669444444444	-0.0814833866096061\\
14.8725	-0.0793651237511452\\
14.8780555555556	-0.0780812787931502\\
14.8836111111111	-0.0771881335477205\\
14.8891666666667	-0.0772960402401875\\
14.8947222222222	-0.0782005174475232\\
14.9002777777778	-0.079928350791733\\
14.9058333333333	-0.0795409592286741\\
14.9113888888889	-0.0788901694528683\\
14.9169444444444	-0.0786921567781997\\
14.9225	-0.0786984783013674\\
14.9280555555556	-0.0780267269088928\\
14.9336111111111	-0.0788690200778943\\
14.9391666666667	-0.0785719024921045\\
14.9447222222222	-0.0795707640733349\\
14.9502777777778	-0.0792637682855142\\
14.9558333333333	-0.0790563231823068\\
14.9613888888889	-0.0806669306260645\\
14.9669444444444	-0.080696959792837\\
14.9725	-0.080006712100634\\
14.9780555555556	-0.0803963329794996\\
14.9836111111111	-0.0793949589915027\\
14.9891666666667	-0.0795361941887209\\
14.9947222222222	-0.079342668520874\\
15.0002777777778	-0.0799877883935348\\
15.0058333333333	-0.0797867875281501\\
15.0113888888889	-0.0800653217657787\\
15.0169444444444	-0.0802801016032062\\
15.0225	-0.0804458908280726\\
15.0280555555556	-0.0817593129594662\\
15.0336111111111	-0.0824248887338161\\
15.0391666666667	-0.0827136861974892\\
15.0447222222222	-0.0836053050775627\\
15.0502777777778	-0.0833955350607562\\
15.0558333333333	-0.0824600162045055\\
15.0613888888889	-0.0825192446165361\\
15.0669444444444	-0.0834844615008929\\
15.0725	-0.0832457723604821\\
15.0780555555556	-0.0834840894245255\\
15.0836111111111	-0.0825086395349927\\
15.0891666666667	-0.0812790387722386\\
15.0947222222222	-0.0814092760029722\\
15.1002777777778	-0.0816020058593472\\
15.1058333333333	-0.0796873244423199\\
15.1113888888889	-0.0790039641385756\\
15.1169444444444	-0.0789922581330752\\
15.1225	-0.0778800520712588\\
15.1280555555556	-0.0778848840407538\\
15.1336111111111	-0.0784948560187094\\
15.1391666666667	-0.0798964170340362\\
15.1447222222222	-0.0798808865299951\\
15.1502777777778	-0.0799228942036311\\
15.1558333333333	-0.0796057868324889\\
15.1613888888889	-0.0792962037331182\\
15.1669444444444	-0.0800752719469728\\
15.1725	-0.0807768607437727\\
15.1780555555556	-0.0817754612615772\\
15.1836111111111	-0.0820157443875185\\
15.1891666666667	-0.0830338891749011\\
15.1947222222222	-0.0832437261048107\\
15.2002777777778	-0.0834448772358914\\
15.2058333333333	-0.0834676912031682\\
15.2113888888889	-0.0830251143766601\\
15.2169444444444	-0.0843323839327402\\
15.2225	-0.0847790293155578\\
15.2280555555556	-0.0839690719268639\\
15.2336111111111	-0.0858084605727029\\
15.2391666666667	-0.0853281701932917\\
15.2447222222222	-0.0855247985880124\\
15.2502777777778	-0.0846950932268422\\
15.2558333333333	-0.0846448948005506\\
15.2613888888889	-0.0844197728091066\\
15.2669444444444	-0.0834889035391566\\
15.2725	-0.0837767646928071\\
15.2780555555556	-0.083811513014972\\
15.2836111111111	-0.082786143254199\\
15.2891666666667	-0.0834195169848455\\
15.2947222222222	-0.0830957831058893\\
15.3002777777778	-0.0854900764478025\\
15.3058333333333	-0.0833408041919995\\
15.3113888888889	-0.0841927675657316\\
15.3169444444444	-0.0841379969402841\\
15.3225	-0.0845847318734943\\
15.3280555555556	-0.0847535052962738\\
15.3336111111111	-0.0843875657828192\\
15.3391666666667	-0.0859398822253603\\
15.3447222222222	-0.0863687677453495\\
15.3502777777778	-0.0883467747009118\\
15.3558333333333	-0.0881317374465639\\
15.3613888888889	-0.0885896903998032\\
15.3669444444444	-0.0895593327153575\\
15.3725	-0.0898883290081223\\
15.3780555555556	-0.0897039140852546\\
15.3836111111111	-0.0903646328974719\\
15.3891666666667	-0.0918671321132976\\
15.3947222222222	-0.0917018517618418\\
15.4002777777778	-0.091498746483856\\
15.4058333333333	-0.0911778938671677\\
15.4113888888889	-0.092391708185239\\
15.4169444444444	-0.0934483669765591\\
15.4225	-0.0941217557883472\\
15.4280555555556	-0.0935835769855285\\
15.4336111111111	-0.093768116361697\\
15.4391666666667	-0.0939413025570906\\
15.4447222222222	-0.094217371299071\\
15.4502777777778	-0.0944626102486037\\
15.4558333333333	-0.0921530289999922\\
15.4613888888889	-0.0927869211735006\\
15.4669444444444	-0.0937377338122922\\
15.4725	-0.0940699720504729\\
15.4780555555556	-0.0930049741448728\\
15.4836111111111	-0.0932232518293883\\
15.4891666666667	-0.0933532954879783\\
15.4947222222222	-0.0945075770190818\\
15.5002777777778	-0.0956626143589157\\
15.5058333333333	-0.095034679204647\\
15.5113888888889	-0.0950898483372592\\
15.5169444444444	-0.0969829827068665\\
15.5225	-0.0974209343664247\\
15.5280555555556	-0.0958025455211681\\
15.5336111111111	-0.0952856512777657\\
15.5391666666667	-0.0952445576290833\\
15.5447222222222	-0.0955469977637035\\
15.5502777777778	-0.0944265926209305\\
15.5558333333333	-0.0948486473645762\\
15.5613888888889	-0.0950074686934211\\
15.5669444444444	-0.0945804043961418\\
15.5725	-0.0955971282734111\\
15.5780555555556	-0.0956102171140665\\
15.5836111111111	-0.0962807649035686\\
15.5891666666667	-0.0961967068765315\\
15.5947222222222	-0.0956542035421593\\
15.6002777777778	-0.094021753355824\\
15.6058333333333	-0.0954898229152126\\
15.6113888888889	-0.0982018355432433\\
15.6169444444444	-0.0985697796873279\\
15.6225	-0.0976043889930037\\
15.6280555555556	-0.0972235001783282\\
15.6336111111111	-0.0983346818341518\\
15.6391666666667	-0.0975742149692289\\
15.6447222222222	-0.096508292561958\\
15.6502777777778	-0.0964516101209452\\
15.6558333333333	-0.0954229262958454\\
15.6613888888889	-0.0948647141314252\\
15.6669444444444	-0.0965607384673679\\
15.6725	-0.0951064761073389\\
15.6780555555556	-0.0953043330173036\\
15.6836111111111	-0.0931763181190557\\
15.6891666666667	-0.0939548673087373\\
15.6947222222222	-0.0952148135896858\\
15.7002777777778	-0.0950331789429494\\
15.7058333333333	-0.0943545772615463\\
15.7113888888889	-0.0942587876767185\\
15.7169444444444	-0.0954559044408337\\
15.7225	-0.0972078900893633\\
15.7280555555556	-0.0973645917293701\\
15.7336111111111	-0.0970356830609171\\
15.7391666666667	-0.0989234437310984\\
15.7447222222222	-0.0995578524031838\\
15.7502777777778	-0.0994887123654646\\
15.7558333333333	-0.1003438669202\\
15.7613888888889	-0.100364495807324\\
15.7669444444444	-0.100451321346036\\
15.7725	-0.101493396534183\\
15.7780555555556	-0.101373239112211\\
15.7836111111111	-0.101679127552579\\
15.7891666666667	-0.10095767420416\\
15.7947222222222	-0.100128779057908\\
15.8002777777778	-0.10165505824092\\
15.8058333333333	-0.101660395647695\\
15.8113888888889	-0.102847390416297\\
15.8169444444444	-0.101951547339286\\
15.8225	-0.101826955828665\\
15.8280555555556	-0.101800890006972\\
15.8336111111111	-0.103552130648242\\
15.8391666666667	-0.101019153668823\\
15.8447222222222	-0.101184389718045\\
15.8502777777778	-0.101542621875465\\
15.8558333333333	-0.100582924454575\\
15.8613888888889	-0.101500678177704\\
15.8669444444444	-0.0972110156899742\\
15.8725	-0.0981917034515377\\
15.8780555555556	-0.0989666073839171\\
15.8836111111111	-0.0971590730827816\\
15.8891666666667	-0.0968301722010294\\
15.8947222222222	-0.0978191891697474\\
15.9002777777778	-0.0957294395262694\\
15.9058333333333	-0.0977740230849544\\
15.9113888888889	-0.0964836568131977\\
15.9169444444444	-0.0979774760238133\\
15.9225	-0.0983942728236662\\
15.9280555555556	-0.0982841841035903\\
15.9336111111111	-0.0989860303399422\\
15.9391666666667	-0.0995784533994186\\
15.9447222222222	-0.0994542841175827\\
15.9502777777778	-0.0991853830565622\\
15.9558333333333	-0.0986293660174248\\
15.9613888888889	-0.0969168020292623\\
15.9669444444444	-0.0983979689008865\\
15.9725	-0.0980560462941144\\
15.9780555555556	-0.0984458649900924\\
15.9836111111111	-0.0981043187414739\\
15.9891666666667	-0.0995803843963307\\
15.9947222222222	-0.104323641436333\\
};
\addplot [color=mycolor2,solid,line width=1.5pt,forget plot]
  table[row sep=crcr]{%
9.50027777777778	0\\
9.50583333333333	0.00485164792522314\\
9.51138888888889	0.0075784879670516\\
9.51694444444444	0.0134342266331915\\
9.5225	0.0182217144354031\\
9.52805555555556	0.0196361123789192\\
9.53361111111111	0.0229159601868659\\
9.53916666666667	0.0237622573605291\\
9.54472222222222	0.024014259691032\\
9.55027777777778	0.0251480848113398\\
9.55583333333333	0.0280542367438851\\
9.56138888888889	0.0275575747889492\\
9.56694444444444	0.0275999242947551\\
9.5725	0.0282136076277146\\
9.57805555555555	0.0300148349479661\\
9.58361111111111	0.0305410007123942\\
9.58916666666667	0.0322303466567827\\
9.59472222222222	0.0378003719706107\\
9.60027777777778	0.0404814519423022\\
9.60583333333333	0.0408633665551747\\
9.61138888888889	0.0423210573321853\\
9.61694444444444	0.0436914074294192\\
9.6225	0.047124594838483\\
9.62805555555556	0.0435058148332068\\
9.63361111111111	0.0408028464736128\\
9.63916666666667	0.0404795250028122\\
9.64472222222222	0.0440096393944113\\
9.65027777777778	0.041347991030625\\
9.65583333333333	0.0406097606371278\\
9.66138888888889	0.0369306089930566\\
9.66694444444444	0.0332141983089662\\
9.6725	0.0347305997121738\\
9.67805555555555	0.0333476222704198\\
9.68361111111111	0.0305283717416686\\
9.68916666666667	0.0326616708537242\\
9.69472222222222	0.0328350887091318\\
9.70027777777778	0.0322094236784287\\
9.70583333333333	0.0385376010939309\\
9.71138888888889	0.0437595274044391\\
9.71694444444444	0.0424918588714545\\
9.7225	0.0447483980247772\\
9.72805555555555	0.0528243182216742\\
9.73361111111111	0.0566079902571038\\
9.73916666666667	0.0522590759434442\\
9.74472222222222	0.0506539156859712\\
9.75027777777778	0.0556412933137385\\
9.75583333333333	0.057816525643415\\
9.76138888888889	0.0544185465665795\\
9.76694444444444	0.0609643025539293\\
9.7725	0.0655228119472488\\
9.77805555555556	0.0626321671420609\\
9.78361111111111	0.0610897269291088\\
9.78916666666667	0.0618460692548473\\
9.79472222222222	0.0658480977167808\\
9.80027777777778	0.0676259213111019\\
9.80583333333333	0.0732874866219007\\
9.81138888888889	0.0751737295001619\\
9.81694444444444	0.0725049281568781\\
9.8225	0.0786280800884834\\
9.82805555555555	0.0809137455875843\\
9.83361111111111	0.0829871026717791\\
9.83916666666667	0.0783717583467376\\
9.84472222222222	0.0788945055431264\\
9.85027777777778	0.0791371415396653\\
9.85583333333333	0.0784414997917807\\
9.86138888888889	0.0775637077796375\\
9.86694444444444	0.0741559354510811\\
9.8725	0.0714839097962679\\
9.87805555555556	0.073778013746214\\
9.88361111111111	0.0717968251755401\\
9.88916666666667	0.0762114497502502\\
9.89472222222222	0.0773893953200907\\
9.90027777777778	0.0784039517633384\\
9.90583333333333	0.0761016886996109\\
9.91138888888889	0.0744940144356958\\
9.91694444444444	0.076038112502231\\
9.9225	0.0757536217597375\\
9.92805555555555	0.073130117685335\\
9.93361111111111	0.0687481616862542\\
9.93916666666667	0.0741892059784643\\
9.94472222222222	0.0765992269202137\\
9.95027777777778	0.0732396721847427\\
9.95583333333333	0.0762494280156387\\
9.96138888888889	0.0786690577382619\\
9.96694444444444	0.078084375711623\\
9.9725	0.0734712103125288\\
9.97805555555555	0.0735259894552297\\
9.98361111111111	0.0737602692129506\\
9.98916666666667	0.0768474465279629\\
9.99472222222222	0.0839973654155332\\
10.0002777777778	0.0956309385774191\\
10.0058333333333	0.0979349931238792\\
10.0113888888889	0.102392480284926\\
10.0169444444444	0.104478150194754\\
10.0225	0.113110446791419\\
10.0280555555556	0.116902481256407\\
10.0336111111111	0.119479276657117\\
10.0391666666667	0.114209710477654\\
10.0447222222222	0.118289026243073\\
10.0502777777778	0.118279530948097\\
10.0558333333333	0.114766931368352\\
10.0613888888889	0.117193538342873\\
10.0669444444444	0.115567277464849\\
10.0725	0.115889139675737\\
10.0780555555556	0.112054426235494\\
10.0836111111111	0.113191363410196\\
10.0891666666667	0.11488874152259\\
10.0947222222222	0.121197893236287\\
10.1002777777778	0.122125660914088\\
10.1058333333333	0.119169756334056\\
10.1113888888889	0.117289828061028\\
10.1169444444444	0.120431528046678\\
10.1225	0.126444249044754\\
10.1280555555556	0.128678164787811\\
10.1336111111111	0.130299396061918\\
10.1391666666667	0.133089512733705\\
10.1447222222222	0.141658335921608\\
10.1502777777778	0.145339903057804\\
10.1558333333333	0.144500677146283\\
10.1613888888889	0.140596008525351\\
10.1669444444444	0.139700474513387\\
10.1725	0.136017044502535\\
10.1780555555556	0.133497945577285\\
10.1836111111111	0.133787703775377\\
10.1891666666667	0.131137869097167\\
10.1947222222222	0.131635410046942\\
10.2002777777778	0.134209559697297\\
10.2058333333333	0.135357828958881\\
10.2113888888889	0.133301254299259\\
10.2169444444444	0.128694767114576\\
10.2225	0.129188902126586\\
10.2280555555556	0.125650160846914\\
10.2336111111111	0.123929522908158\\
10.2391666666667	0.121810774295719\\
10.2447222222222	0.117460916982044\\
10.2502777777778	0.113748609431632\\
10.2558333333333	0.10971370416013\\
10.2613888888889	0.110770624702687\\
10.2669444444444	0.11117624971069\\
10.2725	0.108113106240399\\
10.2780555555556	0.116232707865364\\
10.2836111111111	0.11735834648598\\
10.2891666666667	0.114535940166185\\
10.2947222222222	0.118400959599813\\
10.3002777777778	0.11548321186543\\
10.3058333333333	0.113293438771651\\
10.3113888888889	0.105932018971054\\
10.3169444444444	0.0989958307647941\\
10.3225	0.0927403780754437\\
10.3280555555556	0.0891302228632225\\
10.3336111111111	0.0875261556123595\\
10.3391666666667	0.0820823589036893\\
10.3447222222222	0.0828679594660756\\
10.3502777777778	0.0888545391540862\\
10.3558333333333	0.0883634832518864\\
10.3613888888889	0.0856804188481666\\
10.3669444444444	0.0828285587265733\\
10.3725	0.0789889728004345\\
10.3780555555556	0.0763833718909614\\
10.3836111111111	0.0753994941740911\\
10.3891666666667	0.0763976111339388\\
10.3947222222222	0.0763382723449119\\
10.4002777777778	0.0786299726917089\\
10.4058333333333	0.0825645661719746\\
10.4113888888889	0.0859477812023285\\
10.4169444444444	0.0864349329912533\\
10.4225	0.0771278405210502\\
10.4280555555556	0.0758672170910897\\
10.4336111111111	0.0736218480182944\\
10.4391666666667	0.0744273613623602\\
10.4447222222222	0.0728908662915322\\
10.4502777777778	0.0789960898867605\\
10.4558333333333	0.0778608628673104\\
10.4613888888889	0.0767596833279851\\
10.4669444444444	0.0782187562996478\\
10.4725	0.0784387340171308\\
10.4780555555556	0.0809919487536005\\
10.4836111111111	0.0807484374411158\\
10.4891666666667	0.07743536045485\\
10.4947222222222	0.0753148718389632\\
10.5002777777778	0.0755766160888329\\
10.5058333333333	0.0781590050814717\\
10.5113888888889	0.0812941542880272\\
10.5169444444444	0.0840596022644537\\
10.5225	0.0872682695987834\\
10.5280555555556	0.0895618116585067\\
10.5336111111111	0.0946648582836974\\
10.5391666666667	0.107380294545065\\
10.5447222222222	0.112006530007361\\
10.5502777777778	0.109661423535038\\
10.5558333333333	0.109923884193932\\
10.5613888888889	0.109484417336855\\
10.5669444444444	0.105920548645951\\
10.5725	0.108875084610648\\
10.5780555555556	0.105672632198188\\
10.5836111111111	0.109525494954104\\
10.5891666666667	0.109338179951787\\
10.5947222222222	0.108974724398379\\
10.6002777777778	0.103767393839452\\
10.6058333333333	0.103299236876314\\
10.6113888888889	0.105498470163159\\
10.6169444444444	0.105221324070304\\
10.6225	0.104343258626838\\
10.6280555555556	0.0998446867966215\\
10.6336111111111	0.10213721419211\\
10.6391666666667	0.100289763786209\\
10.6447222222222	0.110342961324035\\
10.6502777777778	0.113195246973569\\
10.6558333333333	0.110594633751464\\
10.6613888888889	0.107844109044841\\
10.6669444444444	0.106876506359704\\
10.6725	0.10761784467895\\
10.6780555555556	0.111604877300612\\
10.6836111111111	0.112368098626403\\
10.6891666666667	0.105381191428932\\
10.6947222222222	0.104039925876854\\
10.7002777777778	0.104602898739982\\
10.7058333333333	0.101158678872428\\
10.7113888888889	0.105760961571093\\
10.7169444444444	0.104849234760683\\
10.7225	0.103583897320637\\
10.7280555555556	0.103548757151502\\
10.7336111111111	0.0994378216487075\\
10.7391666666667	0.104532093342005\\
10.7447222222222	0.102538470347982\\
10.7502777777778	0.0975343770363308\\
10.7558333333333	0.0979959504523184\\
10.7613888888889	0.106631543925753\\
10.7669444444444	0.108645469982254\\
10.7725	0.108885866141823\\
10.7780555555556	0.108449312942782\\
10.7836111111111	0.106109680590629\\
10.7891666666667	0.104127846242214\\
10.7947222222222	0.112698230347806\\
10.8002777777778	0.115134215034457\\
10.8058333333333	0.108636793106631\\
10.8113888888889	0.105181282479541\\
10.8169444444444	0.111112843067855\\
10.8225	0.113839087487499\\
10.8280555555556	0.117701858601328\\
10.8336111111111	0.113995570035829\\
10.8391666666667	0.120053698626088\\
10.8447222222222	0.122170331193172\\
10.8502777777778	0.116207017134365\\
10.8558333333333	0.119743610879606\\
10.8613888888889	0.121496835498937\\
10.8669444444444	0.114722764845271\\
10.8725	0.11256371245586\\
10.8780555555556	0.116196404367797\\
10.8836111111111	0.112032814583928\\
10.8891666666667	0.110447568419547\\
10.8947222222222	0.107426434031341\\
10.9002777777778	0.110455921223236\\
10.9058333333333	0.10856117635964\\
10.9113888888889	0.104751505873965\\
10.9169444444444	0.107805396218887\\
10.9225	0.101736770341673\\
10.9280555555556	0.100938041837274\\
10.9336111111111	0.103049614734\\
10.9391666666667	0.100140158880707\\
10.9447222222222	0.0891908168316028\\
10.9502777777778	0.091621686319658\\
10.9558333333333	0.0867024498014243\\
10.9613888888889	0.085925989712468\\
10.9669444444444	0.0899925488216603\\
10.9725	0.083278151899491\\
10.9780555555556	0.082353352358119\\
10.9836111111111	0.0814663302092872\\
10.9891666666667	0.0864593661566415\\
10.9947222222222	0.0884509947079415\\
11.0002777777778	0.0958340752288968\\
11.0058333333333	0.0957726537418903\\
11.0113888888889	0.0948224939568777\\
11.0169444444444	0.0874268478181602\\
11.0225	0.0926703575611559\\
11.0280555555556	0.092723221009748\\
11.0336111111111	0.10025544623684\\
11.0391666666667	0.105275716407881\\
11.0447222222222	0.105067451664523\\
11.0502777777778	0.102978750857971\\
11.0558333333333	0.102971470763686\\
11.0613888888889	0.104304318882306\\
11.0669444444444	0.100722422412651\\
11.0725	0.0925750019989922\\
11.0780555555556	0.0908501130521561\\
11.0836111111111	0.0875839767283741\\
11.0891666666667	0.0942575524263586\\
11.0947222222222	0.101121686437686\\
11.1002777777778	0.0950047573946874\\
11.1058333333333	0.0868477998481899\\
11.1113888888889	0.0852025313736595\\
11.1169444444444	0.0838457277636785\\
11.1225	0.0817130527374835\\
11.1280555555556	0.0835551775842127\\
11.1336111111111	0.0848893277051574\\
11.1391666666667	0.081360773830934\\
11.1447222222222	0.0839452430340264\\
11.1502777777778	0.0816842190545072\\
11.1558333333333	0.0804819478293712\\
11.1613888888889	0.0892695282757654\\
11.1669444444444	0.0882743674907638\\
11.1725	0.0884078595742832\\
11.1780555555556	0.0863873423725884\\
11.1836111111111	0.0846662238376099\\
11.1891666666667	0.0835922458055669\\
11.1947222222222	0.0890147324460544\\
11.2002777777778	0.0996687960317394\\
11.2058333333333	0.10090408629749\\
11.2113888888889	0.093720099756253\\
11.2169444444444	0.0997875043637794\\
11.2225	0.097081761217342\\
11.2280555555556	0.101133996731871\\
11.2336111111111	0.107289704810156\\
11.2391666666667	0.102202889046738\\
11.2447222222222	0.103992526772552\\
11.2502777777778	0.107310669893951\\
11.2558333333333	0.10614225554097\\
11.2613888888889	0.0993153748772568\\
11.2669444444444	0.104337197393124\\
11.2725	0.101072967658486\\
11.2780555555556	0.100846445611977\\
11.2836111111111	0.095988954398895\\
11.2891666666667	0.10570529680116\\
11.2947222222222	0.102453213785114\\
11.3002777777778	0.0990092174874407\\
11.3058333333333	0.0982890231746748\\
11.3113888888889	0.0960268119884442\\
11.3169444444444	0.0935137844016186\\
11.3225	0.0925354909455965\\
11.3280555555556	0.0908618613085688\\
11.3336111111111	0.0921656556073258\\
11.3391666666667	0.0953395215768405\\
11.3447222222222	0.0935471761703462\\
11.3502777777778	0.0893307993207433\\
11.3558333333333	0.0874397947932448\\
11.3613888888889	0.0866154176523158\\
11.3669444444444	0.0794646193567764\\
11.3725	0.0728258075558125\\
11.3780555555556	0.0781289517625398\\
11.3836111111111	0.0820262711046424\\
11.3891666666667	0.0822105426040872\\
11.3947222222222	0.0782034519230947\\
11.4002777777778	0.0756016963981503\\
11.4058333333333	0.0812066833161987\\
11.4113888888889	0.0845972691111304\\
11.4169444444444	0.0847213914119739\\
11.4225	0.0887889283396889\\
11.4280555555556	0.0955328072543749\\
11.4336111111111	0.0895367594463167\\
11.4391666666667	0.0919246494584867\\
11.4447222222222	0.0890877261774848\\
11.4502777777778	0.0910968963005203\\
11.4558333333333	0.0949630452846145\\
11.4613888888889	0.0894125592234954\\
11.4669444444444	0.0875813475242177\\
11.4725	0.0825777413642345\\
11.4780555555556	0.0791732971791092\\
11.4836111111111	0.0726074341703797\\
11.4891666666667	0.0717802520458374\\
11.4947222222222	0.0754714699018959\\
11.5002777777778	0.0742890022567402\\
11.5058333333333	0.0698185630878746\\
11.5113888888889	0.0735011658952407\\
11.5169444444444	0.0701816487413216\\
11.5225	0.0686666085147577\\
11.5280555555556	0.0649447826795153\\
11.5336111111111	0.0660452574709701\\
11.5391666666667	0.0583045671377285\\
11.5447222222222	0.0611114691068469\\
11.5502777777778	0.0640346133230526\\
11.5558333333333	0.064121533389649\\
11.5613888888889	0.0590865773116029\\
11.5669444444444	0.0525167008590885\\
11.5725	0.0476471054624562\\
11.5780555555556	0.0547960087166856\\
11.5836111111111	0.0446118332427327\\
11.5891666666667	0.0437804878947767\\
11.5947222222222	0.0401190805018869\\
11.6002777777778	0.0459899914757949\\
11.6058333333333	0.0503194339142998\\
11.6113888888889	0.0568970244021815\\
11.6169444444444	0.0554196600302932\\
11.6225	0.0583561997845769\\
11.6280555555556	0.0573771897445478\\
11.6336111111111	0.060234555921184\\
11.6391666666667	0.0581476570805299\\
11.6447222222222	0.0624571068188967\\
11.6502777777778	0.0614506443743991\\
11.6558333333333	0.0619735354798036\\
11.6613888888889	0.0573117196931949\\
11.6669444444444	0.0572691068814095\\
11.6725	0.0637862592656553\\
11.6780555555556	0.0607745954221353\\
11.6836111111111	0.0636156015131292\\
11.6891666666667	0.0607375653533669\\
11.6947222222222	0.0612459687047303\\
11.7002777777778	0.0580402632196505\\
11.7058333333333	0.0593933015850666\\
11.7113888888889	0.0579729769185493\\
11.7169444444444	0.0653629887090363\\
11.7225	0.0592615862514447\\
11.7280555555556	0.0538917591862616\\
11.7336111111111	0.0475854280876448\\
11.7391666666667	0.04555916197059\\
11.7447222222222	0.043379789086983\\
11.7502777777778	0.0467197710640208\\
11.7558333333333	0.0392888455595659\\
11.7613888888889	0.0331720313921019\\
11.7669444444444	0.0316213831648924\\
11.7725	0.0341498389762762\\
11.7780555555556	0.0331029078434669\\
11.7836111111111	0.0346770304876323\\
11.7891666666667	0.0367616915568872\\
11.7947222222222	0.0322954951661109\\
11.8002777777778	0.0336163893848981\\
11.8058333333333	0.0375299808172963\\
11.8113888888889	0.0473315857046057\\
11.8169444444444	0.0461465768074769\\
11.8225	0.0501086563175954\\
11.8280555555556	0.0493634563800778\\
11.8336111111111	0.0540793236183287\\
11.8391666666667	0.0595727372699577\\
11.8447222222222	0.0663836880349622\\
11.8502777777778	0.0738785828794921\\
11.8558333333333	0.0745243424276739\\
11.8613888888889	0.0729548797270339\\
11.8669444444444	0.0697990643901492\\
11.8725	0.0700036990708102\\
11.8780555555556	0.0696062413998942\\
11.8836111111111	0.0727242687433257\\
11.8891666666667	0.0672075290817016\\
11.8947222222222	0.0676327704318699\\
11.9002777777778	0.0648755570688145\\
11.9058333333333	0.0705388985744482\\
11.9113888888889	0.076717400824467\\
11.9169444444444	0.0782962754093061\\
11.9225	0.0759189375657286\\
11.9280555555556	0.0804915466107431\\
11.9336111111111	0.0783383340856817\\
11.9391666666667	0.0727017871806404\\
11.9447222222222	0.0775149394974615\\
11.9502777777778	0.0789138713705012\\
11.9558333333333	0.0721133195658786\\
11.9613888888889	0.0657645642759909\\
11.9669444444444	0.0665045919834549\\
11.9725	0.0737488472825074\\
11.9780555555556	0.0798812817188791\\
11.9836111111111	0.0817316189995114\\
11.9891666666667	0.0795778950382487\\
11.9947222222222	0.0775892204299141\\
12.0002777777778	0.0784610992124162\\
12.0058333333333	0.0728209080581037\\
12.0113888888889	0.0695048586684\\
12.0169444444444	0.0767976367290582\\
12.0225	0.0817859526183129\\
12.0280555555556	0.0807792964588061\\
12.0336111111111	0.084577467510486\\
12.0391666666667	0.0852571721957348\\
12.0447222222222	0.0931528142190628\\
12.0502777777778	0.0984876192676614\\
12.0558333333333	0.102426284779388\\
12.0613888888889	0.108033892080471\\
12.0669444444444	0.104401185000165\\
12.0725	0.109542733991318\\
12.0780555555556	0.106883418226445\\
12.0836111111111	0.116156912620441\\
12.0891666666667	0.111476035646028\\
12.0947222222222	0.116926719513162\\
12.1002777777778	0.115455193218613\\
12.1058333333333	0.118907935996596\\
12.1113888888889	0.122007668435699\\
12.1169444444444	0.121002251260082\\
12.1225	0.121950557920608\\
12.1280555555556	0.116698329379214\\
12.1336111111111	0.11222269630938\\
12.1391666666667	0.117587773750806\\
12.1447222222222	0.121677911150022\\
12.1502777777778	0.119869530716873\\
12.1558333333333	0.120334975813244\\
12.1613888888889	0.112581105764762\\
12.1669444444444	0.116320564623273\\
12.1725	0.119843639033627\\
12.1780555555556	0.125651883921527\\
12.1836111111111	0.126948054219874\\
12.1891666666667	0.128821894856816\\
12.1947222222222	0.126974877927868\\
12.2002777777778	0.129292097857176\\
12.2058333333333	0.131491876375972\\
12.2113888888889	0.128547906462601\\
12.2169444444444	0.13392356458499\\
12.2225	0.134388876097087\\
12.2280555555556	0.133832655629564\\
12.2336111111111	0.134958273308347\\
12.2391666666667	0.13993172769899\\
12.2447222222222	0.143101406746674\\
12.2502777777778	0.143030311970107\\
12.2558333333333	0.134963632604473\\
12.2613888888889	0.136515133797862\\
12.2669444444444	0.141167732082612\\
12.2725	0.136574616567997\\
12.2780555555556	0.138541566360232\\
12.2836111111111	0.132779338790781\\
12.2891666666667	0.13405355515367\\
12.2947222222222	0.125214314642996\\
12.3002777777778	0.128181959173111\\
12.3058333333333	0.126335160017742\\
12.3113888888889	0.126215757724406\\
12.3169444444444	0.120324857615528\\
12.3225	0.123189205045213\\
12.3280555555556	0.129757355526115\\
12.3336111111111	0.133501885363646\\
12.3391666666667	0.132171918928682\\
12.3447222222222	0.127754257225814\\
12.3502777777778	0.130034570647324\\
12.3558333333333	0.127370446787175\\
12.3613888888889	0.124011532375818\\
12.3669444444444	0.12581691503536\\
12.3725	0.128650088141001\\
12.3780555555556	0.130077639521289\\
12.3836111111111	0.134962288608746\\
12.3891666666667	0.141160136474198\\
12.3947222222222	0.139542636402443\\
12.4002777777778	0.134859945126003\\
12.4058333333333	0.140616327945698\\
12.4113888888889	0.146544667572599\\
12.4169444444444	0.149267653126374\\
12.4225	0.151647318254016\\
12.4280555555556	0.152342538890208\\
12.4336111111111	0.154186900080342\\
12.4391666666667	0.158512568205449\\
12.4447222222222	0.161670858913934\\
12.4502777777778	0.161190171309619\\
12.4558333333333	0.1605864620702\\
12.4613888888889	0.162028446048917\\
12.4669444444444	0.158543480400801\\
12.4725	0.16214238369661\\
12.4780555555556	0.16528158741617\\
12.4836111111111	0.169336036989749\\
12.4891666666667	0.16589613366058\\
12.4947222222222	0.160159306952421\\
12.5002777777778	0.154168418155056\\
12.5058333333333	0.150209013207232\\
12.5113888888889	0.143520575408939\\
12.5169444444444	0.14035384843856\\
12.5225	0.139027885551588\\
12.5280555555556	0.140837497570601\\
12.5336111111111	0.141874717144675\\
12.5391666666667	0.141513622323991\\
12.5447222222222	0.144301227513534\\
12.5502777777778	0.143608073320283\\
12.5558333333333	0.141768072966483\\
12.5613888888889	0.134297201830678\\
12.5669444444444	0.136369076674557\\
12.5725	0.13200569889982\\
12.5780555555556	0.131067231684795\\
12.5836111111111	0.125180676299562\\
12.5891666666667	0.127752475456989\\
12.5947222222222	0.12221877812209\\
12.6002777777778	0.123361013154298\\
12.6058333333333	0.1246188313608\\
12.6113888888889	0.121082479100428\\
12.6169444444444	0.118956507522701\\
12.6225	0.11934482533698\\
12.6280555555556	0.125727517114916\\
12.6336111111111	0.129623593786674\\
12.6391666666667	0.133606858000151\\
12.6447222222222	0.137515084772277\\
12.6502777777778	0.139045803109582\\
12.6558333333333	0.140900228139099\\
12.6613888888889	0.142288478362196\\
12.6669444444444	0.150431044346551\\
12.6725	0.146903092011795\\
12.6780555555556	0.143852294597441\\
12.6836111111111	0.149300441662281\\
12.6891666666667	0.147464314784741\\
12.6947222222222	0.145646496443402\\
12.7002777777778	0.142659795532488\\
12.7058333333333	0.144385873158549\\
12.7113888888889	0.139820761887546\\
12.7169444444444	0.136068513159886\\
12.7225	0.138579636411147\\
12.7280555555556	0.139363899756521\\
12.7336111111111	0.13993303038443\\
12.7391666666667	0.126230418931708\\
12.7447222222222	0.130542987051057\\
12.7502777777778	0.138918816955116\\
12.7558333333333	0.133599970401947\\
12.7613888888889	0.13522547356846\\
12.7669444444444	0.13428313127401\\
12.7725	0.130516975925766\\
12.7780555555556	0.128247069462995\\
12.7836111111111	0.125291168771643\\
12.7891666666667	0.1133184952163\\
12.7947222222222	0.112902937590007\\
12.8002777777778	0.112231473746729\\
12.8058333333333	0.107102311309403\\
12.8113888888889	0.108944920554808\\
12.8169444444444	0.110431937880142\\
12.8225	0.11965253587196\\
12.8280555555556	0.116329975655851\\
12.8336111111111	0.121352921794657\\
12.8391666666667	0.11913047209126\\
12.8447222222222	0.116157713814132\\
12.8502777777778	0.116851416304237\\
12.8558333333333	0.116373896667267\\
12.8613888888889	0.116308024250343\\
12.8669444444444	0.114508431834986\\
12.8725	0.112625705111725\\
12.8780555555556	0.109207078823137\\
12.8836111111111	0.10555269974132\\
12.8891666666667	0.110477389646925\\
12.8947222222222	0.117326881708896\\
12.9002777777778	0.111864967618386\\
12.9058333333333	0.109379534433851\\
12.9113888888889	0.113023889728291\\
12.9169444444444	0.114481106113961\\
12.9225	0.122926080699899\\
12.9280555555556	0.126522287218069\\
12.9336111111111	0.133896431876648\\
12.9391666666667	0.141126314102021\\
12.9447222222222	0.137880493228515\\
12.9502777777778	0.140219635598548\\
12.9558333333333	0.146199832167506\\
12.9613888888889	0.145302505409853\\
12.9669444444444	0.154289318207345\\
12.9725	0.149638086259594\\
12.9780555555556	0.150613800471375\\
12.9836111111111	0.145109215504186\\
12.9891666666667	0.137866218931524\\
12.9947222222222	0.137887608705749\\
13.0002777777778	0.13455871849229\\
13.0058333333333	0.124455114515352\\
13.0113888888889	0.122537902474118\\
13.0169444444444	0.129572933026114\\
13.0225	0.124686123282436\\
13.0280555555556	0.117587431186515\\
13.0336111111111	0.116413239054236\\
13.0391666666667	0.116757084092067\\
13.0447222222222	0.112906578916387\\
13.0502777777778	0.113072015323674\\
13.0558333333333	0.108077311809959\\
13.0613888888889	0.109526683532317\\
13.0669444444444	0.107833696081864\\
13.0725	0.105364562977033\\
13.0780555555556	0.105530086635602\\
13.0836111111111	0.107693294449002\\
13.0891666666667	0.111025691495197\\
13.0947222222222	0.113200150527278\\
13.1002777777778	0.115560633571708\\
13.1058333333333	0.118478800145879\\
13.1113888888889	0.114653711778448\\
13.1169444444444	0.110767040924285\\
13.1225	0.112577414494163\\
13.1280555555556	0.112723379030612\\
13.1336111111111	0.115977181016305\\
13.1391666666667	0.117567904733546\\
13.1447222222222	0.119136874609972\\
13.1502777777778	0.129664104487118\\
13.1558333333333	0.126401428115911\\
13.1613888888889	0.131070407852089\\
13.1669444444444	0.127119385964244\\
13.1725	0.126566552251507\\
13.1780555555556	0.13232689160726\\
13.1836111111111	0.13700671046298\\
13.1891666666667	0.13594056714179\\
13.1947222222222	0.137521270451043\\
13.2002777777778	0.14660508736405\\
13.2058333333333	0.138465154452368\\
13.2113888888889	0.135582255576092\\
13.2169444444444	0.132844065212958\\
13.2225	0.129472037697547\\
13.2280555555556	0.12530337360329\\
13.2336111111111	0.118293562070564\\
13.2391666666667	0.119307700877319\\
13.2447222222222	0.119443169340166\\
13.2502777777778	0.113866945724131\\
13.2558333333333	0.118221811228769\\
13.2613888888889	0.117791970687508\\
13.2669444444444	0.108683969665128\\
13.2725	0.118267270439709\\
13.2780555555556	0.118316535645006\\
13.2836111111111	0.116611277695666\\
13.2891666666667	0.112371714237584\\
13.2947222222222	0.112750911987234\\
13.3002777777778	0.109584310742204\\
13.3058333333333	0.105587355525324\\
13.3113888888889	0.100518734368029\\
13.3169444444444	0.10643531347215\\
13.3225	0.0966813000663944\\
13.3280555555556	0.0948599651598849\\
13.3336111111111	0.0954877002377133\\
13.3391666666667	0.0895525814240516\\
13.3447222222222	0.0831989533276979\\
13.3502777777778	0.0870541326936018\\
13.3558333333333	0.0786593643169187\\
13.3613888888889	0.0817460286912631\\
13.3669444444444	0.0786991262094582\\
13.3725	0.0800376015866655\\
13.3780555555556	0.0842295189288622\\
13.3836111111111	0.0880895372417732\\
13.3891666666667	0.0843694744267724\\
13.3947222222222	0.0790232888550907\\
13.4002777777778	0.0787678670632005\\
13.4058333333333	0.0808002040122929\\
13.4113888888889	0.0817579162710663\\
13.4169444444444	0.0795498097992045\\
13.4225	0.078920240590407\\
13.4280555555556	0.0780162938809962\\
13.4336111111111	0.0748648947054995\\
13.4391666666667	0.0797502453505937\\
13.4447222222222	0.079022356934916\\
13.4502777777778	0.0757783214264066\\
13.4558333333333	0.0729727985116466\\
13.4613888888889	0.0706159974903544\\
13.4669444444444	0.0769992915059013\\
13.4725	0.0793057638646479\\
13.4780555555556	0.0764603892667851\\
13.4836111111111	0.0761324806610268\\
13.4891666666667	0.07203542844545\\
13.4947222222222	0.0717985375161602\\
13.5002777777778	0.0800782405904466\\
13.5058333333333	0.0773607602960839\\
13.5113888888889	0.0740475938566934\\
13.5169444444444	0.0795300037177398\\
13.5225	0.0721710972370308\\
13.5280555555556	0.0806857976462397\\
13.5336111111111	0.0754334851267028\\
13.5391666666667	0.0727410692061497\\
13.5447222222222	0.0731840561697071\\
13.5502777777778	0.0732673922355758\\
13.5558333333333	0.0662953696278976\\
13.5613888888889	0.0596492710400146\\
13.5669444444444	0.0610918434891982\\
13.5725	0.0613000282717782\\
13.5780555555556	0.0641329478876539\\
13.5836111111111	0.0686532128753029\\
13.5891666666667	0.0693500987899021\\
13.5947222222222	0.0704353063642123\\
13.6002777777778	0.0662014563554738\\
13.6058333333333	0.0681508563727221\\
13.6113888888889	0.0671134389381566\\
13.6169444444444	0.0746867603209038\\
13.6225	0.0705994657211504\\
13.6280555555556	0.0756003352845729\\
13.6336111111111	0.0702213034491349\\
13.6391666666667	0.0743789384119386\\
13.6447222222222	0.0841005085545814\\
13.6502777777778	0.0856247114363465\\
13.6558333333333	0.0825914498766608\\
13.6613888888889	0.0777170286885675\\
13.6669444444444	0.069237092818701\\
13.6725	0.0576296974323677\\
13.6780555555556	0.062433193694221\\
13.6836111111111	0.0719790511134006\\
13.6891666666667	0.0707287849164211\\
13.6947222222222	0.0699712174993682\\
13.7002777777778	0.0813761559462863\\
13.7058333333333	0.0868458127463938\\
13.7113888888889	0.073421159111625\\
13.7169444444444	0.0745540257184957\\
13.7225	0.0722651853253868\\
13.7280555555556	0.0635457279783648\\
13.7336111111111	0.0580691182512959\\
13.7391666666667	0.046701338201268\\
13.7447222222222	0.0494257817341025\\
13.7502777777778	0.0498432964908379\\
13.7558333333333	0.0504497833221036\\
13.7613888888889	0.0478268418207706\\
13.7669444444444	0.0385784346153162\\
13.7725	0.0365784087367276\\
13.7780555555556	0.0297000850948608\\
13.7836111111111	0.0291050258008282\\
13.7891666666667	0.0295698520147745\\
13.7947222222222	0.0347301780469722\\
13.8002777777778	0.0364523981110788\\
13.8058333333333	0.0421917198496113\\
13.8113888888889	0.0389239554352877\\
13.8169444444444	0.0465663596195012\\
13.8225	0.0466062535380861\\
13.8280555555556	0.0545748910254804\\
13.8336111111111	0.0564798523474054\\
13.8391666666667	0.0519793047802464\\
13.8447222222222	0.0434636595171147\\
13.8502777777778	0.0442363630962976\\
13.8558333333333	0.0400441401260476\\
13.8613888888889	0.0403111743682201\\
13.8669444444444	0.0372263826751539\\
13.8725	0.043397303549211\\
13.8780555555556	0.0441250209638486\\
13.8836111111111	0.0428546404143956\\
13.8891666666667	0.0358959601952505\\
13.8947222222222	0.0308059503054591\\
13.9002777777778	0.0264546932574854\\
13.9058333333333	0.0289571669368168\\
13.9113888888889	0.023831686593464\\
13.9169444444444	0.0283555095075775\\
13.9225	0.0328056694809105\\
13.9280555555556	0.0389537167566846\\
13.9336111111111	0.0272994717441006\\
13.9391666666667	0.021781015526752\\
13.9447222222222	0.0249481200154889\\
13.9502777777778	0.020143202523269\\
13.9558333333333	0.0224799191031424\\
13.9613888888889	0.024407307804735\\
13.9669444444444	0.025162274827029\\
13.9725	0.0259576338705061\\
13.9780555555556	0.0372167464548726\\
13.9836111111111	0.0357603379394235\\
13.9891666666667	0.0396916507077747\\
13.9947222222222	0.0368192561001859\\
14.0002777777778	0.0310375168017142\\
14.0058333333333	0.0457161898885363\\
14.0113888888889	0.0383773862850169\\
14.0169444444444	0.0253034313353609\\
14.0225	0.0288783599167627\\
14.0280555555556	0.0334697788171729\\
14.0336111111111	0.0307919141978519\\
14.0391666666667	0.0364381336352441\\
14.0447222222222	0.0357948224631308\\
14.0502777777778	0.024608691815874\\
14.0558333333333	0.0319954153228962\\
14.0613888888889	0.0181337759355883\\
14.0669444444444	0.0190763531671016\\
14.0725	0.0284386833619623\\
14.0780555555556	0.0347751913708001\\
14.0836111111111	0.0272169645570614\\
14.0891666666667	0.0215658083353181\\
14.0947222222222	0.0212529200024371\\
14.1002777777778	0.0272497960041825\\
14.1058333333333	0.0295901028708302\\
14.1113888888889	0.035663917358396\\
14.1169444444444	0.0333692113231326\\
14.1225	0.0352386512948553\\
14.1280555555556	0.0447925626954536\\
14.1336111111111	0.036898710589122\\
14.1391666666667	0.0265962604380669\\
14.1447222222222	0.0291604989815576\\
14.1502777777778	0.0279774789338742\\
14.1558333333333	0.0251916434499441\\
14.1613888888889	0.0240228876405161\\
14.1669444444444	0.0282632123836611\\
14.1725	0.0202646016398441\\
14.1780555555556	0.0180484014835038\\
14.1836111111111	0.0176331719895212\\
14.1891666666667	0.019783735981671\\
14.1947222222222	0.0216433814352379\\
14.2002777777778	0.0260035476453397\\
14.2058333333333	0.0172805225892317\\
14.2113888888889	0.0116333056692214\\
14.2169444444444	0.0169116782469554\\
14.2225	0.017645211695524\\
14.2280555555556	0.0202958199120459\\
14.2336111111111	0.0143612348009651\\
14.2391666666667	0.00789283792601894\\
14.2447222222222	0.0104350359049268\\
14.2502777777778	0.0102875483664044\\
14.2558333333333	0.0147873334515336\\
14.2613888888889	0.0173526344727607\\
14.2669444444444	0.0166555695583923\\
14.2725	0.0128892109693389\\
14.2780555555556	0.0122491957160192\\
14.2836111111111	0.0173978360134302\\
14.2891666666667	0.0125307470366827\\
14.2947222222222	0.00219107590165984\\
14.3002777777778	-0.000782037344449873\\
14.3058333333333	-0.010086785718456\\
14.3113888888889	-0.011524951186027\\
14.3169444444444	-0.00893276392609332\\
14.3225	-0.0130389776842566\\
14.3280555555556	-0.0110886129646851\\
14.3336111111111	-0.0143455783650537\\
14.3391666666667	-0.00739599457055028\\
14.3447222222222	-0.0131608054339267\\
14.3502777777778	-0.00838266006723897\\
14.3558333333333	0.000203339504515561\\
14.3613888888889	-0.0063762907327267\\
14.3669444444444	-0.00569464317919133\\
14.3725	-0.00888108860515356\\
14.3780555555556	-0.000552652283083815\\
14.3836111111111	0.013819860041028\\
14.3891666666667	0.00786598622040502\\
14.3947222222222	0.00184158911881112\\
14.4002777777778	0.00978266607113945\\
14.4058333333333	0.00189900802582879\\
14.4113888888889	-0.00273850885150377\\
14.4169444444444	-0.000720544167797759\\
14.4225	-0.00759270219829943\\
14.4280555555556	-0.0181425750982583\\
14.4336111111111	-0.017090783997326\\
14.4391666666667	-0.0189207887427845\\
14.4447222222222	-0.0224734414617741\\
14.4502777777778	-0.0168384275374676\\
14.4558333333333	-0.0206356924099591\\
14.4613888888889	-0.0197108327421579\\
14.4669444444444	-0.0176271078641427\\
14.4725	-0.020072896062713\\
14.4780555555556	-0.024312868507806\\
14.4836111111111	-0.0216833149010276\\
14.4891666666667	-0.0117438896709842\\
14.4947222222222	0.00171989616739825\\
14.5002777777778	0.00916423956423551\\
14.5058333333333	0.0104942168623608\\
14.5113888888889	0.020546637147288\\
14.5169444444444	0.012846193988365\\
14.5225	0.0241507695465971\\
14.5280555555556	0.0248267189730663\\
14.5336111111111	0.0275512124560267\\
14.5391666666667	0.0341188594640499\\
14.5447222222222	0.0387151847728512\\
14.5502777777778	0.0281457502466555\\
14.5558333333333	0.0306435727465608\\
14.5613888888889	0.0339840213427139\\
14.5669444444444	0.0300783133966705\\
14.5725	0.0308224750537466\\
14.5780555555556	0.0293118069423599\\
14.5836111111111	0.0476323388963035\\
14.5891666666667	0.0432169628754773\\
14.5947222222222	0.0294273215769082\\
14.6002777777778	0.0436145087967422\\
14.6058333333333	0.0395745730394035\\
14.6113888888889	0.0509704257086272\\
14.6169444444444	0.0530168691031822\\
14.6225	0.0451768899072571\\
14.6280555555556	0.0349290815458897\\
14.6336111111111	0.0268785281369869\\
14.6391666666667	0.0332495094746283\\
14.6447222222222	0.031490712272655\\
14.6502777777778	0.0461994427118706\\
14.6558333333333	0.0367002527624323\\
14.6613888888889	0.0267367828259054\\
14.6669444444444	0.0244046928214823\\
14.6725	0.0248480158192365\\
14.6780555555556	0.0299486024432031\\
14.6836111111111	0.0322960155651898\\
14.6891666666667	0.0303989813655196\\
14.6947222222222	0.0393213874560235\\
14.7002777777778	0.0531284298512168\\
14.7058333333333	0.0593720834893567\\
14.7113888888889	0.0607219917345649\\
14.7169444444444	0.0556709195243621\\
14.7225	0.0493053172404798\\
14.7280555555556	0.053552009692359\\
14.7336111111111	0.0514090518975427\\
14.7391666666667	0.050168925735416\\
14.7447222222222	0.0525418936802426\\
14.7502777777778	0.0436997898895198\\
14.7558333333333	0.0435936335839579\\
14.7613888888889	0.0497507782253658\\
14.7669444444444	0.0565692350634988\\
14.7725	0.0605446061148385\\
14.7780555555556	0.049842022633149\\
14.7836111111111	0.0567587762834111\\
14.7891666666667	0.053602456748119\\
14.7947222222222	0.0441466773666663\\
14.8002777777778	0.0506894638676724\\
14.8058333333333	0.0478215439409221\\
14.8113888888889	0.0468052698654385\\
14.8169444444444	0.0536654131661158\\
14.8225	0.055889732057757\\
14.8280555555556	0.0486577467213356\\
14.8336111111111	0.0443754362946511\\
14.8391666666667	0.0515479412737399\\
14.8447222222222	0.0469593618492758\\
14.8502777777778	0.0586967415995148\\
14.8558333333333	0.0595110001865988\\
14.8613888888889	0.0609131399633549\\
14.8669444444444	0.0664405063769006\\
14.8725	0.0657363657100554\\
14.8780555555556	0.0615112215080887\\
14.8836111111111	0.067845794065683\\
14.8891666666667	0.0716287088135091\\
14.8947222222222	0.0699123325050584\\
14.9002777777778	0.0742319712860422\\
14.9058333333333	0.0732920448375669\\
14.9113888888889	0.0779381432277334\\
14.9169444444444	0.0876408629599933\\
14.9225	0.0806219784047554\\
14.9280555555556	0.0873466267890055\\
14.9336111111111	0.0891247107107467\\
14.9391666666667	0.0862024134665654\\
14.9447222222222	0.087469976901397\\
14.9502777777778	0.0854050954209945\\
14.9558333333333	0.0895789898296858\\
14.9613888888889	0.0794359065136235\\
14.9669444444444	0.0657047987996858\\
14.9725	0.0723654302683805\\
14.9780555555556	0.0805482564897448\\
14.9836111111111	0.0882228742107839\\
14.9891666666667	0.0845921965743358\\
14.9947222222222	0.0731181563454152\\
15.0002777777778	0.0807819888598576\\
15.0058333333333	0.0718447463889372\\
15.0113888888889	0.0720195442187615\\
15.0169444444444	0.0843384005265487\\
15.0225	0.0812228445690115\\
15.0280555555556	0.0728941618447476\\
15.0336111111111	0.065616663696118\\
15.0391666666667	0.0693757712517347\\
15.0447222222222	0.0698165932369195\\
15.0502777777778	0.066864295609353\\
15.0558333333333	0.0498115215929784\\
15.0613888888889	0.0433204453918823\\
15.0669444444444	0.0498678825101883\\
15.0725	0.0449045788839696\\
15.0780555555556	0.0465780338332977\\
15.0836111111111	0.0365175559888539\\
15.0891666666667	0.0260360118032976\\
15.0947222222222	0.0319269853438645\\
15.1002777777778	0.0309564833316097\\
15.1058333333333	0.0282948542668394\\
15.1113888888889	0.0207779243800553\\
15.1169444444444	0.0145379209600762\\
15.1225	0.00959481029932168\\
15.1280555555556	0.0108377610161748\\
15.1336111111111	0.0100407134037267\\
15.1391666666667	0.0098506419902292\\
15.1447222222222	0.0106380254333355\\
15.1502777777778	0.00453355305772456\\
15.1558333333333	0.00225879346648166\\
15.1613888888889	0.00600662633383502\\
15.1669444444444	0.00752758531218323\\
15.1725	0.0167250779223925\\
15.1780555555556	0.0209602581251105\\
15.1836111111111	0.0210714010345273\\
15.1891666666667	0.026380867379281\\
15.1947222222222	0.0231606118400427\\
15.2002777777778	0.0176211785049869\\
15.2058333333333	0.00817551505031138\\
15.2113888888889	0.000399078317786662\\
15.2169444444444	0.00477214759882383\\
15.2225	0.00333738225543407\\
15.2280555555556	0.00138136201738725\\
15.2336111111111	0.00875043820253996\\
15.2391666666667	0.0129894794613698\\
15.2447222222222	0.0155309238680097\\
15.2502777777778	0.028213260081516\\
15.2558333333333	0.0249795492095544\\
15.2613888888889	0.0357141455206282\\
15.2669444444444	0.0386890804842569\\
15.2725	0.039664432932311\\
15.2780555555556	0.0396820035449535\\
15.2836111111111	0.0521265793758859\\
15.2891666666667	0.0624403971769824\\
15.2947222222222	0.0702572799836867\\
15.3002777777778	0.0809995414712804\\
15.3058333333333	0.0687922950251105\\
15.3113888888889	0.0655454648246318\\
15.3169444444444	0.0601788910576588\\
15.3225	0.0577344972060378\\
15.3280555555556	0.0450041380597124\\
15.3336111111111	0.0334994043765338\\
15.3391666666667	0.0276564791521593\\
15.3447222222222	0.0309772733643179\\
15.3502777777778	0.0389315971932557\\
15.3558333333333	0.0310890951027953\\
15.3613888888889	0.0374259200727149\\
15.3669444444444	0.0277188923269725\\
15.3725	0.0260341146320067\\
15.3780555555556	0.0208973371880404\\
15.3836111111111	0.02740970677296\\
15.3891666666667	0.0311184367705555\\
15.3947222222222	0.0289233379460867\\
15.4002777777778	0.026550038974372\\
15.4058333333333	0.0278996029677617\\
15.4113888888889	0.0509181327824846\\
15.4169444444444	0.0579359821828345\\
15.4225	0.0551660644834701\\
15.4280555555556	0.0508568089698801\\
15.4336111111111	0.0540289784464138\\
15.4391666666667	0.0521362654838987\\
15.4447222222222	0.0521770987266432\\
15.4502777777778	0.0603851978283794\\
15.4558333333333	0.0470602356805008\\
15.4613888888889	0.0579014830508044\\
15.4669444444444	0.0508225588165137\\
15.4725	0.0501645891394217\\
15.4780555555556	0.0424946958012749\\
15.4836111111111	0.0476398372328613\\
15.4891666666667	0.0555166082615644\\
15.4947222222222	0.0394390943569142\\
15.5002777777778	0.0486385004093349\\
15.5058333333333	0.0453881200078994\\
15.5113888888889	0.0336906993626112\\
15.5169444444444	0.0356923226262331\\
15.5225	0.0428161477116864\\
15.5280555555556	0.0332558526064181\\
15.5336111111111	0.0182102444218523\\
15.5391666666667	0.0300578663623032\\
15.5447222222222	0.0361739241575202\\
15.5502777777778	0.0219533539849083\\
15.5558333333333	0.0284418043182431\\
15.5613888888889	0.0296386055763487\\
15.5669444444444	0.0337210880483422\\
15.5725	0.0325881555093375\\
15.5780555555556	0.0437696206209817\\
15.5836111111111	0.0417076875064812\\
15.5891666666667	0.0347053573342229\\
15.5947222222222	0.0469219096142835\\
15.6002777777778	0.0322436412190766\\
15.6058333333333	0.0332239238707418\\
15.6113888888889	0.0414054918366066\\
15.6169444444444	0.0485562923325904\\
15.6225	0.0469841382827624\\
15.6280555555556	0.0518945561668787\\
15.6336111111111	0.051618027886822\\
15.6391666666667	0.0421803405387214\\
15.6447222222222	0.0301377240725315\\
15.6502777777778	0.0249688528049024\\
15.6558333333333	0.0102946064099091\\
15.6613888888889	0.0166470270096985\\
15.6669444444444	0.0168834651524319\\
15.6725	0.0239607312774189\\
15.6780555555556	0.0200136431189353\\
15.6836111111111	0.0140027983346522\\
15.6891666666667	0.0177774259982896\\
15.6947222222222	0.0283738837882204\\
15.7002777777778	0.021823990340515\\
15.7058333333333	0.0148646395410519\\
15.7113888888889	0.0116897598083099\\
15.7169444444444	0.0175065759342849\\
15.7225	0.0159064860927504\\
15.7280555555556	0.0300141147272391\\
15.7336111111111	0.0261568751195495\\
15.7391666666667	0.0204179682460436\\
15.7447222222222	0.0284900468797546\\
15.7502777777778	0.033826251819352\\
15.7558333333333	0.0389692655048372\\
15.7613888888889	0.0470505249307873\\
15.7669444444444	0.0485246288003064\\
15.7725	0.0409826852900815\\
15.7780555555556	0.0486016612185502\\
15.7836111111111	0.0665618377761152\\
15.7891666666667	0.0819809460660162\\
15.7947222222222	0.0836594695262569\\
15.8002777777778	0.10016382642855\\
15.8058333333333	0.0919459502407068\\
15.8113888888889	0.0995409021085981\\
15.8169444444444	0.092958143235044\\
15.8225	0.103158456215091\\
15.8280555555556	0.10179058548521\\
15.8336111111111	0.065379932219286\\
15.8391666666667	0.0778890567457442\\
15.8447222222222	0.078705373170792\\
15.8502777777778	0.066925974429208\\
15.8558333333333	0.0727940269139718\\
15.8613888888889	0.0690781435117841\\
15.8669444444444	0.0437401261253115\\
15.8725	0.0441582829743875\\
15.8780555555556	0.0534724343476718\\
15.8836111111111	0.0635447908147795\\
15.8891666666667	0.0659630834598811\\
15.8947222222222	0.0733489346329038\\
15.9002777777778	0.0692375637145567\\
15.9058333333333	0.071092351072871\\
15.9113888888889	0.0646399305726267\\
15.9169444444444	0.0747348968753649\\
15.9225	0.0885525227150588\\
15.9280555555556	0.0887317489152738\\
15.9336111111111	0.08016411998333\\
15.9391666666667	0.08665751155829\\
15.9447222222222	0.083366192906552\\
15.9502777777778	0.0908783851171676\\
15.9558333333333	0.0858058305855456\\
15.9613888888889	0.0731761278308082\\
15.9669444444444	0.0783873746703661\\
15.9725	0.0683640489262526\\
15.9780555555556	0.0405929878506045\\
15.9836111111111	0.0319568297361209\\
15.9891666666667	0.0193986308303998\\
15.9947222222222	0.0438261169173193\\
};
\addplot [color=mycolor3,solid,line width=1.5pt,forget plot]
  table[row sep=crcr]{%
9.50027777777778	0\\
9.50583333333333	-0.000514323449969491\\
9.51138888888889	-0.000953796703887714\\
9.51694444444444	0.000334851858194769\\
9.5225	0.00173610970089218\\
9.52805555555556	0.00333871656275489\\
9.53361111111111	0.00226112695475088\\
9.53916666666667	0.00210054497201958\\
9.54472222222222	0.0032234290552511\\
9.55027777777778	0.00270296967269441\\
9.55583333333333	0.00279458904079758\\
9.56138888888889	0.00239230355235489\\
9.56694444444444	0.0027341853624351\\
9.5725	0.0028085357182076\\
9.57805555555555	0.00326154420315498\\
9.58361111111111	0.0026191809224295\\
9.58916666666667	0.0032187822365513\\
9.59472222222222	0.00398877221812599\\
9.60027777777778	0.0046930397459423\\
9.60583333333333	0.00544230516958842\\
9.61138888888889	0.00561097306991204\\
9.61694444444444	0.00504955849481572\\
9.6225	0.00614183826196727\\
9.62805555555556	0.00745928163852233\\
9.63361111111111	0.00836211535358001\\
9.63916666666667	0.00848122596635007\\
9.64472222222222	0.00784931634349396\\
9.65027777777778	0.00691374509248608\\
9.65583333333333	0.00787933978238592\\
9.66138888888889	0.00891259045304982\\
9.66694444444444	0.00895058126190382\\
9.6725	0.0095951365728451\\
9.67805555555555	0.00792548693257025\\
9.68361111111111	0.00668272549567445\\
9.68916666666667	0.00604286125054573\\
9.69472222222222	0.0062582905646282\\
9.70027777777778	0.0071056378250876\\
9.70583333333333	0.00812537789273704\\
9.71138888888889	0.00711506621095832\\
9.71694444444444	0.00710995208251005\\
9.7225	0.00537297781593626\\
9.72805555555555	0.00327280315861581\\
9.73361111111111	0.00294806698110815\\
9.73916666666667	0.00366829823665274\\
9.74472222222222	0.00459376620361191\\
9.75027777777778	0.00533571316062172\\
9.75583333333333	0.00587949480984469\\
9.76138888888889	0.0054387425526415\\
9.76694444444444	0.0063346171868101\\
9.7725	0.00589729862294016\\
9.77805555555556	0.00497648059441248\\
9.78361111111111	0.00560420655801174\\
9.78916666666667	0.00716030332500714\\
9.79472222222222	0.00680080333508892\\
9.80027777777778	0.00775915057587853\\
9.80583333333333	0.00730624769257062\\
9.81138888888889	0.00794522902179774\\
9.81694444444444	0.00908307166732239\\
9.8225	0.00906375804246266\\
9.82805555555555	0.00821836444294748\\
9.83361111111111	0.00768419547641777\\
9.83916666666667	0.00702543228391035\\
9.84472222222222	0.00726242821072885\\
9.85027777777778	0.00700014857604868\\
9.85583333333333	0.00668465397214575\\
9.86138888888889	0.00611729341710733\\
9.86694444444444	0.0052209565254972\\
9.8725	0.00563801813708396\\
9.87805555555556	0.00435502022139905\\
9.88361111111111	0.00440651662660563\\
9.88916666666667	0.00457166051830471\\
9.89472222222222	0.00413329801228182\\
9.90027777777778	0.00329618611258542\\
9.90583333333333	0.00425130014865926\\
9.91138888888889	0.00293984144468339\\
9.91694444444444	0.00371543706614026\\
9.9225	0.00400835275705147\\
9.92805555555555	0.00367305477671613\\
9.93361111111111	0.00417689119350184\\
9.93916666666667	0.00325065949498152\\
9.94472222222222	0.0028017946851421\\
9.95027777777778	0.00348676613672472\\
9.95583333333333	0.00298002801979522\\
9.96138888888889	0.00287490648724485\\
9.96694444444444	0.00209952067896102\\
9.9725	0.00273655900359049\\
9.97805555555555	0.00357262024789071\\
9.98361111111111	0.0039673282421574\\
9.98916666666667	0.00325757939484976\\
9.99472222222222	0.00220392889113425\\
10.0002777777778	0.00202909455097532\\
10.0058333333333	0.00106184368017085\\
10.0113888888889	0.00164877739187273\\
10.0169444444444	0.00339469333083765\\
10.0225	0.00229742885304633\\
10.0280555555556	0.000511785306572835\\
10.0336111111111	-0.000400816291577977\\
10.0391666666667	0.000523195380324142\\
10.0447222222222	0.0011036069248859\\
10.0502777777778	0.000938332607455848\\
10.0558333333333	0.000330049073785826\\
10.0613888888889	-0.000280455853064683\\
10.0669444444444	-0.00115300622741333\\
10.0725	-0.00163263194154712\\
10.0780555555556	-0.00137596218230339\\
10.0836111111111	-0.00193651431881533\\
10.0891666666667	-0.00222034875394383\\
10.0947222222222	-0.00304030384124559\\
10.1002777777778	-0.00355719955916098\\
10.1058333333333	-0.00351057691282123\\
10.1113888888889	-0.00239810355923741\\
10.1169444444444	-0.00306895369402347\\
10.1225	-0.00282660954786107\\
10.1280555555556	-0.00311068748988479\\
10.1336111111111	-0.0031737428302805\\
10.1391666666667	-0.00319010904151401\\
10.1447222222222	-0.00417764511663682\\
10.1502777777778	-0.00302026720841286\\
10.1558333333333	-0.00207098858193846\\
10.1613888888889	-0.00134021746421996\\
10.1669444444444	-0.000637534458501456\\
10.1725	-0.00180930714871088\\
10.1780555555556	-0.00193275892943097\\
10.1836111111111	-0.00189960476618406\\
10.1891666666667	-0.00187358215193106\\
10.1947222222222	-0.00251926622653683\\
10.2002777777778	-0.00195052877251844\\
10.2058333333333	-0.00139428834974707\\
10.2113888888889	-0.00083795126495933\\
10.2169444444444	0.000568652407651073\\
10.2225	-0.000452402581677369\\
10.2280555555556	5.93454188428159e-05\\
10.2336111111111	0.000512881499036346\\
10.2391666666667	0.00138337615090949\\
10.2447222222222	0.00131240058833869\\
10.2502777777778	0.00161917332573847\\
10.2558333333333	0.00193722155138597\\
10.2613888888889	0.00143686412459956\\
10.2669444444444	3.10103071758022e-05\\
10.2725	0.000543268848685064\\
10.2780555555556	-0.000207097278069947\\
10.2836111111111	0.000335958443004816\\
10.2891666666667	9.71603550784752e-05\\
10.2947222222222	-0.000528785406510739\\
10.3002777777778	0.00105145848687225\\
10.3058333333333	0.000979928311864253\\
10.3113888888889	0.000996984275536937\\
10.3169444444444	0.000799094234967594\\
10.3225	0.00132315987080399\\
10.3280555555556	0.000647457350904668\\
10.3336111111111	-0.00052269974833123\\
10.3391666666667	7.75352190110753e-05\\
10.3447222222222	-0.000640743978571017\\
10.3502777777778	-0.000131688511230958\\
10.3558333333333	0.00108083344399644\\
10.3613888888889	0.00136269133534968\\
10.3669444444444	0.00168333122167011\\
10.3725	0.00199672904035026\\
10.3780555555556	0.00222014930206979\\
10.3836111111111	0.00130942688496467\\
10.3891666666667	0.000451642560088244\\
10.3947222222222	0.00130208312562107\\
10.4002777777778	0.00175348534696059\\
10.4058333333333	0.00109673129730853\\
10.4113888888889	0.00146374511588473\\
10.4169444444444	0.00117333965397893\\
10.4225	0.00313153194430086\\
10.4280555555556	0.0035247500063082\\
10.4336111111111	0.00282099058712552\\
10.4391666666667	0.00208730082997179\\
10.4447222222222	0.00116861628919404\\
10.4502777777778	0.00092069914787488\\
10.4558333333333	0.000585846009205214\\
10.4613888888889	0.000708153909063101\\
10.4669444444444	0.00100236456451344\\
10.4725	0.00110800030319136\\
10.4780555555556	0.00100722924495195\\
10.4836111111111	0.00057858689221735\\
10.4891666666667	0.00110420710884132\\
10.4947222222222	0.00202056923826959\\
10.5002777777778	0.00260646742718608\\
10.5058333333333	0.00258810845106345\\
10.5113888888889	0.00313731946030732\\
10.5169444444444	0.00339917175047428\\
10.5225	0.00277832502668657\\
10.5280555555556	0.00331441074483909\\
10.5336111111111	0.00340689499276442\\
10.5391666666667	0.00153035041120327\\
10.5447222222222	0.000730024351054569\\
10.5502777777778	8.67059038754904e-05\\
10.5558333333333	0.000404862522194649\\
10.5613888888889	0.00102730732833796\\
10.5669444444444	0.000906889301237233\\
10.5725	0.000722652966854406\\
10.5780555555556	0.000244936866204114\\
10.5836111111111	-0.00129289027382304\\
10.5891666666667	-0.00212779882514784\\
10.5947222222222	-0.00230540654908301\\
10.6002777777778	-0.00273061511751002\\
10.6058333333333	-0.00296744180606355\\
10.6113888888889	-0.000830948693579888\\
10.6169444444444	-0.000808985608619436\\
10.6225	-0.00103221876949175\\
10.6280555555556	-0.00217216418543497\\
10.6336111111111	-0.00197010475959127\\
10.6391666666667	-0.00276010093665955\\
10.6447222222222	-0.00321710407810304\\
10.6502777777778	-0.00292019157905183\\
10.6558333333333	-0.00354196357429561\\
10.6613888888889	-0.00299219838637072\\
10.6669444444444	-0.00304377969818609\\
10.6725	-0.00305514776832895\\
10.6780555555556	-0.00297177496939623\\
10.6836111111111	-0.00207743323260892\\
10.6891666666667	-0.00285203035530663\\
10.6947222222222	-0.00340773393504451\\
10.7002777777778	-0.0023421793754583\\
10.7058333333333	-0.00285977392286845\\
10.7113888888889	-0.00249725366592663\\
10.7169444444444	-0.00307983741723883\\
10.7225	-0.00343085325419332\\
10.7280555555556	-0.00377096717048118\\
10.7336111111111	-0.00434586950423335\\
10.7391666666667	-0.00404446955413961\\
10.7447222222222	-0.00500666023251572\\
10.7502777777778	-0.00329606212365465\\
10.7558333333333	-0.00499838791791259\\
10.7613888888889	-0.00505063364699053\\
10.7669444444444	-0.00518439249558539\\
10.7725	-0.00455529191343263\\
10.7780555555556	-0.00463390315825499\\
10.7836111111111	-0.00475647284730394\\
10.7891666666667	-0.00538638545232014\\
10.7947222222222	-0.00558820599793399\\
10.8002777777778	-0.00540140472990573\\
10.8058333333333	-0.00395427576519999\\
10.8113888888889	-0.00311850057789599\\
10.8169444444444	-0.00309575637308954\\
10.8225	-0.00237039621676413\\
10.8280555555556	-0.00196755602642298\\
10.8336111111111	-0.00147438161006142\\
10.8391666666667	-0.00215379903318291\\
10.8447222222222	-0.00197801824175888\\
10.8502777777778	-0.00167362348021321\\
10.8558333333333	-0.00216890407107024\\
10.8613888888889	-0.000715140784968601\\
10.8669444444444	-0.000646708672186621\\
10.8725	-0.00117771365773942\\
10.8780555555556	-0.000916618813420614\\
10.8836111111111	-0.000342171382463971\\
10.8891666666667	-0.000134263562808969\\
10.8947222222222	-0.00127056414628324\\
10.9002777777778	-0.00108253563245643\\
10.9058333333333	-0.00148895189606656\\
10.9113888888889	-0.00170528254693469\\
10.9169444444444	-0.00090131807312143\\
10.9225	-0.0013902047274937\\
10.9280555555556	-0.00120958012080103\\
10.9336111111111	-0.00240298826780172\\
10.9391666666667	-0.0026972646935621\\
10.9447222222222	-0.0037121917895074\\
10.9502777777778	-0.00373097234453652\\
10.9558333333333	-0.00440643074501488\\
10.9613888888889	-0.00387012893110107\\
10.9669444444444	-0.00397219917004007\\
10.9725	-0.00341021950557059\\
10.9780555555556	-0.00364682357272878\\
10.9836111111111	-0.00417424257283511\\
10.9891666666667	-0.00449786022227876\\
10.9947222222222	-0.00388604927315871\\
11.0002777777778	-0.00451107616832883\\
11.0058333333333	-0.00456711056712452\\
11.0113888888889	-0.00419066445910775\\
11.0169444444444	-0.00474006437572352\\
11.0225	-0.00448579369138958\\
11.0280555555556	-0.0043981209859593\\
11.0336111111111	-0.00435357376780143\\
11.0391666666667	-0.00411288244046344\\
11.0447222222222	-0.00358171163734486\\
11.0502777777778	-0.00376178360108921\\
11.0558333333333	-0.00338953962674397\\
11.0613888888889	-0.00372670424771562\\
11.0669444444444	-0.0040282638776499\\
11.0725	-0.00405024427019434\\
11.0780555555556	-0.00359419891296412\\
11.0836111111111	-0.00398375565525548\\
11.0891666666667	-0.00438062767309403\\
11.0947222222222	-0.00446196474699881\\
11.1002777777778	-0.00371314383838111\\
11.1058333333333	-0.00276358305590919\\
11.1113888888889	-0.00284272094583016\\
11.1169444444444	-0.00268307188124679\\
11.1225	-0.00393133157990598\\
11.1280555555556	-0.004795059460723\\
11.1336111111111	-0.00480468385970219\\
11.1391666666667	-0.00510532107956524\\
11.1447222222222	-0.00443407523578523\\
11.1502777777778	-0.00376528649715135\\
11.1558333333333	-0.00392588835171269\\
11.1613888888889	-0.00359742134723179\\
11.1669444444444	-0.00412089992688608\\
11.1725	-0.00415080692841353\\
11.1780555555556	-0.00385039701519258\\
11.1836111111111	-0.00380874261699726\\
11.1891666666667	-0.00287091387456235\\
11.1947222222222	-0.00222199884047038\\
11.2002777777778	-0.00373918878294194\\
11.2058333333333	-0.00331230012291397\\
11.2113888888889	-0.00320835579308831\\
11.2169444444444	-0.0032839240154154\\
11.2225	-0.00339549807408648\\
11.2280555555556	-0.00371960237756008\\
11.2336111111111	-0.00338306257338845\\
11.2391666666667	-0.00303339276040778\\
11.2447222222222	-0.00426868423466914\\
11.2502777777778	-0.00474708669964328\\
11.2558333333333	-0.00398055228942462\\
11.2613888888889	-0.00405498149317655\\
11.2669444444444	-0.00346387178915933\\
11.2725	-0.0028320195867481\\
11.2780555555556	-0.00298404752436164\\
11.2836111111111	-0.00359455487165071\\
11.2891666666667	-0.00305160989668804\\
11.2947222222222	-0.00307114708821594\\
11.3002777777778	-0.00299231491694586\\
11.3058333333333	-0.00267051566616024\\
11.3113888888889	-0.00354926547084324\\
11.3169444444444	-0.00402951747118577\\
11.3225	-0.00361011113683003\\
11.3280555555556	-0.00312911278356937\\
11.3336111111111	-0.00358535794358126\\
11.3391666666667	-0.00344439844925154\\
11.3447222222222	-0.00397724318617014\\
11.3502777777778	-0.00319267171743474\\
11.3558333333333	-0.00320050329488587\\
11.3613888888889	-0.00363704938969682\\
11.3669444444444	-0.00351531624614965\\
11.3725	-0.00409653267716046\\
11.3780555555556	-0.00509124180332824\\
11.3836111111111	-0.00467843361192792\\
11.3891666666667	-0.00431205946690563\\
11.3947222222222	-0.00355312001626555\\
11.4002777777778	-0.00408795659305621\\
11.4058333333333	-0.00406035510802667\\
11.4113888888889	-0.00425097641452456\\
11.4169444444444	-0.00427117095076923\\
11.4225	-0.00391918350666621\\
11.4280555555556	-0.00458165797180532\\
11.4336111111111	-0.00461565405240559\\
11.4391666666667	-0.0037114415770205\\
11.4447222222222	-0.00443321931040318\\
11.4502777777778	-0.00413041698486839\\
11.4558333333333	-0.00388871213390429\\
11.4613888888889	-0.00360263495630563\\
11.4669444444444	-0.00406521770166873\\
11.4725	-0.00444128948797901\\
11.4780555555556	-0.00493996289968573\\
11.4836111111111	-0.00450944013609246\\
11.4891666666667	-0.00500092630821396\\
11.4947222222222	-0.00413742784326549\\
11.5002777777778	-0.0035783674094354\\
11.5058333333333	-0.00322962141930372\\
11.5113888888889	-0.0034569554752818\\
11.5169444444444	-0.00364033202001096\\
11.5225	-0.00307012553866634\\
11.5280555555556	-0.00340511225639435\\
11.5336111111111	-0.00324831918624207\\
11.5391666666667	-0.00379405353306122\\
11.5447222222222	-0.00415515941672539\\
11.5502777777778	-0.00393974488732673\\
11.5558333333333	-0.00482503594329752\\
11.5613888888889	-0.00447183359730841\\
11.5669444444444	-0.00449843525393923\\
11.5725	-0.00308805589509268\\
11.5780555555556	-0.00335002758156673\\
11.5836111111111	-0.00263809438904907\\
11.5891666666667	-0.00275873099041976\\
11.5947222222222	-0.00220007080452372\\
11.6002777777778	-0.00289192048962013\\
11.6058333333333	-0.00353879346451784\\
11.6113888888889	-0.00409159303547808\\
11.6169444444444	-0.00498710954495597\\
11.6225	-0.00462772068490029\\
11.6280555555556	-0.00475184850232987\\
11.6336111111111	-0.00372631691665359\\
11.6391666666667	-0.00396076882070051\\
11.6447222222222	-0.00381321710519898\\
11.6502777777778	-0.00345978373298005\\
11.6558333333333	-0.00349223474725775\\
11.6613888888889	-0.00245390054780564\\
11.6669444444444	-0.00228439822166245\\
11.6725	-0.00205788473412373\\
11.6780555555556	-0.00271939561013628\\
11.6836111111111	-0.00291301036115492\\
11.6891666666667	-0.00263182508350726\\
11.6947222222222	-0.00266685466672641\\
11.7002777777778	-0.00272817980697309\\
11.7058333333333	-0.0029573276948259\\
11.7113888888889	-0.00250722019770502\\
11.7169444444444	-0.00249517699021672\\
11.7225	-0.00198440569143693\\
11.7280555555556	-0.00138505562989477\\
11.7336111111111	-0.00114913609269353\\
11.7391666666667	-0.00104959311799123\\
11.7447222222222	-0.00119696499569726\\
11.7502777777778	-0.00153770596714459\\
11.7558333333333	-0.000803944742622873\\
11.7613888888889	-0.000434804264217817\\
11.7669444444444	-0.000541637436446167\\
11.7725	-0.00131763390956066\\
11.7780555555556	-0.000374723579397624\\
11.7836111111111	-0.000646980656942176\\
11.7891666666667	-0.000373914625787851\\
11.7947222222222	-0.000895495770882461\\
11.8002777777778	-0.000663265495336508\\
11.8058333333333	-0.000938193866938737\\
11.8113888888889	-0.000971144160097292\\
11.8169444444444	-0.00040285977698967\\
11.8225	-0.000891179418544735\\
11.8280555555556	-0.000897480581246329\\
11.8336111111111	-0.00124467817564643\\
11.8391666666667	-0.00165856810267154\\
11.8447222222222	-0.00162376737837132\\
11.8502777777778	-0.00146887013746037\\
11.8558333333333	-0.00176582176170758\\
11.8613888888889	-0.00111889561277239\\
11.8669444444444	-0.0010058526426267\\
11.8725	-0.000928332092103924\\
11.8780555555556	-0.00164788867096572\\
11.8836111111111	-0.0017383787686402\\
11.8891666666667	-0.00143264978311108\\
11.8947222222222	-0.00101451137183777\\
11.9002777777778	-0.000214917914998029\\
11.9058333333333	-0.000128413619050799\\
11.9113888888889	0.000219223022232961\\
11.9169444444444	0.000231372082771318\\
11.9225	-0.000200072059324341\\
11.9280555555556	-0.000723189162581987\\
11.9336111111111	-0.00068342195132658\\
11.9391666666667	-0.000672775038222043\\
11.9447222222222	-0.00144235838885412\\
11.9502777777778	-0.00187871140780705\\
11.9558333333333	-0.00220542723482163\\
11.9613888888889	-0.00242553982032643\\
11.9669444444444	-0.00182519988535106\\
11.9725	-0.0013660917506501\\
11.9780555555556	-0.00145165998296609\\
11.9836111111111	-0.00161717281646857\\
11.9891666666667	-0.0024779835583859\\
11.9947222222222	-0.00241909034120792\\
12.0002777777778	-0.00229265813506301\\
12.0058333333333	-0.0020937234144083\\
12.0113888888889	-0.00179384242341655\\
12.0169444444444	-0.00183930652676679\\
12.0225	-0.00198171817224842\\
12.0280555555556	-0.00166692683000048\\
12.0336111111111	-0.00126916084492117\\
12.0391666666667	-0.00150031909029751\\
12.0447222222222	-0.00192056962361216\\
12.0502777777778	-0.00208863742749706\\
12.0558333333333	-0.0012885878808747\\
12.0613888888889	-0.000884624726185533\\
12.0669444444444	-0.0004165135649759\\
12.0725	0.000255595953296309\\
12.0780555555556	-0.000204247802592151\\
12.0836111111111	0.000222179004898403\\
12.0891666666667	0.00126500813339086\\
12.0947222222222	0.00091096359356528\\
12.1002777777778	0.00147066798170104\\
12.1058333333333	0.00131072209557609\\
12.1113888888889	0.00074565969678472\\
12.1169444444444	-0.000429320080122356\\
12.1225	-0.000310362121974572\\
12.1280555555556	-0.000220305415439308\\
12.1336111111111	-0.000456779085410312\\
12.1391666666667	-6.30989885087545e-05\\
12.1447222222222	-0.000150638407164694\\
12.1502777777778	7.06167741325278e-05\\
12.1558333333333	-0.000682474303933579\\
12.1613888888889	-0.000280309301537101\\
12.1669444444444	-0.00104982602207813\\
12.1725	-0.000995593038368983\\
12.1780555555556	-0.000568586794463484\\
12.1836111111111	-0.000347058560259437\\
12.1891666666667	-0.000386791973813333\\
12.1947222222222	-0.000757448418703948\\
12.2002777777778	-0.00107218353854171\\
12.2058333333333	-0.00139515141524932\\
12.2113888888889	-0.00150226740779753\\
12.2169444444444	-0.00163444549620245\\
12.2225	-0.00178061500556922\\
12.2280555555556	-0.00171359903430213\\
12.2336111111111	-0.00167014070637566\\
12.2391666666667	-0.00206576968937661\\
12.2447222222222	-0.00291302787075855\\
12.2502777777778	-0.00200005187234586\\
12.2558333333333	-0.00206079713780551\\
12.2613888888889	-0.00237566712307966\\
12.2669444444444	-0.00281503049755257\\
12.2725	-0.00242293895501066\\
12.2780555555556	-0.00223301185082998\\
12.2836111111111	-0.00183341742695626\\
12.2891666666667	-0.00198983322364402\\
12.2947222222222	-0.00230152324018941\\
12.3002777777778	-0.00227354847975625\\
12.3058333333333	-0.00206833156876848\\
12.3113888888889	-0.00177968625945346\\
12.3169444444444	-0.00212420240357872\\
12.3225	-0.00159811624272372\\
12.3280555555556	-0.00110705583983675\\
12.3336111111111	-0.000502306745389804\\
12.3391666666667	-0.000791386672270856\\
12.3447222222222	-0.000844379365797749\\
12.3502777777778	-0.00040266769026333\\
12.3558333333333	-0.000728741248930731\\
12.3613888888889	-0.000573608990755844\\
12.3669444444444	-4.10347233598793e-05\\
12.3725	0.000763630560544456\\
12.3780555555556	0.000746516795422171\\
12.3836111111111	0.000590962450338472\\
12.3891666666667	0.000886618438572135\\
12.3947222222222	0.000941349959500322\\
12.4002777777778	0.000505757374885282\\
12.4058333333333	0.00105907024376567\\
12.4113888888889	0.00194733856145276\\
12.4169444444444	0.00205428871578923\\
12.4225	0.00232899754743657\\
12.4280555555556	0.00266189659578774\\
12.4336111111111	0.0025474217139877\\
12.4391666666667	0.00208832166105436\\
12.4447222222222	0.00211024745534459\\
12.4502777777778	0.00192540379737591\\
12.4558333333333	0.00162835088890143\\
12.4613888888889	0.00199964170514374\\
12.4669444444444	0.00224499622745884\\
12.4725	0.0013691001044272\\
12.4780555555556	0.00113484885721774\\
12.4836111111111	0.00053899168637164\\
12.4891666666667	0.000207636773507346\\
12.4947222222222	0.000733688639602513\\
12.5002777777778	0.00119435969574318\\
12.5058333333333	0.000759897341182097\\
12.5113888888889	0.000355682577312742\\
12.5169444444444	0.00126687656118803\\
12.5225	0.00135911796652488\\
12.5280555555556	0.000826260646560746\\
12.5336111111111	0.00121892718649409\\
12.5391666666667	0.00146984090014445\\
12.5447222222222	0.00175042633166851\\
12.5502777777778	0.00180916527965349\\
12.5558333333333	0.00102605155577368\\
12.5613888888889	0.000870926326233866\\
12.5669444444444	0.00167409807570616\\
12.5725	0.000753833156835435\\
12.5780555555556	0.00156888704628761\\
12.5836111111111	0.000942511180810228\\
12.5891666666667	0.00129698845401929\\
12.5947222222222	0.000935851259697722\\
12.6002777777778	0.000149650639678769\\
12.6058333333333	-4.34320585376657e-05\\
12.6113888888889	6.46459329833888e-05\\
12.6169444444444	5.77514642934257e-05\\
12.6225	4.41046768134284e-05\\
12.6280555555556	-1.81277243802497e-05\\
12.6336111111111	0.000315807044655716\\
12.6391666666667	0.000138582983253524\\
12.6447222222222	0.000273992023072941\\
12.6502777777778	0.000348391047100769\\
12.6558333333333	0.0004336222009377\\
12.6613888888889	0.000649490827692925\\
12.6669444444444	0.000957796069862329\\
12.6725	0.000656568051977881\\
12.6780555555556	0.000733518107889326\\
12.6836111111111	0.00084954678109828\\
12.6891666666667	0.000689635029455062\\
12.6947222222222	0.00122306544225585\\
12.7002777777778	0.00133311034724773\\
12.7058333333333	0.000871245701961286\\
12.7113888888889	0.000789352985497431\\
12.7169444444444	0.00122646499159639\\
12.7225	0.000314704227544306\\
12.7280555555556	-0.000235483980342699\\
12.7336111111111	-0.000301698265780389\\
12.7391666666667	-0.000125177945393641\\
12.7447222222222	-4.54136559053407e-05\\
12.7502777777778	0.0010503592334094\\
12.7558333333333	0.00169998996355891\\
12.7613888888889	0.00190554046145781\\
12.7669444444444	0.00205995580075468\\
12.7725	0.00173190076116451\\
12.7780555555556	0.00153773638719803\\
12.7836111111111	0.00199510217573822\\
12.7891666666667	0.00366292925552954\\
12.7947222222222	0.00417897937935696\\
12.8002777777778	0.00461537943607099\\
12.8058333333333	0.0045407850163154\\
12.8113888888889	0.00487493844822226\\
12.8169444444444	0.00520033847753506\\
12.8225	0.00450500603854626\\
12.8280555555556	0.00481189530700263\\
12.8336111111111	0.00470702392483625\\
12.8391666666667	0.00474347817229072\\
12.8447222222222	0.00533353834179454\\
12.8502777777778	0.00517134721282308\\
12.8558333333333	0.00519385647944865\\
12.8613888888889	0.00527145326278269\\
12.8669444444444	0.00554156297147885\\
12.8725	0.0070423436219469\\
12.8780555555556	0.00692404956260254\\
12.8836111111111	0.0066496519789046\\
12.8891666666667	0.00734100167696068\\
12.8947222222222	0.00668287786660119\\
12.9002777777778	0.00762633646297252\\
12.9058333333333	0.00843048516751244\\
12.9113888888889	0.00813363922954594\\
12.9169444444444	0.00777254896290263\\
12.9225	0.00860988360479605\\
12.9280555555556	0.00856487056100628\\
12.9336111111111	0.0086684680070721\\
12.9391666666667	0.00884919431858227\\
12.9447222222222	0.00826818102175614\\
12.9502777777778	0.00713870141537799\\
12.9558333333333	0.00679186622542132\\
12.9613888888889	0.0062251409581923\\
12.9669444444444	0.00622178273235538\\
12.9725	0.00593394559015184\\
12.9780555555556	0.00687343706888512\\
12.9836111111111	0.00666872640428531\\
12.9891666666667	0.00685305631108925\\
12.9947222222222	0.00657469196174992\\
13.0002777777778	0.00601450031661244\\
13.0058333333333	0.00652926710731919\\
13.0113888888889	0.00649425796553834\\
13.0169444444444	0.00595515103055291\\
13.0225	0.00691398765200182\\
13.0280555555556	0.00689513800193023\\
13.0336111111111	0.00711862313395017\\
13.0391666666667	0.00752522794765022\\
13.0447222222222	0.00833291982540689\\
13.0502777777778	0.00806143995072438\\
13.0558333333333	0.00790712628680757\\
13.0613888888889	0.00856990742405771\\
13.0669444444444	0.00996864343005732\\
13.0725	0.0101794221803449\\
13.0780555555556	0.0116844254274207\\
13.0836111111111	0.0119133730278813\\
13.0891666666667	0.0117360829696836\\
13.0947222222222	0.0120077913525607\\
13.1002777777778	0.012522732187493\\
13.1058333333333	0.012681423051581\\
13.1113888888889	0.0128427552253694\\
13.1169444444444	0.0126341023102352\\
13.1225	0.0126129464151963\\
13.1280555555556	0.0117687919028646\\
13.1336111111111	0.0123396733620045\\
13.1391666666667	0.0119096869983708\\
13.1447222222222	0.0126653614306159\\
13.1502777777778	0.0112447400548392\\
13.1558333333333	0.0114571642785594\\
13.1613888888889	0.0110693481885899\\
13.1669444444444	0.0124000665606977\\
13.1725	0.0108027729435702\\
13.1780555555556	0.00981796318649151\\
13.1836111111111	0.00932902621026006\\
13.1891666666667	0.00733987031373441\\
13.1947222222222	0.00745505301392764\\
13.2002777777778	0.0063051212530093\\
13.2058333333333	0.00607025610474532\\
13.2113888888889	0.0065846396954885\\
13.2169444444444	0.00650722758817559\\
13.2225	0.0057462345451459\\
13.2280555555556	0.00556800732574582\\
13.2336111111111	0.00611775520850956\\
13.2391666666667	0.00650571597709435\\
13.2447222222222	0.00639616392332461\\
13.2502777777778	0.00736285287035793\\
13.2558333333333	0.0078338217051407\\
13.2613888888889	0.0073451321214126\\
13.2669444444444	0.00791114843890321\\
13.2725	0.00839625408837652\\
13.2780555555556	0.00766594561973817\\
13.2836111111111	0.00783141711432423\\
13.2891666666667	0.00717022635629653\\
13.2947222222222	0.00725251605091229\\
13.3002777777778	0.00753581731811748\\
13.3058333333333	0.00778630113123818\\
13.3113888888889	0.00785636834248759\\
13.3169444444444	0.0085362668459436\\
13.3225	0.008623131248298\\
13.3280555555556	0.00873320760193112\\
13.3336111111111	0.00857267851600846\\
13.3391666666667	0.00840320662989313\\
13.3447222222222	0.00870585170058807\\
13.3502777777778	0.00868083423064256\\
13.3558333333333	0.00866531657107652\\
13.3613888888889	0.00899988367931832\\
13.3669444444444	0.00850850942732209\\
13.3725	0.00830252911973189\\
13.3780555555556	0.00812262950644973\\
13.3836111111111	0.00781366399130874\\
13.3891666666667	0.00697360816921218\\
13.3947222222222	0.00739789960799658\\
13.4002777777778	0.00754957663197511\\
13.4058333333333	0.00799686247051489\\
13.4113888888889	0.00710020462230856\\
13.4169444444444	0.00735607931193933\\
13.4225	0.00756404236917816\\
13.4280555555556	0.00738324973332903\\
13.4336111111111	0.00704737897609267\\
13.4391666666667	0.00702791387338939\\
13.4447222222222	0.00752906320489005\\
13.4502777777778	0.00803477209888354\\
13.4558333333333	0.00788616076370381\\
13.4613888888889	0.00767557376541902\\
13.4669444444444	0.00685755450741103\\
13.4725	0.00679597845110351\\
13.4780555555556	0.00578448143811335\\
13.4836111111111	0.00518016400000892\\
13.4891666666667	0.00612929424160148\\
13.4947222222222	0.00609172295955909\\
13.5002777777778	0.0054518401523471\\
13.5058333333333	0.00539566910186714\\
13.5113888888889	0.00591591803346037\\
13.5169444444444	0.0050330930003075\\
13.5225	0.00459015382537986\\
13.5280555555556	0.00500908192042878\\
13.5336111111111	0.00555137988609304\\
13.5391666666667	0.00546731683164193\\
13.5447222222222	0.00609051610366323\\
13.5502777777778	0.00563224837040981\\
13.5558333333333	0.00524535327965438\\
13.5613888888889	0.00577992620956416\\
13.5669444444444	0.00508138185911758\\
13.5725	0.00497079735910263\\
13.5780555555556	0.00593133734733866\\
13.5836111111111	0.00586721707657595\\
13.5891666666667	0.00658390346330367\\
13.5947222222222	0.00711850709570091\\
13.6002777777778	0.00681798929104116\\
13.6058333333333	0.00675506703217093\\
13.6113888888889	0.00639314976610206\\
13.6169444444444	0.00688964673214071\\
13.6225	0.00701242431811709\\
13.6280555555556	0.0075111058561206\\
13.6336111111111	0.00686095737566102\\
13.6391666666667	0.0065745875232661\\
13.6447222222222	0.00705510313022185\\
13.6502777777778	0.00726684315953444\\
13.6558333333333	0.006970445437551\\
13.6613888888889	0.00700775855131996\\
13.6669444444444	0.00727325657528081\\
13.6725	0.00702870750777247\\
13.6780555555556	0.00681999812978367\\
13.6836111111111	0.0073491412840738\\
13.6891666666667	0.00675937144495022\\
13.6947222222222	0.00687744901962066\\
13.7002777777778	0.00673596047651301\\
13.7058333333333	0.00633768463151941\\
13.7113888888889	0.00753053070595939\\
13.7169444444444	0.00748162320504592\\
13.7225	0.00762819178554391\\
13.7280555555556	0.00808543048853486\\
13.7336111111111	0.00830461049150756\\
13.7391666666667	0.00952533122928821\\
13.7447222222222	0.00928099637521054\\
13.7502777777778	0.00929661647808905\\
13.7558333333333	0.00951733705672445\\
13.7613888888889	0.00969289768054034\\
13.7669444444444	0.00986774856152734\\
13.7725	0.00937204615598904\\
13.7780555555556	0.00954856106860219\\
13.7836111111111	0.0101580470381501\\
13.7891666666667	0.0106721354956111\\
13.7947222222222	0.00953033892550296\\
13.8002777777778	0.00877915559526661\\
13.8058333333333	0.00916593043099845\\
13.8113888888889	0.00884705735139583\\
13.8169444444444	0.00842795756092683\\
13.8225	0.00910903337715533\\
13.8280555555556	0.00871897453746093\\
13.8336111111111	0.00805621207552339\\
13.8391666666667	0.00818158992931828\\
13.8447222222222	0.00842755674241804\\
13.8502777777778	0.00793365819921953\\
13.8558333333333	0.00784449570109041\\
13.8613888888889	0.00795628122935708\\
13.8669444444444	0.00803648575678138\\
13.8725	0.00758996624327999\\
13.8780555555556	0.00745648233331888\\
13.8836111111111	0.00742062200033154\\
13.8891666666667	0.00732871880689599\\
13.8947222222222	0.00834679834797077\\
13.9002777777778	0.0084312710690607\\
13.9058333333333	0.00938006485520979\\
13.9113888888889	0.009004514486752\\
13.9169444444444	0.00933910828130915\\
13.9225	0.0088508467636331\\
13.9280555555556	0.009085877568669\\
13.9336111111111	0.00908659284680546\\
13.9391666666667	0.00875032562867711\\
13.9447222222222	0.00929404094294285\\
13.9502777777778	0.00974874473913157\\
13.9558333333333	0.00923244473659489\\
13.9613888888889	0.00962758714349288\\
13.9669444444444	0.00971230902662437\\
13.9725	0.00981789439820843\\
13.9780555555556	0.00993679729257664\\
13.9836111111111	0.00954969829108065\\
13.9891666666667	0.00965403552316911\\
13.9947222222222	0.00947594252824929\\
14.0002777777778	0.00934139547000273\\
14.0058333333333	0.0101060765517199\\
14.0113888888889	0.0107787159255272\\
14.0169444444444	0.0108112361608688\\
14.0225	0.0100750127911365\\
14.0280555555556	0.0102786032277065\\
14.0336111111111	0.0111644453971083\\
14.0391666666667	0.0107987236172361\\
14.0447222222222	0.00985631284018591\\
14.0502777777778	0.00916665985791173\\
14.0558333333333	0.00861062360757813\\
14.0613888888889	0.0078796304734511\\
14.0669444444444	0.00745806249814136\\
14.0725	0.00720125715532579\\
14.0780555555556	0.00654185451371459\\
14.0836111111111	0.00570116075286341\\
14.0891666666667	0.00623184210083214\\
14.0947222222222	0.00612480552954576\\
14.1002777777778	0.00629438954259701\\
14.1058333333333	0.0066722196466595\\
14.1113888888889	0.00698181013064669\\
14.1169444444444	0.00724129185007394\\
14.1225	0.00717094006408553\\
14.1280555555556	0.00725193029288644\\
14.1336111111111	0.00645256362758691\\
14.1391666666667	0.00573142316238214\\
14.1447222222222	0.00605117772946644\\
14.1502777777778	0.00648794573987318\\
14.1558333333333	0.00653751234043066\\
14.1613888888889	0.00585046770614353\\
14.1669444444444	0.0058368367405854\\
14.1725	0.00518064461849305\\
14.1780555555556	0.00494273557183318\\
14.1836111111111	0.00465793377716042\\
14.1891666666667	0.00444714258052441\\
14.1947222222222	0.00433462429495734\\
14.2002777777778	0.00406956292534186\\
14.2058333333333	0.00464836057848394\\
14.2113888888889	0.00416447967506709\\
14.2169444444444	0.00332867164915859\\
14.2225	0.00351290045366913\\
14.2280555555556	0.00419174544258059\\
14.2336111111111	0.00472256895606001\\
14.2391666666667	0.00406514229925658\\
14.2447222222222	0.00431069752421558\\
14.2502777777778	0.00446901448121608\\
14.2558333333333	0.00457069732321879\\
14.2613888888889	0.00469905353018494\\
14.2669444444444	0.00457671903932943\\
14.2725	0.00529139842791206\\
14.2780555555556	0.00611596776029792\\
14.2836111111111	0.00583055802067502\\
14.2891666666667	0.00638850911345862\\
14.2947222222222	0.00694422039417363\\
14.3002777777778	0.00735208945977254\\
14.3058333333333	0.00673620248884645\\
14.3113888888889	0.00665715737731592\\
14.3169444444444	0.0066165091737259\\
14.3225	0.00632574987988629\\
14.3280555555556	0.00571685169168583\\
14.3336111111111	0.00596034994387274\\
14.3391666666667	0.00623502210044693\\
14.3447222222222	0.00538988610998734\\
14.3502777777778	0.00589579937049161\\
14.3558333333333	0.00639548417285899\\
14.3613888888889	0.00658024875706834\\
14.3669444444444	0.00601240963671024\\
14.3725	0.00524839177535361\\
14.3780555555556	0.00570764952878586\\
14.3836111111111	0.00584908530279169\\
14.3891666666667	0.00689739581873195\\
14.3947222222222	0.00802918120709529\\
14.4002777777778	0.00768587558250153\\
14.4058333333333	0.00659944312814234\\
14.4113888888889	0.00654525653727831\\
14.4169444444444	0.00618162956677782\\
14.4225	0.00667431032014041\\
14.4280555555556	0.00605176239938945\\
14.4336111111111	0.00599801700559144\\
14.4391666666667	0.00594087851833564\\
14.4447222222222	0.00566751823662606\\
14.4502777777778	0.00580258095032661\\
14.4558333333333	0.00600765064079484\\
14.4613888888889	0.00676511905930883\\
14.4669444444444	0.00573128306896835\\
14.4725	0.00539219155677326\\
14.4780555555556	0.00610408275599515\\
14.4836111111111	0.00569349766049855\\
14.4891666666667	0.00537865733937709\\
14.4947222222222	0.00498982160973081\\
14.5002777777778	0.00481066573846144\\
14.5058333333333	0.00486005783800563\\
14.5113888888889	0.00442683375043072\\
14.5169444444444	0.00425204843808722\\
14.5225	0.00423356014446613\\
14.5280555555556	0.00408920346795073\\
14.5336111111111	0.00344851049574089\\
14.5391666666667	0.0038136221193255\\
14.5447222222222	0.00384009690260726\\
14.5502777777778	0.00433877847999584\\
14.5558333333333	0.00543961526811189\\
14.5613888888889	0.00482411469506751\\
14.5669444444444	0.00495758637227748\\
14.5725	0.0053879679794908\\
14.5780555555556	0.00512224432546834\\
14.5836111111111	0.00449286067950914\\
14.5891666666667	0.00452010188284538\\
14.5947222222222	0.00371658628266672\\
14.6002777777778	0.00384385174824601\\
14.6058333333333	0.00488140644680714\\
14.6113888888889	0.00457998925817242\\
14.6169444444444	0.00509240140168832\\
14.6225	0.00463277354130636\\
14.6280555555556	0.0045768695063156\\
14.6336111111111	0.0048755767687265\\
14.6391666666667	0.00469238487214515\\
14.6447222222222	0.00400469702424021\\
14.6502777777778	0.00366305758131413\\
14.6558333333333	0.00376981394301333\\
14.6613888888889	0.00437518983253928\\
14.6669444444444	0.00384753299893519\\
14.6725	0.00331645310545464\\
14.6780555555556	0.00305976893530262\\
14.6836111111111	0.00372379698678398\\
14.6891666666667	0.0034932879653616\\
14.6947222222222	0.00333240908720044\\
14.7002777777778	0.00312225307647728\\
14.7058333333333	0.00249492177584873\\
14.7113888888889	0.0028002324646221\\
14.7169444444444	0.00297424735281507\\
14.7225	0.00348027638999667\\
14.7280555555556	0.00361987373066094\\
14.7336111111111	0.00390026340647486\\
14.7391666666667	0.00405895457097699\\
14.7447222222222	0.00437698446005062\\
14.7502777777778	0.00430943004855854\\
14.7558333333333	0.00456466225173591\\
14.7613888888889	0.00432654925606591\\
14.7669444444444	0.0048865919884305\\
14.7725	0.00430412286124937\\
14.7780555555556	0.00479762438799344\\
14.7836111111111	0.00499163177912513\\
14.7891666666667	0.00510253116949976\\
14.7947222222222	0.00459031689671852\\
14.8002777777778	0.00456537564160766\\
14.8058333333333	0.00419899234943286\\
14.8113888888889	0.00402644359734927\\
14.8169444444444	0.00366855476994559\\
14.8225	0.00387023092880461\\
14.8280555555556	0.00369302735499386\\
14.8336111111111	0.00363286222425126\\
14.8391666666667	0.00350366028208726\\
14.8447222222222	0.00337835561049324\\
14.8502777777778	0.00330204731139506\\
14.8558333333333	0.00381681821485884\\
14.8613888888889	0.00374386308513354\\
14.8669444444444	0.00342807157282452\\
14.8725	0.00396623162423832\\
14.8780555555556	0.00413630211176881\\
14.8836111111111	0.00505446942622284\\
14.8891666666667	0.00508422077381504\\
14.8947222222222	0.00486696815394319\\
14.9002777777778	0.00491695948144172\\
14.9058333333333	0.00520677880097667\\
14.9113888888889	0.00548141687178108\\
14.9169444444444	0.00518110318549123\\
14.9225	0.00522346650231653\\
14.9280555555556	0.0060604331956149\\
14.9336111111111	0.00631836514735429\\
14.9391666666667	0.0057987507460059\\
14.9447222222222	0.00561849824099694\\
14.9502777777778	0.00612694584966235\\
14.9558333333333	0.00653311643792511\\
14.9613888888889	0.00577516883104985\\
14.9669444444444	0.00556234875662154\\
14.9725	0.00599930519570484\\
14.9780555555556	0.00589358107700596\\
14.9836111111111	0.00621819296602132\\
14.9891666666667	0.00608739744232448\\
14.9947222222222	0.0047299292835045\\
15.0002777777778	0.00498217288387881\\
15.0058333333333	0.00430955041171626\\
15.0113888888889	0.00357057539636121\\
15.0169444444444	0.003688895073088\\
15.0225	0.00396155182766184\\
15.0280555555556	0.00421127637023093\\
15.0336111111111	0.00343618885970347\\
15.0391666666667	0.00384185019709173\\
15.0447222222222	0.0043358639893643\\
15.0502777777778	0.00452569411091784\\
15.0558333333333	0.00453762804100629\\
15.0613888888889	0.00424188280145415\\
15.0669444444444	0.00449384845163288\\
15.0725	0.00459970610153808\\
15.0780555555556	0.00490293140522476\\
15.0836111111111	0.00457576335823739\\
15.0891666666667	0.00421588515088014\\
15.0947222222222	0.00457631308219482\\
15.1002777777778	0.00463297789859536\\
15.1058333333333	0.00454721210079408\\
15.1113888888889	0.00442067835340837\\
15.1169444444444	0.00415820766431321\\
15.1225	0.00402239405563782\\
15.1280555555556	0.00409456966194358\\
15.1336111111111	0.00462482358811969\\
15.1391666666667	0.00441584258748603\\
15.1447222222222	0.00468438145044371\\
15.1502777777778	0.00446258369274875\\
15.1558333333333	0.00431633317881334\\
15.1613888888889	0.00536260364750673\\
15.1669444444444	0.00527158469913087\\
15.1725	0.00509915187274222\\
15.1780555555556	0.00444362465643159\\
15.1836111111111	0.00454741457178301\\
15.1891666666667	0.00464096022736985\\
15.1947222222222	0.00503951508759616\\
15.2002777777778	0.00480081800314465\\
15.2058333333333	0.00424351109594262\\
15.2113888888889	0.00408886775881991\\
15.2169444444444	0.00379371762584988\\
15.2225	0.00412099946513543\\
15.2280555555556	0.00453906040459064\\
15.2336111111111	0.00407048480888604\\
15.2391666666667	0.00458287730815166\\
15.2447222222222	0.00482402961185701\\
15.2502777777778	0.00579008060270819\\
15.2558333333333	0.00590704836743781\\
15.2613888888889	0.00640188127837896\\
15.2669444444444	0.00577657190841908\\
15.2725	0.00590620009644705\\
15.2780555555556	0.00555310022513179\\
15.2836111111111	0.00629039730387668\\
15.2891666666667	0.00673998113591957\\
15.2947222222222	0.00653864505359226\\
15.3002777777778	0.0047717617500194\\
15.3058333333333	0.00531953862461779\\
15.3113888888889	0.00407632088444645\\
15.3169444444444	0.00430103478388467\\
15.3225	0.00408390872844296\\
15.3280555555556	0.00361037642734325\\
15.3336111111111	0.00385591263431732\\
15.3391666666667	0.004072750189098\\
15.3447222222222	0.00394824367805092\\
15.3502777777778	0.00338073324702568\\
15.3558333333333	0.00223992740689255\\
15.3613888888889	0.00204091745099399\\
15.3669444444444	0.00174278892978111\\
15.3725	0.00169456605739987\\
15.3780555555556	0.00197143705285832\\
15.3836111111111	0.00146920036546582\\
15.3891666666667	0.00248602699361772\\
15.3947222222222	0.0027310596373762\\
15.4002777777778	0.00235260198897675\\
15.4058333333333	0.00228106250268823\\
15.4113888888889	0.00238791667601453\\
15.4169444444444	0.00212665340731907\\
15.4225	0.00175315252512903\\
15.4280555555556	0.00178588130967719\\
15.4336111111111	0.00111730148798872\\
15.4391666666667	0.00145860621207187\\
15.4447222222222	0.00165578860503986\\
15.4502777777778	0.00154786987685249\\
15.4558333333333	0.00313118280317737\\
15.4613888888889	0.00334867702003635\\
15.4669444444444	0.00314442008714938\\
15.4725	0.00335879379066545\\
15.4780555555556	0.00410396130526624\\
15.4836111111111	0.00483106823164994\\
15.4891666666667	0.00517247627974423\\
15.4947222222222	0.00516381167231321\\
15.5002777777778	0.00435068585919617\\
15.5058333333333	0.00446065874997277\\
15.5113888888889	0.00405761794581775\\
15.5169444444444	0.00337653975316321\\
15.5225	0.00311001861089337\\
15.5280555555556	0.00378956168476303\\
15.5336111111111	0.00418533370663734\\
15.5391666666667	0.00375470396241705\\
15.5447222222222	0.00397295785328655\\
15.5502777777778	0.00405448927305826\\
15.5558333333333	0.00393392394127643\\
15.5613888888889	0.00391370959402614\\
15.5669444444444	0.00461233600021788\\
15.5725	0.00436580805734037\\
15.5780555555556	0.00469715160900224\\
15.5836111111111	0.00429689390987487\\
15.5891666666667	0.0037481142913141\\
15.5947222222222	0.00345647010347337\\
15.6002777777778	0.00481938571300956\\
15.6058333333333	0.00388436981469255\\
15.6113888888889	0.00319854727582333\\
15.6169444444444	0.00335615808263843\\
15.6225	0.00384221404412373\\
15.6280555555556	0.00429313492535888\\
15.6336111111111	0.00411476362414688\\
15.6391666666667	0.00426365713122972\\
15.6447222222222	0.00434242508219468\\
15.6502777777778	0.00438213048286313\\
15.6558333333333	0.00505990543394837\\
15.6613888888889	0.00454398721847149\\
15.6669444444444	0.00458146071034714\\
15.6725	0.00499073217268938\\
15.6780555555556	0.00534746291601459\\
15.6836111111111	0.00597749584502682\\
15.6891666666667	0.00584798254206727\\
15.6947222222222	0.00525193521813752\\
15.7002777777778	0.00494051250534749\\
15.7058333333333	0.00562515899998729\\
15.7113888888889	0.00614593362517847\\
15.7169444444444	0.00555718217764897\\
15.7225	0.00493504354483104\\
15.7280555555556	0.00372885934479663\\
15.7336111111111	0.00417023991194019\\
15.7391666666667	0.00314171357896459\\
15.7447222222222	0.00293507920042881\\
15.7502777777778	0.00192738019724077\\
15.7558333333333	0.00153802884458476\\
15.7613888888889	0.00253462028871906\\
15.7669444444444	0.00254567013650782\\
15.7725	0.00275288967712962\\
15.7780555555556	0.00317542491900814\\
15.7836111111111	0.00156420042861355\\
15.7891666666667	0.00192956516632051\\
15.7947222222222	0.00104613027159342\\
15.8002777777778	0.00100429796889769\\
15.8058333333333	0.00167896153137424\\
15.8113888888889	0.00140942028934961\\
15.8169444444444	0.00128257730699704\\
15.8225	0.00110829744203331\\
15.8280555555556	0.00130641655170102\\
15.8336111111111	0.000251209596930347\\
15.8391666666667	0.0010462025528925\\
15.8447222222222	0.00103903481431772\\
15.8502777777778	0.000913908992682474\\
15.8558333333333	0.000658091195317515\\
15.8613888888889	-9.64703067832501e-05\\
15.8669444444444	0.000769577838071813\\
15.8725	0.000407238839840506\\
15.8780555555556	0.00150659194601882\\
15.8836111111111	0.00184167797446206\\
15.8891666666667	0.00190819639367487\\
15.8947222222222	0.00135298831691065\\
15.9002777777778	0.00246889920702419\\
15.9058333333333	0.00139533653615184\\
15.9113888888889	0.00211853835214144\\
15.9169444444444	0.00176986643186039\\
15.9225	0.00235106874192113\\
15.9280555555556	0.00209607359632609\\
15.9336111111111	0.00261991075471825\\
15.9391666666667	0.0026253631894149\\
15.9447222222222	0.00263945389689536\\
15.9502777777778	0.00216818421749466\\
15.9558333333333	0.00189533609334857\\
15.9613888888889	0.00243483833022285\\
15.9669444444444	0.00127685293674779\\
15.9725	0.00207761723839843\\
15.9780555555556	0.00247882603654894\\
15.9836111111111	0.00210416002582442\\
15.9891666666667	0.00312371619661696\\
15.9947222222222	0.00143374136618994\\
};
\end{axis}
\end{tikzpicture}%
%
\end{subfigure}%
\hspace{1.5cm}%
\begin{subfigure}[b]{.4\linewidth}%
  %\centering
  \setlength\figureheight{\linewidth}%
  \setlength\figurewidth{\linewidth}%
  \tikzsetnextfilename{Ch2/INTC_naive_strat_comp}%
  % This file was created by matlab2tikz.
%
%The latest updates can be retrieved from
%  http://www.mathworks.com/matlabcentral/fileexchange/22022-matlab2tikz-matlab2tikz
%where you can also make suggestions and rate matlab2tikz.
%
\definecolor{mycolor1}{rgb}{0.00000,0.00000,0.00000}%
\definecolor{mycolor2}{rgb}{0.40000,0.40000,0.40000}%
\definecolor{mycolor3}{rgb}{0.70000,0.70000,0.70000}%
%
\begin{tikzpicture}[trim axis left, trim axis right]

\begin{axis}[%
width=\figurewidth,
height=\figureheight,
at={(0\figurewidth,0\figureheight)},
scale only axis,
every outer x axis line/.append style={black},
every x tick label/.append style={font=\color{black}},
xmin=9.5,
xmax=16,
every outer y axis line/.append style={black},
every y tick label/.append style={font=\color{black}},
ymin=-0.25,
ymax=0.5,
title={INTC},
axis background/.style={fill=white},
axis x line*=bottom,
axis y line*=left
]
\addplot [color=mycolor1,solid,line width=1.5pt,forget plot]
  table[row sep=crcr]{%
9.50027777777778	0\\
9.50583333333333	-0.0011202069801848\\
9.51138888888889	-0.00185610152028889\\
9.51694444444444	-0.00287631260642222\\
9.5225	-0.00318421543477387\\
9.52805555555556	-0.0043359352652091\\
9.53361111111111	-0.00451165123009948\\
9.53916666666667	-0.0049625390209242\\
9.54472222222222	-0.00547022605072493\\
9.55027777777778	-0.00589403520997499\\
9.55583333333333	-0.00664415913200614\\
9.56138888888889	-0.0070450586578701\\
9.56694444444444	-0.00770574857108564\\
9.5725	-0.00841006164055116\\
9.57805555555555	-0.00875943489900924\\
9.58361111111111	-0.00829962910569239\\
9.58916666666667	-0.00838830617023795\\
9.59472222222222	-0.00940804072439133\\
9.60027777777778	-0.0104906456427089\\
9.60583333333333	-0.0112703667541564\\
9.61138888888889	-0.0109184199420426\\
9.61694444444444	-0.010314222722415\\
9.6225	-0.0101326148541956\\
9.62805555555556	-0.0104177612873686\\
9.63361111111111	-0.0105305390091521\\
9.63916666666667	-0.010670552998351\\
9.64472222222222	-0.0119297574055315\\
9.65027777777778	-0.0125544124268184\\
9.65583333333333	-0.0128072630342936\\
9.66138888888889	-0.0136228897678175\\
9.66694444444444	-0.0138697340300197\\
9.6725	-0.0129050432185947\\
9.67805555555555	-0.0128793410977404\\
9.68361111111111	-0.0131887441986988\\
9.68916666666667	-0.0137703508868109\\
9.69472222222222	-0.013664773968354\\
9.70027777777778	-0.0137741651869353\\
9.70583333333333	-0.0134870556429028\\
9.71138888888889	-0.014432247714516\\
9.71694444444444	-0.0148261000517353\\
9.7225	-0.0141218620349769\\
9.72805555555555	-0.0136264385727116\\
9.73361111111111	-0.0136727033675693\\
9.73916666666667	-0.0144227183902356\\
9.74472222222222	-0.0138457237707773\\
9.75027777777778	-0.013681878793486\\
9.75583333333333	-0.0139637514828544\\
9.76138888888889	-0.0153156858430135\\
9.76694444444444	-0.0159090349480801\\
9.7725	-0.0159062963429498\\
9.77805555555556	-0.0156626542755022\\
9.78361111111111	-0.0157100831086272\\
9.78916666666667	-0.0159699298214632\\
9.79472222222222	-0.0164330548590992\\
9.80027777777778	-0.0156066647272141\\
9.80583333333333	-0.0153505434271171\\
9.81138888888889	-0.0150689094694367\\
9.81694444444444	-0.015343096095052\\
9.8225	-0.0155133236684891\\
9.82805555555555	-0.0154867579018357\\
9.83361111111111	-0.015386777372504\\
9.83916666666667	-0.0163693613525599\\
9.84472222222222	-0.0163031331446796\\
9.85027777777778	-0.0161091545538191\\
9.85583333333333	-0.0163314168132864\\
9.86138888888889	-0.0171662561682676\\
9.86694444444444	-0.0178081319109839\\
9.8725	-0.0183277047555141\\
9.87805555555556	-0.0183596708722408\\
9.88361111111111	-0.0181077862313897\\
9.88916666666667	-0.0182335217855267\\
9.89472222222222	-0.0192160062870956\\
9.90027777777778	-0.0194013953645353\\
9.90583333333333	-0.0195360046216585\\
9.91138888888889	-0.0189921926112669\\
9.91694444444444	-0.0185785697289188\\
9.9225	-0.0182179366686986\\
9.92805555555555	-0.0183647211478372\\
9.93361111111111	-0.0174625782302524\\
9.93916666666667	-0.0174650074584662\\
9.94472222222222	-0.0188637777674197\\
9.95027777777778	-0.0194201811596273\\
9.95583333333333	-0.0197466735072698\\
9.96138888888889	-0.0196497886125199\\
9.96694444444444	-0.0195060491711076\\
9.9725	-0.0195351646027217\\
9.97805555555555	-0.020038691913523\\
9.98361111111111	-0.0199701205312001\\
9.98916666666667	-0.0198740022378106\\
9.99472222222222	-0.0207157108529657\\
10.0002777777778	-0.0222294506651098\\
10.0058333333333	-0.0218130351618767\\
10.0113888888889	-0.021563768858802\\
10.0169444444444	-0.0211092660704346\\
10.0225	-0.0200952191459798\\
10.0280555555556	-0.0204598750307475\\
10.0336111111111	-0.0208707080819973\\
10.0391666666667	-0.0209422029217664\\
10.0447222222222	-0.0210617780285185\\
10.0502777777778	-0.0207520810894855\\
10.0558333333333	-0.0214867742838595\\
10.0613888888889	-0.0224468835381958\\
10.0669444444444	-0.0211898670008902\\
10.0725	-0.0211893234561544\\
10.0780555555556	-0.0212477333048836\\
10.0836111111111	-0.0209005849365109\\
10.0891666666667	-0.0208257670026119\\
10.0947222222222	-0.0212435531906665\\
10.1002777777778	-0.0211415855088132\\
10.1058333333333	-0.0213055068565725\\
10.1113888888889	-0.0208176830327345\\
10.1169444444444	-0.0208475537506699\\
10.1225	-0.0213766855250393\\
10.1280555555556	-0.0219790874273651\\
10.1336111111111	-0.0217962303515835\\
10.1391666666667	-0.0222215219110844\\
10.1447222222222	-0.022402324424338\\
10.1502777777778	-0.0219818377223625\\
10.1558333333333	-0.0218039999800029\\
10.1613888888889	-0.0214505330675429\\
10.1669444444444	-0.0209401645497035\\
10.1725	-0.0221868585614573\\
10.1780555555556	-0.022968826532874\\
10.1836111111111	-0.0241566800632564\\
10.1891666666667	-0.0246111239669729\\
10.1947222222222	-0.0246788584840415\\
10.2002777777778	-0.025052726428195\\
10.2058333333333	-0.025234046298229\\
10.2113888888889	-0.0257813787838347\\
10.2169444444444	-0.0263886799168201\\
10.2225	-0.0261960920963823\\
10.2280555555556	-0.0260803004094261\\
10.2336111111111	-0.0253194558417974\\
10.2391666666667	-0.0250916088625672\\
10.2447222222222	-0.0244323104510756\\
10.2502777777778	-0.0248468471257171\\
10.2558333333333	-0.0260793548126648\\
10.2613888888889	-0.0255323104951196\\
10.2669444444444	-0.0253703057469226\\
10.2725	-0.0252311384391566\\
10.2780555555556	-0.0263542383636283\\
10.2836111111111	-0.026750428888829\\
10.2891666666667	-0.0262394134075225\\
10.2947222222222	-0.0269172448533619\\
10.3002777777778	-0.0273566598225294\\
10.3058333333333	-0.0268763361149364\\
10.3113888888889	-0.0262994632018158\\
10.3169444444444	-0.0268192760774934\\
10.3225	-0.025959408620839\\
10.3280555555556	-0.0260388136971611\\
10.3336111111111	-0.0266778657684088\\
10.3391666666667	-0.025665352161116\\
10.3447222222222	-0.0268036311186012\\
10.3502777777778	-0.0281515005527146\\
10.3558333333333	-0.028992510431405\\
10.3613888888889	-0.0286964249794221\\
10.3669444444444	-0.0292852183641436\\
10.3725	-0.0289549350434474\\
10.3780555555556	-0.0292415315201489\\
10.3836111111111	-0.0290814446477471\\
10.3891666666667	-0.0283936765013517\\
10.3947222222222	-0.0285888032847507\\
10.4002777777778	-0.0293844102970485\\
10.4058333333333	-0.0300551975168709\\
10.4113888888889	-0.0285518203253014\\
10.4169444444444	-0.029470355319186\\
10.4225	-0.0298013750701714\\
10.4280555555556	-0.0298232664908138\\
10.4336111111111	-0.0299574645693138\\
10.4391666666667	-0.0291614396487683\\
10.4447222222222	-0.0288828078664938\\
10.4502777777778	-0.0295987160524334\\
10.4558333333333	-0.0291741661041972\\
10.4613888888889	-0.0297827718001201\\
10.4669444444444	-0.0300316698904045\\
10.4725	-0.0293478020682829\\
10.4780555555556	-0.0296197756378493\\
10.4836111111111	-0.0289697248164852\\
10.4891666666667	-0.0286416279265724\\
10.4947222222222	-0.027190369052147\\
10.5002777777778	-0.0273704155165486\\
10.5058333333333	-0.0269951178678197\\
10.5113888888889	-0.0261656912512619\\
10.5169444444444	-0.0272725987227383\\
10.5225	-0.0275020163973968\\
10.5280555555556	-0.0275127723534734\\
10.5336111111111	-0.0272587689908187\\
10.5391666666667	-0.0276507069685226\\
10.5447222222222	-0.0279925870347241\\
10.5502777777778	-0.0282349833533235\\
10.5558333333333	-0.0286223615786877\\
10.5613888888889	-0.0293229845342217\\
10.5669444444444	-0.0303508993753004\\
10.5725	-0.0301615134142954\\
10.5780555555556	-0.0302246282364238\\
10.5836111111111	-0.0301403378294434\\
10.5891666666667	-0.0309027607338413\\
10.5947222222222	-0.0303343668750384\\
10.6002777777778	-0.0299418480955644\\
10.6058333333333	-0.0305425416340619\\
10.6113888888889	-0.0313744243928714\\
10.6169444444444	-0.0307818402688015\\
10.6225	-0.0315779096269313\\
10.6280555555556	-0.030596288541909\\
10.6336111111111	-0.0303894558249698\\
10.6391666666667	-0.0303256053381047\\
10.6447222222222	-0.0319197468411164\\
10.6502777777778	-0.0324619917354902\\
10.6558333333333	-0.0342506663344279\\
10.6613888888889	-0.0344028522847711\\
10.6669444444444	-0.0341503782587952\\
10.6725	-0.0334227395348971\\
10.6780555555556	-0.0330924000912414\\
10.6836111111111	-0.0329570336084791\\
10.6891666666667	-0.0327044528510899\\
10.6947222222222	-0.0339427425512227\\
10.7002777777778	-0.0345811462723813\\
10.7058333333333	-0.0344197799682151\\
10.7113888888889	-0.0340818734866517\\
10.7169444444444	-0.0336768405215907\\
10.7225	-0.0337949436770452\\
10.7280555555556	-0.0341586352505128\\
10.7336111111111	-0.0330336071095672\\
10.7391666666667	-0.0331588954772484\\
10.7447222222222	-0.0340100214400673\\
10.7502777777778	-0.0342056597080031\\
10.7558333333333	-0.0351224127985511\\
10.7613888888889	-0.0356018451845387\\
10.7669444444444	-0.0362278844788984\\
10.7725	-0.0359987992463263\\
10.7780555555556	-0.035402113984486\\
10.7836111111111	-0.0359775654000531\\
10.7891666666667	-0.0359927777133113\\
10.7947222222222	-0.0361107498520657\\
10.8002777777778	-0.0369241315582299\\
10.8058333333333	-0.036867914922491\\
10.8113888888889	-0.0364798944472421\\
10.8169444444444	-0.035525719544823\\
10.8225	-0.0353207258085275\\
10.8280555555556	-0.0370540607531448\\
10.8336111111111	-0.037397115910956\\
10.8391666666667	-0.0368794414614863\\
10.8447222222222	-0.0358289375624801\\
10.8502777777778	-0.0359869168026306\\
10.8558333333333	-0.0370018311107623\\
10.8613888888889	-0.03720859969889\\
10.8669444444444	-0.0370996405089825\\
10.8725	-0.0359366713004175\\
10.8780555555556	-0.0362456551736893\\
10.8836111111111	-0.036487981667913\\
10.8891666666667	-0.0360440334922148\\
10.8947222222222	-0.0362553626556753\\
10.9002777777778	-0.0352981133610157\\
10.9058333333333	-0.0353582853136468\\
10.9113888888889	-0.0360547051467222\\
10.9169444444444	-0.0356335274330186\\
10.9225	-0.0370269183515833\\
10.9280555555556	-0.0371560596456037\\
10.9336111111111	-0.038026407004434\\
10.9391666666667	-0.0384303356608813\\
10.9447222222222	-0.0365927343821943\\
10.9502777777778	-0.0368632592539773\\
10.9558333333333	-0.0371017642252666\\
10.9613888888889	-0.0373289251880893\\
10.9669444444444	-0.0383757584691097\\
10.9725	-0.0393856446088018\\
10.9780555555556	-0.0397353465507004\\
10.9836111111111	-0.0413687005976655\\
10.9891666666667	-0.0403658149818026\\
10.9947222222222	-0.0405973156032989\\
11.0002777777778	-0.0403308132093143\\
11.0058333333333	-0.0410341365647939\\
11.0113888888889	-0.0411951137587152\\
11.0169444444444	-0.0411865040377497\\
11.0225	-0.0416916619837389\\
11.0280555555556	-0.0406823072921072\\
11.0336111111111	-0.0412952124106996\\
11.0391666666667	-0.041204656976666\\
11.0447222222222	-0.0408772716831607\\
11.0502777777778	-0.0405131447393124\\
11.0558333333333	-0.0411064424218794\\
11.0613888888889	-0.0408300124303472\\
11.0669444444444	-0.0420095523900861\\
11.0725	-0.0419490891234733\\
11.0780555555556	-0.0408964445853349\\
11.0836111111111	-0.041905778542894\\
11.0891666666667	-0.0422209340753963\\
11.0947222222222	-0.0429743872630018\\
11.1002777777778	-0.043800816052448\\
11.1058333333333	-0.0429796044781959\\
11.1113888888889	-0.0427605307410261\\
11.1169444444444	-0.0433991441848479\\
11.1225	-0.0433734741260849\\
11.1280555555556	-0.0437143423022691\\
11.1336111111111	-0.0427546064142629\\
11.1391666666667	-0.0440813514721689\\
11.1447222222222	-0.0447472031372207\\
11.1502777777778	-0.0444720027062459\\
11.1558333333333	-0.0449691049273396\\
11.1613888888889	-0.0451333259437319\\
11.1669444444444	-0.0451051877881439\\
11.1725	-0.0461857076201611\\
11.1780555555556	-0.0463084811638476\\
11.1836111111111	-0.0469723283305404\\
11.1891666666667	-0.0466228771293875\\
11.1947222222222	-0.0465443104527377\\
11.2002777777778	-0.046294750074389\\
11.2058333333333	-0.0462905475425158\\
11.2113888888889	-0.0468185167231019\\
11.2169444444444	-0.0465905999889129\\
11.2225	-0.0464337889238656\\
11.2280555555556	-0.0466295287925654\\
11.2336111111111	-0.0479160341264885\\
11.2391666666667	-0.0485305231259556\\
11.2447222222222	-0.0479678197360233\\
11.2502777777778	-0.0480646163733728\\
11.2558333333333	-0.0480590740683809\\
11.2613888888889	-0.048276046793589\\
11.2669444444444	-0.0487104267381025\\
11.2725	-0.04880954237273\\
11.2780555555556	-0.0487289395243183\\
11.2836111111111	-0.0500408554444967\\
11.2891666666667	-0.0487743677291317\\
11.2947222222222	-0.0477400257767485\\
11.3002777777778	-0.0478566079663395\\
11.3058333333333	-0.0464766543984985\\
11.3113888888889	-0.0464599739835111\\
11.3169444444444	-0.0465272005646011\\
11.3225	-0.0460032013879745\\
11.3280555555556	-0.0470166356055471\\
11.3336111111111	-0.0466592146217856\\
11.3391666666667	-0.0463286358637343\\
11.3447222222222	-0.0471052026683629\\
11.3502777777778	-0.0474191412693604\\
11.3558333333333	-0.0474979808836201\\
11.3613888888889	-0.0472634889259837\\
11.3669444444444	-0.047001976776088\\
11.3725	-0.0472867209119147\\
11.3780555555556	-0.0469323052483943\\
11.3836111111111	-0.0461425110253458\\
11.3891666666667	-0.0472412086332235\\
11.3947222222222	-0.0474146148085933\\
11.4002777777778	-0.0476701304763765\\
11.4058333333333	-0.0474014514520896\\
11.4113888888889	-0.046746402596999\\
11.4169444444444	-0.0472442750512187\\
11.4225	-0.0481173560193329\\
11.4280555555556	-0.0490418029953529\\
11.4336111111111	-0.0494745613771897\\
11.4391666666667	-0.0498891534632572\\
11.4447222222222	-0.0497218559456796\\
11.4502777777778	-0.0501693224548028\\
11.4558333333333	-0.0504103523610614\\
11.4613888888889	-0.0500644535403813\\
11.4669444444444	-0.0502943967634612\\
11.4725	-0.0508164319860858\\
11.4780555555556	-0.0501726541284549\\
11.4836111111111	-0.0499726148750889\\
11.4891666666667	-0.0498923819575489\\
11.4947222222222	-0.04960815765938\\
11.5002777777778	-0.049007610646069\\
11.5058333333333	-0.0487728678243445\\
11.5113888888889	-0.0496969473543015\\
11.5169444444444	-0.049865694833747\\
11.5225	-0.050457306773267\\
11.5280555555556	-0.0503502254228203\\
11.5336111111111	-0.0500096724126056\\
11.5391666666667	-0.0509666090662969\\
11.5447222222222	-0.0511154416363923\\
11.5502777777778	-0.051441779297382\\
11.5558333333333	-0.0523814764551376\\
11.5613888888889	-0.0510600853268182\\
11.5669444444444	-0.0512305573915756\\
11.5725	-0.0499820480633225\\
11.5780555555556	-0.0509176409265371\\
11.5836111111111	-0.0511928272424376\\
11.5891666666667	-0.0518987744346693\\
11.5947222222222	-0.0509528080865852\\
11.6002777777778	-0.0522211052425101\\
11.6058333333333	-0.0516801928364896\\
11.6113888888889	-0.0511442329915976\\
11.6169444444444	-0.0508740261866848\\
11.6225	-0.051204237625552\\
11.6280555555556	-0.0514882107254955\\
11.6336111111111	-0.0518391195344435\\
11.6391666666667	-0.0507929869840695\\
11.6447222222222	-0.0515062258825112\\
11.6502777777778	-0.0505741265933963\\
11.6558333333333	-0.0510444747313364\\
11.6613888888889	-0.0507469902637307\\
11.6669444444444	-0.051984815204816\\
11.6725	-0.0517788287232404\\
11.6780555555556	-0.0524485893080799\\
11.6836111111111	-0.0516204888165343\\
11.6891666666667	-0.0519599084999748\\
11.6947222222222	-0.0529322763079661\\
11.7002777777778	-0.0525822052840467\\
11.7058333333333	-0.0513261730277346\\
11.7113888888889	-0.0510477556078436\\
11.7169444444444	-0.0508161345685852\\
11.7225	-0.0513931908421034\\
11.7280555555556	-0.0511078189500675\\
11.7336111111111	-0.0503182232662485\\
11.7391666666667	-0.0501981054549675\\
11.7447222222222	-0.0504095926971789\\
11.7502777777778	-0.0498613651722974\\
11.7558333333333	-0.0491650134641391\\
11.7613888888889	-0.0489622457177849\\
11.7669444444444	-0.0481004443233264\\
11.7725	-0.0476115293526913\\
11.7780555555556	-0.0479811302897249\\
11.7836111111111	-0.0474347605224877\\
11.7891666666667	-0.0485544808877424\\
11.7947222222222	-0.0478607192722369\\
11.8002777777778	-0.0478286774645499\\
11.8058333333333	-0.0488188668258088\\
11.8113888888889	-0.0480753631184184\\
11.8169444444444	-0.049195214270109\\
11.8225	-0.0499757689636765\\
11.8280555555556	-0.0498034045471082\\
11.8336111111111	-0.0505251679138008\\
11.8391666666667	-0.0518334655505958\\
11.8447222222222	-0.0530175919764101\\
11.8502777777778	-0.0532812179889051\\
11.8558333333333	-0.0519346010947353\\
11.8613888888889	-0.0516511320598015\\
11.8669444444444	-0.0519983069132364\\
11.8725	-0.0528517144035233\\
11.8780555555556	-0.0534304072007696\\
11.8836111111111	-0.0537680994347839\\
11.8891666666667	-0.0541228769341893\\
11.8947222222222	-0.0538363629551507\\
11.9002777777778	-0.0544821729939362\\
11.9058333333333	-0.0547210816684427\\
11.9113888888889	-0.0549931047069304\\
11.9169444444444	-0.0546464149092327\\
11.9225	-0.0542688483199669\\
11.9280555555556	-0.0542221774087702\\
11.9336111111111	-0.0542471790597184\\
11.9391666666667	-0.0543038915101265\\
11.9447222222222	-0.054704794381833\\
11.9502777777778	-0.0556778704332251\\
11.9558333333333	-0.0554398069643751\\
11.9613888888889	-0.0556074627629495\\
11.9669444444444	-0.0551567036360463\\
11.9725	-0.0555986828553964\\
11.9780555555556	-0.0560730388870272\\
11.9836111111111	-0.0566939584305121\\
11.9891666666667	-0.0571165531200518\\
11.9947222222222	-0.0575437400536473\\
12.0002777777778	-0.0569386902517321\\
12.0058333333333	-0.0568755237531752\\
12.0113888888889	-0.058037369439674\\
12.0169444444444	-0.0590534845582421\\
12.0225	-0.0594210979395462\\
12.0280555555556	-0.0591298344762548\\
12.0336111111111	-0.0590480592090758\\
12.0391666666667	-0.0587135816270683\\
12.0447222222222	-0.0584175022096912\\
12.0502777777778	-0.0584768819658233\\
12.0558333333333	-0.057552339922274\\
12.0613888888889	-0.0584659262454815\\
12.0669444444444	-0.0590798130929063\\
12.0725	-0.0590510759630584\\
12.0780555555556	-0.0598343353697809\\
12.0836111111111	-0.0599298093979016\\
12.0891666666667	-0.0611088516557471\\
12.0947222222222	-0.0609994934031539\\
12.1002777777778	-0.0614539861223049\\
12.1058333333333	-0.06010835295959\\
12.1113888888889	-0.0602634630595081\\
12.1169444444444	-0.0608112682915369\\
12.1225	-0.0599468248231713\\
12.1280555555556	-0.0593357816934427\\
12.1336111111111	-0.0587195743160276\\
12.1391666666667	-0.0583534785756864\\
12.1447222222222	-0.0583398924906914\\
12.1502777777778	-0.0584675806700209\\
12.1558333333333	-0.0587302642479048\\
12.1613888888889	-0.0588809938271668\\
12.1669444444444	-0.0588276430553793\\
12.1725	-0.0592832375570306\\
12.1780555555556	-0.0599507128619723\\
12.1836111111111	-0.0600611317539168\\
12.1891666666667	-0.0610401677558309\\
12.1947222222222	-0.0608428099816138\\
12.2002777777778	-0.0598606900318025\\
12.2058333333333	-0.0598867299928621\\
12.2113888888889	-0.0596647108865968\\
12.2169444444444	-0.0601486448029994\\
12.2225	-0.0598008361457875\\
12.2280555555556	-0.0599889223940581\\
12.2336111111111	-0.0597890128794107\\
12.2391666666667	-0.0592560250551072\\
12.2447222222222	-0.0596736582590488\\
12.2502777777778	-0.0586798041368189\\
12.2558333333333	-0.0597313522688973\\
12.2613888888889	-0.0599240907029126\\
12.2669444444444	-0.0592788993512436\\
12.2725	-0.0589667614134462\\
12.2780555555556	-0.0586408184844487\\
12.2836111111111	-0.0580408586126308\\
12.2891666666667	-0.057748277841263\\
12.2947222222222	-0.0583493771458471\\
12.3002777777778	-0.0582255378548154\\
12.3058333333333	-0.0581767861852111\\
12.3113888888889	-0.057308728356045\\
12.3169444444444	-0.0578921056118494\\
12.3225	-0.0585989484631551\\
12.3280555555556	-0.058569582638202\\
12.3336111111111	-0.0593486654724989\\
12.3391666666667	-0.0591577668262655\\
12.3447222222222	-0.059324462331449\\
12.3502777777778	-0.0592013205679013\\
12.3558333333333	-0.0595448689955347\\
12.3613888888889	-0.0589204134835653\\
12.3669444444444	-0.0592793049369758\\
12.3725	-0.0585300244354832\\
12.3780555555556	-0.0583687509705321\\
12.3836111111111	-0.0576760133061911\\
12.3891666666667	-0.0586195709447711\\
12.3947222222222	-0.0587320146152397\\
12.4002777777778	-0.0581684548655021\\
12.4058333333333	-0.0582073790103473\\
12.4113888888889	-0.0579116886566088\\
12.4169444444444	-0.0579009274676793\\
12.4225	-0.057673985664973\\
12.4280555555556	-0.0580361414704937\\
12.4336111111111	-0.0587180302713521\\
12.4391666666667	-0.0587405777962945\\
12.4447222222222	-0.058787548163845\\
12.4502777777778	-0.0584669574519134\\
12.4558333333333	-0.058759269943256\\
12.4613888888889	-0.059328064536915\\
12.4669444444444	-0.0589244148912301\\
12.4725	-0.058174969404789\\
12.4780555555556	-0.0584696677552939\\
12.4836111111111	-0.0588889543658093\\
12.4891666666667	-0.059379458605614\\
12.4947222222222	-0.0593513270142871\\
12.5002777777778	-0.0592675496757557\\
12.5058333333333	-0.058699852197001\\
12.5113888888889	-0.059380514597629\\
12.5169444444444	-0.0594493655369207\\
12.5225	-0.0589766227443331\\
12.5280555555556	-0.0583080026408936\\
12.5336111111111	-0.0590165242339464\\
12.5391666666667	-0.0591193574877803\\
12.5447222222222	-0.0588020467108458\\
12.5502777777778	-0.059044187450216\\
12.5558333333333	-0.0592315445230141\\
12.5613888888889	-0.0597738013701239\\
12.5669444444444	-0.0599251736564912\\
12.5725	-0.0599481580347026\\
12.5780555555556	-0.060824188604999\\
12.5836111111111	-0.0610874329952385\\
12.5891666666667	-0.0610849877037914\\
12.5947222222222	-0.0610599895944864\\
12.6002777777778	-0.061810617126042\\
12.6058333333333	-0.0610988766793171\\
12.6113888888889	-0.060728386073579\\
12.6169444444444	-0.0612737211016437\\
12.6225	-0.0618343247027585\\
12.6280555555556	-0.0625185433529459\\
12.6336111111111	-0.0628129094056548\\
12.6391666666667	-0.0628565640996717\\
12.6447222222222	-0.0619124842355175\\
12.6502777777778	-0.0626444348264932\\
12.6558333333333	-0.0627113616307527\\
12.6613888888889	-0.0623543561843743\\
12.6669444444444	-0.0618532015813325\\
12.6725	-0.06189153529053\\
12.6780555555556	-0.0609998133733487\\
12.6836111111111	-0.0605541870912703\\
12.6891666666667	-0.0604688912645115\\
12.6947222222222	-0.0605853883084135\\
12.7002777777778	-0.0605796465958862\\
12.7058333333333	-0.060514362469497\\
12.7113888888889	-0.0607415013572914\\
12.7169444444444	-0.0606744110394452\\
12.7225	-0.0612994037259825\\
12.7280555555556	-0.0607516859980249\\
12.7336111111111	-0.0608597756362168\\
12.7391666666667	-0.0605740176648123\\
12.7447222222222	-0.0604358416192553\\
12.7502777777778	-0.0604483689151163\\
12.7558333333333	-0.0604981446790238\\
12.7613888888889	-0.0601620015874523\\
12.7669444444444	-0.0600567456418767\\
12.7725	-0.0598834619451564\\
12.7780555555556	-0.060469619356373\\
12.7836111111111	-0.0601127406123234\\
12.7891666666667	-0.0607965768493059\\
12.7947222222222	-0.0604546784837016\\
12.8002777777778	-0.0612141675194113\\
12.8058333333333	-0.0606806571418545\\
12.8113888888889	-0.0605535102233174\\
12.8169444444444	-0.0617154654334456\\
12.8225	-0.0621748748699497\\
12.8280555555556	-0.0625551468839091\\
12.8336111111111	-0.0627732797386052\\
12.8391666666667	-0.0632695905004585\\
12.8447222222222	-0.0635545923393683\\
12.8502777777778	-0.0638365160794151\\
12.8558333333333	-0.0639223883572926\\
12.8613888888889	-0.0643726896147403\\
12.8669444444444	-0.0636315557405823\\
12.8725	-0.0630025271698868\\
12.8780555555556	-0.0619301686787402\\
12.8836111111111	-0.0614559318960151\\
12.8891666666667	-0.0628179510462742\\
12.8947222222222	-0.0632207819087124\\
12.9002777777778	-0.0639298671389806\\
12.9058333333333	-0.0631703480131095\\
12.9113888888889	-0.0627267998305729\\
12.9169444444444	-0.0624919960835946\\
12.9225	-0.0629867581115384\\
12.9280555555556	-0.0631234161590851\\
12.9336111111111	-0.0627660226947577\\
12.9391666666667	-0.0628017592988584\\
12.9447222222222	-0.0631111066464691\\
12.9502777777778	-0.0633421242917554\\
12.9558333333333	-0.063619648419595\\
12.9613888888889	-0.0632091273265566\\
12.9669444444444	-0.0622659030519929\\
12.9725	-0.0638895790683496\\
12.9780555555556	-0.0633196979344757\\
12.9836111111111	-0.0627553300646355\\
12.9891666666667	-0.0628233490007879\\
12.9947222222222	-0.0626371441829489\\
13.0002777777778	-0.0622550385457772\\
13.0058333333333	-0.0609520976095403\\
13.0113888888889	-0.060688737126809\\
13.0169444444444	-0.0613424680864613\\
13.0225	-0.0615396613109035\\
13.0280555555556	-0.0605206019590795\\
13.0336111111111	-0.0606100324626569\\
13.0391666666667	-0.0610690004678875\\
13.0447222222222	-0.0606815535913354\\
13.0502777777778	-0.0598184562279345\\
13.0558333333333	-0.0608813256677639\\
13.0613888888889	-0.0603605206112474\\
13.0669444444444	-0.0604373739859593\\
13.0725	-0.0605800456445075\\
13.0780555555556	-0.0612060655361713\\
13.0836111111111	-0.0617363700681883\\
13.0891666666667	-0.0614853649532139\\
13.0947222222222	-0.0616365488608441\\
13.1002777777778	-0.061477695660227\\
13.1058333333333	-0.0609038918426864\\
13.1113888888889	-0.0610036510687393\\
13.1169444444444	-0.0608317947181714\\
13.1225	-0.0602606701926261\\
13.1280555555556	-0.0601907721593442\\
13.1336111111111	-0.0610052790248636\\
13.1391666666667	-0.0612929650608491\\
13.1447222222222	-0.0614291702379749\\
13.1502777777778	-0.0610275036684836\\
13.1558333333333	-0.0606328648045323\\
13.1613888888889	-0.0606029067634144\\
13.1669444444444	-0.0607765578450504\\
13.1725	-0.0609716147447917\\
13.1780555555556	-0.0614195818389265\\
13.1836111111111	-0.0611560859189027\\
13.1891666666667	-0.0606986457867803\\
13.1947222222222	-0.0611596605081939\\
13.2002777777778	-0.0607555399836767\\
13.2058333333333	-0.0595342823498901\\
13.2113888888889	-0.0604921796868753\\
13.2169444444444	-0.0603077670755748\\
13.2225	-0.0597139731155515\\
13.2280555555556	-0.0592143522816002\\
13.2336111111111	-0.0599293320018248\\
13.2391666666667	-0.0603630278645665\\
13.2447222222222	-0.0601518287408323\\
13.2502777777778	-0.0608301358009813\\
13.2558333333333	-0.0601524404014431\\
13.2613888888889	-0.0611102173121313\\
13.2669444444444	-0.0602776367614751\\
13.2725	-0.0607470890009665\\
13.2780555555556	-0.0614042499511831\\
13.2836111111111	-0.0602795192356161\\
13.2891666666667	-0.0596030998366369\\
13.2947222222222	-0.0588447084067719\\
13.3002777777778	-0.057734950233805\\
13.3058333333333	-0.0571703310119131\\
13.3113888888889	-0.0574072120382207\\
13.3169444444444	-0.0571805080101636\\
13.3225	-0.0563379705291085\\
13.3280555555556	-0.05543084011856\\
13.3336111111111	-0.0546992230960871\\
13.3391666666667	-0.0536521276300094\\
13.3447222222222	-0.0532743333735952\\
13.3502777777778	-0.0535431654842555\\
13.3558333333333	-0.0533048366994892\\
13.3613888888889	-0.0538560870930207\\
13.3669444444444	-0.0547994028154674\\
13.3725	-0.0550415322344135\\
13.3780555555556	-0.0542249728353806\\
13.3836111111111	-0.0542505353064335\\
13.3891666666667	-0.0539908783957929\\
13.3947222222222	-0.0540054539472395\\
13.4002777777778	-0.054700268699087\\
13.4058333333333	-0.0553715890298834\\
13.4113888888889	-0.0551834829296391\\
13.4169444444444	-0.0550947542119736\\
13.4225	-0.0550870375331279\\
13.4280555555556	-0.0552786860669196\\
13.4336111111111	-0.0548201449495184\\
13.4391666666667	-0.055252673070794\\
13.4447222222222	-0.0569114446163458\\
13.4502777777778	-0.0561024558241548\\
13.4558333333333	-0.0560945703698972\\
13.4613888888889	-0.0558082331130328\\
13.4669444444444	-0.0567213386990881\\
13.4725	-0.0561408443131258\\
13.4780555555556	-0.0565500236443993\\
13.4836111111111	-0.0573162620381407\\
13.4891666666667	-0.056580886935569\\
13.4947222222222	-0.0567148287571688\\
13.5002777777778	-0.057551658906217\\
13.5058333333333	-0.058041044206307\\
13.5113888888889	-0.0575311988775794\\
13.5169444444444	-0.0582315648135094\\
13.5225	-0.0577732547897683\\
13.5280555555556	-0.0574809211969269\\
13.5336111111111	-0.0593782574583393\\
13.5391666666667	-0.0588569318292652\\
13.5447222222222	-0.0570925334969453\\
13.5502777777778	-0.0567050582479635\\
13.5558333333333	-0.0571496993297822\\
13.5613888888889	-0.0566435137485963\\
13.5669444444444	-0.0569153990496208\\
13.5725	-0.0567598151018074\\
13.5780555555556	-0.0570265914955575\\
13.5836111111111	-0.0568764631553596\\
13.5891666666667	-0.0572318645048773\\
13.5947222222222	-0.057668493399292\\
13.6002777777778	-0.0576992951647239\\
13.6058333333333	-0.0577411764040421\\
13.6113888888889	-0.0576846039044446\\
13.6169444444444	-0.0575261082573716\\
13.6225	-0.0576718382182161\\
13.6280555555556	-0.05727162798819\\
13.6336111111111	-0.0573822505681137\\
13.6391666666667	-0.0567355571445445\\
13.6447222222222	-0.0563052999620349\\
13.6502777777778	-0.0560584592174575\\
13.6558333333333	-0.0547308851828524\\
13.6613888888889	-0.0555915039939752\\
13.6669444444444	-0.055447389329803\\
13.6725	-0.0555083276788732\\
13.6780555555556	-0.0564340721808768\\
13.6836111111111	-0.0556801449262097\\
13.6891666666667	-0.055167248802104\\
13.6947222222222	-0.0554476096607248\\
13.7002777777778	-0.0559057290261385\\
13.7058333333333	-0.0552869929245523\\
13.7113888888889	-0.054242944745762\\
13.7169444444444	-0.0541581429992969\\
13.7225	-0.0546906889785426\\
13.7280555555556	-0.0547510006134394\\
13.7336111111111	-0.0555466775605018\\
13.7391666666667	-0.0556657398042417\\
13.7447222222222	-0.0565736464338024\\
13.7502777777778	-0.0570905884568493\\
13.7558333333333	-0.0573754181281097\\
13.7613888888889	-0.057136055420677\\
13.7669444444444	-0.0580911807129701\\
13.7725	-0.0581810863412914\\
13.7780555555556	-0.0580486198646627\\
13.7836111111111	-0.0573294914567387\\
13.7891666666667	-0.0572210273127349\\
13.7947222222222	-0.0574090842868039\\
13.8002777777778	-0.0577996503197833\\
13.8058333333333	-0.0576751941362014\\
13.8113888888889	-0.058318344872862\\
13.8169444444444	-0.0579686393963736\\
13.8225	-0.0581364562841453\\
13.8280555555556	-0.0590559483773553\\
13.8336111111111	-0.0584144453315142\\
13.8391666666667	-0.0590344808202118\\
13.8447222222222	-0.0593858715346539\\
13.8502777777778	-0.0599698176862187\\
13.8558333333333	-0.0602938509462092\\
13.8613888888889	-0.0593483851707195\\
13.8669444444444	-0.0594847919965948\\
13.8725	-0.0594348796227943\\
13.8780555555556	-0.0583243957169051\\
13.8836111111111	-0.0589162422336963\\
13.8891666666667	-0.0591577549092434\\
13.8947222222222	-0.0600564247268884\\
13.9002777777778	-0.0597753550693505\\
13.9058333333333	-0.0588002177074354\\
13.9113888888889	-0.0589203543227758\\
13.9169444444444	-0.0589225494136482\\
13.9225	-0.0588725991022981\\
13.9280555555556	-0.0590984648182704\\
13.9336111111111	-0.0577095948127753\\
13.9391666666667	-0.0583036295379827\\
13.9447222222222	-0.0583087777647329\\
13.9502777777778	-0.0581343140896921\\
13.9558333333333	-0.0578447719540841\\
13.9613888888889	-0.0575470572626461\\
13.9669444444444	-0.057443571283009\\
13.9725	-0.0585206561839444\\
13.9780555555556	-0.0569254822330187\\
13.9836111111111	-0.0578464799729452\\
13.9891666666667	-0.0587329179517724\\
13.9947222222222	-0.0586981123525947\\
14.0002777777778	-0.059122862395747\\
14.0058333333333	-0.0569403434463772\\
14.0113888888889	-0.0565273949444188\\
14.0169444444444	-0.0588671945448963\\
14.0225	-0.0589404976010172\\
14.0280555555556	-0.0585726381579676\\
14.0336111111111	-0.0587400853987608\\
14.0391666666667	-0.0591779978425625\\
14.0447222222222	-0.0600147710025296\\
14.0502777777778	-0.0589941871094845\\
14.0558333333333	-0.0571032324851896\\
14.0613888888889	-0.0575792952762002\\
14.0669444444444	-0.0571471987711809\\
14.0725	-0.057598686200584\\
14.0780555555556	-0.0576096487564394\\
14.0836111111111	-0.0574786933785563\\
14.0891666666667	-0.0559424507937124\\
14.0947222222222	-0.0558954838411016\\
14.1002777777778	-0.0552793437646939\\
14.1058333333333	-0.0551772588375984\\
14.1113888888889	-0.0552221768288595\\
14.1169444444444	-0.0549766094590274\\
14.1225	-0.0548285777833999\\
14.1280555555556	-0.0558511099832802\\
14.1336111111111	-0.0548167361773297\\
14.1391666666667	-0.0552151661820543\\
14.1447222222222	-0.0546114850680269\\
14.1502777777778	-0.0552365925839669\\
14.1558333333333	-0.0555558958600042\\
14.1613888888889	-0.0554796875531261\\
14.1669444444444	-0.0554701449746798\\
14.1725	-0.0542625814513987\\
14.1780555555556	-0.0535171812867467\\
14.1836111111111	-0.0530333931203031\\
14.1891666666667	-0.0540828706288349\\
14.1947222222222	-0.0538572590230114\\
14.2002777777778	-0.0534254800466858\\
14.2058333333333	-0.0525318921078373\\
14.2113888888889	-0.0531729841376367\\
14.2169444444444	-0.0518901625553856\\
14.2225	-0.0519210597349024\\
14.2280555555556	-0.0520736711873099\\
14.2336111111111	-0.0533105264096012\\
14.2391666666667	-0.0529749325578568\\
14.2447222222222	-0.0533903864521117\\
14.2502777777778	-0.0533285709242889\\
14.2558333333333	-0.0529499348329739\\
14.2613888888889	-0.0538209912333364\\
14.2669444444444	-0.0545380533398092\\
14.2725	-0.0545803735363565\\
14.2780555555556	-0.0541733345811757\\
14.2836111111111	-0.0543144566847782\\
14.2891666666667	-0.055296416240139\\
14.2947222222222	-0.0572445087671953\\
14.3002777777778	-0.0562767380572305\\
14.3058333333333	-0.0561505391518591\\
14.3113888888889	-0.0560143743860845\\
14.3169444444444	-0.0552434328087037\\
14.3225	-0.0550676753701808\\
14.3280555555556	-0.0550688277446731\\
14.3336111111111	-0.0556260744722305\\
14.3391666666667	-0.0563112257055063\\
14.3447222222222	-0.0549948204212069\\
14.3502777777778	-0.0545334500344948\\
14.3558333333333	-0.0556352288223222\\
14.3613888888889	-0.0548967510871001\\
14.3669444444444	-0.054415061361463\\
14.3725	-0.0548237650817286\\
14.3780555555556	-0.0546892189891038\\
14.3836111111111	-0.0546488514915273\\
14.3891666666667	-0.0550615211911728\\
14.3947222222222	-0.053715364497244\\
14.4002777777778	-0.0544918831674519\\
14.4058333333333	-0.0559868929158621\\
14.4113888888889	-0.0551916961857116\\
14.4169444444444	-0.0572418807854169\\
14.4225	-0.0571184188490397\\
14.4280555555556	-0.0568198533986712\\
14.4336111111111	-0.0570776518321382\\
14.4391666666667	-0.056808290812175\\
14.4447222222222	-0.0563535438261415\\
14.4502777777778	-0.0569855167626808\\
14.4558333333333	-0.0563102883328055\\
14.4613888888889	-0.0573674286195452\\
14.4669444444444	-0.0570806353163214\\
14.4725	-0.0578101985343494\\
14.4780555555556	-0.0583468116689462\\
14.4836111111111	-0.0572321589393891\\
14.4891666666667	-0.0572503452537786\\
14.4947222222222	-0.0577297216714946\\
14.5002777777778	-0.0578412797523974\\
14.5058333333333	-0.0588164945168285\\
14.5113888888889	-0.0604018820247601\\
14.5169444444444	-0.0600995996029598\\
14.5225	-0.0590322140590751\\
14.5280555555556	-0.0602951848374908\\
14.5336111111111	-0.0608653689317202\\
14.5391666666667	-0.0596252940415176\\
14.5447222222222	-0.0600110777893099\\
14.5502777777778	-0.0586871151027049\\
14.5558333333333	-0.0591920197984268\\
14.5613888888889	-0.0590656011518611\\
14.5669444444444	-0.0585620913577143\\
14.5725	-0.0590350863970346\\
14.5780555555556	-0.0594609384764253\\
14.5836111111111	-0.0585790479933302\\
14.5891666666667	-0.0592752067358941\\
14.5947222222222	-0.0605917669578498\\
14.6002777777778	-0.0624638978833153\\
14.6058333333333	-0.0620633671342137\\
14.6113888888889	-0.0615655829352984\\
14.6169444444444	-0.0603968991055268\\
14.6225	-0.0610302382718273\\
14.6280555555556	-0.0598691255081997\\
14.6336111111111	-0.0589249178847863\\
14.6391666666667	-0.0609767949269681\\
14.6447222222222	-0.0604169043663909\\
14.6502777777778	-0.0594824506541672\\
14.6558333333333	-0.0589875355684367\\
14.6613888888889	-0.0585857308995216\\
14.6669444444444	-0.0591095772625277\\
14.6725	-0.0597896282508501\\
14.6780555555556	-0.0594204384965603\\
14.6836111111111	-0.0603168755643089\\
14.6891666666667	-0.0589452275389526\\
14.6947222222222	-0.059006021218467\\
14.7002777777778	-0.0599914923798408\\
14.7058333333333	-0.0608576435993678\\
14.7113888888889	-0.0620511055339942\\
14.7169444444444	-0.0617335583518042\\
14.7225	-0.0614090517792874\\
14.7280555555556	-0.0608947108557269\\
14.7336111111111	-0.0609760216216283\\
14.7391666666667	-0.061640021372576\\
14.7447222222222	-0.063125401061188\\
14.7502777777778	-0.0623530286621733\\
14.7558333333333	-0.0623624815221908\\
14.7613888888889	-0.0619181299505662\\
14.7669444444444	-0.0622544137992836\\
14.7725	-0.062189854193282\\
14.7780555555556	-0.0619309199093609\\
14.7836111111111	-0.0613683229091045\\
14.7891666666667	-0.0605486956805442\\
14.7947222222222	-0.0590261325117557\\
14.8002777777778	-0.0599882420131757\\
14.8058333333333	-0.05946061413193\\
14.8113888888889	-0.0596758639386715\\
14.8169444444444	-0.06027448353581\\
14.8225	-0.0604915181862858\\
14.8280555555556	-0.0602071542825063\\
14.8336111111111	-0.0589809132327494\\
14.8391666666667	-0.0580716849114039\\
14.8447222222222	-0.0573868333063332\\
14.8502777777778	-0.0576245954236106\\
14.8558333333333	-0.0582204254265915\\
14.8613888888889	-0.0575326334313808\\
14.8669444444444	-0.0586499195260167\\
14.8725	-0.0604522842574143\\
14.8780555555556	-0.0595450238484838\\
14.8836111111111	-0.0592825079059758\\
14.8891666666667	-0.0593152510984226\\
14.8947222222222	-0.0593633561496108\\
14.9002777777778	-0.0587049625507645\\
14.9058333333333	-0.0586832000321457\\
14.9113888888889	-0.0588551632910454\\
14.9169444444444	-0.0602967738823941\\
14.9225	-0.0594602101119684\\
14.9280555555556	-0.0588659571508317\\
14.9336111111111	-0.0592480795831348\\
14.9391666666667	-0.0590975206399357\\
14.9447222222222	-0.0605214189017353\\
14.9502777777778	-0.0617541231779672\\
14.9558333333333	-0.0612404806086697\\
14.9613888888889	-0.0609996282557042\\
14.9669444444444	-0.060921546893974\\
14.9725	-0.0609221736236527\\
14.9780555555556	-0.0616476755792951\\
14.9836111111111	-0.0620681310312275\\
14.9891666666667	-0.0609717630115616\\
14.9947222222222	-0.0614882852578789\\
15.0002777777778	-0.0605272542114194\\
15.0058333333333	-0.0615230958674546\\
15.0113888888889	-0.0609658227859753\\
15.0169444444444	-0.0608768615182958\\
15.0225	-0.0608163327307823\\
15.0280555555556	-0.0600711452308178\\
15.0336111111111	-0.061037658628388\\
15.0391666666667	-0.0594272470604648\\
15.0447222222222	-0.0604603352895311\\
15.0502777777778	-0.0621695625451882\\
15.0558333333333	-0.063707995156492\\
15.0613888888889	-0.0631526192166504\\
15.0669444444444	-0.0635406432666694\\
15.0725	-0.0644263105119525\\
15.0780555555556	-0.0638554778300155\\
15.0836111111111	-0.0644501334114494\\
15.0891666666667	-0.0634624942961931\\
15.0947222222222	-0.0644244215386323\\
15.1002777777778	-0.063100861421058\\
15.1058333333333	-0.0618436296730851\\
15.1113888888889	-0.0620423926973271\\
15.1169444444444	-0.0632532247369333\\
15.1225	-0.0645395255706285\\
15.1280555555556	-0.0636753054754708\\
15.1336111111111	-0.0633886520903235\\
15.1391666666667	-0.0641033973492985\\
15.1447222222222	-0.065668865058279\\
15.1502777777778	-0.0639230105767415\\
15.1558333333333	-0.0629319254619988\\
15.1613888888889	-0.0625509665342964\\
15.1669444444444	-0.0620296733151862\\
15.1725	-0.0616773327025906\\
15.1780555555556	-0.0615622614168445\\
15.1836111111111	-0.0607918004018604\\
15.1891666666667	-0.0622920127205345\\
15.1947222222222	-0.0619843585664519\\
15.2002777777778	-0.0619245848069626\\
15.2058333333333	-0.0606819229522173\\
15.2113888888889	-0.0611976135444857\\
15.2169444444444	-0.0614833072122041\\
15.2225	-0.0622923399131422\\
15.2280555555556	-0.0634890310854408\\
15.2336111111111	-0.064108259800923\\
15.2391666666667	-0.0651648608050405\\
15.2447222222222	-0.0648048026813821\\
15.2502777777778	-0.0656670310874422\\
15.2558333333333	-0.0656178718883405\\
15.2613888888889	-0.0645372770463062\\
15.2669444444444	-0.0661588882350874\\
15.2725	-0.0665632336983049\\
15.2780555555556	-0.0677748179251896\\
15.2836111111111	-0.0675189582899055\\
15.2891666666667	-0.0670081989885275\\
15.2947222222222	-0.0670965017553713\\
15.3002777777778	-0.067208803866032\\
15.3058333333333	-0.0655330244954078\\
15.3113888888889	-0.0659465838181277\\
15.3169444444444	-0.0648111445473271\\
15.3225	-0.0644646274565385\\
15.3280555555556	-0.0632986399556256\\
15.3336111111111	-0.0618322523032093\\
15.3391666666667	-0.0609274933234693\\
15.3447222222222	-0.0605103992437244\\
15.3502777777778	-0.0595856483872291\\
15.3558333333333	-0.0581193806412519\\
15.3613888888889	-0.0579534777416802\\
15.3669444444444	-0.0578819432180645\\
15.3725	-0.0575840237976731\\
15.3780555555556	-0.0560795890662515\\
15.3836111111111	-0.0560775004201043\\
15.3891666666667	-0.0559904731382868\\
15.3947222222222	-0.0552585381997964\\
15.4002777777778	-0.0548507187329549\\
15.4058333333333	-0.0544225041603003\\
15.4113888888889	-0.0560644730739649\\
15.4169444444444	-0.0593388623991715\\
15.4225	-0.0607124606287536\\
15.4280555555556	-0.0615465210638888\\
15.4336111111111	-0.0622328573785402\\
15.4391666666667	-0.0634203531121457\\
15.4447222222222	-0.0622772960120153\\
15.4502777777778	-0.06418492006931\\
15.4558333333333	-0.0647471746454056\\
15.4613888888889	-0.0668317528736403\\
15.4669444444444	-0.0676759401141742\\
15.4725	-0.0678930621487436\\
15.4780555555556	-0.0691291598701663\\
15.4836111111111	-0.0694639023263979\\
15.4891666666667	-0.070359443575246\\
15.4947222222222	-0.0695141594536845\\
15.5002777777778	-0.0710184626838916\\
15.5058333333333	-0.0709259221388119\\
15.5113888888889	-0.0725905848057884\\
15.5169444444444	-0.0741410315155283\\
15.5225	-0.0735021799459984\\
15.5280555555556	-0.0713120474544068\\
15.5336111111111	-0.0729701216577866\\
15.5391666666667	-0.073702185445077\\
15.5447222222222	-0.0738829293389591\\
15.5502777777778	-0.0741508718334397\\
15.5558333333333	-0.0737560200629354\\
15.5613888888889	-0.0720742342746623\\
15.5669444444444	-0.0715379888784622\\
15.5725	-0.0712371238826689\\
15.5780555555556	-0.0716170247358051\\
15.5836111111111	-0.0720780180786267\\
15.5891666666667	-0.0711708571866159\\
15.5947222222222	-0.06978831525285\\
15.6002777777778	-0.0691339206602724\\
15.6058333333333	-0.0702541334686819\\
15.6113888888889	-0.068986594061239\\
15.6169444444444	-0.0679153043639383\\
15.6225	-0.0699209504981724\\
15.6280555555556	-0.0701340200091915\\
15.6336111111111	-0.0707313300390954\\
15.6391666666667	-0.0739546086272273\\
15.6447222222222	-0.0754496611361343\\
15.6502777777778	-0.0756328285606356\\
15.6558333333333	-0.0748187810825245\\
15.6613888888889	-0.0734449367046408\\
15.6669444444444	-0.0737354697278207\\
15.6725	-0.0746750273352685\\
15.6780555555556	-0.0736755903766412\\
15.6836111111111	-0.0723501247302066\\
15.6891666666667	-0.0713054358232801\\
15.6947222222222	-0.0707691896925918\\
15.7002777777778	-0.0711086350112015\\
15.7058333333333	-0.0710643247164719\\
15.7113888888889	-0.0708866480746186\\
15.7169444444444	-0.072270923150101\\
15.7225	-0.0731648652220381\\
15.7280555555556	-0.0742073989600535\\
15.7336111111111	-0.0744413229158143\\
15.7391666666667	-0.0748140015496555\\
15.7447222222222	-0.0751000747041668\\
15.7502777777778	-0.0747518802131151\\
15.7558333333333	-0.0778238476262251\\
15.7613888888889	-0.0777582106052786\\
15.7669444444444	-0.0788006787213916\\
15.7725	-0.0793033181775411\\
15.7780555555556	-0.0790332118706622\\
15.7836111111111	-0.0778135767880798\\
15.7891666666667	-0.0800838213855539\\
15.7947222222222	-0.0795694530758209\\
15.8002777777778	-0.0786775476135379\\
15.8058333333333	-0.0797688519250705\\
15.8113888888889	-0.0789540377302983\\
15.8169444444444	-0.0802039775872506\\
15.8225	-0.0828673571805893\\
15.8280555555556	-0.0826531348968443\\
15.8336111111111	-0.0882578267491706\\
15.8391666666667	-0.0880540666042517\\
15.8447222222222	-0.0860874038002432\\
15.8502777777778	-0.0878266194434926\\
15.8558333333333	-0.0887979440564385\\
15.8613888888889	-0.0882748642715292\\
15.8669444444444	-0.0892019298391242\\
15.8725	-0.08779623975604\\
15.8780555555556	-0.0888668165061335\\
15.8836111111111	-0.0889468862684475\\
15.8891666666667	-0.0891061102913712\\
15.8947222222222	-0.0863420953522386\\
15.9002777777778	-0.0848543376063407\\
15.9058333333333	-0.0866091526810095\\
15.9113888888889	-0.0840874967239541\\
15.9169444444444	-0.0847881836900438\\
15.9225	-0.0846295935385751\\
15.9280555555556	-0.083072608309628\\
15.9336111111111	-0.0819726649256676\\
15.9391666666667	-0.0823573067744303\\
15.9447222222222	-0.0815325391667921\\
15.9502777777778	-0.081983385475832\\
15.9558333333333	-0.080528744078133\\
15.9613888888889	-0.0834436529505516\\
15.9669444444444	-0.0817755877103098\\
15.9725	-0.0821736240795195\\
15.9780555555556	-0.0831579925306571\\
15.9836111111111	-0.0827531844317855\\
15.9891666666667	-0.0799652572707796\\
15.9947222222222	-0.0815705861845765\\
};
\addplot [color=mycolor2,solid,line width=1.5pt,forget plot]
  table[row sep=crcr]{%
9.50027777777778	0\\
9.50583333333333	0.00699800525802049\\
9.51138888888889	0.0111147174978733\\
9.51694444444444	0.0155794638003464\\
9.5225	0.0188441811092928\\
9.52805555555556	0.0157388167460201\\
9.53361111111111	0.013912659137186\\
9.53916666666667	0.0105807887454685\\
9.54472222222222	0.0100639683665867\\
9.55027777777778	0.00962039792497916\\
9.55583333333333	0.0125038887243493\\
9.56138888888889	0.0136032154549372\\
9.56694444444444	0.0146269354623092\\
9.5725	0.0151316348950605\\
9.57805555555555	0.0137202618274813\\
9.58361111111111	0.0193037286774595\\
9.58916666666667	0.0199024304873992\\
9.59472222222222	0.0225291534813575\\
9.60027777777778	0.0238111573611265\\
9.60583333333333	0.0304694978326834\\
9.61138888888889	0.0368513624737481\\
9.61694444444444	0.0452917449778676\\
9.6225	0.0456912013839841\\
9.62805555555556	0.0505721778888803\\
9.63361111111111	0.0552111726250077\\
9.63916666666667	0.0597771642689127\\
9.64472222222222	0.0646211828373701\\
9.65027777777778	0.0625090456294405\\
9.65583333333333	0.0663473065579525\\
9.66138888888889	0.0657749230253193\\
9.66694444444444	0.0640817345320617\\
9.6725	0.0654770970176162\\
9.67805555555555	0.0701531718721737\\
9.68361111111111	0.0709074791889163\\
9.68916666666667	0.0721829963357894\\
9.69472222222222	0.0773552861411045\\
9.70027777777778	0.0803240363484366\\
9.70583333333333	0.0819860115826491\\
9.71138888888889	0.0860515911576295\\
9.71694444444444	0.0878897052195199\\
9.7225	0.0910769533900715\\
9.72805555555555	0.0952096042404503\\
9.73361111111111	0.0944368700004569\\
9.73916666666667	0.090659146630518\\
9.74472222222222	0.0938659936971711\\
9.75027777777778	0.0909195428420679\\
9.75583333333333	0.0878297488563608\\
9.76138888888889	0.0894843809782424\\
9.76694444444444	0.0938316113253781\\
9.7725	0.0962558265852492\\
9.77805555555556	0.0983287840427874\\
9.78361111111111	0.097040387857992\\
9.78916666666667	0.0942685225101049\\
9.79472222222222	0.0904061464919331\\
9.80027777777778	0.0951991033062452\\
9.80583333333333	0.0993538752208655\\
9.81138888888889	0.09951993496132\\
9.81694444444444	0.101637626579451\\
9.8225	0.101319070953318\\
9.82805555555555	0.102552756136826\\
9.83361111111111	0.103142415627821\\
9.83916666666667	0.105434486590643\\
9.84472222222222	0.104850300342417\\
9.85027777777778	0.104743031717476\\
9.85583333333333	0.103644904366618\\
9.86138888888889	0.105300527566894\\
9.86694444444444	0.104248247220232\\
9.8725	0.107333735797329\\
9.87805555555556	0.107304988255318\\
9.88361111111111	0.102472935219309\\
9.88916666666667	0.0950417850640111\\
9.89472222222222	0.0944202659323696\\
9.90027777777778	0.0949733425463338\\
9.90583333333333	0.0931758951940733\\
9.91138888888889	0.0959453847874559\\
9.91694444444444	0.0947972522665502\\
9.9225	0.0863151132071537\\
9.92805555555555	0.0828815413900828\\
9.93361111111111	0.0812505100492088\\
9.93916666666667	0.0772078922830347\\
9.94472222222222	0.0800381436716131\\
9.95027777777778	0.0867144880813085\\
9.95583333333333	0.089071420732895\\
9.96138888888889	0.0879179235222906\\
9.96694444444444	0.0897584284840106\\
9.9725	0.0894301861610319\\
9.97805555555555	0.0938535353039317\\
9.98361111111111	0.0951938562437547\\
9.98916666666667	0.0986467110821856\\
9.99472222222222	0.104763481019468\\
10.0002777777778	0.113054635643303\\
10.0058333333333	0.107777309124313\\
10.0113888888889	0.103593685110762\\
10.0169444444444	0.109716164964855\\
10.0225	0.104057473853931\\
10.0280555555556	0.107575873381085\\
10.0336111111111	0.114569904287889\\
10.0391666666667	0.113618477856143\\
10.0447222222222	0.111710785633461\\
10.0502777777778	0.110221485589784\\
10.0558333333333	0.112588711580883\\
10.0613888888889	0.114436198246066\\
10.0669444444444	0.110708433627882\\
10.0725	0.11356712105185\\
10.0780555555556	0.109081652580079\\
10.0836111111111	0.108366331290943\\
10.0891666666667	0.108827019955538\\
10.0947222222222	0.113679532171708\\
10.1002777777778	0.11797029134676\\
10.1058333333333	0.118565739189598\\
10.1113888888889	0.119478337402821\\
10.1169444444444	0.126174608984808\\
10.1225	0.116756684127315\\
10.1280555555556	0.111362263257655\\
10.1336111111111	0.115051643589446\\
10.1391666666667	0.112311474417563\\
10.1447222222222	0.115503437715207\\
10.1502777777778	0.111664661424293\\
10.1558333333333	0.111850861536351\\
10.1613888888889	0.110324044845268\\
10.1669444444444	0.113119794029326\\
10.1725	0.110488834117241\\
10.1780555555556	0.112107623788209\\
10.1836111111111	0.119334113793565\\
10.1891666666667	0.127471481514602\\
10.1947222222222	0.134816765322904\\
10.2002777777778	0.131763635701253\\
10.2058333333333	0.129770743692639\\
10.2113888888889	0.124782151590054\\
10.2169444444444	0.122127360099021\\
10.2225	0.119007354425059\\
10.2280555555556	0.120443988770163\\
10.2336111111111	0.119973137247075\\
10.2391666666667	0.122015945719394\\
10.2447222222222	0.120490338681828\\
10.2502777777778	0.12377823827583\\
10.2558333333333	0.129305083977555\\
10.2613888888889	0.141104675633065\\
10.2669444444444	0.139462689779565\\
10.2725	0.14400625434277\\
10.2780555555556	0.152990311604322\\
10.2836111111111	0.145967366689301\\
10.2891666666667	0.146625351592803\\
10.2947222222222	0.138821138532114\\
10.3002777777778	0.141679495232642\\
10.3058333333333	0.142606835937219\\
10.3113888888889	0.137641571742204\\
10.3169444444444	0.139792214668886\\
10.3225	0.142155931755219\\
10.3280555555556	0.13507394052459\\
10.3336111111111	0.138773787760056\\
10.3391666666667	0.141354594002918\\
10.3447222222222	0.140802128176442\\
10.3502777777778	0.145424236443838\\
10.3558333333333	0.143568730034878\\
10.3613888888889	0.147132195149975\\
10.3669444444444	0.154447766409646\\
10.3725	0.152953317672923\\
10.3780555555556	0.149508195305133\\
10.3836111111111	0.148937058872206\\
10.3891666666667	0.149532413365739\\
10.3947222222222	0.144963126156571\\
10.4002777777778	0.144979638445319\\
10.4058333333333	0.142930199732257\\
10.4113888888889	0.143503065464101\\
10.4169444444444	0.147282536034256\\
10.4225	0.144982948311929\\
10.4280555555556	0.144022191169106\\
10.4336111111111	0.144404629771922\\
10.4391666666667	0.140180273498714\\
10.4447222222222	0.145649561694995\\
10.4502777777778	0.154873088877617\\
10.4558333333333	0.154278746975342\\
10.4613888888889	0.149971481150009\\
10.4669444444444	0.152842284866769\\
10.4725	0.154466906916283\\
10.4780555555556	0.155874687077024\\
10.4836111111111	0.160027172213106\\
10.4891666666667	0.153085059887857\\
10.4947222222222	0.150728207622944\\
10.5002777777778	0.14651056977322\\
10.5058333333333	0.147739646445339\\
10.5113888888889	0.146136994862982\\
10.5169444444444	0.143728045132038\\
10.5225	0.149901193101464\\
10.5280555555556	0.152848668109037\\
10.5336111111111	0.150963262087309\\
10.5391666666667	0.15624474754801\\
10.5447222222222	0.153097515845525\\
10.5502777777778	0.148753987116501\\
10.5558333333333	0.150420855547066\\
10.5613888888889	0.146820748542939\\
10.5669444444444	0.143897435355776\\
10.5725	0.140554069212482\\
10.5780555555556	0.140906667321767\\
10.5836111111111	0.149660669352616\\
10.5891666666667	0.149908785196808\\
10.5947222222222	0.152998915030355\\
10.6002777777778	0.144342780134385\\
10.6058333333333	0.144545995502702\\
10.6113888888889	0.150476788745342\\
10.6169444444444	0.152929019954945\\
10.6225	0.15116802358119\\
10.6280555555556	0.156849406370012\\
10.6336111111111	0.148840659021764\\
10.6391666666667	0.143530485247736\\
10.6447222222222	0.147848821971187\\
10.6502777777778	0.14390542535499\\
10.6558333333333	0.144414743539725\\
10.6613888888889	0.146092806161257\\
10.6669444444444	0.154554332446836\\
10.6725	0.156503396071931\\
10.6780555555556	0.157191062441943\\
10.6836111111111	0.159649491266181\\
10.6891666666667	0.161790177536518\\
10.6947222222222	0.158977446587207\\
10.7002777777778	0.163140627710561\\
10.7058333333333	0.162934258261843\\
10.7113888888889	0.157424968407153\\
10.7169444444444	0.159038045955385\\
10.7225	0.148898150597609\\
10.7280555555556	0.150865782079544\\
10.7336111111111	0.14898221254162\\
10.7391666666667	0.149808691693231\\
10.7447222222222	0.152761114663386\\
10.7502777777778	0.148258607812921\\
10.7558333333333	0.152424850561957\\
10.7613888888889	0.14876141345511\\
10.7669444444444	0.136688638967959\\
10.7725	0.128609318514529\\
10.7780555555556	0.127210408730934\\
10.7836111111111	0.128562288990876\\
10.7891666666667	0.126274683538456\\
10.7947222222222	0.129397327193076\\
10.8002777777778	0.131687657054046\\
10.8058333333333	0.124312214072594\\
10.8113888888889	0.125343188442419\\
10.8169444444444	0.11758941878778\\
10.8225	0.116101305313921\\
10.8280555555556	0.12751458523155\\
10.8336111111111	0.121019280899882\\
10.8391666666667	0.117197620921426\\
10.8447222222222	0.119901742219339\\
10.8502777777778	0.118093182786526\\
10.8558333333333	0.119775285302199\\
10.8613888888889	0.124937539693658\\
10.8669444444444	0.131280194410372\\
10.8725	0.126603412027387\\
10.8780555555556	0.122853204707174\\
10.8836111111111	0.12500747768749\\
10.8891666666667	0.139076255920492\\
10.8947222222222	0.13990880719101\\
10.9002777777778	0.143258965774035\\
10.9058333333333	0.139572368648144\\
10.9113888888889	0.145691026200623\\
10.9169444444444	0.143134922852344\\
10.9225	0.138225407705327\\
10.9280555555556	0.134351271788567\\
10.9336111111111	0.140487793161129\\
10.9391666666667	0.143643310345455\\
10.9447222222222	0.139261952812687\\
10.9502777777778	0.148616649584012\\
10.9558333333333	0.149260886376621\\
10.9613888888889	0.149840745922945\\
10.9669444444444	0.152358505456321\\
10.9725	0.150330037551968\\
10.9780555555556	0.149088398974388\\
10.9836111111111	0.146025861978068\\
10.9891666666667	0.156099374904602\\
10.9947222222222	0.161358468905346\\
11.0002777777778	0.153763876472935\\
11.0058333333333	0.15982520791285\\
11.0113888888889	0.158036850311478\\
11.0169444444444	0.157258334700595\\
11.0225	0.15909926918358\\
11.0280555555556	0.160366135768288\\
11.0336111111111	0.162016552407434\\
11.0391666666667	0.161259112798753\\
11.0447222222222	0.154785921022033\\
11.0502777777778	0.155598155168377\\
11.0558333333333	0.156735951216629\\
11.0613888888889	0.158868974563049\\
11.0669444444444	0.160858625683311\\
11.0725	0.164166305056208\\
11.0780555555556	0.165857955786719\\
11.0836111111111	0.165601386237412\\
11.0891666666667	0.167125591496471\\
11.0947222222222	0.162129652339973\\
11.1002777777778	0.155637292324836\\
11.1058333333333	0.154449768978226\\
11.1113888888889	0.154665935306662\\
11.1169444444444	0.156534913987205\\
11.1225	0.157858251438226\\
11.1280555555556	0.160282694592665\\
11.1336111111111	0.153279011491462\\
11.1391666666667	0.147354160819828\\
11.1447222222222	0.146415960067966\\
11.1502777777778	0.149483683805302\\
11.1558333333333	0.146559777057125\\
11.1613888888889	0.150456129139307\\
11.1669444444444	0.148389038612083\\
11.1725	0.151293731270858\\
11.1780555555556	0.164284394931412\\
11.1836111111111	0.163884412458099\\
11.1891666666667	0.156813620466625\\
11.1947222222222	0.154678661882814\\
11.2002777777778	0.161386393880614\\
11.2058333333333	0.165068133445995\\
11.2113888888889	0.163013446624839\\
11.2169444444444	0.159831765586934\\
11.2225	0.167102426738016\\
11.2280555555556	0.167760746733558\\
11.2336111111111	0.17076610860729\\
11.2391666666667	0.169249420773432\\
11.2447222222222	0.170762384023363\\
11.2502777777778	0.168125876253955\\
11.2558333333333	0.168751692320164\\
11.2613888888889	0.16081848201669\\
11.2669444444444	0.16510149017077\\
11.2725	0.166599465056365\\
11.2780555555556	0.169788241802019\\
11.2836111111111	0.171638621231958\\
11.2891666666667	0.17464865756787\\
11.2947222222222	0.176974431527234\\
11.3002777777778	0.179821306603477\\
11.3058333333333	0.170299348030945\\
11.3113888888889	0.176387343549651\\
11.3169444444444	0.173044612441166\\
11.3225	0.180032371208913\\
11.3280555555556	0.178566737528588\\
11.3336111111111	0.18498922489083\\
11.3391666666667	0.18672979070144\\
11.3447222222222	0.188368012091165\\
11.3502777777778	0.186837293050792\\
11.3558333333333	0.189763016579484\\
11.3613888888889	0.193341862410879\\
11.3669444444444	0.190126345103722\\
11.3725	0.183870151558157\\
11.3780555555556	0.186206159892941\\
11.3836111111111	0.186442993407226\\
11.3891666666667	0.18784838231923\\
11.3947222222222	0.191062617636048\\
11.4002777777778	0.187133597632832\\
11.4058333333333	0.196549225339299\\
11.4113888888889	0.196222083730392\\
11.4169444444444	0.195388138550297\\
11.4225	0.202526366871784\\
11.4280555555556	0.201902699202349\\
11.4336111111111	0.194211293698765\\
11.4391666666667	0.189493540931399\\
11.4447222222222	0.186602814502675\\
11.4502777777778	0.193741213022316\\
11.4558333333333	0.188601972411547\\
11.4613888888889	0.188420227408393\\
11.4669444444444	0.190100860349419\\
11.4725	0.189162396146237\\
11.4780555555556	0.182060419416668\\
11.4836111111111	0.17572006301897\\
11.4891666666667	0.180624886219854\\
11.4947222222222	0.181022144034466\\
11.5002777777778	0.180935950981118\\
11.5058333333333	0.193075951245277\\
11.5113888888889	0.193181126139693\\
11.5169444444444	0.198929526039912\\
11.5225	0.20423803619366\\
11.5280555555556	0.206568950380932\\
11.5336111111111	0.207131452105825\\
11.5391666666667	0.210660831399679\\
11.5447222222222	0.213257290745843\\
11.5502777777778	0.221947203901971\\
11.5558333333333	0.221114371797909\\
11.5613888888889	0.229527359419744\\
11.5669444444444	0.225346957625156\\
11.5725	0.220919996558718\\
11.5780555555556	0.224219298912955\\
11.5836111111111	0.227825628160758\\
11.5891666666667	0.222506535033619\\
11.5947222222222	0.218198631469568\\
11.6002777777778	0.220577438821612\\
11.6058333333333	0.217530806223133\\
11.6113888888889	0.212407376592596\\
11.6169444444444	0.208815924785004\\
11.6225	0.211179122665065\\
11.6280555555556	0.215910725929412\\
11.6336111111111	0.21648064656854\\
11.6391666666667	0.215401585999385\\
11.6447222222222	0.218046456598574\\
11.6502777777778	0.223504564840207\\
11.6558333333333	0.226039468642947\\
11.6613888888889	0.228030015179283\\
11.6669444444444	0.234985402340316\\
11.6725	0.235517404600192\\
11.6780555555556	0.234227548455283\\
11.6836111111111	0.233095131373287\\
11.6891666666667	0.232776258010736\\
11.6947222222222	0.222003670056125\\
11.7002777777778	0.224033271765105\\
11.7058333333333	0.226330189661251\\
11.7113888888889	0.223264677592672\\
11.7169444444444	0.223474965363529\\
11.7225	0.228621615064505\\
11.7280555555556	0.221005953751005\\
11.7336111111111	0.214778251110535\\
11.7391666666667	0.208642599714902\\
11.7447222222222	0.205798241695531\\
11.7502777777778	0.202756974654947\\
11.7558333333333	0.192307388809928\\
11.7613888888889	0.184334028554371\\
11.7669444444444	0.181167709731879\\
11.7725	0.185784482373418\\
11.7780555555556	0.188554801493873\\
11.7836111111111	0.185601064109305\\
11.7891666666667	0.182831596592895\\
11.7947222222222	0.186174549207203\\
11.8002777777778	0.186452343660876\\
11.8058333333333	0.191619691847938\\
11.8113888888889	0.196549844451739\\
11.8169444444444	0.202732572649296\\
11.8225	0.210901723638162\\
11.8280555555556	0.21424205329354\\
11.8336111111111	0.222531933522697\\
11.8391666666667	0.217950490025543\\
11.8447222222222	0.224192493995698\\
11.8502777777778	0.226863064527515\\
11.8558333333333	0.229997793106273\\
11.8613888888889	0.234472424032154\\
11.8669444444444	0.228515861674238\\
11.8725	0.236420114735227\\
11.8780555555556	0.235710265778168\\
11.8836111111111	0.237046356551607\\
11.8891666666667	0.235193525220168\\
11.8947222222222	0.23766598584857\\
11.9002777777778	0.237440970747562\\
11.9058333333333	0.231170148982638\\
11.9113888888889	0.232884037919448\\
11.9169444444444	0.233363483506881\\
11.9225	0.234165172116141\\
11.9280555555556	0.234492092396063\\
11.9336111111111	0.237531186181599\\
11.9391666666667	0.223941396964195\\
11.9447222222222	0.232529456912446\\
11.9502777777778	0.231094750488036\\
11.9558333333333	0.228914290755345\\
11.9613888888889	0.232817895808771\\
11.9669444444444	0.236761523046563\\
11.9725	0.233121516316019\\
11.9780555555556	0.23754368898178\\
11.9836111111111	0.248398071776079\\
11.9891666666667	0.256103320363287\\
11.9947222222222	0.256582726488212\\
12.0002777777778	0.257709209468414\\
12.0058333333333	0.262011887582836\\
12.0113888888889	0.261645712107306\\
12.0169444444444	0.2608835971618\\
12.0225	0.26643697336609\\
12.0280555555556	0.269440895304925\\
12.0336111111111	0.273702711328926\\
12.0391666666667	0.267839301171702\\
12.0447222222222	0.264083141393685\\
12.0502777777778	0.257901223177421\\
12.0558333333333	0.259213595117379\\
12.0613888888889	0.251469845820124\\
12.0669444444444	0.249364850207975\\
12.0725	0.251732344834214\\
12.0780555555556	0.249411192325655\\
12.0836111111111	0.23886558420067\\
12.0891666666667	0.239980520668773\\
12.0947222222222	0.23836889523647\\
12.1002777777778	0.233138231294479\\
12.1058333333333	0.232633236909149\\
12.1113888888889	0.234508771235956\\
12.1169444444444	0.238869939335236\\
12.1225	0.239434259274408\\
12.1280555555556	0.239364436144219\\
12.1336111111111	0.242414373368918\\
12.1391666666667	0.23941546960435\\
12.1447222222222	0.242593255314521\\
12.1502777777778	0.23564928075426\\
12.1558333333333	0.229091577736513\\
12.1613888888889	0.22535931177566\\
12.1669444444444	0.234353306000641\\
12.1725	0.234388114822205\\
12.1780555555556	0.230758275733366\\
12.1836111111111	0.235023612539164\\
12.1891666666667	0.230334692105223\\
12.1947222222222	0.222395226876212\\
12.2002777777778	0.224352869870268\\
12.2058333333333	0.222929388667066\\
12.2113888888889	0.21708623743433\\
12.2169444444444	0.220778785251692\\
12.2225	0.216513985627392\\
12.2280555555556	0.224462952075997\\
12.2336111111111	0.228620585559778\\
12.2391666666667	0.233165427589882\\
12.2447222222222	0.228558928447048\\
12.2502777777778	0.224934949977701\\
12.2558333333333	0.224811036122641\\
12.2613888888889	0.227243909731383\\
12.2669444444444	0.225225846150356\\
12.2725	0.212398167750281\\
12.2780555555556	0.218120874334311\\
12.2836111111111	0.2161257500683\\
12.2891666666667	0.21749134005585\\
12.2947222222222	0.214476943938183\\
12.3002777777778	0.21163888849193\\
12.3058333333333	0.208349324919507\\
12.3113888888889	0.209078028853988\\
12.3169444444444	0.210010009660374\\
12.3225	0.215553425639993\\
12.3280555555556	0.221435735538418\\
12.3336111111111	0.224942882045503\\
12.3391666666667	0.220220887905873\\
12.3447222222222	0.22907629766387\\
12.3502777777778	0.225047840754997\\
12.3558333333333	0.232402752683783\\
12.3613888888889	0.232689501173964\\
12.3669444444444	0.231739101035762\\
12.3725	0.22693092150932\\
12.3780555555556	0.229334713661171\\
12.3836111111111	0.229043008928704\\
12.3891666666667	0.234182394126723\\
12.3947222222222	0.232962201857377\\
12.4002777777778	0.23239830296712\\
12.4058333333333	0.237942154467566\\
12.4113888888889	0.231833116421077\\
12.4169444444444	0.233315660476932\\
12.4225	0.228108739287876\\
12.4280555555556	0.23161789765622\\
12.4336111111111	0.238440715402899\\
12.4391666666667	0.236104417815034\\
12.4447222222222	0.237565229780459\\
12.4502777777778	0.245337154401163\\
12.4558333333333	0.240301346383523\\
12.4613888888889	0.239422577491499\\
12.4669444444444	0.238495164533927\\
12.4725	0.240789299928992\\
12.4780555555556	0.236406319931152\\
12.4836111111111	0.238580782470998\\
12.4891666666667	0.234675509705575\\
12.4947222222222	0.245218230074929\\
12.5002777777778	0.24610885710425\\
12.5058333333333	0.248738469499172\\
12.5113888888889	0.255546171779436\\
12.5169444444444	0.265087505688512\\
12.5225	0.263688405166419\\
12.5280555555556	0.265046841107413\\
12.5336111111111	0.262664274624736\\
12.5391666666667	0.26850857519219\\
12.5447222222222	0.272916492648535\\
12.5502777777778	0.277643741601269\\
12.5558333333333	0.272624272196898\\
12.5613888888889	0.279803514129267\\
12.5669444444444	0.273043851336336\\
12.5725	0.274567506838103\\
12.5780555555556	0.276582339797774\\
12.5836111111111	0.280765518678821\\
12.5891666666667	0.279363794000605\\
12.5947222222222	0.282327384436217\\
12.6002777777778	0.284779277378048\\
12.6058333333333	0.289493143724803\\
12.6113888888889	0.292105884785709\\
12.6169444444444	0.296245103075279\\
12.6225	0.290245398378333\\
12.6280555555556	0.292146736516955\\
12.6336111111111	0.287459168964242\\
12.6391666666667	0.286021075723809\\
12.6447222222222	0.288966771057416\\
12.6502777777778	0.280617444503248\\
12.6558333333333	0.286686324874128\\
12.6613888888889	0.287063073424498\\
12.6669444444444	0.282738537228437\\
12.6725	0.284958980774641\\
12.6780555555556	0.273297172477916\\
12.6836111111111	0.275651473564343\\
12.6891666666667	0.279885803482639\\
12.6947222222222	0.284069063336739\\
12.7002777777778	0.284725533836946\\
12.7058333333333	0.280220763846589\\
12.7113888888889	0.272359058031602\\
12.7169444444444	0.273027131996881\\
12.7225	0.27215170120136\\
12.7280555555556	0.267428078215365\\
12.7336111111111	0.263168836855739\\
12.7391666666667	0.263033578223188\\
12.7447222222222	0.266014909940517\\
12.7502777777778	0.264222309080766\\
12.7558333333333	0.267393266568355\\
12.7613888888889	0.275630433910907\\
12.7669444444444	0.273814290386613\\
12.7725	0.267018100896405\\
12.7780555555556	0.259558786681112\\
12.7836111111111	0.257168486211915\\
12.7891666666667	0.252289982813494\\
12.7947222222222	0.255138543129195\\
12.8002777777778	0.256271978564294\\
12.8058333333333	0.251987916191766\\
12.8113888888889	0.247390000381498\\
12.8169444444444	0.255429712250682\\
12.8225	0.25572991651026\\
12.8280555555556	0.259119323583618\\
12.8336111111111	0.26087391417849\\
12.8391666666667	0.266091500634448\\
12.8447222222222	0.259185987139931\\
12.8502777777778	0.260571418912163\\
12.8558333333333	0.268925824692509\\
12.8613888888889	0.26915315705729\\
12.8669444444444	0.268485054690181\\
12.8725	0.275340035499274\\
12.8780555555556	0.278981162919657\\
12.8836111111111	0.285311468633555\\
12.8891666666667	0.283332705681263\\
12.8947222222222	0.280975693552192\\
12.9002777777778	0.282953989287089\\
12.9058333333333	0.279531840688376\\
12.9113888888889	0.286651287110321\\
12.9169444444444	0.293785987089808\\
12.9225	0.294196626109618\\
12.9280555555556	0.300592905621284\\
12.9336111111111	0.300877481111212\\
12.9391666666667	0.303083317128656\\
12.9447222222222	0.29614225503649\\
12.9502777777778	0.290334306353858\\
12.9558333333333	0.2969675253931\\
12.9613888888889	0.287840599767864\\
12.9669444444444	0.280642261629766\\
12.9725	0.27717944394416\\
12.9780555555556	0.276521556302524\\
12.9836111111111	0.27153210097483\\
12.9891666666667	0.274757275926351\\
12.9947222222222	0.275186196231804\\
13.0002777777778	0.249332192312403\\
13.0058333333333	0.253515385942712\\
13.0113888888889	0.250039158101582\\
13.0169444444444	0.255073410376251\\
13.0225	0.25306455128556\\
13.0280555555556	0.254626865955951\\
13.0336111111111	0.254394834816504\\
13.0391666666667	0.259155661443268\\
13.0447222222222	0.260019096033702\\
13.0502777777778	0.268546276163172\\
13.0558333333333	0.260276484239078\\
13.0613888888889	0.263712440424919\\
13.0669444444444	0.265379454962588\\
13.0725	0.272590703628622\\
13.0780555555556	0.281516833608726\\
13.0836111111111	0.27980825670709\\
13.0891666666667	0.274802000810487\\
13.0947222222222	0.274023504513642\\
13.1002777777778	0.278000737903675\\
13.1058333333333	0.273012075843295\\
13.1113888888889	0.266401385057313\\
13.1169444444444	0.265035690852504\\
13.1225	0.26674863407369\\
13.1280555555556	0.265896661249778\\
13.1336111111111	0.272201329980953\\
13.1391666666667	0.264574045972947\\
13.1447222222222	0.267413701267675\\
13.1502777777778	0.263296984040181\\
13.1558333333333	0.260450922038304\\
13.1613888888889	0.253487995031183\\
13.1669444444444	0.258565794891786\\
13.1725	0.260180000958212\\
13.1780555555556	0.259981070801428\\
13.1836111111111	0.256512271330912\\
13.1891666666667	0.254161673991466\\
13.1947222222222	0.246035393309763\\
13.2002777777778	0.239094379297379\\
13.2058333333333	0.238486658158533\\
13.2113888888889	0.234924179290257\\
13.2169444444444	0.232394167577861\\
13.2225	0.238118617309126\\
13.2280555555556	0.239559487715968\\
13.2336111111111	0.23818007749604\\
13.2391666666667	0.240903133932463\\
13.2447222222222	0.229606008551427\\
13.2502777777778	0.236535491754698\\
13.2558333333333	0.235077230311563\\
13.2613888888889	0.234939416105535\\
13.2669444444444	0.23471316508014\\
13.2725	0.232159514632501\\
13.2780555555556	0.237775158230138\\
13.2836111111111	0.239058094415171\\
13.2891666666667	0.229901009714041\\
13.2947222222222	0.231258178041509\\
13.3002777777778	0.232055133706898\\
13.3058333333333	0.230234905450542\\
13.3113888888889	0.235077198009679\\
13.3169444444444	0.238924382501162\\
13.3225	0.234926200066342\\
13.3280555555556	0.2212747305\\
13.3336111111111	0.231245346703106\\
13.3391666666667	0.231673458968312\\
13.3447222222222	0.234471783300209\\
13.3502777777778	0.232785459101\\
13.3558333333333	0.238001079746856\\
13.3613888888889	0.228452055779758\\
13.3669444444444	0.225470944604648\\
13.3725	0.224806855315257\\
13.3780555555556	0.218050151249007\\
13.3836111111111	0.215294082420252\\
13.3891666666667	0.212240492537038\\
13.3947222222222	0.217137602285038\\
13.4002777777778	0.218576580436811\\
13.4058333333333	0.219077156383637\\
13.4113888888889	0.215570657130181\\
13.4169444444444	0.21855488773891\\
13.4225	0.223372886321655\\
13.4280555555556	0.220775020925433\\
13.4336111111111	0.221664940188965\\
13.4391666666667	0.217427471388359\\
13.4447222222222	0.219210014752129\\
13.4502777777778	0.219273505383292\\
13.4558333333333	0.217568659011153\\
13.4613888888889	0.2217054920536\\
13.4669444444444	0.223647227046738\\
13.4725	0.225293377596311\\
13.4780555555556	0.228165903039982\\
13.4836111111111	0.223809077079326\\
13.4891666666667	0.217658780452461\\
13.4947222222222	0.227100135210856\\
13.5002777777778	0.233047884670911\\
13.5058333333333	0.235463307054653\\
13.5113888888889	0.242543277878769\\
13.5169444444444	0.250488751286924\\
13.5225	0.252775390021894\\
13.5280555555556	0.254619394055151\\
13.5336111111111	0.264034341600998\\
13.5391666666667	0.258954435446779\\
13.5447222222222	0.24860824630619\\
13.5502777777778	0.245645702480699\\
13.5558333333333	0.248688090381142\\
13.5613888888889	0.241545366387957\\
13.5669444444444	0.236435937100896\\
13.5725	0.231536843312953\\
13.5780555555556	0.232092079951408\\
13.5836111111111	0.2414345372916\\
13.5891666666667	0.238722585063851\\
13.5947222222222	0.24242631665679\\
13.6002777777778	0.245007758080825\\
13.6058333333333	0.241924924395555\\
13.6113888888889	0.242313252897355\\
13.6169444444444	0.249431479555497\\
13.6225	0.241605257098916\\
13.6280555555556	0.243367949726083\\
13.6336111111111	0.239655551293556\\
13.6391666666667	0.236318846596545\\
13.6447222222222	0.236883001979598\\
13.6502777777778	0.236538953401543\\
13.6558333333333	0.240235259706578\\
13.6613888888889	0.244442563381712\\
13.6669444444444	0.249645160307406\\
13.6725	0.253636778606241\\
13.6780555555556	0.260553023741013\\
13.6836111111111	0.251611779789094\\
13.6891666666667	0.255482936972068\\
13.6947222222222	0.255666101353934\\
13.7002777777778	0.248360571177112\\
13.7058333333333	0.2457222713725\\
13.7113888888889	0.245957583419998\\
13.7169444444444	0.241654880413572\\
13.7225	0.241645231176445\\
13.7280555555556	0.245610659323395\\
13.7336111111111	0.249223551161879\\
13.7391666666667	0.243599326206924\\
13.7447222222222	0.24621716234821\\
13.7502777777778	0.253370838357969\\
13.7558333333333	0.265311179715299\\
13.7613888888889	0.262088863322512\\
13.7669444444444	0.262015507670335\\
13.7725	0.256366959978725\\
13.7780555555556	0.258991048007008\\
13.7836111111111	0.251352031199792\\
13.7891666666667	0.257536004874006\\
13.7947222222222	0.256749236830723\\
13.8002777777778	0.258312613088962\\
13.8058333333333	0.267906876384741\\
13.8113888888889	0.261610176949905\\
13.8169444444444	0.265248783089294\\
13.8225	0.272077971166764\\
13.8280555555556	0.27858781336521\\
13.8336111111111	0.277286482237415\\
13.8391666666667	0.269646889216125\\
13.8447222222222	0.265987555030728\\
13.8502777777778	0.270072214115965\\
13.8558333333333	0.263143136957248\\
13.8613888888889	0.266388888657553\\
13.8669444444444	0.262094273889736\\
13.8725	0.26577595557436\\
13.8780555555556	0.263266315820537\\
13.8836111111111	0.265671526117175\\
13.8891666666667	0.264623237612864\\
13.8947222222222	0.268110239563066\\
13.9002777777778	0.275766627652313\\
13.9058333333333	0.268555913941946\\
13.9113888888889	0.273118037409088\\
13.9169444444444	0.271987994785206\\
13.9225	0.276180634783281\\
13.9280555555556	0.279126793718039\\
13.9336111111111	0.273742558617374\\
13.9391666666667	0.270897981418897\\
13.9447222222222	0.269701375126875\\
13.9502777777778	0.269070474750049\\
13.9558333333333	0.26844899681932\\
13.9613888888889	0.267612620998253\\
13.9669444444444	0.262069596280954\\
13.9725	0.269886185317906\\
13.9780555555556	0.265983686963361\\
13.9836111111111	0.271485489573865\\
13.9891666666667	0.279607725577681\\
13.9947222222222	0.285805501854477\\
14.0002777777778	0.285344276983855\\
14.0058333333333	0.288149783797045\\
14.0113888888889	0.282340750921584\\
14.0169444444444	0.288497314205312\\
14.0225	0.289600339592562\\
14.0280555555556	0.287772530635267\\
14.0336111111111	0.285243225361522\\
14.0391666666667	0.285107344962228\\
14.0447222222222	0.297819081914688\\
14.0502777777778	0.293194763388976\\
14.0558333333333	0.298198563078953\\
14.0613888888889	0.300430710830283\\
14.0669444444444	0.307458259042903\\
14.0725	0.31169484482237\\
14.0780555555556	0.308653129596629\\
14.0836111111111	0.303277262170689\\
14.0891666666667	0.303031779753601\\
14.0947222222222	0.304648815754725\\
14.1002777777778	0.313647904953584\\
14.1058333333333	0.314969023455506\\
14.1113888888889	0.300647859894196\\
14.1169444444444	0.298717855801724\\
14.1225	0.285930667364085\\
14.1280555555556	0.292854851789304\\
14.1336111111111	0.292059020270324\\
14.1391666666667	0.294015069153934\\
14.1447222222222	0.29349244079321\\
14.1502777777778	0.289171771786452\\
14.1558333333333	0.28765016398479\\
14.1613888888889	0.282782715746429\\
14.1669444444444	0.282237530577706\\
14.1725	0.290763988813793\\
14.1780555555556	0.292336318077809\\
14.1836111111111	0.290713531046547\\
14.1891666666667	0.288516838350668\\
14.1947222222222	0.283542702468363\\
14.2002777777778	0.28821175948354\\
14.2058333333333	0.292806469761712\\
14.2113888888889	0.294477249465563\\
14.2169444444444	0.29335831251219\\
14.2225	0.294503003237282\\
14.2280555555556	0.297884550552312\\
14.2336111111111	0.307563613629343\\
14.2391666666667	0.309903228339782\\
14.2447222222222	0.307408606427922\\
14.2502777777778	0.303290006939909\\
14.2558333333333	0.308582744812407\\
14.2613888888889	0.30979371239873\\
14.2669444444444	0.307387727738165\\
14.2725	0.291846915145083\\
14.2780555555556	0.298028135179448\\
14.2836111111111	0.303101473703751\\
14.2891666666667	0.295712313224449\\
14.2947222222222	0.298143666558004\\
14.3002777777778	0.290785606448585\\
14.3058333333333	0.286786872469299\\
14.3113888888889	0.294521198984905\\
14.3169444444444	0.282472636778388\\
14.3225	0.283416312969341\\
14.3280555555556	0.270548263398427\\
14.3336111111111	0.269165584075642\\
14.3391666666667	0.257578666213721\\
14.3447222222222	0.266324476048902\\
14.3502777777778	0.255523035356934\\
14.3558333333333	0.255033450193509\\
14.3613888888889	0.248416797332717\\
14.3669444444444	0.250237691915346\\
14.3725	0.257275228118015\\
14.3780555555556	0.253670508586666\\
14.3836111111111	0.249162053758153\\
14.3891666666667	0.257590861720961\\
14.3947222222222	0.250127857824948\\
14.4002777777778	0.252734600782855\\
14.4058333333333	0.252696286291487\\
14.4113888888889	0.25380375352757\\
14.4169444444444	0.262816042715925\\
14.4225	0.26005077192527\\
14.4280555555556	0.252952863446492\\
14.4336111111111	0.257503834750346\\
14.4391666666667	0.261618764790629\\
14.4447222222222	0.253631394349994\\
14.4502777777778	0.259988914689109\\
14.4558333333333	0.262090845999436\\
14.4613888888889	0.271276916570932\\
14.4669444444444	0.276664988842757\\
14.4725	0.269610901251223\\
14.4780555555556	0.276758744650711\\
14.4836111111111	0.280554914935336\\
14.4891666666667	0.285082486421057\\
14.4947222222222	0.289149242979385\\
14.5002777777778	0.294909130589848\\
14.5058333333333	0.290411793930574\\
14.5113888888889	0.306848320604884\\
14.5169444444444	0.318029074429129\\
14.5225	0.314242814388416\\
14.5280555555556	0.317326558052154\\
14.5336111111111	0.315335436287413\\
14.5391666666667	0.310084766720704\\
14.5447222222222	0.309946224158643\\
14.5502777777778	0.315888253641791\\
14.5558333333333	0.316429140104327\\
14.5613888888889	0.320864014954803\\
14.5669444444444	0.326182995251144\\
14.5725	0.333008724591315\\
14.5780555555556	0.339979661713884\\
14.5836111111111	0.344705798057627\\
14.5891666666667	0.345860919465025\\
14.5947222222222	0.341994308073984\\
14.6002777777778	0.324815033441737\\
14.6058333333333	0.324792794665862\\
14.6113888888889	0.330221024664691\\
14.6169444444444	0.32619891010618\\
14.6225	0.326624436224579\\
14.6280555555556	0.323242788541878\\
14.6336111111111	0.312464742695661\\
14.6391666666667	0.302813505288528\\
14.6447222222222	0.299697676055737\\
14.6502777777778	0.301403282230151\\
14.6558333333333	0.297771996223196\\
14.6613888888889	0.304199237000501\\
14.6669444444444	0.290375772658803\\
14.6725	0.282776399056594\\
14.6780555555556	0.286296422353364\\
14.6836111111111	0.29604687882365\\
14.6891666666667	0.287336453351764\\
14.6947222222222	0.287535207004236\\
14.7002777777778	0.283139939565182\\
14.7058333333333	0.28392036918837\\
14.7113888888889	0.287338394656494\\
14.7169444444444	0.296676993975935\\
14.7225	0.28920045183783\\
14.7280555555556	0.295997878642638\\
14.7336111111111	0.293516865570676\\
14.7391666666667	0.304019146255438\\
14.7447222222222	0.307600558909559\\
14.7502777777778	0.308483634146818\\
14.7558333333333	0.306980499722594\\
14.7613888888889	0.310309631991865\\
14.7669444444444	0.315012503279077\\
14.7725	0.310814166911645\\
14.7780555555556	0.312796023240524\\
14.7836111111111	0.325877778755343\\
14.7891666666667	0.331177937995437\\
14.7947222222222	0.326507651355541\\
14.8002777777778	0.323374324814574\\
14.8058333333333	0.311851638428736\\
14.8113888888889	0.318750561959114\\
14.8169444444444	0.312038723607285\\
14.8225	0.312769723597691\\
14.8280555555556	0.319948083683515\\
14.8336111111111	0.315299811192222\\
14.8391666666667	0.306706935881552\\
14.8447222222222	0.298372200686778\\
14.8502777777778	0.30092572447514\\
14.8558333333333	0.306267325904648\\
14.8613888888889	0.303800997904189\\
14.8669444444444	0.30976287953769\\
14.8725	0.311710367932942\\
14.8780555555556	0.320215260777781\\
14.8836111111111	0.318780364748563\\
14.8891666666667	0.325498397515195\\
14.8947222222222	0.334939436952845\\
14.9002777777778	0.337322564684404\\
14.9058333333333	0.337100134440209\\
14.9113888888889	0.345590374299937\\
14.9169444444444	0.349619133489444\\
14.9225	0.349387669091231\\
14.9280555555556	0.351011039261074\\
14.9336111111111	0.360899300275657\\
14.9391666666667	0.357989732643187\\
14.9447222222222	0.359777878618278\\
14.9502777777778	0.363935347799821\\
14.9558333333333	0.351850473427783\\
14.9613888888889	0.358122822041899\\
14.9669444444444	0.365769804809568\\
14.9725	0.372599666949577\\
14.9780555555556	0.368806063016461\\
14.9836111111111	0.370835832238404\\
14.9891666666667	0.370994306861811\\
14.9947222222222	0.374785812968208\\
15.0002777777778	0.364600932183738\\
15.0058333333333	0.369978323068304\\
15.0113888888889	0.382896605816535\\
15.0169444444444	0.376081858310585\\
15.0225	0.377235271725913\\
15.0280555555556	0.377556484281957\\
15.0336111111111	0.392880545530201\\
15.0391666666667	0.384407714448108\\
15.0447222222222	0.385793531317804\\
15.0502777777778	0.385903043426255\\
15.0558333333333	0.383913352847215\\
15.0613888888889	0.379897896007339\\
15.0669444444444	0.390853863757121\\
15.0725	0.397645317403982\\
15.0780555555556	0.409717344949292\\
15.0836111111111	0.415035769436484\\
15.0891666666667	0.420295346748837\\
15.0947222222222	0.421181484500148\\
15.1002777777778	0.429478498920923\\
15.1058333333333	0.426239073676527\\
15.1113888888889	0.425167826427016\\
15.1169444444444	0.42913281163109\\
15.1225	0.429772693197313\\
15.1280555555556	0.429344020952305\\
15.1336111111111	0.436051235142702\\
15.1391666666667	0.423556481156392\\
15.1447222222222	0.425686786245235\\
15.1502777777778	0.429326265990757\\
15.1558333333333	0.430710135608648\\
15.1613888888889	0.433472799170946\\
15.1669444444444	0.443327051850205\\
15.1725	0.437355179760855\\
15.1780555555556	0.445133982387873\\
15.1836111111111	0.433957989981078\\
15.1891666666667	0.443215561962051\\
15.1947222222222	0.438679083547935\\
15.2002777777778	0.439377180852986\\
15.2058333333333	0.429192910630442\\
15.2113888888889	0.422242874348044\\
15.2169444444444	0.421073599838043\\
15.2225	0.425094717182473\\
15.2280555555556	0.411600893838198\\
15.2336111111111	0.409102051386518\\
15.2391666666667	0.405403343332996\\
15.2447222222222	0.400905895499748\\
15.2502777777778	0.412067505836362\\
15.2558333333333	0.392892427088104\\
15.2613888888889	0.406985293840852\\
15.2669444444444	0.404784207045827\\
15.2725	0.399074566930437\\
15.2780555555556	0.403955687314294\\
15.2836111111111	0.395297658752918\\
15.2891666666667	0.406087898685185\\
15.2947222222222	0.393731251626664\\
15.3002777777778	0.391727678105173\\
15.3058333333333	0.398935227257759\\
15.3113888888889	0.400659481255635\\
15.3169444444444	0.401324282625217\\
15.3225	0.399469678011974\\
15.3280555555556	0.396009428207714\\
15.3336111111111	0.393747507549479\\
15.3391666666667	0.396330182387282\\
15.3447222222222	0.392631949207979\\
15.3502777777778	0.398060320379443\\
15.3558333333333	0.393433746025905\\
15.3613888888889	0.399530966227524\\
15.3669444444444	0.390762412771581\\
15.3725	0.38397718386898\\
15.3780555555556	0.38352341004263\\
15.3836111111111	0.375753198866038\\
15.3891666666667	0.379611919019136\\
15.3947222222222	0.368009246997742\\
15.4002777777778	0.366181049926904\\
15.4058333333333	0.359338722149457\\
15.4113888888889	0.359156560035117\\
15.4169444444444	0.340152320256741\\
15.4225	0.341828522941717\\
15.4280555555556	0.345195420672676\\
15.4336111111111	0.350970734732586\\
15.4391666666667	0.36219161397698\\
15.4447222222222	0.365915117596418\\
15.4502777777778	0.376167445488164\\
15.4558333333333	0.365094611499081\\
15.4613888888889	0.372579645638237\\
15.4669444444444	0.369926221701071\\
15.4725	0.366904411451034\\
15.4780555555556	0.369825403666648\\
15.4836111111111	0.368306515521815\\
15.4891666666667	0.372954323248897\\
15.4947222222222	0.36635823513779\\
15.5002777777778	0.360115712681187\\
15.5058333333333	0.359677435134966\\
15.5113888888889	0.365432372807399\\
15.5169444444444	0.360014596132793\\
15.5225	0.355254202850194\\
15.5280555555556	0.338582148512786\\
15.5336111111111	0.32353525781473\\
15.5391666666667	0.335168580291336\\
15.5447222222222	0.334391825243584\\
15.5502777777778	0.330517327897488\\
15.5558333333333	0.325006913063457\\
15.5613888888889	0.32237661440238\\
15.5669444444444	0.324505014819251\\
15.5725	0.322982393382841\\
15.5780555555556	0.315631905411849\\
15.5836111111111	0.31467776815211\\
15.5891666666667	0.319157981424987\\
15.5947222222222	0.306820309169063\\
15.6002777777778	0.299242295927237\\
15.6058333333333	0.295618591639069\\
15.6113888888889	0.291534110888289\\
15.6169444444444	0.288721209383409\\
15.6225	0.285779464967098\\
15.6280555555556	0.290114461174419\\
15.6336111111111	0.296789924935367\\
15.6391666666667	0.299445958579855\\
15.6447222222222	0.307119957643951\\
15.6502777777778	0.310345621398185\\
15.6558333333333	0.309587198573678\\
15.6613888888889	0.315315542432046\\
15.6669444444444	0.310460551708082\\
15.6725	0.312527122564226\\
15.6780555555556	0.320321052820259\\
15.6836111111111	0.312086504130312\\
15.6891666666667	0.303275888514856\\
15.6947222222222	0.309980712514894\\
15.7002777777778	0.292302136974588\\
15.7058333333333	0.296016462353337\\
15.7113888888889	0.290262180960377\\
15.7169444444444	0.287801545981188\\
15.7225	0.291615427548885\\
15.7280555555556	0.291549948385685\\
15.7336111111111	0.297240346692263\\
15.7391666666667	0.290077633184704\\
15.7447222222222	0.303195298770402\\
15.7502777777778	0.302974152140184\\
15.7558333333333	0.319236331184199\\
15.7613888888889	0.342112228740187\\
15.7669444444444	0.35286121105456\\
15.7725	0.351166397941391\\
15.7780555555556	0.340494757593451\\
15.7836111111111	0.335382979939895\\
15.7891666666667	0.337557433956299\\
15.7947222222222	0.347157037598583\\
15.8002777777778	0.348523262295334\\
15.8058333333333	0.342921714946348\\
15.8113888888889	0.341595389201082\\
15.8169444444444	0.341676653997535\\
15.8225	0.342442661894565\\
15.8280555555556	0.345684473066749\\
15.8336111111111	0.36088250258946\\
15.8391666666667	0.382293478113177\\
15.8447222222222	0.37494398775186\\
15.8502777777778	0.390204589650284\\
15.8558333333333	0.401718399070259\\
15.8613888888889	0.396033252985583\\
15.8669444444444	0.409982061452276\\
15.8725	0.408403650682967\\
15.8780555555556	0.43306607677229\\
15.8836111111111	0.432295598598746\\
15.8891666666667	0.457678129417771\\
15.8947222222222	0.448158957811116\\
15.9002777777778	0.446614702650631\\
15.9058333333333	0.437213662886332\\
15.9113888888889	0.433598233448683\\
15.9169444444444	0.43711244891424\\
15.9225	0.432598808229401\\
15.9280555555556	0.428945313605798\\
15.9336111111111	0.43513134596268\\
15.9391666666667	0.42944732712866\\
15.9447222222222	0.424193334078483\\
15.9502777777778	0.437709052823568\\
15.9558333333333	0.420398784658409\\
15.9613888888889	0.433587907715018\\
15.9669444444444	0.425372846982581\\
15.9725	0.424182355637395\\
15.9780555555556	0.419760946145329\\
15.9836111111111	0.40855134284465\\
15.9891666666667	0.387769909844712\\
15.9947222222222	0.387851242125163\\
};
\addplot [color=mycolor3,solid,line width=1.5pt,forget plot]
  table[row sep=crcr]{%
9.50027777777778	0\\
9.50583333333333	0.00105510490137203\\
9.51138888888889	0.00187713438083597\\
9.51694444444444	0.00099142050580654\\
9.5225	0.00108632770578239\\
9.52805555555556	-0.000706212987721958\\
9.53361111111111	-0.00196163743916607\\
9.53916666666667	-0.00354024715176882\\
9.54472222222222	-0.00394926590634644\\
9.55027777777778	-0.00453608957817277\\
9.55583333333333	-0.00633800039868441\\
9.56138888888889	-0.00609636982476073\\
9.56694444444444	-0.00643674474204986\\
9.5725	-0.00671742505150852\\
9.57805555555555	-0.00667648204849717\\
9.58361111111111	-0.00467854128196624\\
9.58916666666667	-0.00410216972214507\\
9.59472222222222	-0.00508940427861309\\
9.60027777777778	-0.0064462845424984\\
9.60583333333333	-0.00647658148585988\\
9.61138888888889	-0.00669264939678745\\
9.61694444444444	-0.00596288419162119\\
9.6225	-0.00616453761238081\\
9.62805555555556	-0.00696926068421175\\
9.63361111111111	-0.00691969122261969\\
9.63916666666667	-0.00681958381931043\\
9.64472222222222	-0.00693858454011985\\
9.65027777777778	-0.00747068857720691\\
9.65583333333333	-0.00735650650607516\\
9.66138888888889	-0.00744913475488472\\
9.66694444444444	-0.00788016857514948\\
9.6725	-0.00650837005713482\\
9.67805555555555	-0.00601375727792518\\
9.68361111111111	-0.00661329708315352\\
9.68916666666667	-0.0066599418226282\\
9.69472222222222	-0.00603565826844498\\
9.70027777777778	-0.00538895691029195\\
9.70583333333333	-0.00484389638049027\\
9.71138888888889	-0.00546867301528917\\
9.71694444444444	-0.00526637516844474\\
9.7225	-0.00467517434022665\\
9.72805555555555	-0.00382092964129354\\
9.73361111111111	-0.00359967498188794\\
9.73916666666667	-0.00396583283457221\\
9.74472222222222	-0.00345577928761646\\
9.75027777777778	-0.00341586568172531\\
9.75583333333333	-0.00428074871909077\\
9.76138888888889	-0.00565967993094331\\
9.76694444444444	-0.00581008712894325\\
9.7725	-0.00531826344439258\\
9.77805555555556	-0.00442314635278204\\
9.78361111111111	-0.00419449902440855\\
9.78916666666667	-0.00457069633474722\\
9.79472222222222	-0.00495651925011638\\
9.80027777777778	-0.00392966766314484\\
9.80583333333333	-0.00363702305980866\\
9.81138888888889	-0.00329666584120828\\
9.81694444444444	-0.00299659594951874\\
9.8225	-0.0020563765795773\\
9.82805555555555	-0.00177489813951767\\
9.83361111111111	-0.00153358826496892\\
9.83916666666667	-0.00133611802214707\\
9.84472222222222	-0.00103168221316236\\
9.85027777777778	-0.000544579364506018\\
9.85583333333333	-0.000849299792819774\\
9.86138888888889	-0.00148865705305795\\
9.86694444444444	-0.00216496439810932\\
9.8725	-0.00228571670182423\\
9.87805555555556	-0.00159321491007906\\
9.88361111111111	-0.00124482480973709\\
9.88916666666667	-0.00187349507453669\\
9.89472222222222	-0.00263690428363833\\
9.90027777777778	-0.00270457422107282\\
9.90583333333333	-0.00240963906854491\\
9.91138888888889	-0.00241577596704018\\
9.91694444444444	-0.00243622843795877\\
9.9225	-0.00189065864417209\\
9.92805555555555	-0.00187514342857254\\
9.93361111111111	-0.00130060202493207\\
9.93916666666667	-0.00190403844830032\\
9.94472222222222	-0.00185813841225205\\
9.95027777777778	-0.0015823295976341\\
9.95583333333333	-0.00185741647350771\\
9.96138888888889	-0.00162107081735119\\
9.96694444444444	-0.0010595521821258\\
9.9725	-0.00108805488754703\\
9.97805555555555	-0.000586460339062314\\
9.98361111111111	-0.000197314428623274\\
9.98916666666667	-6.14342267024757e-05\\
9.99472222222222	-1.58760969060893e-05\\
10.0002777777778	3.21917421008411e-05\\
10.0058333333333	-0.000580909950094987\\
10.0113888888889	-0.000267787261229586\\
10.0169444444444	-4.41802766241268e-05\\
10.0225	0.00101779279815079\\
10.0280555555556	0.000853452186187302\\
10.0336111111111	0.000870231123188706\\
10.0391666666667	0.000748414679146672\\
10.0447222222222	0.000903651565286474\\
10.0502777777778	0.000451488543819041\\
10.0558333333333	-0.000484894458930331\\
10.0613888888889	-0.00132406472797092\\
10.0669444444444	-0.000707744092593276\\
10.0725	-0.000312670563642663\\
10.0780555555556	-0.000646168003771232\\
10.0836111111111	-0.000697761198970801\\
10.0891666666667	-0.000595302976760395\\
10.0947222222222	8.47069792620983e-05\\
10.1002777777778	0.000452516965817156\\
10.1058333333333	0.000840289619888054\\
10.1113888888889	0.000899548043797418\\
10.1169444444444	0.000627182948820596\\
10.1225	-0.000332333840314005\\
10.1280555555556	-0.000742352265125481\\
10.1336111111111	-0.000178405874173873\\
10.1391666666667	-0.000520328628552873\\
10.1447222222222	-0.000758052359901946\\
10.1502777777778	0.000228931036781232\\
10.1558333333333	0.00046645417122651\\
10.1613888888889	0.00108905035425142\\
10.1669444444444	0.00225153596872074\\
10.1725	0.00161953730388863\\
10.1780555555556	0.00107658922684381\\
10.1836111111111	0.00103024755235685\\
10.1891666666667	0.000973630038406047\\
10.1947222222222	0.00129466577536929\\
10.2002777777778	0.00152083868053972\\
10.2058333333333	0.00122715583708811\\
10.2113888888889	0.000916044665051332\\
10.2169444444444	0.000822867870617303\\
10.2225	0.000459705419275842\\
10.2280555555556	0.000100254950770782\\
10.2336111111111	-0.000227251824708981\\
10.2391666666667	9.695387411235e-05\\
10.2447222222222	0.000661363243548243\\
10.2502777777778	0.000481676146999846\\
10.2558333333333	-0.000261993364717442\\
10.2613888888889	0.000888356789601973\\
10.2669444444444	0.000896405180160239\\
10.2725	0.00180369796815615\\
10.2780555555556	0.00213296377688179\\
10.2836111111111	0.00128847181744964\\
10.2891666666667	0.00155424111107365\\
10.2947222222222	0.000797479873341894\\
10.3002777777778	0.000863578965327492\\
10.3058333333333	0.000832387808147388\\
10.3113888888889	0.00070129636308038\\
10.3169444444444	0.000886872203454095\\
10.3225	0.00130025663147137\\
10.3280555555556	0.00129572669031123\\
10.3336111111111	0.000518262790024925\\
10.3391666666667	0.00138039824687054\\
10.3447222222222	0.00126866156751121\\
10.3502777777778	9.02542592420981e-05\\
10.3558333333333	-6.62096442582002e-06\\
10.3613888888889	0.000927863892734845\\
10.3669444444444	0.000566207628404592\\
10.3725	0.00129605910591721\\
10.3780555555556	0.00100490671099412\\
10.3836111111111	0.00124484442655112\\
10.3891666666667	0.00145342270860973\\
10.3947222222222	0.0015619072251619\\
10.4002777777778	0.00056572492761471\\
10.4058333333333	-0.000608130179224609\\
10.4113888888889	0.000534601500460208\\
10.4169444444444	0.000924997409545339\\
10.4225	0.00066325863244754\\
10.4280555555556	0.000586873053067177\\
10.4336111111111	0.000731562537033278\\
10.4391666666667	0.00112367791507366\\
10.4447222222222	0.000922457901530244\\
10.4502777777778	0.000392199207336609\\
10.4558333333333	0.000737826411469922\\
10.4613888888889	0.000396066925091134\\
10.4669444444444	-0.000165219591727662\\
10.4725	0.000410780113854518\\
10.4780555555556	0.000229334664657227\\
10.4836111111111	0.000928685361844399\\
10.4891666666667	0.00167405651767021\\
10.4947222222222	0.00234518772215859\\
10.5002777777778	0.00206507300560728\\
10.5058333333333	0.00206435734449126\\
10.5113888888889	0.00211302686809811\\
10.5169444444444	0.00177411117496893\\
10.5225	0.00258870450040023\\
10.5280555555556	0.0019518005052309\\
10.5336111111111	0.00146599471198995\\
10.5391666666667	0.00169884833813631\\
10.5447222222222	0.00131494435799627\\
10.5502777777778	0.00124423508288142\\
10.5558333333333	0.00149252316996643\\
10.5613888888889	0.00153834114957257\\
10.5669444444444	0.00130060236156471\\
10.5725	0.00138917279654001\\
10.5780555555556	0.00113666708280864\\
10.5836111111111	0.00133696526001739\\
10.5891666666667	0.00161872336199964\\
10.5947222222222	0.00223572409127095\\
10.6002777777778	0.00206593765686888\\
10.6058333333333	0.00179844473768033\\
10.6113888888889	0.000156043123178978\\
10.6169444444444	0.000623458000974501\\
10.6225	0.000105159537333694\\
10.6280555555556	0.000821324220666513\\
10.6336111111111	0.000765217518655134\\
10.6391666666667	0.00092922838298112\\
10.6447222222222	-3.70331699428757e-05\\
10.6502777777778	-0.000676592467494106\\
10.6558333333333	-0.00134435502009194\\
10.6613888888889	-0.00129028589275136\\
10.6669444444444	-0.0011702525884435\\
10.6725	-0.000761223231886302\\
10.6780555555556	3.97020333316676e-05\\
10.6836111111111	0.000371594815621773\\
10.6891666666667	0.00120777977260782\\
10.6947222222222	0.0010880348530735\\
10.7002777777778	-6.18686669558018e-05\\
10.7058333333333	0.00017150235533575\\
10.7113888888889	-0.00022190997003052\\
10.7169444444444	0.000491949910576358\\
10.7225	0.000204090678821415\\
10.7280555555556	4.99103453739136e-05\\
10.7336111111111	0.000400252969681827\\
10.7391666666667	0.000121468310855856\\
10.7447222222222	-2.96243950550644e-05\\
10.7502777777778	-0.000806287387437796\\
10.7558333333333	-0.00121157684890314\\
10.7613888888889	-0.00101066270949787\\
10.7669444444444	-0.00161010621017871\\
10.7725	-0.00183190731293864\\
10.7780555555556	-0.00201252061427152\\
10.7836111111111	-0.00228650491118496\\
10.7891666666667	-0.00261338393041608\\
10.7947222222222	-0.00226616364565852\\
10.8002777777778	-0.00254065161032586\\
10.8058333333333	-0.00326637721572431\\
10.8113888888889	-0.00335920227871189\\
10.8169444444444	-0.00316426055753407\\
10.8225	-0.0029889320153662\\
10.8280555555556	-0.00334092388823856\\
10.8336111111111	-0.00348139827378043\\
10.8391666666667	-0.00313463258124587\\
10.8447222222222	-0.00233505396191278\\
10.8502777777778	-0.00242904817782887\\
10.8558333333333	-0.00251052030554847\\
10.8613888888889	-0.00246689153081722\\
10.8669444444444	-0.00206533066561295\\
10.8725	-0.00120348463074929\\
10.8780555555556	-0.00209804791003971\\
10.8836111111111	-0.00233217894617278\\
10.8891666666667	-0.00280214455117497\\
10.8947222222222	-0.00265273448387529\\
10.9002777777778	-0.00193097542283444\\
10.9058333333333	-0.00150900313198718\\
10.9113888888889	-0.00202489266830069\\
10.9169444444444	-0.00179152945448936\\
10.9225	-0.00220136459340092\\
10.9280555555556	-0.00246244491218362\\
10.9336111111111	-0.00226988185184168\\
10.9391666666667	-0.00256802619028137\\
10.9447222222222	-0.00205107580186797\\
10.9502777777778	-0.00212708609807536\\
10.9558333333333	-0.00224255163361243\\
10.9613888888889	-0.00210656278426422\\
10.9669444444444	-0.00209057470148742\\
10.9725	-0.00262780496897923\\
10.9780555555556	-0.0023607366787764\\
10.9836111111111	-0.00244565793847332\\
10.9891666666667	-0.0018987414347471\\
10.9947222222222	-0.00188717165716874\\
11.0002777777778	-0.00207359391417714\\
11.0058333333333	-0.00242790600487638\\
11.0113888888889	-0.00292757653195908\\
11.0169444444444	-0.00287404176350171\\
11.0225	-0.00225650640716685\\
11.0280555555556	-0.00158796119330282\\
11.0336111111111	-0.00196009183563225\\
11.0391666666667	-0.00166130734173788\\
11.0447222222222	-0.001739719237407\\
11.0502777777778	-0.00165398790672717\\
11.0558333333333	-0.00209345303439801\\
11.0613888888889	-0.00220433177512737\\
11.0669444444444	-0.00271647393924528\\
11.0725	-0.0021553257023344\\
11.0780555555556	-0.00151799735175663\\
11.0836111111111	-0.00238562894823606\\
11.0891666666667	-0.00226724639867309\\
11.0947222222222	-0.00246820356309984\\
11.1002777777778	-0.00354714753591916\\
11.1058333333333	-0.00284180398137526\\
11.1113888888889	-0.00317005139261674\\
11.1169444444444	-0.00295814665246373\\
11.1225	-0.00274902821864856\\
11.1280555555556	-0.00277750422277773\\
11.1336111111111	-0.00264700345523354\\
11.1391666666667	-0.00310763287013174\\
11.1447222222222	-0.00382345636373973\\
11.1502777777778	-0.00384081276235374\\
11.1558333333333	-0.00409231556414718\\
11.1613888888889	-0.00426608655201973\\
11.1669444444444	-0.00467136815654306\\
11.1725	-0.00489446619639998\\
11.1780555555556	-0.00471265233181442\\
11.1836111111111	-0.00555508144503174\\
11.1891666666667	-0.00498689379901169\\
11.1947222222222	-0.00489486929080026\\
11.2002777777778	-0.0043370977729299\\
11.2058333333333	-0.00412278280760936\\
11.2113888888889	-0.00462357771924625\\
11.2169444444444	-0.0043219409323326\\
11.2225	-0.00354112968740365\\
11.2280555555556	-0.00320928226762492\\
11.2336111111111	-0.00339163661298105\\
11.2391666666667	-0.00333369499940832\\
11.2447222222222	-0.00330868357488615\\
11.2502777777778	-0.0034847050439609\\
11.2558333333333	-0.00350105830205986\\
11.2613888888889	-0.00408609551888199\\
11.2669444444444	-0.00374024374293122\\
11.2725	-0.00360841440101851\\
11.2780555555556	-0.00372597386920546\\
11.2836111111111	-0.00409105716204832\\
11.2891666666667	-0.00350374041714249\\
11.2947222222222	-0.00291128096237439\\
11.3002777777778	-0.00253200552445096\\
11.3058333333333	-0.00179117649575804\\
11.3113888888889	-0.00168787425115357\\
11.3169444444444	-0.00215622862662215\\
11.3225	-0.00141653048992316\\
11.3280555555556	-0.0018624566650524\\
11.3336111111111	-0.0019161891032039\\
11.3391666666667	-0.00211253759610617\\
11.3447222222222	-0.00249585864125029\\
11.3502777777778	-0.00238966538857307\\
11.3558333333333	-0.00207560974309982\\
11.3613888888889	-0.00160598789670436\\
11.3669444444444	-0.000979225789626567\\
11.3725	-0.00154887966160343\\
11.3780555555556	-0.00140816551161149\\
11.3836111111111	-0.000807855223682554\\
11.3891666666667	-0.00136771115806945\\
11.3947222222222	-0.000756542856142191\\
11.4002777777778	-0.00102531703262834\\
11.4058333333333	-0.000686149560749374\\
11.4113888888889	-0.000353988286507684\\
11.4169444444444	-0.000421697107145064\\
11.4225	-0.000728710195489695\\
11.4280555555556	-0.00157689490347871\\
11.4336111111111	-0.00153336172907631\\
11.4391666666667	-0.00174354158104591\\
11.4447222222222	-0.00221162102612168\\
11.4502777777778	-0.00219402424484457\\
11.4558333333333	-0.00241743435331342\\
11.4613888888889	-0.00261483529941945\\
11.4669444444444	-0.00287039755627901\\
11.4725	-0.003319081150096\\
11.4780555555556	-0.00329020630397406\\
11.4836111111111	-0.00331216153850325\\
11.4891666666667	-0.00322106311913489\\
11.4947222222222	-0.00332237231208587\\
11.5002777777778	-0.00279627732926569\\
11.5058333333333	-0.00230902111330847\\
11.5113888888889	-0.00346826421326499\\
11.5169444444444	-0.00403319864898899\\
11.5225	-0.00387490299059452\\
11.5280555555556	-0.00331886605038377\\
11.5336111111111	-0.0033008729189154\\
11.5391666666667	-0.00345984057093291\\
11.5447222222222	-0.00290627831808743\\
11.5502777777778	-0.00304329072142367\\
11.5558333333333	-0.00355927151217109\\
11.5613888888889	-0.0027521440932366\\
11.5669444444444	-0.00279252976494532\\
11.5725	-0.00212250374028736\\
11.5780555555556	-0.00265555544097434\\
11.5836111111111	-0.00225139502783684\\
11.5891666666667	-0.00313095422365619\\
11.5947222222222	-0.0027153192585063\\
11.6002777777778	-0.00339987701910372\\
11.6058333333333	-0.00339745517897921\\
11.6113888888889	-0.00301605959371107\\
11.6169444444444	-0.00330128836848215\\
11.6225	-0.00360012536406034\\
11.6280555555556	-0.00391332664948066\\
11.6336111111111	-0.00402987413889833\\
11.6391666666667	-0.00343160836164373\\
11.6447222222222	-0.00337230652823958\\
11.6502777777778	-0.00324915071019172\\
11.6558333333333	-0.00360164041671559\\
11.6613888888889	-0.00297146408581778\\
11.6669444444444	-0.00302195462032564\\
11.6725	-0.00266493582039228\\
11.6780555555556	-0.00328223174485839\\
11.6836111111111	-0.00227238455322631\\
11.6891666666667	-0.00226376000790849\\
11.6947222222222	-0.00275042109825108\\
11.7002777777778	-0.00235171879088708\\
11.7058333333333	-0.00202837227714795\\
11.7113888888889	-0.00182023114896623\\
11.7169444444444	-0.00158739345561005\\
11.7225	-0.00175088309824662\\
11.7280555555556	-0.00187404119401991\\
11.7336111111111	-0.00141838487886301\\
11.7391666666667	-0.00131651558349108\\
11.7447222222222	-0.00146534801192379\\
11.7502777777778	-0.00128081823119992\\
11.7558333333333	-0.00157634192273254\\
11.7613888888889	-0.00197019112614141\\
11.7669444444444	-0.00187053846701119\\
11.7725	-0.00200259498647847\\
11.7780555555556	-0.00229454255521241\\
11.7836111111111	-0.00179493367727911\\
11.7891666666667	-0.00237341530451037\\
11.7947222222222	-0.00159197936295001\\
11.8002777777778	-0.00210505527170803\\
11.8058333333333	-0.00250932555319436\\
11.8113888888889	-0.00275829239657627\\
11.8169444444444	-0.00238064732916393\\
11.8225	-0.00295803787465328\\
11.8280555555556	-0.00253709764505193\\
11.8336111111111	-0.00266112969690463\\
11.8391666666667	-0.00352531524798319\\
11.8447222222222	-0.00336999042970155\\
11.8502777777778	-0.00341882035967923\\
11.8558333333333	-0.00293821512970818\\
11.8613888888889	-0.00275576059612211\\
11.8669444444444	-0.00300936839804913\\
11.8725	-0.0029936270814919\\
11.8780555555556	-0.00346590103040119\\
11.8836111111111	-0.00358526592909629\\
11.8891666666667	-0.00344908176813573\\
11.8947222222222	-0.00307240006172294\\
11.9002777777778	-0.00310956558326048\\
11.9058333333333	-0.00333987317242913\\
11.9113888888889	-0.00326041350231964\\
11.9169444444444	-0.00351010448738736\\
11.9225	-0.00302621712996578\\
11.9280555555556	-0.00305844462280244\\
11.9336111111111	-0.00308902441201925\\
11.9391666666667	-0.00355043463472672\\
11.9447222222222	-0.00357275908272985\\
11.9502777777778	-0.00398214038843046\\
11.9558333333333	-0.003981136983476\\
11.9613888888889	-0.00429700243539295\\
11.9669444444444	-0.00416111786154994\\
11.9725	-0.00454420531608424\\
11.9780555555556	-0.00442158792800725\\
11.9836111111111	-0.00407136536030626\\
11.9891666666667	-0.00410390954792059\\
11.9947222222222	-0.00429446473388492\\
12.0002777777778	-0.00367022727485878\\
12.0058333333333	-0.00461948188413346\\
12.0113888888889	-0.00492399284783995\\
12.0169444444444	-0.00547942220125023\\
12.0225	-0.00501730017741617\\
12.0280555555556	-0.00474673995822263\\
12.0336111111111	-0.00466356471563505\\
12.0391666666667	-0.00437896979823246\\
12.0447222222222	-0.0045715968254916\\
12.0502777777778	-0.00442601742594896\\
12.0558333333333	-0.00405954563361193\\
12.0613888888889	-0.00424219645827504\\
12.0669444444444	-0.0041298428021025\\
12.0725	-0.00418709801863685\\
12.0780555555556	-0.004578865182923\\
12.0836111111111	-0.00533020452213601\\
12.0891666666667	-0.00595880343653866\\
12.0947222222222	-0.00604747355284558\\
12.1002777777778	-0.00646171611353613\\
12.1058333333333	-0.00545153672965198\\
12.1113888888889	-0.00537167044548065\\
12.1169444444444	-0.00551842522938655\\
12.1225	-0.00539769842162958\\
12.1280555555556	-0.00533213416752749\\
12.1336111111111	-0.00540319275267247\\
12.1391666666667	-0.00530153951489575\\
12.1447222222222	-0.00584310526736921\\
12.1502777777778	-0.00672482308617277\\
12.1558333333333	-0.00718217766255474\\
12.1613888888889	-0.00729407720710462\\
12.1669444444444	-0.00654118207055634\\
12.1725	-0.00649716896878673\\
12.1780555555556	-0.00674429491129298\\
12.1836111111111	-0.00684073199144384\\
12.1891666666667	-0.0078186741964887\\
12.1947222222222	-0.00788930807217331\\
12.2002777777778	-0.00722736290202903\\
12.2058333333333	-0.00701142922397031\\
12.2113888888889	-0.00749381708708132\\
12.2169444444444	-0.00755142804917152\\
12.2225	-0.00734932846400287\\
12.2280555555556	-0.00709070106452136\\
12.2336111111111	-0.00691570570292963\\
12.2391666666667	-0.00662810255208385\\
12.2447222222222	-0.00755480963853709\\
12.2502777777778	-0.00724525882753051\\
12.2558333333333	-0.00806579527235643\\
12.2613888888889	-0.00830222055729709\\
12.2669444444444	-0.00813723108079458\\
12.2725	-0.00797471707246856\\
12.2780555555556	-0.00809151593572046\\
12.2836111111111	-0.0079682668240348\\
12.2891666666667	-0.0077545533060414\\
12.2947222222222	-0.00755700680630854\\
12.3002777777778	-0.00738473217957086\\
12.3058333333333	-0.00733034770954917\\
12.3113888888889	-0.00718969136672639\\
12.3169444444444	-0.00791414918733174\\
12.3225	-0.00776073653007719\\
12.3280555555556	-0.0074911444574531\\
12.3336111111111	-0.00784239071435808\\
12.3391666666667	-0.00823211654941165\\
12.3447222222222	-0.00814672087217471\\
12.3502777777778	-0.00780636708952129\\
12.3558333333333	-0.00759999838695221\\
12.3613888888889	-0.00691316414346735\\
12.3669444444444	-0.00683471551831401\\
12.3725	-0.00664499975216761\\
12.3780555555556	-0.00673149246311662\\
12.3836111111111	-0.00633796084267286\\
12.3891666666667	-0.00636396430297958\\
12.3947222222222	-0.00656639067539567\\
12.4002777777778	-0.00619754127744964\\
12.4058333333333	-0.00637079414009452\\
12.4113888888889	-0.00650540020147176\\
12.4169444444444	-0.00642850687156109\\
12.4225	-0.00629642789625369\\
12.4280555555556	-0.00622439896045629\\
12.4336111111111	-0.00644041043964889\\
12.4391666666667	-0.0064756184014187\\
12.4447222222222	-0.00652563718678743\\
12.4502777777778	-0.00626916086498833\\
12.4558333333333	-0.00688822072461581\\
12.4613888888889	-0.00703405320628882\\
12.4669444444444	-0.00682162550938922\\
12.4725	-0.00646731302064068\\
12.4780555555556	-0.00699707106355496\\
12.4836111111111	-0.00717146007853853\\
12.4891666666667	-0.00741347268120376\\
12.4947222222222	-0.00727956176321858\\
12.5002777777778	-0.00753410362081518\\
12.5058333333333	-0.00775331148598662\\
12.5113888888889	-0.00811247353323185\\
12.5169444444444	-0.00828012369342491\\
12.5225	-0.00774470379949713\\
12.5280555555556	-0.00759912356449128\\
12.5336111111111	-0.00724361686863114\\
12.5391666666667	-0.0069907429035519\\
12.5447222222222	-0.00668201059666704\\
12.5502777777778	-0.0062991145368757\\
12.5558333333333	-0.00575944024961383\\
12.5613888888889	-0.00534085243780581\\
12.5669444444444	-0.00530485926059974\\
12.5725	-0.00586504448362377\\
12.5780555555556	-0.00583008216854878\\
12.5836111111111	-0.00563530336571198\\
12.5891666666667	-0.00578056319533235\\
12.5947222222222	-0.00534322995556281\\
12.6002777777778	-0.00552007787743497\\
12.6058333333333	-0.00534900940498902\\
12.6113888888889	-0.00486389798625352\\
12.6169444444444	-0.00513833945655226\\
12.6225	-0.00564734655238119\\
12.6280555555556	-0.00570305310019806\\
12.6336111111111	-0.00592443416617206\\
12.6391666666667	-0.00568976402362951\\
12.6447222222222	-0.00567336795588753\\
12.6502777777778	-0.00590037686119689\\
12.6558333333333	-0.00620729124007732\\
12.6613888888889	-0.00556416206686711\\
12.6669444444444	-0.00576624520439321\\
12.6725	-0.00576524498990361\\
12.6780555555556	-0.00521334214161975\\
12.6836111111111	-0.00528420122023005\\
12.6891666666667	-0.00513742090199846\\
12.6947222222222	-0.0053609793371435\\
12.7002777777778	-0.00542939021727779\\
12.7058333333333	-0.00561117847338917\\
12.7113888888889	-0.0058821395093861\\
12.7169444444444	-0.00611623679674888\\
12.7225	-0.00674314093759998\\
12.7280555555556	-0.00661449050019676\\
12.7336111111111	-0.00670720151666193\\
12.7391666666667	-0.00688039533017181\\
12.7447222222222	-0.00673980555100281\\
12.7502777777778	-0.0067523965857355\\
12.7558333333333	-0.00614953869412352\\
12.7613888888889	-0.00632680705891274\\
12.7669444444444	-0.00662702423424788\\
12.7725	-0.00701616251656929\\
12.7780555555556	-0.00698974545221071\\
12.7836111111111	-0.00686198900125347\\
12.7891666666667	-0.00747256360462439\\
12.7947222222222	-0.00710458988403691\\
12.8002777777778	-0.00747505411532079\\
12.8058333333333	-0.00724063678090796\\
12.8113888888889	-0.00725313901195211\\
12.8169444444444	-0.00719355362925353\\
12.8225	-0.00720051323322716\\
12.8280555555556	-0.00720321816242391\\
12.8336111111111	-0.00704033523586396\\
12.8391666666667	-0.00719381780865839\\
12.8447222222222	-0.00718686439319524\\
12.8502777777778	-0.00685579731594089\\
12.8558333333333	-0.00669063071580153\\
12.8613888888889	-0.00699822626800276\\
12.8669444444444	-0.00651849214207505\\
12.8725	-0.00588066133296297\\
12.8780555555556	-0.00549764702901659\\
12.8836111111111	-0.00544015970664337\\
12.8891666666667	-0.00590555165388464\\
12.8947222222222	-0.0063760477406515\\
12.9002777777778	-0.0063616876415155\\
12.9058333333333	-0.00547679177265959\\
12.9113888888889	-0.00494383186424105\\
12.9169444444444	-0.00506253304925585\\
12.9225	-0.00520446700706743\\
12.9280555555556	-0.00494077943437658\\
12.9336111111111	-0.00464255518070488\\
12.9391666666667	-0.00431301111105924\\
12.9447222222222	-0.00472012975947739\\
12.9502777777778	-0.00537685347977137\\
12.9558333333333	-0.005240126100915\\
12.9613888888889	-0.00549790884604836\\
12.9669444444444	-0.00503642422633873\\
12.9725	-0.00604969824165108\\
12.9780555555556	-0.00506150338218448\\
12.9836111111111	-0.00520668023407424\\
12.9891666666667	-0.00511458297383666\\
12.9947222222222	-0.00531424411312264\\
13.0002777777778	-0.00674266328273718\\
13.0058333333333	-0.00606839200847735\\
13.0113888888889	-0.00602651506769553\\
13.0169444444444	-0.00612768743210678\\
13.0225	-0.00593628290198049\\
13.0280555555556	-0.00562494014872548\\
13.0336111111111	-0.00563558414270239\\
13.0391666666667	-0.00627022544456133\\
13.0447222222222	-0.00613975442058044\\
13.0502777777778	-0.00592519225530459\\
13.0558333333333	-0.00675924156767602\\
13.0613888888889	-0.00695909075937071\\
13.0669444444444	-0.0067698832991645\\
13.0725	-0.00654575878591084\\
13.0780555555556	-0.00589813139025097\\
13.0836111111111	-0.0064666126505792\\
13.0891666666667	-0.00634101818859934\\
13.0947222222222	-0.0063592962285315\\
13.1002777777778	-0.00558138333117789\\
13.1058333333333	-0.00548214733877288\\
13.1113888888889	-0.00567382027054082\\
13.1169444444444	-0.00582724547260757\\
13.1225	-0.00535753373673883\\
13.1280555555556	-0.00542019721761132\\
13.1336111111111	-0.00581983673673677\\
13.1391666666667	-0.00597950845491898\\
13.1447222222222	-0.00592500424821805\\
13.1502777777778	-0.00570281849805723\\
13.1558333333333	-0.00566725413893095\\
13.1613888888889	-0.00641952025697416\\
13.1669444444444	-0.00654115550317854\\
13.1725	-0.00651208998776887\\
13.1780555555556	-0.00676745183675509\\
13.1836111111111	-0.00671860354948793\\
13.1891666666667	-0.00630578336787362\\
13.1947222222222	-0.00678345351803947\\
13.2002777777778	-0.00665073653505327\\
13.2058333333333	-0.0062792984542006\\
13.2113888888889	-0.00683076202313702\\
13.2169444444444	-0.00676492644419039\\
13.2225	-0.00645831979826832\\
13.2280555555556	-0.00664209748303688\\
13.2336111111111	-0.00679308817745033\\
13.2391666666667	-0.00730425657039282\\
13.2447222222222	-0.00725141090050886\\
13.2502777777778	-0.00718841361089149\\
13.2558333333333	-0.00670510396857338\\
13.2613888888889	-0.00710780763221525\\
13.2669444444444	-0.006674013599078\\
13.2725	-0.00662417359285752\\
13.2780555555556	-0.00657023741498685\\
13.2836111111111	-0.0067514350233511\\
13.2891666666667	-0.00697414650317356\\
13.2947222222222	-0.00659273360495277\\
13.3002777777778	-0.00617495272476834\\
13.3058333333333	-0.00595170881206759\\
13.3113888888889	-0.00640673583752901\\
13.3169444444444	-0.00636201024002785\\
13.3225	-0.00624115749574544\\
13.3280555555556	-0.00599318467875971\\
13.3336111111111	-0.00583178024477258\\
13.3391666666667	-0.00486894559137808\\
13.3447222222222	-0.00466202237329041\\
13.3502777777778	-0.00468870731355817\\
13.3558333333333	-0.00500529233117906\\
13.3613888888889	-0.00508015169632627\\
13.3669444444444	-0.00536492617860682\\
13.3725	-0.00579830861447064\\
13.3780555555556	-0.00608062950907205\\
13.3836111111111	-0.00621447931655769\\
13.3891666666667	-0.00603592540443397\\
13.3947222222222	-0.00587095748349517\\
13.4002777777778	-0.00649456888673007\\
13.4058333333333	-0.00660083601450529\\
13.4113888888889	-0.00654057648000052\\
13.4169444444444	-0.00627380656297579\\
13.4225	-0.00591119937577246\\
13.4280555555556	-0.00586532594007134\\
13.4336111111111	-0.00560279245640692\\
13.4391666666667	-0.00571566818416068\\
13.4447222222222	-0.00597032079581068\\
13.4502777777778	-0.00571228282941806\\
13.4558333333333	-0.00535425111091148\\
13.4613888888889	-0.00509893052989728\\
13.4669444444444	-0.00563723206053494\\
13.4725	-0.00492045085952295\\
13.4780555555556	-0.00466557392503345\\
13.4836111111111	-0.00504356102091499\\
13.4891666666667	-0.00518115559295868\\
13.4947222222222	-0.00519806062333191\\
13.5002777777778	-0.0051011796493618\\
13.5058333333333	-0.00507454485926343\\
13.5113888888889	-0.00451447141166255\\
13.5169444444444	-0.00499202175962157\\
13.5225	-0.00485616435451909\\
13.5280555555556	-0.00472315263200575\\
13.5336111111111	-0.00519370263274405\\
13.5391666666667	-0.00499769928519492\\
13.5447222222222	-0.00449475302934396\\
13.5502777777778	-0.00417822516666222\\
13.5558333333333	-0.00423177746165506\\
13.5613888888889	-0.00342144163936717\\
13.5669444444444	-0.00365394369943667\\
13.5725	-0.00324344202397391\\
13.5780555555556	-0.00335170654188708\\
13.5836111111111	-0.00334726970800189\\
13.5891666666667	-0.0031623805356124\\
13.5947222222222	-0.00323503701023162\\
13.6002777777778	-0.00293819385601561\\
13.6058333333333	-0.00317653995181109\\
13.6113888888889	-0.00307400403088199\\
13.6169444444444	-0.00262069092240872\\
13.6225	-0.00263047608762805\\
13.6280555555556	-0.0024996944278674\\
13.6336111111111	-0.00264569299116933\\
13.6391666666667	-0.00235161595650067\\
13.6447222222222	-0.00185880193811623\\
13.6502777777778	-0.00204095196062137\\
13.6558333333333	-0.00172254070120088\\
13.6613888888889	-0.00208682667175104\\
13.6669444444444	-0.00192628319528963\\
13.6725	-0.00226718833575247\\
13.6780555555556	-0.00263410065556192\\
13.6836111111111	-0.00255890491638493\\
13.6891666666667	-0.0025654804596113\\
13.6947222222222	-0.00234610645481841\\
13.7002777777778	-0.00255250036190271\\
13.7058333333333	-0.00265227951321571\\
13.7113888888889	-0.00231613687319366\\
13.7169444444444	-0.00266183856861006\\
13.7225	-0.00290188178293359\\
13.7280555555556	-0.0030961944614716\\
13.7336111111111	-0.00356923215295211\\
13.7391666666667	-0.00359211555092288\\
13.7447222222222	-0.00406059859540585\\
13.7502777777778	-0.00372621374754809\\
13.7558333333333	-0.00396004119695898\\
13.7613888888889	-0.00373493812670719\\
13.7669444444444	-0.00345796616568298\\
13.7725	-0.00371583932640803\\
13.7780555555556	-0.00298098522879809\\
13.7836111111111	-0.00259659944321224\\
13.7891666666667	-0.00290400769900493\\
13.7947222222222	-0.00280724370646535\\
13.8002777777778	-0.00306064913489112\\
13.8058333333333	-0.00314940460728079\\
13.8113888888889	-0.00317996663394376\\
13.8169444444444	-0.00301082483195603\\
13.8225	-0.00311509584087935\\
13.8280555555556	-0.0032276532405997\\
13.8336111111111	-0.00272251065992181\\
13.8391666666667	-0.00301512157123655\\
13.8447222222222	-0.00315381363276775\\
13.8502777777778	-0.0030150113084014\\
13.8558333333333	-0.00328521079509662\\
13.8613888888889	-0.00271380684019413\\
13.8669444444444	-0.00262562564782389\\
13.8725	-0.00219163684800635\\
13.8780555555556	-0.00142701021732627\\
13.8836111111111	-0.00195443702480312\\
13.8891666666667	-0.00243168278613569\\
13.8947222222222	-0.00235106777517362\\
13.9002777777778	-0.00231437422759642\\
13.9058333333333	-0.0023210873755917\\
13.9113888888889	-0.00238802543860542\\
13.9169444444444	-0.00220786855960184\\
13.9225	-0.00192576397714168\\
13.9280555555556	-0.00223988860613754\\
13.9336111111111	-0.0018477975913177\\
13.9391666666667	-0.00205247516210993\\
13.9447222222222	-0.00198763355785051\\
13.9502777777778	-0.00197550668788747\\
13.9558333333333	-0.00196221177360271\\
13.9613888888889	-0.00156943683628898\\
13.9669444444444	-0.00147287740394541\\
13.9725	-0.0019206653876686\\
13.9780555555556	-0.00127639203711731\\
13.9836111111111	-0.00136853307558257\\
13.9891666666667	-0.00217212743093773\\
13.9947222222222	-0.00253248679477352\\
14.0002777777778	-0.00223862620266543\\
14.0058333333333	-0.000825808237261107\\
14.0113888888889	-0.000448414260197241\\
14.0169444444444	-0.000866744550012253\\
14.0225	-0.00149444359712434\\
14.0280555555556	-0.00132934601061586\\
14.0336111111111	-0.00125504403852656\\
14.0391666666667	-0.00129654827499113\\
14.0447222222222	-0.0019165284079094\\
14.0502777777778	-0.000789032443631325\\
14.0558333333333	-6.04845220905135e-05\\
14.0613888888889	-0.000710220787692124\\
14.0669444444444	-0.000332948607733726\\
14.0725	-0.000129976467186299\\
14.0780555555556	-0.000278409649700793\\
14.0836111111111	-0.000147960272522401\\
14.0891666666667	0.000270465840011995\\
14.0947222222222	-0.000245818421384593\\
14.1002777777778	0.000131834611002585\\
14.1058333333333	-0.000277276601618745\\
14.1113888888889	-0.000425505911439506\\
14.1169444444444	-0.000403704301935715\\
14.1225	-0.00041916896079957\\
14.1280555555556	-0.000535036912311065\\
14.1336111111111	-0.000219190732197225\\
14.1391666666667	-0.000307491038057705\\
14.1447222222222	-0.000180745277825997\\
14.1502777777778	-0.000318737706227642\\
14.1558333333333	-0.000225290945858049\\
14.1613888888889	-0.000177835635168956\\
14.1669444444444	-0.000243043306792065\\
14.1725	6.40195586318993e-05\\
14.1780555555556	0.000730786443799467\\
14.1836111111111	0.00116311227555202\\
14.1891666666667	0.00131464312197704\\
14.1947222222222	0.00161725978850796\\
14.2002777777778	0.00138033381191662\\
14.2058333333333	0.00215120020675901\\
14.2113888888889	0.00149934826247982\\
14.2169444444444	0.00219952884857111\\
14.2225	0.00192994092125615\\
14.2280555555556	0.00232414414300568\\
14.2336111111111	0.00250309835310127\\
14.2391666666667	0.00290604735588882\\
14.2447222222222	0.00269718176997283\\
14.2502777777778	0.0031059403968509\\
14.2558333333333	0.00362705824512811\\
14.2613888888889	0.00303138515320232\\
14.2669444444444	0.00292801476719045\\
14.2725	0.00289920438664094\\
14.2780555555556	0.00292743290223992\\
14.2836111111111	0.00309929700757222\\
14.2891666666667	0.00206012813922381\\
14.2947222222222	0.00147693747644152\\
14.3002777777778	0.00229434504857998\\
14.3058333333333	0.00187520897015614\\
14.3113888888889	0.0020696227913045\\
14.3169444444444	0.00189082994777013\\
14.3225	0.00191134841571921\\
14.3280555555556	0.0013932658607132\\
14.3336111111111	0.00137713119877273\\
14.3391666666667	0.00119342573422904\\
14.3447222222222	0.00136051226822304\\
14.3502777777778	0.00195192001467932\\
14.3558333333333	0.00122697557303141\\
14.3613888888889	0.00174686249790185\\
14.3669444444444	0.00221549121853477\\
14.3725	0.00186354945671867\\
14.3780555555556	0.00188150328189016\\
14.3836111111111	0.00185892823355704\\
14.3891666666667	0.00176853161709832\\
14.3947222222222	0.00198845236796153\\
14.4002777777778	0.00148125952021046\\
14.4058333333333	0.000822433110068685\\
14.4113888888889	0.000400598693667818\\
14.4169444444444	-0.000317968784793502\\
14.4225	-0.000409675189735743\\
14.4280555555556	-0.000307594596889109\\
14.4336111111111	-0.000651996637032194\\
14.4391666666667	-0.000782417905524244\\
14.4447222222222	-0.000858480612417008\\
14.4502777777778	-0.00105340321010804\\
14.4558333333333	-0.000565214293432088\\
14.4613888888889	-0.0010945123233728\\
14.4669444444444	-0.000935509848916229\\
14.4725	-0.0011396475901497\\
14.4780555555556	-0.00120216032599859\\
14.4836111111111	-0.000163589099381901\\
14.4891666666667	0.000176162962450809\\
14.4947222222222	8.10651367532322e-06\\
14.5002777777778	-0.00028408137898971\\
14.5058333333333	-0.000661113007558667\\
14.5113888888889	-0.00073887955730407\\
14.5169444444444	-0.000693757320046465\\
14.5225	-0.000269912606847844\\
14.5280555555556	-0.000951901352722758\\
14.5336111111111	-0.000806766763808484\\
14.5391666666667	-0.000441013380285867\\
14.5447222222222	-0.000687202533958698\\
14.5502777777778	-0.000309305126563642\\
14.5558333333333	-0.000529308636250617\\
14.5613888888889	-0.000725577284937937\\
14.5669444444444	-0.000356894388137838\\
14.5725	-0.000482000439131545\\
14.5780555555556	-0.000600463812680473\\
14.5836111111111	-0.00030721016185227\\
14.5891666666667	-0.000473911985785789\\
14.5947222222222	-0.000763327864454989\\
14.6002777777778	-0.00169144225149101\\
14.6058333333333	-0.00225109467530694\\
14.6113888888889	-0.00158747207530159\\
14.6169444444444	-0.00132215179579281\\
14.6225	-0.00142390668386022\\
14.6280555555556	-0.000570146967062488\\
14.6336111111111	-0.000290913790791452\\
14.6391666666667	-0.00110586564401453\\
14.6447222222222	-0.00154318183255352\\
14.6502777777778	-0.00150532136871957\\
14.6558333333333	-0.00137918137088373\\
14.6613888888889	-0.000998443512611094\\
14.6669444444444	-0.00149122512069739\\
14.6725	-0.00135024726159122\\
14.6780555555556	-0.00138538601714688\\
14.6836111111111	-0.000926416173318245\\
14.6891666666667	-0.000390734323668872\\
14.6947222222222	-0.00051794217275978\\
14.7002777777778	-0.000852489810769617\\
14.7058333333333	-0.00106637607256883\\
14.7113888888889	-0.00126713371432075\\
14.7169444444444	-0.00128633302699236\\
14.7225	-0.00151906827254398\\
14.7280555555556	-0.00121241208079449\\
14.7336111111111	-0.00145040296212633\\
14.7391666666667	-0.00160114831372675\\
14.7447222222222	-0.00143530725876599\\
14.7502777777778	-0.0014675279804849\\
14.7558333333333	-0.00137856506065998\\
14.7613888888889	-0.000827583729784872\\
14.7669444444444	-0.00109945552471309\\
14.7725	-0.00182101134674518\\
14.7780555555556	-0.00144142320736952\\
14.7836111111111	-0.0013831294118033\\
14.7891666666667	-0.000764209002148827\\
14.7947222222222	-7.73080371501522e-05\\
14.8002777777778	-0.000888251643158421\\
14.8058333333333	-0.000782716384168665\\
14.8113888888889	-0.00051343898091114\\
14.8169444444444	-0.000863742334729912\\
14.8225	-0.00151122644153752\\
14.8280555555556	-0.00118787665699552\\
14.8336111111111	-0.000418430727615876\\
14.8391666666667	-0.000610798112415516\\
14.8447222222222	-0.0005818143229706\\
14.8502777777778	-0.000167866411300952\\
14.8558333333333	-0.000755595567821645\\
14.8613888888889	-0.000253102605393162\\
14.8669444444444	-0.000381179871605801\\
14.8725	-0.000641977177844919\\
14.8780555555556	-0.000467812102274727\\
14.8836111111111	-0.00026849185507102\\
14.8891666666667	-0.000770726700626225\\
14.8947222222222	-0.000589075756291307\\
14.9002777777778	-0.000186930252750465\\
14.9058333333333	0.000184080335228293\\
14.9113888888889	5.91672874086642e-05\\
14.9169444444444	-0.000364693963403472\\
14.9225	-0.000631691191405986\\
14.9280555555556	-0.000400295842684844\\
14.9336111111111	-0.00062300124971729\\
14.9391666666667	-0.000594108461415168\\
14.9447222222222	-0.000872876939317527\\
14.9502777777778	-0.00108720770083168\\
14.9558333333333	-0.00126138259175758\\
14.9613888888889	-0.00111224417376843\\
14.9669444444444	-0.000795669307822165\\
14.9725	-0.000531671605088502\\
14.9780555555556	-0.000638766901323465\\
14.9836111111111	-0.00105228744070632\\
14.9891666666667	-0.000554466551384562\\
14.9947222222222	-0.000700354002704177\\
15.0002777777778	-0.000628307409482369\\
15.0058333333333	-0.00162071212686934\\
15.0113888888889	-0.00128719884076\\
15.0169444444444	-0.000971185680577546\\
15.0225	-0.000574047620044499\\
15.0280555555556	-0.00066963753972673\\
15.0336111111111	-0.000476972610872077\\
15.0391666666667	-8.55347084515294e-05\\
15.0447222222222	-0.000427053917450563\\
15.0502777777778	-0.00129660834469507\\
15.0558333333333	-0.00201757088487976\\
15.0613888888889	-0.00189950535871275\\
15.0669444444444	-0.00155157959570687\\
15.0725	-0.00192790461440197\\
15.0780555555556	-0.00138908915148012\\
15.0836111111111	-0.0012416309719092\\
15.0891666666667	-0.00146427401699286\\
15.0947222222222	-0.00223957309047808\\
15.1002777777778	-0.00200835556468963\\
15.1058333333333	-0.00129023957049029\\
15.1113888888889	-0.00129962314920604\\
15.1169444444444	-0.00175020809630784\\
15.1225	-0.00285988045605599\\
15.1280555555556	-0.00272605518431812\\
15.1336111111111	-0.001995241580536\\
15.1391666666667	-0.00267944089101845\\
15.1447222222222	-0.00299395186391418\\
15.1502777777778	-0.00240209544791513\\
15.1558333333333	-0.00222816569349845\\
15.1613888888889	-0.00177459543277783\\
15.1669444444444	-0.00272470599579659\\
15.1725	-0.00233983922648634\\
15.1780555555556	-0.00280181996384124\\
15.1836111111111	-0.00212790001552902\\
15.1891666666667	-0.00266029183180898\\
15.1947222222222	-0.00228062023614784\\
15.2002777777778	-0.00216183743196691\\
15.2058333333333	-0.00237841399235928\\
15.2113888888889	-0.00225825109079587\\
15.2169444444444	-0.00229586091266269\\
15.2225	-0.00268616234620257\\
15.2280555555556	-0.00324525127121844\\
15.2336111111111	-0.00308433374915995\\
15.2391666666667	-0.00375398534812507\\
15.2447222222222	-0.00348248341992989\\
15.2502777777778	-0.00341966328449319\\
15.2558333333333	-0.00370843701640623\\
15.2613888888889	-0.0025549368478949\\
15.2669444444444	-0.00332064530573379\\
15.2725	-0.00313883035374028\\
15.2780555555556	-0.00358079543031177\\
15.2836111111111	-0.00294197421767562\\
15.2891666666667	-0.00282417968750899\\
15.2947222222222	-0.00280864679895522\\
15.3002777777778	-0.0024065876455407\\
15.3058333333333	-0.00292201807848314\\
15.3113888888889	-0.00278012215654112\\
15.3169444444444	-0.00264716119724804\\
15.3225	-0.0023123037270505\\
15.3280555555556	-0.00187926638195607\\
15.3336111111111	-0.000814561214639939\\
15.3391666666667	0.0101312161216089\\
15.3447222222222	0.00985895132137382\\
15.3502777777778	0.0102252766763586\\
15.3558333333333	0.0103827487650521\\
15.3613888888889	0.0107416164882773\\
15.3669444444444	0.0113262694461758\\
15.3725	0.0112203242836376\\
15.3780555555556	0.011770977456657\\
15.3836111111111	0.0117029221266406\\
15.3891666666667	0.0118164456527531\\
15.3947222222222	0.0119819215694067\\
15.4002777777778	0.0119575829703416\\
15.4058333333333	0.0118840265346545\\
15.4113888888889	0.0116660215420487\\
15.4169444444444	0.00194768733190078\\
15.4225	0.00155964448154419\\
15.4280555555556	0.00111268105030351\\
15.4336111111111	0.00140872556630313\\
15.4391666666667	0.00123222410030688\\
15.4447222222222	0.000963024178538347\\
15.4502777777778	0.000629488800598088\\
15.4558333333333	0.000578500092765631\\
15.4613888888889	0.000236413276255964\\
15.4669444444444	5.96799990822737e-05\\
15.4725	-0.000344448035935011\\
15.4780555555556	-0.000877218496219356\\
15.4836111111111	-0.00124749933327631\\
15.4891666666667	-0.000994216755147496\\
15.4947222222222	-0.00035234426012301\\
15.5002777777778	-0.000668376524678524\\
15.5058333333333	-0.0010437926532521\\
15.5113888888889	-0.001063230753982\\
15.5169444444444	-0.00154857106738122\\
15.5225	-0.00171475832179258\\
15.5280555555556	-0.00157008677377193\\
15.5336111111111	-0.00240186976335028\\
15.5391666666667	-0.00243778705329513\\
15.5447222222222	-0.00248856920016912\\
15.5502777777778	-0.00246338150489783\\
15.5558333333333	-0.00207181276787465\\
15.5613888888889	-0.000819601671798731\\
15.5669444444444	-0.000200579495993931\\
15.5725	-0.000109527240877652\\
15.5780555555556	-0.000222106351705112\\
15.5836111111111	-0.000100993732512722\\
15.5891666666667	-4.68111283932809e-05\\
15.5947222222222	0.000271635309118683\\
15.6002777777778	0.00023983368242079\\
15.6058333333333	-0.000216243862089613\\
15.6113888888889	4.62803034179767e-05\\
15.6169444444444	0.000549756268064812\\
15.6225	-5.67977244605511e-05\\
15.6280555555556	-9.32299989666441e-05\\
15.6336111111111	-0.000255093086168826\\
15.6391666666667	-0.00192704730397217\\
15.6447222222222	-0.00232073453421303\\
15.6502777777778	-0.00180275464398599\\
15.6558333333333	-0.000590196630134717\\
15.6613888888889	0.000162454010248904\\
15.6669444444444	-0.00027596526634792\\
15.6725	-0.000456050542730106\\
15.6780555555556	0.000109544725657256\\
15.6836111111111	7.30338028484865e-05\\
15.6891666666667	0.000632995277215626\\
15.6947222222222	0.00139651504429849\\
15.7002777777778	0.000782853465299642\\
15.7058333333333	0.000960042539004506\\
15.7113888888889	0.00166340471866333\\
15.7169444444444	0.0010616181369357\\
15.7225	0.00127463152092975\\
15.7280555555556	0.000807490352769059\\
15.7336111111111	0.00107691405677323\\
15.7391666666667	0.000937820344615797\\
15.7447222222222	0.000804203434258168\\
15.7502777777778	0.000856164420660452\\
15.7558333333333	0.000429568431315537\\
15.7613888888889	0.000394640421059994\\
15.7669444444444	-0.000241654463939156\\
15.7725	-0.000470007477230196\\
15.7780555555556	0.000181071716595076\\
15.7836111111111	0.000273571986546861\\
15.7891666666667	-0.000199777980723934\\
15.7947222222222	0.000838269495300001\\
15.8002777777778	0.00163400701761078\\
15.8058333333333	0.00172485661358588\\
15.8113888888889	0.0016536589146436\\
15.8169444444444	0.00167032322165324\\
15.8225	0.000759717594381703\\
15.8280555555556	0.000968714675259453\\
15.8336111111111	0.000792381911278844\\
15.8391666666667	0.00028920853956881\\
15.8447222222222	0.000639464887096642\\
15.8502777777778	8.96864762709054e-05\\
15.8558333333333	-0.000780292667205029\\
15.8613888888889	-0.000880194819433769\\
15.8669444444444	-0.00139771767322089\\
15.8725	-0.000671796711124237\\
15.8780555555556	-0.00119394024988376\\
15.8836111111111	-0.00153996871624618\\
15.8891666666667	-0.0016490075818538\\
15.8947222222222	-0.000373366427785196\\
15.9002777777778	0.000103480566953934\\
15.9058333333333	-0.00077748794149127\\
15.9113888888889	5.64263355870784e-05\\
15.9169444444444	-0.000177608929628512\\
15.9225	-0.000205894261599867\\
15.9280555555556	0.00103512844819666\\
15.9336111111111	0.00164926010239431\\
15.9391666666667	0.00126814854476559\\
15.9447222222222	0.00253421061639874\\
15.9502777777778	-0.000490525444369534\\
15.9558333333333	0.0016734884415519\\
15.9613888888889	-0.00479268946289348\\
15.9669444444444	-0.00260412696926195\\
15.9725	-0.00294718190710439\\
15.9780555555556	-0.00298302185894575\\
15.9836111111111	-0.002337475724238\\
15.9891666666667	-0.000408134680232706\\
15.9947222222222	-0.0015865133178809\\
};
\end{axis}
\end{tikzpicture}%
%
\end{subfigure}\\%
%
\leavevmode\smash{\makebox[0pt]{\hspace{-3em}% HORIZONTAL POSITION           
  \rotatebox[origin=l]{90}{\hspace{13em}% VERTICAL POSITION
	Normalized Profit and Loss (P\&L)}%
}}\hspace{0pt plus 1filll}\null%

Time (h)

\vspace{1cm}%
\begin{subfigure}{\linewidth}%
  \centering%
  \tikzsetnextfilename{Ch2/naivestratlegend}%
  \definecolor{mycolor1}{rgb}{0.00000,0.00000,0.00000}%
\definecolor{mycolor2}{rgb}{0.40000,0.40000,0.40000}%
\definecolor{mycolor3}{rgb}{0.70000,0.70000,0.70000}%
\begin{tikzpicture}
    \begingroup
    % inits/clears the lists (which might be populated from previous
    % axes):
    \csname pgfplots@init@cleared@structures\endcsname
    \pgfplotsset{legend cell align=left,legend columns = -1,legend style={at={(0,1)},anchor=north west,draw=black,column sep=0ex,/tikz/every even column/.append style={column sep=3ex}},
    legend entries={Naive,
    				Naive+,
    				Naive++}}%
    \csname pgfplots@addlegendimage\endcsname{line width=2pt,mycolor1,solid,sharp plot}
    \csname pgfplots@addlegendimage\endcsname{line width=2pt,mycolor2,solid,sharp plot}
    \csname pgfplots@addlegendimage\endcsname{line width=2pt,mycolor3,solid,sharp plot}
    % draws the legend:
    \csname pgfplots@createlegend\endcsname
    \endgroup
\end{tikzpicture}
%
\end{subfigure}%
  \caption[Comparison of the naive trading strategies]{Comparison of Naive (black), Naive+ (dark gray), and Naive++ (light gray) trading strategies. Plotted are normalized book values, averaged across the trading year, between 0930h (market open) and 1600h (market close). Book value is the sum of cash and market value of assets. Each day begins with zero book value, and all book values are normalized by dividing by the stock price at the start of each trading day.}\label{fig:comp}%
\end{figure}

\section{Conclusions from the Naive Trading Strategies}

\autoref{fig:comp} plots the average daily performance of the naive strategies for four stocks listed on the NASDAQ. The performance is normalized via division by the initial stock price on each trading day. The Naive strategy on average lost revenue, the Naive+ strategy (at-the-touch limit orders when no change was expected) on average generated revenue, and the Naive++ strategy (using limit orders to adversely select agents that traded against the price change momentum) on average neither profited nor lost.

{\bf Why is the Naive strategy producing, on average, normalized losses?} On calibration, we see that our intra-day Sharpe ratio is around 0.01 or 0.02 when we choose our optimal parameters, so at the very least on the calibration date the strategy produces positive returns. The remainder of the calendar days are out-of-sample to the calibration, meaning the parameters are not specifically tuned to that data, and thus the parameters are (likely) not optimal. This adds evidence to rejecting the time homogeneity assumption, and in particular suggests that not every day can be modelled by the same Markov chain. The problem may be exaggerated by the fact that we're calibrating on the first trading day of the calendar year, when we might expect reduced, or at least non-representative, trading activity. Further, we are using midprices to obtain the $\mat{Q}$ probability matrix while ignoring the bid-ask spread. Thus predicting a ``price change'' may be insufficient when considering a monetizable opportunity, as we won't be able to profit off a predicted increase followed by a predicted decrease unless the interim mid-price move is greater than the bid-ask spread (assuming constant spread); this flaw affects trading on \texttt{FARO} in particular, which has a spread of about 15 cents.

{\bf Why do the Naive+ and Naive++ strategies outperform the Naive strategy?} This is particularly interesting since the probabilities are being obtained from the same matrix. The obvious difference between the successful and unsuccessful strategies is that the former (a) uses limit orders, and (b) executes when we predict a zero change, whereas the latter uses (a) market orders, and (b) executes when we predict non-zero changes.

(a) leads to a different transaction price being used: a stock purchase with a limit order is executed at the bid price, while a purchase with a market order is at the ask price. Since the asset is marked-to-market at the more conservative price, and the mid price doesn't move as a result of the transaction, then a limit order purchases the share for the same value at which it is marked-to-market, whereas a market order `crosses the spread' and loses value.

(b) seems to be the largest flaw in the Naive strategy, to which there are two factors. One, we are not predicting the magnitude of the price change, only whether it is zero or non-zero. Two, from the probabilities presented above, \emph{we will only predict a price change if we've already seen a price change}. Thus we're effectively reacting too late. \autoref{tbl:adverse} presents a hypothetical series of events demonstrating the adverse effects of this flaw. Since the strategy is reacting to an already observed price change, the adverse effect would be exacerbated if the initial price change at timestep 4 were larger. All these considerations suggest potential modifications to the strategies.

\begin{table}
\centering
\ra{1.2}
\begin{tabular}{@{} *{7}{c} @{}}
\toprule
$t$ & $I(t)$ & Bid/Ask & Prediction & Action & Inv & P\&L \\
\midrule
0 & 1 & \hphantom{1}9.99/10.01 & $\P[\Delta S_{future} = 0] > 0.5$ & None & 0 & 0 \\
1 & 1 & 10.00/10.02 & $\P[\Delta S_{future} > 0] > 0.5$ & BUY @ 10.02 & 1 & -0.02 \\
2 & 0 & 10.01/10.03 & $\P[\Delta S_{future} = 0] > 0.5$ & None & 1 & -0.01 \\
3 & -1 & 10.01/10.03 & $\P[\Delta S_{future} = 0] > 0.5$ & None & 1 & -0.01 \\
4 & -1 & 10.00/10.02 & $\P[\Delta S_{future} < 0] > 0.5$ & SELL @ 10.00 & 0 & -0.02 \\
\bottomrule
\end{tabular}
\caption[Hypothetical timeline of adverse selection with market orders]{Hypothetical timeline of adverse selection with market orders.}
\label{tbl:adverse}
\end{table}